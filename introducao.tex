\chapter*[Introdução]{Introdução}\addcontentsline{toc}{chapter}{Introdução}

A adoção do computador como instrumento musical remontou, neste trabalho, às investigações de Max \citeonline{mathews_digital_1963,mathews_technology_1969,mathews_groove_1970}. Enquanto os dois primeiros apresentam fundamentos do Processamento de Sinais Digitais (DSP) para o fazer musical em \emph{tempo diferido},  o último apresenta a possibilidade de uma performance musical humana mediada pela máquina digital. Através de um sistema de retroalimentação, um ser humano era capaz de sintetizar sons, e manipulá-los em tempo de execução.

De certa forma, tais publicações permitiram a emergência de um \emph{Programa de Investigação Científica} \cite{lakatos_falsification_1970,neto_lakatos_2008} envolvendo Música e Ciências da Computação. Resultados de pesquisas posteriores, foram materializadas na elaboração de diversos ambientes de programação musical, entre eles, o CSound\footnote{Disponível em \url{https://csound.github.io/}.}, Max/MSP\footnote{Disponível em \url{http://cycling74.com/products/max}.}, PureData\footnote{Disponível em \url{http://puredata.info/}.} e SuperCollider\footnote{Disponível em \url{https://supercollider.github.io/}.}.

Deste último \emph{software}, foi possível extrair um \emph{macro-objeto} de pesquisa. Este macro-objeto é uma atividade social, uma prática, e um paradigmática científico, ou nas palavras de \citeonline{wiggins_framework_2006} ou Alex \citeonline{mclean_music_2006}, um \emph{Universo de possibilidades} para improvisações musicais.  Esta atividade peculiar ficou conhecida como \emph{livecoding}, ou:

\begin{citacao}
\emph{Programação imediata} (ou: programação de conversa, programação no fluxo, programação interativa) é um paradigma que inclue a atividade de programação ela mesma como uma operação do programa. Isto significa um programa que não é tomado como ferramenta que cria primeiro, e depois é produtivo, mas um processo de construção dinâmica de descrição e conversação - escrever o código e então se tornar parte da prática musical ou experimental. \cite[Verbete JITLib]{supercollider.org_supercollider_2014}\footnote{Tradução de \emph{Just in time programming (or: conversational programming, live coding , on-the fly-programming, interactive programming) is a paradigm that includes the programming activity itself in the program's operation. This means a program is not taken as a tool that is made first, then to be productive, but a dynamic construction process of description and conversation - writing code thus becoming a closer part of musical or experimental practice.}}
\end{citacao}

A definição acima é apropriada para um verbete de um documento técnico. Porém reduz o \emph{livecoding} ao modo de operação com o computador, e não descreve possíveis resultados musicais. Teoricamente, qualquer estética musical pode ser reproduzida durante uma sessão de improvisação (\emph{livecoding session}). Dependendo de como este universo é configurado, diferentes estéticas musicais podem emergir. Este tema será discutido no \autoref{cap:introducao}. 


No \autoref{cap:metodologia} delimito os conceitos deste universo, e derivo o \emph{espaço conceitual multidimensional} desta pesquisa, que envolve basicamente a correlação entre os \emph{universos de possibilidades} próprios do \emph{livecoding} e da sessão \emph{Study in Keith}.

No \autoref{cap:trabalhos_relacionados}, descrevo espaços conceituais históricos do \emph{universo de conceitos do livecoding}. Mais especificamente, os predecessores da prática, e a emancipação de um Programa de Investigação Científica específico deste universo. Esta investigação permitiu observar sua ideologia e sua heurística.

No \autoref{cap:estudos_de_caso} investigo um caso específico do compositor e programador australiano Andrew Sorensen. Dada a multiplicidade do conceito de \emph{livecoding}, foi necessário reduzir o escopo da pesquisa. Por outro lado, o exemplo de Sorensen é interessante por propor uma \emph{replicação} não só de um gênero musical, mas de um estilo idiossincrático de improvisação \emph{jazz}.