\chapter*[Introdução]{Introdução}\addcontentsline{toc}{chapter}{Introdução} 

Na transição do séc. XX para o XXI,  DJs e programadores ingleses utilizam linguagens de programação para improvisação musical; posteriormente agregam esta prática para uma improvisação audiovisual. 

\citeonline{collins_algorave:_2014} descreve vários momentos para aquilo que ficou conhecido como o lado \emph{algorave} da prática \emph{live coding}; irei pontuar apenas três. No ano de 1997, \emph{Aphex Twin} (Richard David James) utiliza o programa \emph{SuperCollider 1} para produzir um \emph{live club algorithm}; no ano 2000, o duo \emph{Slub} (Adrian Ward, Alex McLean, e poteriormente um trio, com Dave Grifths), utilizam \emph{laptops} para programar uma replicação de estilos como \emph{techno} e \emph{gabba}; em uma boate em Hamburgo, 2004, é fundada uma organização internacional chamada TOPLAP.

Princípios desta organização são publicados e divulgados por \citeonline{ward_live_2004,griffiths_fluxus:_2008,mccallum_show_2011} como regras heurísticas do \emph{live coding}, de como lidar com instruções computacionais, improvisadas ou escritas previamente, para uma performance artística; são aceitas como regras bastante razoáveis para um número de praticantes. No entanto, seria possível discutir se tais prescrições são heranças de um período anterior à formalização do \emph{live coding}, período nomeado como \emph{Live Computer Music}:


\traduzcitacao{A idéia de usar sistemas eletrônicos eles mesmos como atores musicais, oposto a meramente uma ferramenta, iniciou com compositores como David Tudor e Gordon Mumma. Uma continuação natural desses exemplos pode ser encontrada em compositores locais $[$Baía de São Franscisco$]$ que tocaram com sintetizadores analógicos auto-modificados interligados. Um desess executantes foi o falecido Jim Horton (1994-1998)}
{The idea of using the electronic system itself as a musical actor, as opposed to merely a tool, had started with composers like David Tudor and Gordon Mumma. A natural continuation of their example could also be found in the local composers who performed with self-modifying analog synthesizer patches as well. One of these players was the late Jim Horton (1944-1998)}
{online}
{brown_indigenous_2013}

\emph{Water Surface} (1985), de Ron Kuivila  se utiliza da atividade de programação para expor, durante o processo musical, uma falha do sistema FORTH, que finaliza com um sobrecarregamento e um consequente silêncio \cite{toplapcd_001_2007,collins_origins_2014}. 

John Bischoff e Tim Perkis (membros do grupo \emph{the Hub}), no final da década de 70, ajustavam uma rede de microcontroladores como parte de um \emph{happening} \cite{brown_indigenous_2013}.  

``Spiral 5 PTL (Perhaps The Last)'' (1979), de Dan Sandin, Tom DeFanti, e Mimi Shevitz é atualmente discutido como primeira performance de \emph{live coding}, ao gerar interativamente materiais audiovisuais com a linguagem GRASS\footnote{Disponível em \url{http://lurk.org/groups/livecode/messages/topic/5abPazJSxfegYfVFOzN4T6}.}. 

Por outro lado, \citeonline{mori_pietro_2015} apresenta o compositor Pietro Grossi (1917-2002) como um caso histórico de \emph{live coding} anterior aos casos citados. %Neste sentido, seria possível discutir se o problema da performance com o computador já estava sendo discutido por \citeonline{mathews_groove_1970} com seu sistema GROOVE.

\section*{Problema}

Existem múltiplos contextos de definição do que é, e quando começou o \emph{live coding}. Uma definição amplamente divulgada, através do \emph{software} SuperCollider\footnote{Disponíveis em \url{https://supercollider.github.io/}.} apresenta aspectos técnicos, mas não históricos:

\begin{citacao}
\emph{Programação imediata} (ou: programação de conversa, programação no fluxo, programação interativa) é um paradigma que inclue a atividade de programação ela mesma como uma operação do programa. Isto significa um programa que não é tomado como ferramenta que cria primeiro, e depois é produtivo, mas um processo de construção dinâmica de descrição e conversação - escrever o código e então se tornar parte da prática musical ou experimental. \cite[Verbete JITLib]{supercollider.org_supercollider_2014}\footnote{Tradução de \emph{Just in time programming (or: conversational programming, live coding , on-the fly-programming, interactive programming) is a paradigm that includes the programming activity itself in the program's operation. This means a program is not taken as a tool that is made first, then to be productive, but a dynamic construction process of description and conversation - writing code thus becoming a closer part of musical or experimental practice.}}
\end{citacao}

\citeonline{ward_listen_2003} definem o \emph{live coding} como uma Música computacional ao vivo com performance visual, controladas por processos algorítmicos\footnote{Adaptação de \emph{Live computer music and visual performance can now involve interactive control of algorithmic processes.}}. \citeonline{ward_live_2004} definem como \traducao{atividade da escrita (ou partes de) um programa enquanto ele é executado}{Live coding is the activity of writing (parts of ) a program while it runs}. O que concorda com a definição de \citeonline{sorensen_programming_2014}, ``programar sistemas de tempo-real durante o tempo de execução''\footnote{Traduçao adaptada de \emph{programming real-time systems in real-time}. A tradução literal ``programar sistemas de tempo-real em tempo-real'' nos pareceu pouco explicativa. De fato, esta expressão possui um sentido mais completo na língua inglesa, mas pode confundir o leitor de língua portuguesa. Substituímos o \emph{in real-time} por \emph{during runtime} (durante o tempo de execução, de um \emph{software}) para enfatizar a produção de códigos durante o momento em que está sendo feito. Esta questão pode ser melhor compreendida a partir do GROOVE de \citeonline{mathews_groove_1970}.}. \citeonline{mclean_hacking_2006,collins_algorave:_2014} discutem o \emph{live coding} pela perspectiva da música eletrônica de entretenimento, ou  \emph{algorave}. \citeonline{magnusson_algorithms_2011,collins_origins_2014} sintetizam o \emph{live coding} como improvisação audiovisual.  

Por último, \citeonline{mori_analysing_2015,prospero_social_2015} delineiam o estudo do \emph{live coding} como um ramo da etnomusicologia. Os autores discutem a utilização da etnografia (a vivência com uma comunidade em observação, os \emph{live coders}), e a revisão histórica, como métodos para a re-discussão dos limites do que é \emph{live coding}.

\section*{Hipótese}

Neste trabalho definimos o \emph{live coding} como um \emph{Universo de Conceitos}, contendo diversos limites entre seus \emph{Espaços conceituais}, próprios para diferentes \emph{Modelos de Improvisação}. Seus intérpretes programam sistemas criativos durante o tempo de sua execução. 

Como exemplo, selecionamos \emph{Study in Keith} de Andrew \citeonline{sorensen_keith_2009} para análise.

\section*{Estrutura dos capítulos}

No \autoref{cap:introducao} discutimos  várias abordagens do \emph{live coding}, seguido por elementos históricos. \emph{Espaços Conceituais}.  No \autoref{cap:metodologia} a formalização destes espaços conceituais é realizada através  do Modelo de Improvisação de Alex \citeonline{mclean_music_2006}. No \autoref{cap:estudos_de_caso}, \emph{Study in Keith} (2009) de Andrew Sorensen \footnote{Disponível em \url{https://vimeo.com/2433947}.} representa um caso particular que envolve, além de música, \emph{replicação do estilo} de improvisação de \emph{jazz} de Keith Jarret, durante os concertos \emph{Sun Bear} (1976-1979), para uma sessão de improvisação com o computador.

