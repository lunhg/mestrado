\chapter*{Introdução}\addcontentsline{toc}{chapter}{Introdução}\label{cap:intro}

Giovanni \citeonline[p.~117]{mori_analysing_2015} define a improvisação de códigos em relação à Dança, Música, Imagens em Movimento (animações, vídeos) e/ou Tecelagem. É importante esclarecer que essa definição denota a polivalência de uma técnica:

\traduzcitacao{\emph{Live coding} é uma técnica artística de improvisação. Pode ser empregada em muitos contextos diferentes de performance: dança, música, imagens em movimento e mesmo tecelagem. Eu concentrei minha atenção no lado musical, que parece ser o mais proeminente.}{Live coding is an improvisatory artistic technique. It can be employed in many different performative contexts: dance, music, moving images and even weaving. I have concentrated my attention on the music side, which seems to be the most prominente}

Assim como Mori, situamos o \emph{live coding} -- \emph{codificação ao vivo} --, ou como sugerimos chamar, \emph{improvisação de códigos}, como foco de proposições musicais. Porém uma abordagem especialista não condiz com as múltiplas estéticas observadas. Durante a pesquisa, fomos desafiados a encarar uma técnica que permite produzir (predominantemente) Música Eletrônica para Dançar, Música Eletroacústica, Música-Erro, Música-Ruído\footnote{\cfcite{lunhani_apontamentos_2015}.}.

Existir um estudo de caso, como exigido de um programa de pesquisa, é questionável pela própria definição de Mori. Desta forma, buscamos oferecer ao leitor as três perguntas abaixo, que nortearam uma divisão dos capítulos, com base na definição do pesquisador italiano:

• Existe ao menos um exemplo, e seu artista-programador\footnote{\cfcite{McLean2011}}, para cada artefato artístico \footnote{\cfcite{prospero_social_2015}} não-musical?

• Existem exemplos, e artistas programadores, que justifiquem, do ponto de vista histórico, a predominância do caráter musical de improvisações de códigos?

• É possível analisar uma improvisação de códigos, do ponto de vista de suas proposições musicais, materializadas em um algoritmo gerador de sonoridades?

\section*{Capítulo 1}

\subsection*{Tecelagem}

Contextualizar a tecelagem é uma forma de recontar os primeiros códigos de computador \ver{sec:tecelagem}. Nas palavras do improvisador de códigos Dave \citeonline{griffths_weave2_2015},

 \traduzcitacao{Um dos potenciais da tecelagem que eu fiquei mais interessado é a capacidade de demonstrar fundamentos de \emph{softwares} por fios -- parcialmente tornar a natureza física da computação auto-evidente, mas também como uma maneira de modelar novas formas de aprender e a entender o que são os computadores}{One of the potentials of weaving I’m most interested in is being able to demonstrate fundamentals of software in threads – partly to make the physical nature of computation self evident, but also as a way of designing new ways of learning and understanding what computers are.}

Por outro lado, é uma forma de indicar artistas-programadores influentes, entre eles Alex McLean, Adrian Ward e Dave Griffths. Os dois primeiros formaram a banda \emph{Slub} nos anos 2000, com a seguinte premissa: utilizar a atividade de programação para realizar uma Música Eletrônica para Dançar. Sua primeira reunião foi em 2001, no \emph{Paradiso club} em Amsterdã, durante o festival \emph{Sonic Arts}. Em 2005 o duo participa do festival \emph{Sonar}, sendo que Griffths é convidado a ser membro oficial, o que abre espaço para a inclusão de novas formas práticas, como \emph{games} \cite[p.~138--140]{McLean2011}, e tecelagem.

\subsection*{Dança}\addcontentsline{toc}{section}{Dança}

A discussão sobre Dança é uma forma de ilustrar a improvisação de códigos através de organizações, algumas vezes ligadas às universidades, através de dois exemplos aparentemente distintos.

O primeiro é um subgênero da Música Eletrônica para Dançar (\emph{algorave}). Neste contexto, existe uma tendência em colocar a figura do artista-programador, muitas vezes (mas nem sempre) um pesquisador em alguma instituição acadêmica, como um \emph{Disk Jockey}. Na \autoref{sec:laptoptoplap} (ver p.~\pageref{sec:laptoptoplap}), esclarecemos a origem desta representação.

 O segundo exemplo é um caso recente, de forma que não encontramos casos similares. O código de uma coreografia é improvisada por Kate \citeonline{kate_htb_2015}, que controla uma humana com um dispositivo tátil. A novidade deste trabalho nega o som como resultado da improvisação. No entanto, o  ambiente acadêmico \cite{ICLC2015}, e um formato de apresentação tradicional, são mantidos.

%É importante mencionar que, no escopo da pesquisa de Sichio, são raras as referências aos aspectos sonoros. Diferente das indicações estritas de um pentagrama musical, a partitura de dança ofereçe mais um guia para os movimentos do corpo. Isso significa sugerir que, do ponto de vista computacional, a partitura coreográfica é mais próxima do código escrito em uma linguagem de computador do que a partitura musical tradicional. É curioso notar que ocorre como uma assimilação do pensamento algorítmico na Dança, chamado por \citeonline[p.~31]{sichio_hacking_2004}, através de \citeonline[cap.~1, p.~3]{downie_choreography_2005}, de \emph{Sensibilidades Computacionais}. Ou dispositivos metafóricos elaborados por coreógrafos como Merce Cunningham, Trisha Brown, Bill T. Jones, e William Forsythe -- \traducao{mecanismos de generalização e abstração, representação da coreografia e dança como computação}{mechanisms of generalization and abstraction, choreography as representation, dance as computation} \cite[cap.~1, p.~2--4]{downie_choreography_2005}:usica}

%\section*{Capítulo 2:}\addcontentsline{toc}{section}{Música Computacional}

%Do ponto de vista da Música, apresentamos três tendências da música computacional através de improvisadores de códigos brasileiros, fomentados por Universidades Públicas. 

%O primeiro é o grupo \emph{FooBarBaz}, formado por Gilson Beck, Renato Fabbri, Ricardo Fabbri e Vilson Vieira. Consideramos este grupo como representativo, no Brasil, das regras institucionalizadas por \citeonline[ver \protect\autoref{sec:foobarbaz}, p.~\protect\pageref{sec:foobarbaz}]{ward_live_2004}. Sua particularidade toca no fato de serem um grupo híbrido de compositores, físicos e cientistas da computação.

%O segundo exemplo é a performance \emph{screenBashing} de Magno Caliman. Ela ilustra a figura do compositor solista, em um caso particular de uma \traducao{(\ldots) performance de \emph{livecoding} arquetípica $[$que$]$ envolve programadores escrevendo códigos no palco, com suas telas projetadas para a audiência.}{The archetypal live coding performance involves programmers writing code on stage, with their screens projected for an audience.}\cite[p.~1, ver \protect\autoref{sec:concerto}, p.~\protect\pageref{sec:concerto}]{mclean_tidal_2010}. 

%O terceiro apresenta uma abordagem virtual, isto é, uma improvisação de códigos realizada em locais diferentes, com computadores diferentes, através de uma conexão de \emph{internet}. Esta abordagem não é nova, mas ilustra a correlação entre uma universidade pública brasileira e uma instituição estadounidense \ver{sec:telepresenca}.

%FooBarBaz é um grupo de improvisação de códigos formado por Gilson Beck, Renato Fabbri, Ricardo Fabbri e Vilson Vieira. Sua primeira apresentação foi durante o Festival Contato 2011. Os membros são ativos em um laboratório virtual conhecido como \emph{labMacambira}\disponivelem{http://labmacambira.sourceforge.net/}. Além de cientistas, participam \emph{hackativistas}, ex-programadores do Google, músicos e artistas plásticos interessados em processos criativos com assistência computacional. É interessante notar que as atividades do FooBarBaz estão sincronizadas com algumas das ideologias da improvisação de códigos, com base em regras práticas \ver{sec:showusyourscreens}. Outro ponto interessante deste grupo foi a elaboração de um manifesto próprio, chamado de \emph{Manifesto Freakcoding}, que inclue, como parte executável do manifesto, um ambiente de programação audiovisual chamado \emph{Vivace} \cite{vieira_vivace:_2015}.

%A performance \emph{screenBashing} de Magno Caliman (ver \autoref{fig:screenbashing}) foi realizada durante o XIII ENCUN\footnote{Encontro Nacional de Compositores Universitários em Campinas-SP no ano de 2015.}. A performance consiste no seguinte: Caliman senta-se ao computador, lateralmente à tela de projeção, com uma iluminação de penumbra. O projetor expõe o estado atual de seu \emph{laptop}, que apresenta um editor de texto. O executante começa a programar em linguagem C (ver exemplo abaixo).

%Uma performance virtual de \emph{live coding} é aquela em que dois ou mais executantes, em endereços diferentes de uma rede de computadores. Isso situa três casos, do qual especificaremos um: i) uma rede local, com computadores diferentes, mas com os improvisadores fisicamente próximos; ii) uma rede remota, privada, que comunica um conjunto de pessoas fisicamente distantes; iii) a rede mundial de computadores, onde o navegador se torna o ambiente virtual de criação musical \cite{roberts_web_2013}. Nos três casos, a premissa é compartilhar o mesmo código entre improvisadores-programadores. Podem também compartilhar do mesmo som, mas isso depende de implementações técnicas . 

\section*{Capítulo 2}

\subsection*{Proto-História}

Na comunidade de improvisadores de códigos é colocada em discussão a origem da técnica com fins artísticos. Entre elas, obras audiovisuais de Tom de Fanti, em 1976 \disponivelem{http://lurk.org/groups/livecode/messages/topic/5abPazJSxfegYfVFOzN4T6/}. Tais origens determinam um período proto-histórico da técnica \cite{ward_live_2004}. 

Um consenso da origem na Música é relacionado à performance \emph{Water Surfaces}, do compositor estadounidense Ron Kuivila. Uma desconstrução desta idéia, feita por Giovanni Mori, sugere que o compositor italiano Pietro Grossi elaborou, no começo dos anos setenta, as primeiras experiências formais com um paradigma menor da \emph{Computer Music}, em contraste com aquele paradigma maior formado pela divulgação da família MUSIC N de \citeonline{mathews_digital_1963,mathews_technology_1969}. 

Outros paradigmas menores também são formados através da \emph{Live Computer Music} da Baía de São Franscisco, durante o final da década de 1970, e meados da década de 1980, com o grupo \emph{The League of Automatic Composers}, embrião de outro, \emph{The Hub}.
 
Um hiato de duas décadas (1980-2000) não será discutido pois carecemos de informações mais precisas. Ao que foi possível determinar como origem de uma heurística, apontamos um embate acadêmico. Este embate, que surgiu como uma crítica ao aspecto cênico das performances com computadores, foi usado como estímulo a elaboração de um conjunto de regras. Do trabalho publicado por \citeonline{schloss_dilemma_2003}. sete artistas-programadores ingleses McLean, Griffths, Amy Alexander, Adrian Ward, Fredrik Olofsson, Julian Rohrhuber e Nick Collins, responderam através do manifesto \emph{Lubeck04}, ou \emph{Show us your screens}  \cite{ward_live_2004}.

\section*{Capítulo 3} 

\subsection*{A Study in Keith (2009)}

Existem muitos códigos passíveis de análise. A dificuldade principal é, como discutir aspectos estéticos tão diversos entre si? Nosso caminho foi, não analisar a música em si, mas sim sua proposição, e o algoritmo gerador de sonoridades, dialogam.  

Para simplificar o método de análise, selecionamos \emph{A Study in Keith} \cite{sorensen_keith_2009,sorensen_youtube_2014}, por colocar a figura do intérprete concertista de \emph{jazz}, readequado para os propósitos do artista-programador, o que consideramos como um caso excepcional em um meio permeado de sínteses sonoras ou colagens. O registro audiovisual principal segue a seguinte proposição: após a escuta dos Concertos \emph{Sun Bear} de Keith Jarret, é improvisado um código, com o ímpeto de automatizar uma improvisação pianística, fato que se consolidou com suas \emph{Disklavier Sessions} (2011).

%Embora inspiradas pela atividade perceptiva, o resultado de \emph{A Study in Keith} não guarda nenhuma relação harmônica ou melódica com o pianista estadounidense. Nosso interesse está mais em analisar a proposição que resulta no algoritmo gerador dos três primeiros blocos sonoros de \emph{A Study in Keith}.