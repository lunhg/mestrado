\titulo{\emph{Live Coding}: um algoritmo gerador de uma sonoridade tonal no \emph{A Study in Keith} (2009) de Andrew Sorensen}
\autor{Guilherme Martins Lunhani}

\instituicao{Universidade Federal de Juiz De Fora -- UFJF
  \par
  Instituto de Artes e Design -- IAD
  \par
  Programa de Pós-Graduação em Artes Visuais, Música e Tecnologia}

\orientador[Orientador: ]{Prof. Dr. Luiz Eduardo Castelões}

% \changes{Versão inicial }{2013/07/22 }{v0.0.3}
\tipotrabalho{Dissertação (Mestrado)}

% O preambulo deve conter o tipo do trabalho, o objetivo, 
% o nome da instituição e a área de concentração 
\preambulo{Dissertação apresentada ao Programa de Pós-Graduação em Artes, Cultura e Linguagens, Área de Concentração: Teorias e Processos Poéticos Interdisciplinares, Linha de pesquisa: Artes Visuais, Música e Tecnologia, da Universidade Federal de Juiz de Fora, como requisito parcial para obtenção do grau de Mestre}
%\EnableCrossrefs
%\CodelineIndex
%\RecordChanges

% ---
% Configurações de aparência do PDF final

% alterando o aspecto da cor azul
\definecolor{blue}{RGB}{41,5,195}

% informações do PDF
\makeatletter
\hypersetup{
     	%pagebackref=true,
		pdftitle={\@title}, 
		pdfauthor={\@author},
    	pdfsubject={\imprimirpreambulo},
	    pdfcreator={LaTeX with abnTeX2},
		pdfkeywords={abnt}{latex}{abntex}{abntex2}{trabalho acadêmico}, 
		colorlinks=true,       		% false: boxed links; true: colored links
    	linkcolor=blue,          	% color of internal links
    	citecolor=blue,        		% color of links to bibliography
    	filecolor=magenta,      		% color of file links
		urlcolor=blue,
		bookmarksdepth=4
}
\makeatother

%\newcommand{\todosautoresdelivecoding}{\begin{inparaenum}[]\item \citeonline{collins_live_2003},\item \citeonline{collins_generative_2003},\item \citeonline{collins_live_2003-1},\item \citeonline{wang_--fly_2004},\item \citeonline{ward_live_2004},\item \citeonline{blackwell_programming_2005},\item \citeonline{collins_live_2007},\item \citeonline{griffiths_fluxus:_2008},\item \citeonline{mclean_patterns_2009},\item \citeonline{rohrhuber_improvising_2009},\item \citeonline{mclean_visualisation_2010},\item \citeonline{magnusson_algorithms_2011},\item \citeonline{mccallum_show_2011},\item \citeonline{magnusson_herding_2014},\item \citeonline{magnusson_scoring_2014},\item \citeonline{collins_algorave:_2014},\item \citeonline{sorensen_livecodings_2014}\end{inparaenum}}
% --- 
% Espaçamentos entre linhas e parágrafos 
% --- 

% O tamanho do parágrafo é dado por:
\setlength{\parindent}{1.3cm}

% Controle do espaçamento entre um parágrafo e outro:
\setlength{\parskip}{0.2cm}  % tente também \onelineskip

% ---
% compila o indice
% ---
\makeindex
