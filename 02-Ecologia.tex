\subsection{Live Algorithm Programming and Temporary Organization for its Promotion}

O manifesto ``\emph{Live Algorithm Programmin and Temporary Organization for its Promotion}'' \cite{ward_live_2004} fornece informações a respeito das ideologiaS. Um caso interessante ,  realizam o seguinte comentário:

\begin{citacao}
Contudo, alguns músicos exploram suas idéias como processos de \emph{software}, muitas vezes ao ponto que o \emph{software} se torna a essência da música. Neste ponto, os músicos podem ser pensados como programadores explorando seu código manifestado como som. Isso não reduz seu papel principal como um músico, mas complementa, com a perspectiva única na composição de sua música. \textbf{Termos como ``música generativa'' e ``música de processador'' tem sido inventados e apropriados para descrever esta nova perspectiva de composição}. Muita coisa é feita das supostas propriedades da chamada ``música generativa'' que separa o compositor do resultado do seu trabalho. Brian Eno compara o fazer da música generativa com o semear de sementes que são deixadas para crescer, e sugere abrir mão do controle dos nossos processos, deixando eles ``brincarem ao vento''. \footnote{\opcit[p.~245-246]{ward_live_2004}. Tradução nossa de \emph{Indeed, some musicians explore their ideas as software processes, often to the point that a software becomes the essence of the music. At this point, the musicians may also be thought of as programmers exploring their code manifested as sound. This does not reduce their primary role as a musician, but complements it, with unique perspective on the composition of their music. Terms such as “generative music” and “processor music” have been invented and appropriated to describe this new perspective on composition. Much is made of the alleged properties of so called “generative music” that separate the composer from the resulting work. Brian Eno likens making generative music to sowing seeds that are left to grow, and suggests we give up control to our processes, leaving them to “play in the wind”.}}
\end{citacao}

A partir desta citação, é possível incluir outros dois itens aos \emph{1)} e \emph{2}) apresentados no primeiro parágrafo desta seção: \emph{3)} Música Generatva (MG); e \emph{4)} Música de Processador, ao qual tomarei como um empréstimo conceitual de Música Processual (MP).

Admitindo de antemão esta multiplicidade de técnicas criativas e gêneros musicais na relação \emph{live coder} $\Leftrightarrow$ computador $\Leftrightarrow$ público, reformularei novamente perguntas que tenho feito até aqui como, \textbf{por meio de quais técnicas criativas o \emph{livecoding} se manifesta como um universo de conceitos, na forma de gêneros musicais?} 

O método utilizado nessa sessão para investigar técnicas criativas incluídas no universo de conceitos do \emph{live coding}, é procurar por menções em de autores a respeito das \emph{Música Algorítmica} (que será abreviada para MA, ver \autoref{apropriacao_algoritmica}), \emph{Música Generativa}/Música Processual (MG, MP, ver \autoref{musica_generativa}, \autoref{alg_simples} e \autoref{alg_complexo}), e \emph{code DJing} (DJ, ver \autoref{musica_vanguarda_pista}). Estes termos serão contra-argumentados com outros autores, entre eles \citeonline{reich_music_1968}, \citeonline{eno_music_1978}, \citeonline{kramer_sonification_1999}, \citeonline{roads_times_2001}, \citeonline{wooler_framework_2005} \citeonline{malt_concepts_2006}, \citeonline{walker_auditory_2006}, \citeonline{essl_algorithmic_2007},  \citeonline{cope_prefacio_2008}, \citeonline{iazzetta_musica_2009}, \citeonline{mailman_agency_2013} e \citeonline{collins_algorave:_2014} \citeonline{casteloes_conversores_2015}. 