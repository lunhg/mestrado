%\begin{agradecimentos}
\newpage
\begin{flushright}
\huge{\textbf{Agradecimentos}}

\small{Ao Inominável.
\ \\
À minha família, Jair e Júlia, por todo tipo de apoio, amo vocês! 
\ \\
Aos amigxs que estão (ou moraram em Juiz de Fora) e que foram fundamentais (de alguma forma) no caminho: Glerm, Anna Flávia, Tiago Rubini, Aline. Aos amigxs de Campinas e São Paulo que de alguma forma me ajudaram ultrapassar a distâncias, nostalgias e amizades: Celso, Dani, Evandro, Fábio (sivuca), Felício, Frederico (Doshi, Larissa (zé e diva), Rebechi (vivi), Simone (foucault), Tati (Ruan e Ruanzitx).
\ \\
Aos vizinhos que riem apenas de ver a lua, branca, amarela, ou vermelha, Dhiego e Luisa. Ao Gustavo e Patrick (Pretinha). Aos colegas de mestrado (ou relativos), Diego, Nayse, Analu e Paula. Ao velho amigo que não vejo a muito tempo, Bruno.
\ \\
Aos Professores Dr. Luiz Eduardo Castelões, Dr. Alexandre Fenerich e Dr. Flávio Luiz Schiavonni por serem fundamentais, no apoio institucional; na sugestão de leituras; na cobrança de prazos; nas críticas; nas conversas sobre Música, Universidade e Tecnologia; ou até mesmo pelo sanduíche de queijo quando a bolsa não caiu. À FAPEMIG por suprir esta lacuna, em um momento delicado nas finanças da Universidade Brasileira.
\ \\
À Professora Rosane Preciosa. Seus brilhos deram novos significados ao \emph{Momentum} de Stockhausen. Ao Professor Dr. Elpp Aravena e Tonil Braz, Engenheiros da Cumbia.
\ \\
Ao Professor Hans Joachim Koellreutter pelo centenário. Tocar sua peça ajudou a começar a perceber a Via.
\ \\
Àqueles que passaram pelo caminho, que ajudaram ou atrapalharam. Suas potências permitiram um novo espaço conceitual.
\ \\
Às arrudas, alecrins, ipês e cebolinhas que enfeitaram o universo de possibilidades.}
\end{flushright}

\vfil \ 

\newpage