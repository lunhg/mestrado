]%\begin{agradecimentos}
\newpage
\begin{flushright}
\huge{\textbf{Agradecimentos}}

\small{Para aquele que me protegeu em duros momentos, Jesus.  
\ \\
Aos sem nome, anônimos da Rua de Juiz de Fora, que deram sentido à fraqueza da pergunta deste trabalho.
\ \\
Para uma família, pelo apoio, Jair, Olímpia e Júlia. 
\ \\
Aos Professores Dr. Luiz Eduardo Castelões, Dr. Alexandre Fenerich e Dr. Flávio Luiz Schiavonni, fundamentais no apoio institucional; na sugestão de leituras; na cobrança de prazos; nas críticas; nas conversas sobre Música. 
\ \\
À FAPEMIG por suprir a lacuna financeira, em um momento delicado nas economias da Universidade Brasileira.
\ \\
Aos amigxs que estão (ou moraram em Juiz de Fora): Glerm Soares, Tiago Rubini, Anna Flávia. Aos amigxs de Campinas e São Paulo, que estiveram presentes ou na memória: Celso, Dani, Dhiego e Luisa, Evandro, Fábio, Felício, Frederico, Gustavo, Ivan, Israel, Larissa, Rebechi, Simone, Tati,  ao pessoal da república Lado C, João, Heron, Igor. Ao velho amigo Picchi!
\ \\
Aos freakcoders do \emph{labMacambira}, especialmente ao Caleb Luporini, Daniel Penalva, Renato Fabbri e Vilson Vieira pelo estímulo nestes anos. 
\ \\
Aos colegas e amigos do LABICbr, principalmente Angelica Rimenez, Felipe Caracas, Carlos Lobo, Carlos Rivera, 
Ivo Santiago. Lucas Araújo, Pedro Garbelini, Raquel Pires, Rafael Cortez. Sandra Leão, Mario Alzate,  Oda Scaltoni, Vitor Grilo, 
\ \\
Ao Professor Hans Joachim Koellreutter pelo centenário. Mesmo não conhecendo-o pessoalmente, seus escritos foram essenciais.}
\end{flushright}

\newpage