\chapter{Código fonte de um Universo Conceitual como nuvem de palavras sobre o improviso de códigos}\label{app:A}

O tema da improvisação de códigos, como um universo de conceitos, surgiu de uma experiência com um código em linguagem \emph{python}\disponivelem{https://www.python.org/}, útil para gerar um mapa de termos, como ilustrado na figura \autoref{fig:nuvemlivecoding}

\begin{figure}[!h]
\includegraphics[scale=0.8]{imagens/nuvem.png}
\caption{Nuvem de palavras do \citeonline{ICLC2015},  1$^o$ Congresso Internacional de Live Coding. \textbf{Fonte}: autor.}
\label{fig:nuvemlivecoding}
\end{figure}

A imagem acima foi gerada com um código nomeado como \emph{cloupdf.py}, e considera seguinte situação, subdividida em três passos: 

1) Converter um arquivo de texto, ou um conjunto de textos cientítificos, em formato \emph{.pdf} para formato \emph{.txt}; 

2) Feita a conversão, plotar uma imagem com as palavras mais relevantes, do ponto de vista quantitativo; 

3) Desta plotagem, organizar as palavras qualitativamente, de acordo com suas funções gramaticais no texto, em classes quantitativas (ver \autoref{tab:gen1}, p.~\pageref{tab:gen1}; \autoref{tab:gen2}, p.~\pageref{tab:gen2}; \autoref{tab:gen3}, p.~\pageref{tab:gen3}; \autoref{tab:gen4}, p.~\pageref{tab:gen4}; \autoref{tab:gen5}, p.~\pageref{tab:gen5}).

\input{./cloupdf.tex}

\section{Utilização}

Em um terminal Linux (3.16.0-49-generic, Ubuntu 14.04.1, i686), executamos o código como um comando  com opções de: 1) arquivo de entrada (\verb|--entrada| ou \verb|-e|), número de páginas rastreadas (\verb|--paginas| ou \verb|-p|), classes (\verb|--qualidades| ou \verb|-q|), codificação do pdf (\verb|--codec| ou \verb|-c|), criação de uma nuvem de palavras (\verb|--foto| ou \verb|-f|) e organização de uma tabela (\verb|--table| ou \verb|-t|). 

Este código utilizou três bibliotecas auxiliares \emph{pdf2text.py}\disponivelem{https://pypi.python.org/pypi/pdf2text}, \emph{Wordcloud}\disponivelem{https://github.com/amueller/word_cloud} e \emph{NLTK}\disponivelem{http://nltk.org/}. 

A primeira biblioteca permite extrair do pdf caracteres válidos para análise. A segunda biblioteca realiza o levantamento de dados. E a terceira biblioteca auxilia na organização das funções gramaticais. Essas operações serão exemplificadas na próxima seção.

\section{Experiências}

Durante uma primeira experiência, abrimos um arquivo pdf, mais especificamente os anais de um congresso internacional \cite{ICLC2015}, e organizamos algumas páginas pertinentes, neste caso, todas a páginas com o corpo de texto (p. 4 -- 230),  supostamente codificado em um padrão ISO8895-1 \ver{fig:nuvemlivecoding}.

\begin{example}{Código-fonte do \emph{cloud.py}}
\begin{minted}[fontsize=\scriptsize]{bash}
./cloud.py -e ./iclc2015-proceedings.pdf -p 4..231 -f
\end{minted}
\end{example} 

Uma segunda experiência permitiu organizar a mesma nuvem de palavras em um uma tabela de funções gramaticais. As duas experiências dispararam a intenção de realizar uma pesquisa onde fosse possível verificar os mesmos conceitos em uma bibliografia geral, como feita nos \autoref{cap:introducao} e \autoref{sec:protohistoria}.

\begin{example}{Código-fonte do \emph{cloud.py}}
\begin{minted}[fontsize=\scriptsize]{bash}
./cloud.py -e ./iclc2015-proceedings.pdf -q 10 -p 4..231 -c iso8859-1 -t tex
\end{minted}
\end{example} 

A execução do comando acima gera a seguinte saída de texto:

\begin{example}{Saída de texto do \emph{cloud.py}}
\begin{minted}[fontsize=\scriptsize]{bash}
A converter ./iclc2015-proceedings.pdf para ./iclc2015-proceedings.txt  ... 

=> pdf2txt.py -o /home/guilherme/bitbucket/mestrado/iclc2015-proceedings.txt 
-p 4,5,6,7,8,9,10,11,12,13,14,15,16,17,18,19,20,21,22,23,24,25,26,27,28,29,30,31,
32,33,34,35,36,37,38,39,40,41,42,43,44,45,46,47,48,49,50,51,52,53,54,55,56,57,58,
59,60,61,62,63,64,65,66,67,68,69,70,71,72,73,74,75,76,77,78,79,80,81,82,83,84,85,
86,87,88,89,90,91,92,93,94,95,96,97,98,99,100,101,102,103,104,105,106,107,108,109,
110,111,112,113,114,115,116,117,118,119,120,121,122,123,124,125,126,127,128,129,130,
131,132,133,134,135,136,137,138,139,140,141,142,143,144,145,146,147,148,149,150,151,
152,153,154,155,156,157,158,159,160,161,162,163,164,165,166,167,168,169,170,171,172,
173,174,175,176,177,178,179,180,181,182,183,184,185,186,187,188,189,190,191,192,193,
194,195,196,197,198,199,200,201,202,203,204,205,206,207,208,209,210,211,212,213,214,
215,216,217,218,219,220,221,222,223,224,225,226,227,228,229,230 
-c iso8859-1 /home/guilherme/Dropbox/Mestrado/livecoding/iclc2015-proceedings.pdf
=> ... checking ascii characters
=> Feito

A gerar nuvem ...
=> Feito

A classificar e organizar palavras em uma tabela
=> Feito
\end{minted}
\end{example}

\newpage

\begin{table}
\centering
\caption{VB -- Verbo, forma básica. VBZ -- presente na terceira pessoa do singular.}
\label{tab:gen1}
\small
\begin{tabular}{ | p{2.6cm} | p{2.1cm} | p{2.1cm} | p{1.5cm} | p{0.5cm} | p{0.5cm} | p{0.25cm} | p{0.25cm} | p{0.25cm} | p{0.25cm} | p{0.25cm} | p{0.75cm} |}
\hline
\hline
\tiny \textbf{Qualidade/Função}
 & \textbf{0}
 & \textbf{1}
 & \textbf{2}
 & \textbf{3}
 & \textbf{4}
 & \textbf{5}
 & \textbf{6}
 & \textbf{7}
 & \textbf{8}
 & \textbf{9}
 & \textbf{10} \\ 
\hline
\hline
\tiny \textbf{VB} & \tiny See, Take, Allow, Make, Explore, Provide, Change, Support, Result, Become, Play, Create, Set, Laptop, Show, Project, Different, Type, Output, Object, Present, Point, Parameter, Structure, Memory, Need, Feature, Cognitive, Open, Interface, End, Text, C, Working, Control, Musician, Form, Line, Technique, Ensemble, Networked  & \tiny Use, Design, Machine, Work, State, Problem, Experience, Audio  & \tiny Sound, User, Time, Practice  & \tiny E  & \tiny Code  & \tiny --  & \tiny --  & \tiny --  & \tiny --  & \tiny --  & \tiny Live \\ 
\hline
\tiny \textbf{VBZ}
 & \tiny Collins  & \tiny --  & \tiny --  & \tiny --  & \tiny --  & \tiny --  & \tiny --  & \tiny --  & \tiny --  & \tiny --  & \tiny -- \\
 \hline
 \hline
\end{tabular}
\end{table}

\begin{table}
\centering
\caption{VBG -- \emph{present participle}. VBD -- \emph{past tense}. VBN -- \emph{past participle}}
\label{tab:gen2}
\small
\begin{tabular}{ | p{2.6cm} | p{2.6cm} | p{1.75cm} | p{1.75cm} | p{0.25cm} | p{0.25cm} | p{0.25cm} | p{1cm} | p{0.25cm} | p{0.25cm} | p{0.25cm} | p{0.25cm} |}
\hline
\hline
\tiny \textbf{Qualidade/Funcao}
 & \textbf{0}
 & \textbf{1}
 & \textbf{2}
 & \textbf{3}
 & \textbf{4}
 & \textbf{5}
 & \textbf{6}
 & \textbf{7}
 & \textbf{8}
 & \textbf{9}
 & \textbf{10} \\ 
\hline
\hline
\tiny \textbf{VBG}
 & \tiny Working, Making, Livecoding, Solving  & \tiny Using, Writing  & \tiny Programming  & \tiny --  & \tiny --  & \tiny --  & \tiny Coding  & \tiny --  & \tiny --  & \tiny --  & \tiny -- \\
\hline
\tiny \textbf{VBD}
 & \tiny Set, Developed, Made, Concept, Networked, Shared, Output, Become, Dierent  & \tiny Used, Instrument, Based  & \tiny Sound  & \tiny --  & \tiny --  & \tiny --  & \tiny --  & \tiny --  & \tiny --  & \tiny --  & \tiny -- \\
\hline
\tiny \textbf{VBN}
 & \tiny Developed, Made, Become, Shared, Networked, Method, Set, Need, Output  & \tiny Used, Based  & \tiny --  & \tiny --  & \tiny --  & \tiny --  & \tiny --  & \tiny --  & \tiny --  & \tiny --  & \tiny -- \\
 \hline
  \hline
\end{tabular}
\end{table}

\begin{table}
\centering
\caption{VBP -- \emph{past tense}, sem ser 3$^a$ pessoa do singular. JJ -- \emph{adjetivo, numeral ou ordinal}.}
\label{tab:gen3}
\small
\begin{tabular}{ | p{2.6cm} | p{2cm} | p{1.5cm} | p{1cm} | p{0.25cm} | p{0.75cm} | p{1.25cm} | p{0.25cm} | p{0.25cm} | p{0.25cm} | p{0.25cm} | p{0.25cm} |}
\hline
\hline
\tiny \textbf{Qualidade/Funcao}
 & \textbf{0}
 & \textbf{1}
 & \textbf{2}
 & \textbf{3}
 & \textbf{4}
 & \textbf{5}
 & \textbf{6}
 & \textbf{7}
 & \textbf{8}
 & \textbf{9}
 & \textbf{10} \\ 
\hline
\hline
\tiny \textbf{VBP}
 & \tiny Pattern, Framework, See, Create, Show, Need, Mean, Take, Support, Become, Make, Object, Present, Play, Explore, Project, Change, Type, Point, Allow, Digital, Result, Provide, Et, Knowledge, End, Approach, Video, Cognitive, Collaborative, Server, Screen, Free, Algorithm, Dierent  & \tiny Use, Experience, Work, Process, Network, Audio  & \tiny Sound, Practice  & \tiny E  & \tiny Code  & \tiny --  & \tiny --  & \tiny --  & \tiny --  & \tiny --  & \tiny Live \\
 \hline
\tiny \textbf{JJ}
 & \tiny Current, Electronic, Human, First, Possible, Particular, Free, Open, Mean, Virtual, Potential, Present, Visual, Future, Different, Digital, Collaborative, Important, Cognitive, Similar, Real, Musician, Ensemble, Mclean, Dierent, Livecoding, Working, Networked, Sonic, International, Material, Text, Al, Object, Create, Context, Allow, Laptop, Shared, Developed  & \tiny Musical, Many, Used  & \tiny New, Sound  & \tiny E  & \tiny --  & \tiny Music  & \tiny --  & \tiny --  & \tiny --  & \tiny --  & \tiny Live \\
 \hline
 \hline
\end{tabular}
\end{table}

\begin{table}
\centering
\caption{DT -- \emph{determinant}, sem ser 3$^a$ pessoa do singular. NN -- \emph{noun} (substantivo), comum, singular ou de massa.}
\label{tab:gen3}
\small
\begin{tabular}{ | p{1cm} | p{1cm} | p{1cm} | p{1cm} | p{1cm} | p{1cm} | p{1cm} | p{1cm} | p{1cm} | p{1cm} | p{1cm} | p{1cm} |}
\hline
\hline
\tiny \textbf{Qualidade/Funcao}
 & \textbf{0}
 & \textbf{1}
 & \textbf{2}
 & \textbf{3}
 & \textbf{4}
 & \textbf{5}
 & \textbf{6}
 & \textbf{7}
 & \textbf{8}
 & \textbf{9} \\ 
\hline
\hline
\tiny \textbf{DT}
 & \tiny Another  & \tiny --  & \tiny --  & \tiny --  & \tiny --  & \tiny --  & \tiny --  & \tiny --  & \tiny --  & \tiny --  & \tiny -- \\
 \hline
\tiny \textbf{RP}
 & \tiny --  & \tiny --  & \tiny Sound  & \tiny --  & \tiny --  & \tiny --  & \tiny --  & \tiny --  & \tiny --  & \tiny --  & \tiny -- \\
 \hline
\tiny \textbf{NN}
 & \tiny Control, Art, Number, Point, Method, Et, Al, Interface, Paper, Interaction, Analysis, Feature, Part, Information, Present, Output, Video, Collaboration, Result, Case, Source, Algorithm, Session, Piece, Dierent, Line, Supercollider, Provide, Memory, Program, Web, Browser, Function, Node, End, Form, Set, Parameter, Value, Laptop, Allow, Sonic, Text, Environment, Action, Future, Human, Material, Pattern, Framework, Kind, Body, Potential, Concept, Software, Term, Approach, Context, Order, Application, Programmer, Community, Relation, Audience, Screen, Conference, Show, Development, Structure, Technology, Digital, Tool, Aspect, Activity, Change, Support, Play, Group, Space, Idea, Knowledge, Cell, Server, Figure, Object, World, Need, Type, Composition, Orchestra, Musician, Level, Livecoding, Expression, Within, Project, See, Take, Working, B, Less, Become, Make, C, G, J, Member, Technique, Rule, Cognitive, Explore, Current, Making, Different, University, Solving, Ensemble, Well, Rather, Mean, Particular  & \tiny Machine, Research, State, Example, Work, Use, Audio, Problem, Coder, Process, Writing, Experience, Performer, Network, Design, Way, Improvisation, Instrument, Using, Data, Will  & \tiny Computer, Programming, Language, Time, System, Sound, User, Practice, One  & \tiny E  & \tiny Performance, Code  & \tiny Music  & \tiny Coding  & \tiny --  & \tiny --  & \tiny --  & \tiny Live \\
 \hline
 \hline
\end{tabular}
\end{table}
 
\begin{table}
\centering
\caption{NNS -- \emph{noun}, substantivo próprio no plural. NNP -- \emph{noun} substantivo próprio, singular ou de massa.}
\label{tab:gen4}
\small
\begin{tabular}{ | p{1cm} | p{1cm} | p{1cm} | p{1cm} | p{1cm} | p{1cm} | p{1cm} | p{1cm} | p{1cm} | p{1cm} | p{1cm} | p{1cm} |}
\hline
\hline
\tiny \textbf{Qualidade/Funcao}
 & \textbf{0}
 & \textbf{1}
 & \textbf{2}
 & \textbf{3}
 & \textbf{4}
 & \textbf{5}
 & \textbf{6}
 & \textbf{7}
 & \textbf{8}
 & \textbf{9} \\ 
\hline
\hline
\tiny \textbf{NNS}
 & \tiny Processes, Proceedings, People, Davies, Collins  & \tiny Data  & \tiny --  & \tiny --  & \tiny --  & \tiny --  & \tiny --  & \tiny --  & \tiny --  & \tiny --  & \tiny -- \\
 \hline
\tiny \textbf{NNP}
 & \tiny Collins, Davies, University, Blackwell, Mclean, Supercollider, Figure, Line, Web, Set, Well, Value, Group, Information, Future, Electronic, Expression, Laptop, Body, Proceedings, Conference, Level, International, Making, Analysis, Control, Open, Interface, Interaction, Digital, Algorithm, Art, Cognitive, However, Audience, Visual, Activity, Collaboration, Structure, Within, Sonic, Pi, Although, Take, Collaborative, Framework, Software, Community, J, People, Development, Knowledge, Browser, Technology, Gibber, Free, Orchestra, Number, Livecoding, Composition, Acm, Human, Material, Idea, First, Paper, B, Program, Relation, Davies, Context, World, Show, Working, Approach, Case, Space, Video, C, Action, Networked, Real, Text, Screen, Environment, Processes, Rule, Create, Form, Memory, Application, Method, Two, Solving, G, Without, Org, Source, Object, Shared, See, Grossi, Make, Current, Play, Ensemble, Dierent, Rather, Virtual, Parameter, Support, Type  & \tiny Machine, Musical, Audio, Instrument, Research, Experience, Data, Design, Use, Process, Many, Using, Network, Improvisation, Coder, Work, May, Example, Writing, Based, State, Will, Problem, Performer  & \tiny New, Programming, Language, Sound, System, Computer, User, Practice, Time  & \tiny E  & \tiny Performance, Code  & \tiny Music  & \tiny Coding  & \tiny --  & \tiny --  & \tiny --  & \tiny Live \\
\hline
\hline
\end{tabular}
\end{table}
 
\begin{table}
\centering
\caption{RB -- \emph{adverb}. RBR -- advérbio comparativo. CD -- Numeral ou cardinal. MD -- Auxiliar modal. JJR -- Adjetivo comparativo. }
\label{tab:gen5}
\small
\begin{tabular}{ | p{1cm} | p{1cm} | p{1cm} | p{1cm} | p{1cm} | p{1cm} | p{1cm} | p{1cm} | p{1cm} | p{1cm} | p{1cm} | p{1cm} |}
\hline
\hline
\tiny \textbf{Qualidade/Funcao}
 & \textbf{0}
 & \textbf{1}
 & \textbf{2}
 & \textbf{3}
 & \textbf{4}
 & \textbf{5}
 & \textbf{6}
 & \textbf{7}
 & \textbf{8}
 & \textbf{9} \\ 
\hline
\hline
 \tiny \textbf{RB} & \tiny Well, Even, However, Rather, First, Show, Server, Explore, Video  & \tiny Also  & \tiny --  & \tiny --  & \tiny --  & \tiny --  & \tiny --  & \tiny --  & \tiny --  & \tiny --  & \tiny -- \\
 \hline
\tiny \textbf{RBR}
 & \tiny Less  & \tiny --  & \tiny --  & \tiny --  & \tiny --  & \tiny --  & \tiny --  & \tiny --  & \tiny --  & \tiny --  & \tiny -- \\
 \hline
\tiny \textbf{CD}
 & \tiny Two  & \tiny --  & \tiny One  & \tiny --  & \tiny --  & \tiny --  & \tiny --  & \tiny --  & \tiny --  & \tiny --  & \tiny -- \\
 \hline
\tiny \textbf{IN}
 & \tiny Within, Although, Without, Context, Text, Screen, Explore, Output  & \tiny --  & \tiny Sound  & \tiny --  & \tiny --  & \tiny --  & \tiny --  & \tiny --  & \tiny --  & \tiny --  & \tiny -- \\
 \hline
\tiny \textbf{MD}
 & \tiny Might  & \tiny May, Will  & \tiny --  & \tiny --  & \tiny --  & \tiny --  & \tiny --  & \tiny --  & \tiny --  & \tiny --  & \tiny -- \\
 \hline
\tiny \textbf{JJR}
 & \tiny Less, Programmer, Parameter  & \tiny --  & \tiny User  & \tiny --  & \tiny --  & \tiny --  & \tiny --  & \tiny --  & \tiny --  & \tiny --  & \tiny -- \\
 \hline

\hline
\end{tabular}
\end{table}
