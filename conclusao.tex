\chapter*[Conclusão]{Conclusão}\addcontentsline{toc}{chapter}{Conclusão}\label{conclusao}

A pesquisa tomou um rumo diferente da proposta inicial, : um estudo que discutiria a questão da aparente intimidação experenciada por compositores no uso de linguagens de programação para composição musical, especificamente em ambientes de redes de computadores, bem como a elaboração de um \emph{software} original com base nessa reflexão. 

Para chegar neste aplicativo, tomei conhecimento de uma cena emergente no que é conhecido hoje como \emph{live coding} em navegadores de internet\footnote{\emph{Apple Safari}, \emph{Google Chrome}, \emph{Mozzilla Firefox}.}, tais como \emph{Gibber} \cite{roberts_gibber:_2012}, \emph{Vivace} \cite{vieira_vivace:_2015}, \emph{Wavepot}\footnote{\url{http://www.wavepot.com}, acessado em \today}, \emph{Html5Bytebeat}\footnote{\url{https://github.com/greggman/html5bytebeat}, acessado em \today.}. Uma reflexão a respeito das linguagens nestes aplicativos possibilitou o desenvolvimento de um aplicativo \emph{web} em conjunto com Luíz Schiavonni, professor da UFSJ (Universidade Federal de São João del-Rei), que foi chamado de \emph{Termpot}\footnote{\url{http://jahpd.githb.io/termpot}, acessado em \today.}. 

No entanto, o Programa de Pós-graduação em Artes, Cultura e Linguagens  (PPG-ACL/UFJF) proporcionou o contato com obras  de autores como \citeonline{kuhn_structure_1970}, \citeonline{feyerabend_against_1975} e \citeonline{santos_filosofia_2008} que foram significativas para aprimorar a metodologia; arrisquei-me a uma pesquisa orientada a paradigmas, aqui, paradigmas do \emph{livecoding}, considerando a existência cooperativa/competitiva de um conjunto deles em ambientes de produção de conhecimento (incluímos aqui a Universidade). Adicionalmente, a vivência com uma botânica dedicada em problemas de plano de manejo de uma espécie vegetal em Ubatuba/SP e um comunicador debruçado em problemas de gênero em Juiz de Fora/JF, me colocou na posição de refletir sobre uma questão, pessoalmente mais fundamental que a proposta inicial, de uma ecologia de gêneros musicais utilizando um fragmento de produções do \emph{livecoding}\footnote{Em comunicação pessoal com Tiago Rubini, ``a primeira menção à palavra 'gênero' se refere à questão de identidade de gênero. A segunda, sobre dinâmicas de gêneros musicais''. Infiro que uma dinâmica de gêneros musicais pressupõe um conjunto limitado de conhecimentos, que exige ser discutido para oferecer uma idéia das próprias limitações do conhecimento discutido.}. Autores citados acima não foram utilizados neste trabalho para discutir o \emph{livecoding}, mas foram importantes para a a concepção do seguinte triangulamento metodológico realizado: levantamento qualitativo do \emph{livecoding}, a partir de uma bibliografia básica disponível\footnote{\url{http://toplap.org/wiki/Videos,_Articles_and_Papers}. Acessado em \today.} e verificação quantitativa, através de levantamento de dados seletivo na internet que confirmasse o levantamento qualitativo em um nicho específico\footnote{Utilizando um SDK disponibilizado pelo \emph{Soundcloud}.}.


A partir de diferentes textos, derivei palavras-chave que carregam significados do objeto de pesquisa: \emph{live coding}, \emph{live-coding}, \emph{livecoding}, \emph{live code}, \emph{conversational programming}, \emph{on-the-fly programming}, \emph{live algorithm programming}; estas palavras-chaves apontam para uma prática de improvisação utilizando o computador, articulada de maneira colaborativa ou competitiva entre \emph{performers}, mediada por projetores visuais e acústicos (projetores e alto-falantes), e direcionadas para um público parcialmente passivo em ato de ouvir música e ver imagens (codificadas durante a improvisação). Este aspecto de improvisação com \emph{scripts} criados e editados no computador, segundo \citeonline{cox_coding_2004}, reside em uma capacidade de predizer (de forma aproximada) resultados complexos antes de codificar, isto é, o  executante deve ser capaz de realizar em um curto espaço de tempo uma formalização lógica de um comportamento audiovisual antes de sua programação em um \emph{script}: 

\begin{citacao}
Um programador é, portanto, capaz de predizer e especular sobre como o seu código irá se comportar em circunstâncias mais usuais. Como com qualquer coisa que é de autoria, a questão da subjetividade é inevitável, uma vez que qualquer resultado particular pode ser alcançado em diferentes (e muitas vezes concorrentes) maneiras. Nesse sentido, qualquer senso de improvisação depende de um entendimento preditivo de sistemas complexos e geradores. \cite[p.~169]{cox_coding_2004}\footnote{Tradução nossa de: \emph{A programmer is therefore able to predict and speculate upon how their code will behave in most usual circumstances. As with anything that is authored, the issue of subjectivity is unavoidable, since any particular result can be achieved in different (and often competing) ways. In this, any sense of improvisation relies on a preditive understanding of complex and generative systems}.} 
\end{citacao}

Colocado de outra forma por \citeonline{ruthmann_teaching_2010}

\begin{citacao}
Executar de maneira efetiva uma manipulação em tempo-real do código (live coding musical) para criar e formalizar música gerada requer ambos entendimentos musicais e computacionais. De uma perspectiva musical, é necessário entender o fluxo de como uma música generativa deve soar. Da perspectiva computacional, é preciso entender como o código deve ser ajustado e manipulado em tempo real para atingir mudanças aurais e musicais desejadas. \cite[p.~3-4]{ruthmann_teaching_2010}. \footnote{Tradução nossa de \emph{Performing effective real-time manipulation of code (musical live coding) to create and shape generated music requires both musical and computational understanding. From a musical perspective, one needs to understand how the ongoing, generative music should sound. From a computational perspective, one needs to understand how the code can be adjusted and manipulated in real time to achieve the aural and musical changes and outcomes one desires}.}
\end{citacao}

Esses conhecimentos pré-concebidos antes de codificar podem ter origem no conhecimento de mundo da Música daquele que realiza a improvisação; de diferentes maneiras, músicos-programadores adequam programação-partituras para determinados contextos do fazer musical. Expressa-se uma teoria musical que regula os algoritmos, esta por sua vez regulada por uma ecologia de gostos musicais adequadas para cada contexto. Interessado na dinâmica dos gêneros musicais que emergem de diferentes contextos de improvisação, coloquei-me a tarefa de discutir o que diferentes autores debruçados no assunto \emph{livecoding} entendem por música, como realizam, e seus discursos da respeito de realizações em público. 

Decidi recortar o assunto a partir de comparações limitadas ao âmbito musical por julgar-me incapaz de abordar diferentes linguagens artísticas como a literatura generativa e práticas derivadas de exibição simultanea da imagem renderizada e seu respectivo código; algumas menções foram feitas apenas para delimitar o que esté fora e o que está dentro de uma janela de discussão. Uma bibliografia básica foi levantada como forma de esclarecer origens e práticas musicais do \emph{livecoding}: \begin{inparaenum}[]
\item \citeonline{cox_aesthetics_2000},
\item \citeonline{cox_coding_2004},
\item \citeonline{collins_live_2003},
\item \citeonline{mclean_hacking_2004},
\item \citeonline{wang_--fly_2004},
\item \citeonline{ward_live_2004},
\item \citeonline{collins_live_2007},
\item \citeonline{rohrhuber_improvising_2009},
\item \citeonline{ruthmann_teaching_2010},
\item \citeonline{mclean_visualisation_2010},
\item \citeonline{magnusson_algorithms_2011},
\item \citeonline{magnusson_herding_2014},
\item \citeonline{magnusson_scoring_2014}
\end{inparaenum}.


O \emph{livecoding} pode existir em contextos tradicionalmente reservados para concertos ou em contextos informais que estimulam a sociabilização através da dança (os \emph{Night Clubs} de \citeonline{mclean_hacking_2004}); noto que, em um período de aproximadamente dez anos, ocorre um movimento de cooperação entre espaços acadêmicos e de entretenimento, onde emerge o termo \emph{algorave}\footnote{``Nenhuma conferência acadêmica está completa sem uma \emph{algorave}, uma chance de dançar algoritmos com velhos ou novos amigos. Teremos pelo menos um clube de noite, no excelente co-op Wharf Chambers $[$\url{http://www.wharfchambers.org/}$]$. em \url{http://iclc.livecodenetwork.org/cfp.html\#performance}. Tradução nossa de: \emph{No academic conference is complete without an algorave, a chance to dance to algorithms with friends new and old. We will have at least one club night, at the excellent Wharf Chambers co-op. More details to follow.}}. Por outro lado, estes mesmos espaços estimularam o desenvolvimento de novas áreas de pesquisa, entre elas, o \emph{livecoding} de ambientes virtuais, isto é, a prática sendo aplicada em pequenas ou grandes redes de computadores; ao mesmo tempo, o advento da biblioteca \emph{WebAudio API} possibilitou a experimentação do \emph{livecoding} na internet\footnote{Como os aplicativos \emph{Gibber}, \emph{Wavepot} e \emph{Html5Bytebeat}, \emph{Vivace}. Para mais informações a respeito, sugiro a leitura de "The Web Browser As Synthesizer And Interface" de \citeonline{roberts_web_2013} e "The Viability of the Web Browser as a Computer Music Platform"\citeonline{wyse_viability_2014}.}.

%No primeiro modo, existe uma pretensa sociabilização entre agentes. Por agentes me refiro aos membros do grupo de executantes e membros do grupo do público. Por sociabilização, entendo a interação entre membros. Esta separação ilustra uma característica observada (e praticada pelo autor deste trabalho) até o momento: em algumas performances ao vivo e vídeos da internet, ou em apresentações, a sociabilização entre membros executantes se caracteriza como ativa (modificação  da improvisação através da edição do código compartilhado ou por interferências sonoras), e a sociabilização entre executantes/público tem uma tendência a ser passiva (os membros do público são apenas convidados a observar e a escutar o que está sendo realizado). Tais agrupamentos ainda podem ser organizados em: \begin{inparaenum}[\itshape 1)\upshape] \item agrupamentos institucionalizados (como as \emph{Laptop orchestras}); e \item agrupamentos informais (\emph{solos}, \emph{duos}, \emph{trios}) \end{inparaenum}. 

Destes dois modos busquei fazer um mapeamento de pŕaticas musicais mencionadas por autores de manifestos e artigos mencionados no segundo parágrafo desta conclusão, bem como a localização geográfica e que práticas musicais são colocadas no plano de discussão em relação ao \emph{livecoding}: \begin{inparaenum}[\itshape 1)\upshape]
\item Música Algorítmica (MA),
\item Música Processual (MP), 
\item Música Generativa (MG),
\item e Música de Pista (ou o que que denominamos a partir do termo simplificado \emph{Disk Jockey}, (DJ)
\end{inparaenum}. Busquei averiguar até que ponto o uso dos termos referidos estão de acordo com definições compartilhadas, respaldados em um embate com autores como \citeonline{reich_music_1968}, \citeonline{eno_music_1978}, \citeonline{kramer_sonification_1999}, \citeonline{roads_times_2001}, \citeonline{wooler_framework_2005} \citeonline{malt_concepts_2006}, \citeonline{walker_auditory_2006}, \citeonline{essl_algorithmic_2007},  \citeonline{cope_prefacio_2008}, \citeonline{iazzetta_musica_2009}, \citeonline{mailman_agency_2013} e \citeonline{collins_algorave:_2014} \citeonline{casteloes_conversores_2015}. 

Em um segundo momento, levantamos dados pertinentes ao tema na rede social \emph{Soundcloud} como maneira de justificar o primeiro mapeamento das comunidades de gosto.

Realizando uma retrospectiva geopolítica dos autores, isto é, observando as localizações do globo terrestre em que foram escritos os textos supracitados no primeiro parágrafo deste capítulo, outros em \emph{passim}, mais um conjunto de dados que podem ser observados no \autoref{dados_sclivecoding} -- que descrevem essas regiões de maneira quantitativa em um período de tempo entre 2008 e 2015--, é possível confirmar uma tese que foi sendo construída no decorrer da pesquisa, mas que somente pode ser verbalizada no segundo parágrafo de ``A genealogia da moral'' de Nietzsche, ``o privilégio senhorial de dar nomes permitem-nos conceber a origem da linguagem ela mesma como uma manifestação de poder dos governantes''\footnote{Cf. \emph{On the Genealogy of Morality} Edited by Keith Ansell-Pearson. Translated by Carol Diethe. Cambridge. 2006. Tradução nossa do segundo parágrafo do primeiro ensaio \emph{The seigneurial pribilege of giving names even allow us to conceive of the origin of a language itself as a manifestation of the power of the rulers}.}, no sentido de que o \emph{live coding} possue esse nome, como \emph{programa de pesquisa}, que integra todo uma Epistemologia do Norte que que \citeauthoronline{santos_filosofia_2008} tanto comenta.

Na inglaterra identifico os autores como Alex McLean, Nick Collins Adrian Ward, e Dave Griftths. Nos EUA identifico as \emph{Laptop Orchestras}, em Stanford e Princeton, bem como compositores como James Harkins e Joshua Parmenter e Ge Wang;  na Austrália, tem sido notável o papel de Andrew Sorensen na \emph{Queensland University of Technology} utilizando o piano; no Brasil identifico os trabalhos de Bernardo Barros, André Damião, Antônio Goulart, Vilson Vieira, Geraldo Magela de Castro Rocha Junior, Caleb Mascarenhas Luporini, Daniel Penalva, Ricardo Fabbri, Renato Fabbri, Ricardo Brasileiro e Daniel Penalva e Flávio Luiz Schiavonni. 

No levantamento de dados do Soundcloud pude confirmar alguns dos pontos levantados no levantamento bibliográfico e questionar algumas afirmações feitas: por exemplo, na questão de qual país produz grande quantidade de livecoding confirmei posições de países falantes da língua inglesa, como Inglaterra e EUA, mas ao mesmo tempo, notei uma grande quantidade de produções anônimas, isto é, com localizações não identificadas, desacreditando em uma centralização de produção; ademais foi possível perceber uma miríade de produções em países como Alemanha, México e Japão. Na questão de gênero musical, articulado através de \emph{hashtags}\footnote{Para mais informações, sugiro a leitura de \url{https://en.wikipedia.org/wiki/Hashtag} e \url{https://en.wikipedia.org/wiki/Tag_(metadata)}, acessados em \today.}, pude confirmar uma hibridização de práticas musicais consideradas anteriormente distintas, e ao mesmo tempo, questionar uma centralização conceitual em um ou outro termo, como \begin{inparaenum}[\itshape a)\upshape]
\item \emph{algorithmic music}
\item \emph{algorave},
\item \emph{algopop},
\item \emph{bytebeat},
\item \emph{drone}
\item \emph{electronic music},
\item \emph{electroacoustic},
\item \emph{glitch},
\item \emph{noise},
\item \emph{whistling}
\end{inparaenum}.

O seguinte ponto-de-vista foi desenvolvido: cada termo utilizado no \emph{livecoding} carrega uma teoria musical e, por outro lado, costumes distintos do fazer musical no \emph{livecoding} estão relacionados com gêneros musicais, a partir  daquilo que  discute comunidades de gosto; tais comunidades de gosto foram investigadas em um nicho musical virtual específico, a saber, a rede social Soundcloud. 

Percebi o \emph{livecoding} como um campo de estudos em transição entre Música e Ciências da Computação; esse campo tem sido estimulado em países considerados como centros de referência, por exemplo, Inglaterra e EUA. Esse estímulo tem sido acompanhado por colaborações interdisciplinares formalizadas (isto é, reconhecidas institucionalmente); não coincidentemente, isso foi pressuposto por \citeonline{mathews_digital_1963} no desenvolvimento da família MUSIC N como única maneira de avançar neste campo de estudo\footnote{Lembro que o desenvolvimento de estudos musicais com CSIRAC (na Austrália),  antes mesmo do advento do MUSIC N, tem como um dos fatores de seu fracasso, segundo \citeonline{di_nunzio_genesi_2010}, devido a uma falta de cooperação entre músicos e cientistas da computação.}. No Brasil, no entanto, existem barreiras institucionais que impedem a colaboração interdisciplinar formal, porém estas mesmas barreiras forçam músicos interessados na área, a aprender a programar, mesmo com algumas barreiras do jargão técnico (muitas vezes intimidadoras); de maneira semelhante, cientistas (da computação e muitas vezes físicos) e não-acadêmicos brasileiros, com seu conhecimento de programação, exploraram comportamentos musicais em \emph{softwares} como forma de ultrapassar uma barreira impostas pelo jargão técnico dos músicos.

