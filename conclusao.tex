\chapter*[Conclusão]{Conclusão}\addcontentsline{toc}{chapter}{Conclusão}\label{conclusao}

Este documento contextualizou o que é uma improvisação de códigos, e como podemos entender suas manifestações quando se codificam artefatos sonoros, visuais, corporais e têxteis. Diferente de outras definições, buscamos situar o \emph{live coding} como uma técnica de programação que possibilita elaborar proposições artísticas de qualquer tipo. \citeonline{ward_live_2004} definem a improvisação de códigos como \traducao{``atividade da escrita integral (ou partes) de um programa enquanto ele é executado''}{Live coding is the activity of writing (parts of ) a program while it runs}. \citeonline{blackwell_programming_2005} enfatizam a definição do ponto de vista da linguagem de programação como instrumento musical. \citeonline{mclean_hacking_2006} relata o \emph{live coding} como ferramenta para um \emph{Disk Jockey codificado}.  \citeonline{sorensen_keith_2009} definem \traducao{``uma prática de performance para o qual linguagens de computador definem o meio primário de expressão artística''.}{Live coding is a performance pratice for which computer languages define  the primary means of expression.}. Para \citeonline{sorensen_impromptu_2010}, \emph{live coding} envolve a premissa de uma programação-partitura audiovisual reativa: 

\traduzcitacao{
Livecoding é uma prática de arte computacional que envolve criação em tempo-real de programas de audiovisual generativo para performances multimídias interativas. Comumente as ações dos programadores são expostas para uma audiência por projeção do ambiente de edição. Performances de livecoding geralmente envolvem mais de um participante, e são geralmente iniciadas a partir de uma folha conceitual em branco   
\cite[p.~823]{sorensen_impromptu_2010}}{Livecoding [10, 50] is a computational arts practice that involves the real-time creation of generative audiovisual software for interactive multimedia performance. Commonly the programmers’ actions are exposed to the audience by projection of the editing environment. Livecoding performances often involve more than one participant, and are often commenced from a conceptual blank slate}

\citeonline{magnusson_algorithms_2011,collins_origins_2014} sintetizam o \emph{live coding} como improvisação audiovisual. \citeonline{sorensen_programming_2014} define como \traducao{``programar sistemas de tempo-real em tempo real''}{programming real-time systems in real-time}. Uma discussão intitulada ``\emph{Wtf is livecoding}''\disponivelem{http://lurk.org/groups/livecode/messages/topic/ofAxZpxsKFpDRLnoA48Bh} dificultou o próprio processo de definição, onde o compositor Nick Collins diz que o \traducao{``\emph{Live coding} celebra a efemeridade da própria definição''}{Live coding celebrates the ephemerality of definition itself}. Aqui fica explícito uma \emph{bricolagem} da técnica. \citeonline{sorensen_programming_2014} destaca que modificar alguma coisa é próprio da técnica, de forma que é possível extender essa bricolagem para proposições. Nick \citeonline{collins_origins_2014} situa essa questão da seguinte forma:

  \begin{figure}[h]
    \centering
    \includegraphics[scale=0.7]{imagens/live_coding_def.png}
    \caption{Definição de \emph{live coding}: ``Insira a definição aqui''. \textbf{Fonte}: \citeonline{collins_origins_2014}.}
    \label{fig:live_coding_def}
  \end{figure}

Neste ponto encontramos um desafio à metodologia de pesquisa acadêmica: se o termo que contextualiza um estudo de caso é variável por definição, como analisar este caso? 

Foi necessária uma revisão histórica para entender como a atividade de programar, no sentido de \emph{tocar um instrumento}, se construiu historicamente. Alí noções como \emph{reflexividade} e instrumentos acusticamente divorciados mas informacionalmente ligados, possibilitaram o artistas-programadores se auto-organizarem dentro de suas agremiações. Uma dessas agremiações merece destaque, e surgiu no seio acadêmico inglês. De certa forma esta agremiação prescreveu regras de conduta para improvisar códigos, bem aceitas entre diversos artistas-programadores. Desta agremiação que se tornou uma organização, selecionamos um caso exemplar que segue as regras estipuladas. Investigamos uma improvisação de códigos como uma simulação de uma improvisação instrumental. Esta improvisação instrumental, para facilitação do que pode ser harmonicamente simples, possibilitou destacar a diferença entre o que foi proposto e aquilo que foi plenamente executado.

Por outro lado, julgar \emph{A Study in Keith} apenas com base na sua simplicidade harmônica, se comparada ao pensamento harmônico inicial dos Concertos \emph{Sun Bear}, é obscurecer a possibilidade de uma metodologia composicional, neste caso, uma espécie de \traducao{desenvolvimento orientado a testes}{Test-driven development} musicais. Ao aprofundarmo-nos em uma escavação netnográfica \cite{mori_analysing_2015}, foi possível notar que o mesmo mecanismo de códigos é utilizado em uma performance com pianos acústicos, ou \emph{Disklavier Sessions} de 2011. Neste sentido, o recorte desta pesquisa foi determinar como um algoritmo musical é elaborado e codificado por um artista-programador, aplicado ao desenvolvimento de \emph{softwares}.


% Se por um lado a definição agrega definições, o que dificulta a tarefa inicial de descrever os fundamentos do objeto de pesquisa, por outro ilustra a improvisação de códigos como um \emph{Universo de conceitos}. Neste trabalho consideramos que definições ou performances de improvisação de códigos estão contidas em diferentes \emph{Espaços Conceituais} \cite{wiggins_framework_2006,mclean_music_2006}. Artistas-programadores (\emph{live coders}) transitam entre os Espaços Conceituais para criação de Sistemas Criativos (códigos, programas). Estes Sistemas Criativos são representados em diferentes Linguagens de Programação. Regras práticas conduzem o processo de escrita e exposição desta linguagem; mas não restringem o resultado (no caso da pesquisa, musical). Mas algumas categorizações musicais se destacam.  Neste sentido, selecionamos um exemplo simbólico, \emph{A Study in Keith} de Andrew \citeonline{sorensen_keith_2009}\footnote{Disponível em \url{https://vimeo.com/2433947}.}. Representa um caso particular que foge dos exemplos citados anteriormente, mas envolve a manutenção de uma tradição musical tonal através de um interessante esforço de \emph{replicação do estilo}. No \autoref{cap:introducao} selecionamos alguns exemplos afim de ilustrar nossa percepção (conhecimento-psicológico) do imaginário daquilo que \citeonline{McLean2011} chama de artistas-programadores. No \autoref{sec:protohistoria} buscamos levantar um conjunto de conhecimentos históricos. No \autoref{cap:metodologia}, discutimos um modelo de formalização conceitos, observados pelo prisma de Alex \citeonline{mclean_music_2006}, contraposto por \citeonline{thornton_quantitative_2007}, e rediscutido por \citeonline{Forth2010} \citeonline{McLean2011}. No \autoref{cap:estudos_de_caso}, organizamos conceitos de um algoritmo inicial de uma improvisação de códigos, \emph{A Study in Keith} de 2009, segundo este um Quadro Conceitual de Sistemas Criativos. Finalizamos este trabalho com o apêndice \autoref{app:A}, onde descrevemos um processo de organização de tais qualidades da improvisação de códigos. 
