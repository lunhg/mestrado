%% abtex2-modelo-trabalho-academico.tex, v-1.9.2 laurocesar

%%
%% This work may be distributed and/or modified under the
%% conditions of the LaTeX Project Public License, either version 1.3
%% of this license or (at your option) any later version.
%% The latest version of this license is in
%%   http://www.latex-project.org/lppl.txt
%% and version 1.3 or later is part of all distributions of LaTeX
%% version 2005/12/01 or later.
%%

%% This work has the LPPL maintenance status `maintained'.
%% 
%% The Current Maintainer of this work is the abnTeX2 team, led
%% by Lauro César Araujo. Further information are available on 
%% http://abntex2.googlecode.com/
%%
%% This work consists of the files abntex2-modelo-trabalho-academico.tex,
%% abntex2-modelo-include-comandos and abntex2-modelo-references.bib
%%
%------------------------------------------------------------------------
% ------------------------------------------------------------------------ 
% abnTeX2: Modelo de Trabalho Academico (tese de doutorado, dissertacao de
% mestrado e trabalhos monograficos em geral) em conformidade com 
% ABNT NBR 14724:2011: Informacao e documentacao - Trabalhos academicos -
% Apresentacao
% ------------------------------------------------------------------------
% ------------------------------------------------------------------------

\documentclass[
	% -- opções da classe memoir --
	12pt,				% tamanho da fonte
	openright,			% capítulos começam em pág ímpar (insere página vazia caso preciso)
	twoside,			% para impressão em verso e anverso. Oposto a oneside
	a4paper,			% tamanho do papel. 
	% -- opções da classe abntex2 --
	%chapter=TITLE,		        % títulos de capítulos convertidos em letras maiúsculas
	%section=TITLE,		        % títulos de seções convertidos em letras maiúsculas
	%subsection=TITLE,	        % títulos de subseções convertidos em letras maiúsculas
	%subsubsection=TITLE,           % títulos de subsubseções convertidos em letras maiúsculas
	% -- opções do pacote babel --
	english,			% idioma adicional para hifenização
        italian,                        % idioma adicional para hifenização
	brazil				% o último idioma é o principal do documento
	]{abntex2}

% ---
% Pacotes
% ---
% ---
% PACOTES
% ---
\usepackage{lmodern}			% Usa a fonte Latin Modern
\usepackage[T1]{fontenc}		% Selecao de codigos de fonte.
\usepackage[utf8]{inputenc}		% Codificacao do documento (conversão automática dos acentos)
\usepackage{indentfirst}		% Indenta o primeiro parágrafo de cada seção.
\usepackage{nomencl} 			% Lista de simbolos
\usepackage{color}		               	% Controle das cores
\usepackage{fancyvrb}
\usepackage{graphicx}			% Inclusão de gráficos
\usepackage{txfonts}                	% Fontes virtuais 
\usepackage{listings} 
\usepackage{minted}


% ---
% ABNT
% ---
\usepackage[brazilian,hyperpageref]{backref}    % Paginas com as citações na bibl
\usepackage[alf]{abntex2cite}                  	% Citações padrão ABNT
\usepackage[brazil]{babel}	               	% Idioma do documento


%TODO
\usepackage[colorinlistoftodos]{todonotes}

\makeatletter
\hypersetup{
  pdftitle={\@title},
  pdfauthor={\@author},
  pdfsubject={Seminário para curso de metodologia},
  pdfkeywords={Música}{Metodologia}{Paradigma},
  %pdfcreator={\@LaTeX with \@abnTeX},
  colorlinks=true,
  linkcolor=blue,
  citecolor=blue, 
  urlcolor=blue}
\makeatother

% --- 
% CONFIGURAÇÕES DE PACOTES
% --- 

% ---
% Configurações do pacote backref
% Usado sem a opção hyperpageref de backref
\renewcommand{\backrefpagesname}{Citado na(s) página(s):~}
% Texto padrão antes do número das páginas
\renewcommand{\backref}{}
% Define os textos da citação
\renewcommand*{\backrefalt}[4]{
	\ifcase #1 %
		Nenhuma citação no texto.%
	\or
		Citado na página #2.%
	\else
		Citado #1 vezes nas páginas #2.%
	\fi}%
% ---

\usepackage{tikz}
\tikzset{
  treenode/.style = {shape=rectangle, rounded corners,
                     draw, align=center,
                     top color=white, bottom color=blue!20},
  root/.style     = {treenode, font=\Large, bottom color=red!30},
  env/.style      = {treenode, font=\ttfamily\normalsize},
  dummy/.style    = {circle,draw}
}


% ---
% MACROS
% ---
% --- 
% CONFIGURAÇÕES DE PACOTES
% --- 

% ---
% Configurações do pacote backref
% Usado sem a opção hyperpageref de backref
\renewcommand{\backrefpagesname}{Citado na(s) página(s):~}
% Texto padrão antes do número das páginas
\renewcommand{\backref}{}
% Define os textos da citação
\renewcommand*{\backrefalt}[4]{
	\ifcase #1 %
		Nenhuma citação no texto.%
	\or
		Citado na página #2.%
	\else
		Citado #1 vezes nas páginas #2.%
	\fi}%
% ---

\newcommand{\traducao}[2]{``#1''\footnote{Tradução nossa de ``\emph{#2}''}}
\newcommand{\tabletraducao}[2]{``#1'' \tablefootnote{Tradução nossa de \emph{#2}.}}
\newcommand{\traducaoparcial}[2]{``#1'' \footnote{Tradução parcial nossa de \emph{#2}.}}
\newcommand{\csf}[2]{$\mathcal{#1}_\emph{#2}$}
\newcommand{\traduzcitacao}[2]{\begin{citacao}
#1\footnote{Tradução de \emph{#2}}.
\end{citacao}
}

\newcommand{\disponivelem}[1]{\footnote{Disponível em \url{#1}}}
\newcommand{\sorensen}[2]{#1\disponivelem{#2}}
\newcommand{\ver}[1]{(ver \autoref{#1}, p.~\pageref{#1})}
\newcommand{\traducaoapud}[5]{``#1'' \apud[#3]{#4}{#5}\traducao{#2}}
\newcommand{\pressingtwo}[2]{$\mathcal{#1}'_{#2}$}
\newcommand{\pressingthree}[3]{$\mathcal{#1}'^{#3}_{#2}$} 
\newcommand{\met}[1]{\scriptsize \MakeTextUppercase{#1} \normalsize}
\newcommand{\metafora}[2]{\met{#1}, isto é, #2}
\newcommand{\verex}[1]{(ver Exemplo \autoref{#1}, p.~\pageref{#1})}
\newcommand{\idemibdem}{\emph{idem}, \emph{ibdem}}
\newcommand{\tempo}[2]{#1$'$#2$''$}
\newcommand{\exref}[1]{(ver exemplo \ref{#1})}

%%%%%%%%%%%%%%%%%%%%%%%%%%%%%%%%%%%%%%%%%%%%%%%%
% An example environment
%%%%%%%%%%%%%%%%%%%%%%%%%%%%%%%%%%%%%%%%%%%%%%%%
\theoremheaderfont{\normalfont\bfseries}
\theorembodyfont{\normalfont}
\theoremstyle{break}
\def\theoremframecommand{{\color{deepred}\vrule width 5pt \hspace{5pt}}}
\newshadedtheorem{exa}{Exemplo}[chapter]
\newenvironment{example}[1]{%
		\begin{exa}[#1]
}{%
		\end{exa}
}

\newcommand{\csfeq}[2]{
\mathcal{#1}_\emph{#2}
}

\newcommand{\unionspaces}[6]{
\csfeq{#1}{#2} = \csfeq{#3}{#4} \bigcap \csfeq{#5}{#6}
}

\newcommand{\listspaces}[9]{
\csfeq{#1}{#2}~=~[\csfeq{#3}{#2},~\csfeq{#4}{#2},~\csfeq{#5}{#2},~\csfeq{#6}{#2},~\csfeq{#7}{#2},~\csfeq{#8}{#2},~\csfeq{#9}{#2}
}

% ---
% Informações de dados para CAPA e FOLHA DE ROSTO
% ---
\titulo{\emph{Live Coding}: um algoritmo de sonoridade tonal em \emph{A Study in Keith} (2009) de Andrew Sorensen}
\autor{Guilherme Martins Lunhani}

\instituicao{Universidade Federal de Juiz De Fora -- UFJF
  \par
  Instituto de Artes e Design -- IAD
  \par
  Programa de Pós-Graduação em Artes Visuais, Música e Tecnologia}

\orientador[Orientador: ]{Prof. Dr. Luiz Eduardo Castelões}

% \changes{Versão inicial }{2013/07/22 }{v0.0.3}
\tipotrabalho{Dissertação (Mestrado)}

% O preambulo deve conter o tipo do trabalho, o objetivo, 
% o nome da instituição e a área de concentração 
\preambulo{Dissertação corrigida segundo orientações da banca de defesa no Programa de Mestrado em Artes, Cultura e Linguagens do Instituto de Artes e Design da Universidade Federal de Juiz de Fora (UFJF), linha de Artes Visuais, Musica e Tecnologia.}
%\EnableCrossrefs
%\CodelineIndex
%\RecordChanges

% ---
% Configurações de aparência do PDF final

% alterando o aspecto da cor azul
\definecolor{blue}{RGB}{41,5,195}

% informações do PDF
\makeatletter
\hypersetup{
     	%pagebackref=true,
		pdftitle={\@title}, 
		pdfauthor={\@author},
    	pdfsubject={\imprimirpreambulo},
	    pdfcreator={LaTeX with abnTeX2},
		pdfkeywords={abnt}{latex}{abntex}{abntex2}{trabalho acadêmico}, 
		colorlinks=true,       		% false: boxed links; true: colored links
    	linkcolor=blue,          	% color of internal links
    	citecolor=blue,        		% color of links to bibliography
    	filecolor=magenta,      		% color of file links
		urlcolor=blue,
		bookmarksdepth=4
}
\makeatother

%\newcommand{\todosautoresdelivecoding}{\begin{inparaenum}[]\item \citeonline{collins_live_2003},\item \citeonline{collins_generative_2003},\item \citeonline{collins_live_2003-1},\item \citeonline{wang_--fly_2004},\item \citeonline{ward_live_2004},\item \citeonline{blackwell_programming_2005},\item \citeonline{collins_live_2007},\item \citeonline{griffiths_fluxus:_2008},\item \citeonline{mclean_patterns_2009},\item \citeonline{rohrhuber_improvising_2009},\item \citeonline{mclean_visualisation_2010},\item \citeonline{magnusson_algorithms_2011},\item \citeonline{mccallum_show_2011},\item \citeonline{magnusson_herding_2014},\item \citeonline{magnusson_scoring_2014},\item \citeonline{collins_algorave:_2014},\item \citeonline{sorensen_livecodings_2014}\end{inparaenum}}
% --- 
% Espaçamentos entre linhas e parágrafos 
% --- 

% O tamanho do parágrafo é dado por:
\setlength{\parindent}{1.3cm}

% Controle do espaçamento entre um parágrafo e outro:
\setlength{\parskip}{0.2cm}  % tente também \onelineskip

% ---
% compila o indice
% ---
\makeindex
\makeindex

% ----
% Início do documento
% ----
\begin{document}
\pagenumbering{roman}
% Retira espaço extra obsoleto entre as frases.
\frenchspacing 

% ----------------------------------------------------------
% ELEMENTOS PRÉ-TEXTUAIS
% ----------------------------------------------------------
\pretextual

% ---
% Capa
% ---
\imprimircapa
% ---

% ---
% Folha de rosto
% (o * indica que haverá a ficha bibliográfica)
% ---
\imprimirfolhaderosto*
% ---

% ---
% Inserir a ficha bibliografica
% ---
\input{./ficha_catalografica}

% ---
% Inserir errata
% ---
%\begin{errata}
%\end{errata}

% \includepdf{folhadeaprovacao_final.pdf}
%\input{./folha_aprovacao}

% ---
% Dedicatória
% ---
\begin{dedicatoria}
   \vspace*{\fill}
   \vspace*{\fill}
   \vspace*{\fill}
   \vspace*{\fill}
   \vspace*{\fill}
   \vspace*{\fill}
   \vspace*{\fill}
   \vspace*{\fill}
   \vspace*{\fill}
   \vspace*{\fill}
   \vspace*{\fill}
   \vspace*{\fill}
   \vspace*{\fill}
   \vspace*{\fill}
   \vspace*{\fill}
   \vspace*{\fill}
   \vspace*{\fill}
   \vspace*{\fill}
   \vspace*{\fill}
   \vspace*{\fill}
   \vspace*{\fill}
   \vspace*{\fill}
   \vspace*{\fill}
   \vspace*{\fill}
   \vspace*{\fill}
   \vspace*{\fill}
   \vspace*{\fill}
   \vspace*{\fill}
   \vspace*{\fill}
   \vspace*{\fill}
   \vspace*{\fill}
   \vspace*{\fill}
   \centering
   \noindent
   \textit{À Via que gerou o um. Ao um que gerou o dois. Ao dois que gerou o três. E ao tr\^es que gerou as dez mil coisas.} \vspace*{\fill}
\end{dedicatoria}
\newpage
% ---
% Agradecimentos
% ---
]%\begin{agradecimentos}
\newpage
\begin{flushright}
\huge{\textbf{Agradecimentos}}

\small{Ao que é impossível pronunciar o verdadeiro Nome, mas cuja potência é o próprio sentido da palavra criatividade, cujo Verbo é capaz de mover montanhas.
\ \\
Aos sem nome, anônimos da Rua de Juiz de Fora, que deram sentido à fraqueza da pergunta deste trabalho.
\ \\
Para uma família em cada canto do Universo, Jair, Olímpia e Júlia. 
\ \\
Aos Professores Dr. Luiz Eduardo Castelões, Dr. Alexandre Fenerich e Dr. Flávio Luiz Schiavonni, fundamentais no apoio institucional; na sugestão de leituras; na cobrança de prazos; nas críticas; nas conversas sobre Música. À FAPEMIG por suprir esta lacuna, em um momento delicado nas finanças da Universidade Brasileira.
\ \\
Aos amigxs que estão (ou moraram em Juiz de Fora): Glerm Soares, Tiago Rubini, Anna Flávia. Aos amigxs de Campinas e São Paulo, que estiveram presentes ou na memória: Celso, Dani, Dhiego e Luisa, Evandro, Fábio, Felício, Frederico, Gustavo, Ivan, Israel, Larissa, Rebechi, Simone, Tati,  ao pessoal da república Lado C, João, Heron, Igor. Ao velho amigo Picchi!
\ \\
Aos freakcoders do \emph{labMacambira}, especialmente ao Caleb Luporini, Daniel Penalva, Renato Fabbri e Vilson Vieira pelo estímulo nestes anos. 
\ \\
Aos colegas e amigos do LABICbr, principalmente Angelica Rimenez, Felipe Caracas, Carlos Lobo, Carlos Rivera, 
Ivo Santiago. Lucas Araújo, Pedro Garbelini, Raquel Pires, Rafael Cortez. Sandra Leão, Mario Alzate,  Oda Scaltoni, Vitor Grilo, 
\ \\
Ao Professor Hans Joachim Koellreutter pelo centenário.}
\end{flushright}

\newpage
% ---

% ---
% Epígrafe do livecoding
% ---
%\epigraph{Alguém poderia imaginar uma interface musical na qual um músico especifica o som resultante desejado, em uma linguagem descritiva, na qual poderia ser então traduzida em parâmetros de partículas e renderizados em som. Uma alternativa poderia especificar um exemplo: "Faça um som assim (arquivo de som), mas com pouco vibrato''}{Curtis \citeonline{roads_microsound_2001}}
%\newpage

% ---
% RESUMOS
% ---
% resumo em português
\setlength{\absparsep}{18pt} % ajusta o espaçamento dos parágrafos do resumo
\begin{resumo}
Esta pesquisa foi construída a partir da seguinte pergunta: ao discutir uma prática conhecida como \textit{live coding}, quais categorizações sonoras (ou gêneros musicais) são contextualizadas? 

Contudo, a variedade dos diferentes exemplos da pergunta forçou a redução do \emph{universo de conceitos} inerente às categorizações sonoras estudadas. Neste sentido, o texto deste trabalho busca partir de um suposto espaço conceitual generalizado (\autoref{cap:introducao}); um método de pesquisa que contemple as multiplicidades deste espaço conceitual (\autoref{cap:metodologia}); uma contextualização histórica, anterior à concepção deste espaço conceitual (\autoref{cap:trabalhos_relacionados}); apresentação de alguns casos do compositor Australiano Andrew Sorensen (\autoref{cap:estudos_de_caso}).

Estes casos foram escolhidos por representarem, em um mesma pessoa, uma plaralidade de práticas musicais. Não é nossa intenção a estéticas musical, mais sim alguns meios técnicos pelos quais a variedade delas emerge. Mais especificamente, com uma mesma técnica de improvisação (ou \emph{live coding}), Sorensen improvisa desde \emph{Jazz}, \emph{Minimalismo} de um Steve Reich, \emph{Músicas-populares massivas}, e sínteses sonoras. Dois destes casos são de interesse por envolver uma busca pelo controle do Piano MIDI e o Piano Acústico. O terceiro é interessante por lembrar uma mistura das estéticas da \emph{Computer Music} dos anos setenta e 80, e a música de dança para ambientes sociais noturnos, algo que denominado na cena européia por \emph{algorave}. Por outro lado, os três são paradigmáticos por envolver replicação de estilos.

\textbf{Palavras-chaves}: \textit{Live coding}. Categorizações Sonoras. Sorensen
\end{resumo}

%%%%%%%%%% traduçoes resumo
\begin{comment}
% resumo em inglês
\begin{resumo}[Abstract]
 \begin{otherlanguage*}{english}
   This is the english abstract.

   \vspace{\onelineskip}
 
   \noindent 
   \textbf{Key-words}: latex. abntex. text editoration.
 \end{otherlanguage*}
\end{resumo}

% resumo em francês 
\begin{resumo}[Résumé]
 \begi'n{otherlanguage*}{french}
    Il s'agit d'un résumé en français.
 
   \textbf{Mots-clés}: latex. abntex. publication de textes.
 \end{otherlanguage*}
\end{resumo}

% resumo em espanhol
\begin{resumo}[Resumen]
 \begin{otherlanguage*}{spanish}
   Este es el resumen en español.
  
   \textbf{Palabras clave}: latex. abntex. publicación de textos.
 \end{otherlanguage*}
\end{resumo}
% ---
\end{comment}

% ---
% inserir lista de tabelas
% ---
%\pdfbookmark[0]{\listtablename}{lot}
%\listoftables*
%\cleardoublepage
% ---
% inserir lista de abreviaturas e siglas
% ---
%\begin{siglas}\label{siglas}
  \item[AMC] \textit{Composição Musical Automática/Algorítmica}
  \item[CAC] \textit{Composição Assistida por Computador}
  \item[CGA] \textit{Assistência Gerada por Computador}
  \item[CGC] \textit{Composição Gerada por Computador}
  \item[CGM] \textit{Composição Musical Generativa}
  \item[CSG] \textit{Sons Gerados por Computador}
  \item[DJ] \textit{code DJing}
  \item[MA] \textit{Música Algorítmica}
  \item[MG] \textit{Música Generativa}
  \item[MP] \textit{Música Processual}
\end{siglas}\label{siglas}
% ---
% inserir o sumario
% ---

\pdfbookmark[0]{\contentsname}{toc}
\tableofcontents*
\cleardoublepage

% ---
% inserir lista de ilustrações
% ---
\pdfbookmark[0]{\listfigurename}{lof}
\listoffigures*
%\cleardoublepage

% ----------------------------------------------------------
% ELEMENTOS TEXTUAIS
% ----------------------------------------------------------
\textual
\newpage
%\chapter*[Introdução]{Leia me}\addcontentsline{toc}{chapter}{Leia me}

Qualquer critica será bem-vinda. Porém, para organização delas, o autor solicita que sejam realizadas, por meio de comentários e apontamentos específicos, em um sistema de controle de versões, localizado em \url{https://www.bitbucket.com/cravista/mestrado/issues}. 

% ----------------------------------------------------------
% Introdução (exemplo de capítulo sem numeração, mas presente no Sumário)
% ----------------------------------------------------------
\chapter*{Introdução}\addcontentsline{toc}{chapter}{Introdução} 

%``O marquês de X tem, como se sabe, um belo gabinete de física, mas a Eletricidade é sua paixão e, se o paganismo ainda vigorasse, ele decerto ergueria altares elétricos. Ele sabia quais são minhas preferências e não ignorava que também sou fã da \emph{Eletromania}. Convidou-me, portanto, para um jantar onde estariam presentes, segundo ele, os medalhões da ordem dos eletrizantes e das eletrizadas''. Conviria conhecer essa eletricidade falada que, sem dúvida, revelaria muito mais sobre a psicologia da época do que sobre sua ciência\footnote{Bachelard, G. \textit{A formação do espírito científico}. p.~41. Ed. Contraponto. Trad. Estela dos Santos de Abreu 1938 -- 1996}.

Em seu artigo ``A filosofia à venda, a douta ignorância e a aposta de pascal'', \citeonline[p.~11--12]{santos_filosofia_2008} elabora a imagem mental de uma feira do conhecimento, onde teorias  são antropomorfizadas, escravizadas e vendidas:\ ``determinismo, livre arbítrio,universalismo, relativismo, realismo, construtivismo, marxismo, liberalismo, neoliberalismo, estruturalismo, pós-estruturalismo, modernismo, pós-modernismo, colonialismo, pós-colonialismo, etc.''. As idéias perderam a utilidade para os ex-adeptos, que não estão mais interessados em comprá-las. E vendem aos que supõe algum valor. Para efetuar a venda, é necessário estabelecer uma relação de custo-benefício, negociadas através de respostas às perguntas: ``qual a utilidade que esta ou aquela teoria poderá ter para mim? Qual o preço?''. A valorização ocorre quando esta teoria se torna mais apelativa que aquela. Com a concorrência, a livre-associação dos vendedores regulamentará compras e vendas de conhecimentos conforme seu interesse mais fundamental: se todas teorias forem vendidas não existirá teoria para se vender amanhã (o que não exclúi monopolizações). 

É possível pensar que Santos realiza uma  metáfora de um Mercado contemporâneo do conhecimento. Mas Santos esclarece que este tema é anterior à formação do espírito científico moderno: no texto satírico \emph{A venda de filosofias} (165), Luciano de Samósata (125 -- 181?),  escreve sobre um mercado estimulado por Zeus e gerenciado por Hermes:

\begin{citacao}
Hermes atrai os potenciais compradores, todos comerciantes, gritando alto e bom som “À venda! Uma variedade sortida de filosofias vivas! Posições de todo o tipo! Pagamento à vista ou mediante garantia!” (1905: 190). A “mercadoria” vai sendo exposta, os comerciantes vão chegando e têm o direito de interrogar cada uma das filosofias à venda, começando invariavelmente com a pergunta pela utilidade para o comprador e a sua família ou grupo. O preço é estabelecido por Zeus que, por vezes, se limita a aceitar ofertas feitas pelos comerciantes compradores. A venda tem pleno êxito e Hermes termina, ordenando às teorias que deixem de oferecer resistência e sigam com os seus compradores, ao mesmo tempo que avisa o público: “Senhores, esperamos vê-los amanhã. Estaremos oferecendo novos lotes úteis para homens comuns, artistas e comerciantes”   
\end{citacao}

Recolhendo esta imagem mental do Mercado de conhecimentos escravizados, sugerimos espelhar a metáfora para o assunto específico deste documento. As filosofias vendidas estão em uma feira chamada \emph{live coding}, que traduzimos por  improvisação de códigos. Ali vendem o tonalismo, o pós-tonalismo, o \emph{jazz}, a música algorítmica, o minimalismo, \emph{live computer music}, música ambiental, música \emph{rave},  música-ruído. Além disso são vendidas teorias da tecelagem, do audiovisual, da dança, e do lado científico, as ciências históricas e ciências cognitivas. Compramos uma amostra, cujo exemplares foram divididos em três grupos (ver Objetivos, p.~\pageref{sec:objetivos}).


\section*{Conceito de Pensamento Ortopédico}

Alex McLean defende que a metáfora é de importância central para a linguagem, e o faz através da \emph{teoria da Metáfora Conceitual}\apud[p.~32]{lakoff_methaphors_1980}{McLean2011}. Utilizamos a metáfora de Santos para exemplificar o tipo de pensamento analítico empregado por improvisadores-programadores, quando estes se posicionam em um ambiente acadêmico. Para explicar o sentido de uma metáfora, McLean sugere utilizar texto entre aspas, seguido de outro em caixa alta. 

Santos utiliza três metáforas. A primeira metáfora é o \metafora{pensamento ortopédico}{o mesmo processo especialista utilizado pelo médico responsável em corrigir deformidades do corpo}\disponivelem{http://www.priberam.pt/dlpo/ortopedia}. A segunda é a \metafora{a razão indolente}{insensibilidade com respeito às consequências do processo de correção}: ``A carência a respeito da finitude transforma-se num problema técnico-científico, enquanto a carência a respeito da diversidade infinita é ignorada como um não-problema.'' \cite[p.~15]{santos_filosofia_2008}. O terceiro é \metafora{o pensamento abissal}{a percepção de uma distância que delimita conhecimentos} \cite[p.~1--4]{santos_abissal_2007}:

\begin{citacao}
Consiste num sistema de distinções visíveis e invisíveis, sendo que as invisíveis fundamentam as visíveis. As distinções invisíveis são estabelecidas através de linhas radicais que dividem a realidade social em dois universos distintos: o universo  'deste lado da linha' e o universo 'do outro lado da linha'. A divisão é tal que 'o outro lado da linha' desaparece enquanto realidade, torna-se inexistente, e é mesmo produzido como inexistente. (\ldots) \textbf{O pensamento abissal moderno salienta-se pela sua capacidade de produzir e radicalizar distinções.} Contudo, por mais radicais que sejam estas distinções e por mais dramáticas que possam ser as consequências de estar de um ou do outro dos lados destas distinções, elas têm em comum o facto de pertencerem a este lado da linha e de se combinarem para tornar invisível a linha abissal na qual estão fundadas.\footnote{Grifo nosso.} 
\end{citacao}

\section*{Objetivos}\label{sec:objetivos}

• Investigar um Universo de Conceitos sobre a improvisação de códigos (\emph{live coding});

• Investigar um método de análise/criação, ortopédico, para uma improvisação de códigos;

• Investigar um Espaço Conceitual de uma sonoridade de um algoritmo musical de uma improvisação de códigos;

\section*{Estrutura dos Capítulos}

No \autoref{cap:introducao} selecionamos três abordagens, escolhidas por manterem alguma conexão com a improvisação de códigos no contexto musical.  No \autoref{cap:metodologia} apresentamos um modelo de formalização da criatividade, do ponto de vista do Modelo de Improvisação discutido por Alex \citeonline{McLean2011}. No \autoref{cap:estudos_de_caso}, organizamos conceitos para analisar o contexto de uma sonoridade  de \emph{A Study in Keith} (2009) de Andrew Sorensen.  O \autoref{app:A} foi adicionado para expor o material que estimulou o interesse pelo tema discutido. O \autoref{app:B} sugere a inclusão de um trabalho de \citeonline{mathews_groove_1970} no âmbito proto-histórico da improvisação de códigos.
\newpage

%------------------------------------------------------
% PRINCIPAL
% ----------------------------------------------------------
%\part{Ecologia de saberes no \emph{livecoding}}\label{parte1}
\pagenumbering{arabic}
1Giovanni \citeonline{mori_analysing_2015} perspetivou a importância da Música para os \emph{live coders} de um ponto-de-vista etnongráfico, ao apontar locais e costumes onde o \emph{live coding} é praticado:

\begin{citacao}
\emph{Live coding} é uma técnica artística de improvisação. Pode ser empregada em muitos contextos performativos diferentes: dança, música, imagens em movimento e mesmo tecelagem. Concentrei minha atenção no lado musical, que parece ser o mais proeminente. \cite[p.~117]{mori_analysing_2015}\footnote{Tradução de \emph{Live coding is an improvisatory artistic technique. It can be employed in many different performative contexts: dance, music, moving images and even weaving. I have concentrated my attention on the music side, which seems to be the most prominent.}}
\end{citacao}

O fragmento acima sugere que a Música, usada sozinha ou com outra linguagem (artes do corpo e audiovisual), articula os afetos entre intérpretes e público.

Separando a primeira personagem,  intérpretes (\emph{live coders}) idealizam uma comunidade participativa \cite[p.~71]{prospero_social_2015}, através de diversas apresentações, publicações de artigos, e registros em mídias audio/visuais. A permissividade pode ser um carro-chefe que, por sua vez, possibilita  hibridizações entre os locais e os modos de produção.

\section{O universo de conceitos durante uma improvisação}\label{sec:universo}

Uma sessão de \emph{live coding} é uma improvisação. \citeonline{mclean_music_2006}, um dos principais pesquisadores ingleses, no campo musical, articula a improvisação musical como algo computável, organizado por representações lógicas. 

O \emph{conceito} se torna uma instância, uma variável. É representado pela letra $c$. O  \emph{universo de conceitos} se torna um conjunto, representado pela letra $U$, como um agrupamento de instâncias de $c$. Por outro lado, $c$ é definido por conjuntos de ``objetos'' ($O'$), características ($F'$), e processos ($P'$). 

O primeiro contem a premissa para uma improvisação. O segundo é um conjunto de várias premissas, no sentido de uma rede de conceitos que definem o conceito principal. Os três últimos são classes de informações que definem como o conceito deve ser tocado para caracterizar uma linguagem musical específica.

Nesse sentido, diferentes performances de \emph{live coding} possuem diferentes configurações destas classes lógicas. Os objetos, características e processos de uma improvisação de \emph{Jazz} são diferentes daqueles de uma improvisação de \emph{Música eletroacústica} ou de uma improvisação de \emph{Músicas-populares massivas}\footnote{Sobre Músicas-populares massivas e regras de gêneros musicais, \cfcite{sa_se_2009}.}

Esta contraposição, específica entre o \emph{Jazz} e \emph{Músicas-populares massivas}, será trabalhado no \emph{cap:estudos_de_caso}, ao descrever \emph{Study in Keith} (2009)\footnote{Disponível em \url{https://vimeo.com/2433947}.}, \emph{Day Of The Triffords} (2009)\footnote{Disponível em \url{https://www.youtube.com/watch?v=DGcE8P5F29A}.}.

\section{A questão da tradução}\label{sec:traducao}

Giovanni Mori denomina a prática com dois termos separados, isto é, \emph{live coding}. Tomando esta separação, o prefixo \emph{live} é taduzido como ``ao vivo'', e o sufixo \emph{coding} como ``codificar''. Uma performance cujo ato é improvisar leis ou fórmulas dispersas.
-
Porém a tradução falha em identificar questões-satélites, como por exemplo, a própria Música. Para identificar algumas palavras-chaves, uma compilação dos dos anais do ICLC 2015 \cite{ICLC2015} pode ser esclarecedor.

A \autoref{fig:nuvemlivecoding} é uma \emph{nuvem de palavras} \footnote{Disponível em \url{https://github.com/amueller/word_cloud}.}, uma representação visual de dados textuais. Longe de ser uma análise linguística, utilizei esta ferramenta para me auto-induzir a encontrar as questões-satélites do \emph{live coding}.

\begin{figure}[!h]
\begin{center}
\centering
\includegraphics[scale=0.71]{./imagens/livecoding_cloud1.png}
\caption{Nuvem de palavras dos anais ICLC2015 \textbf{Fonte}: autor.}
\label{fig:nuvemlivecoding}
\end{center}
\end{figure}

\begin{table}
\caption{Tabela de classes qualitativas de termos utilizados nos anais do ICLC2015, agrupados por funções textuais.}
\small
    \begin{tabular}{ p{1.6cm} | p{1.4cm} | p{2cm} | p{1.45cm} | p{1.45cm} | p{1.45cm} | p{1.45cm} | p{1.45cm} | p{1.2cm}}
    \hline 
    \hline 

    \tiny \textbf{Número Qualitativo/Função} & \textbf{0} & \textbf{1}  & \textbf{2} & \textbf{3} & \textbf{4}  & \textbf{5} & \textbf{8} & \textbf{9}\\
    \hline 
    \hline 

    \tiny \textbf{Pessoas}  
    & - 
    & \tiny Collins, Blackwell, McLean, Grossi 
    & - 
    & - 
    & - 
    & -  
    & - 
    & - \\
    \hline

    \tiny \textbf{Aplicativos}
    & - 
    & \tiny SuperCollider, Gibber, SonicPi  
    & - 
    & - 
    & - 
    & -  
    & - 
    & \\
    \hline
    
    \tiny \textbf{Verbos}  
    & \tiny take, see, shared, networked, explore, made
    & \tiny make, provide, writing, solving, making
    & \tiny used
    & \tiny using, coding  
    & \tiny performer
    & - 
    & - 
    & -  \\
  \hline

     \tiny \textbf{Adjetivo ou numeral, ordinal}  
    & \tiny less, open, potential, similar, important, cognitive, virtual
    & \tiny first, real, electronic, visual, ensemble, possible, free, livecoding, aspect  
    & \tiny musical, many
    & \tiny new, one
    & - 
    & -  
    & \tiny live 
    & - \\
    \hline

    \tiny \textbf{Substantivo}  
    & \tiny Browser, point, approach, order, node, collaborative, number, source, present, community, server, framework, orchestra, digital, level, kind, type, memory, analysis, line, body, concept, technology, working, org, current, show, mean, end, processes, people, international
    & \tiny University, conference, proceedings, network, interface, environment, text, form, context, musician, space, paper, program, audience, function, change, control, human, laptop, interaction, structure, part, session, tool, result, create, object, case, algorithm, value, development, material, set, technique, parameter, idea, screen, video, application, support, composition, piece, knowledge, feature, cell, activity, art, action, information, method, web, rule, group, need, particular, project, allow, collaboration, programmer, member, play, output 
    & \tiny use, coder, process, state, example, way, software, research, problem, experience, design, improvisation, different, machine, pattern, audio
    & \tiny work, instrument
    & \tiny system, computer, user, language, time, practice, sound
    & \tiny programming
    & \tiny performance, code
    & \tiny ``live coding'', music  \\
    \hline
    \hline
   
    \end{tabular}
\label{tab:comparacao}
\end{table}

Uma breve análise da nuvem de palavras pode elucidar parte das questões-satélites. Na \autoref{tab:comparacao} filtrei parte dos resultados na nuvem de palavras por conjuntos de funções textuais -- sujeitos-humanos, sujeitos-ferramentas, verbos, adjetivos e substantivos -- e quantas vezes foram utilizados, em categorias qualitativas (0, menos usado e 9 o mais usado, sendo que 6 e 7 não apresentaram resultados). \footnote{O método de extração será explicado em um apêndice oportuno.}. 

No caso dos sujeitos-humanos, podemos ver nomes de Nick Collins e Alex McLean, praticantes responsáveis pela criação de um manifesto, cujo um fragmento será discutido no capítulo 2. Pietro Grossi, é um personagem recentemente estudado por \citeonline{mori_pietro_2015} como um caso prematuro de \emph{live coding}, a partir do final da década de sessenta.

No caso dos sujeitos-ferramentas, destacamos o papel do \emph{SuperCollider}, já citado anteiormente, e do \emph{Gibber}\footnote{Disponível em \url{http://gibber.mat.ucsb.edu}}. Ambos são ambiente de programação para de síntese sonora e composição algorítmica. Uma semelhança paradigmática para estes ambientes, é o procedimento de compilação de códigos conhecido como \emph{Just In Time} \cite{aycock_brief_2003}. Enquanto no primeiro \emph{software} a questão está posta em uma máquina -- \emph{laptop} -- local, o \emph{Gibber} representa a viabilidade do navegador de \emph{internet} como plataforma musical \citeonline{roberts_gibber:_2012,wyse_viability_2014}.

Os verbos fornecem informação sobre o comportamento dos improvisadores de códigos. Além da atividades como \emph{performatizar} e \emph{codificar}, é notável atividades sociais ligadas à visão, à escrita, à técnica, à lógica. Embora a Música seja a atividade proeminente do \emph{live coding}, não obtivemos resultados que retornassem, por exemplo, a palavra \emph{hearing}. Isso é significativo, e no \autoref{cap:trabalhos_relacionados}, exploro estas ações sob a ótica da Música de Processos de Steve Reich.

Já os adjetivos destacam características da prática, onde \emph{live} é a palavra-chave. Como será observado no \autoref{cap:trabalhos_relacionados}, a ação pode ocorrer em uma sala de concerto, um espaço público ou em uma casa noturna. Palavras como  \emph{electronic}, podem sugerir tanto uma música ``eletroacústica'', quanto gêneros de música para dançar. \emph{Visual} remete a uma característica tão fundamental quanto a Música. Um sem número de performances utilizam a projeção de telas de computadores como dispositivo de ``transparência''; isto é, uma ideologia de justificação do ato de improvisação. \emph{Ensemble} destaca uma a natureza de grupos. Poucas performances \emph{solo} são realizadas se comparadas às performances de \emph{duos}, \emph{trios}. 

Substantivos relacionam a atividade como processo (\emph{process}). Como será discutido no Capítulo 2, o \emph{live coding} se apropria de conceitos da Música como Processo de Steve Reich e da Música Generativa de Brian Eno para justificar um processo de codificação incessante. Por outro lado, palavras como \emph{university}, \emph{research} e \emph{technology}, e \emph{laptop} acusam não apenas uma prática artística, mas um programa de pesquisa tipo de performance, que utiliza um computador para resultados audiovisuais, realizados por universitários. A esfera de pesquisa acadêmica permitiu ramos de desenvolvimento com linguagens de programação, cognição, inteligência artificial, semiologia, performance musical (improvisação), e mais recentemente, etnologia, conferindo à produção de \emph{live coding} uma aura de legitimidade escolar.
%\newpage
\chapter{Definições Históricas da Improvisação de códigos}\label{sec:protohistoria}

Este capítulo será dedicado à construção de um espaço conceitual histórico, do ponto de vista musical. Isto é, aqueles exemplos citados como ``proto-históricos'' que possuem alguma similaridade com o conjunto de regras práticas publicadas por \citeauthoronline{ward_live_2004} \ver{sec:laptoptoplap}. \citeonline{mori_pietro_2015} descreve um caso prematuro de \emph{live coding} na Itália, com o compositor Pietro Grossi \ver{sec:grossi}. Grossi sacrificou a questão timbrística para trabalhar na questão performática. As atividades de grupos californianos como \emph{The League of Automatic Composers}, e \emph{The Hub}, serão apresentadas no contexto cultural estadounidense, estimuladas pelo mercado emergente de microcontroladores \ver{sec:baiasaofranscisco}. O compositor Ron Kuivila será apresentado a partir de uma performance prototípica (sem projeção) de improvisação de códigos \ver{sec:kuivila}. Com um certo buraco cronológico, pulamos para meados da década de oitenta para o começo dos anos dois mil, período em que programadores ingleses,  comprometidos com as artes visuais e a Música Eletrônica para Dançar, rebatem uma crítica sobre o papel cênico do músico durante uma apresentação com computadores \ver{sec:laptoptoplap}. 


\section{Pietro Grossi}\label{sec:grossi}

Pouco conhecido no contexto geral da música européia, o compositor veneziano Pietro Grossi foi  um dos pioneiros da \emph{Computer Music} Italiana. De seu interesse pelos computadores como instrumento musical, o pensamento que rege seus programas de computador é bastante prático, e sacrifica questões timbrísticas para concentrar na performance. Isto é, o problema da diversidade timbrística é reduzido para uma única forma de onda, e o problema de performance é colocado em primeiro plano. Além disso, é possível notar a prática de transcrição de peças conhecidas. Nas palavras de \citeonline[p.~126]{mori_pietro_2015}:

\begin{citacao}
\traducao{Grossi começou a se interessar por música computacional durante a primeira metade doas anos 60, quando ele quando ele organizou um programa de rádio centrado em torno da "música inovadora" em geral \cite{giomi_conversasioni_1999}. Contudo a primeira experiência de Grossi com um computador foi em Milão, no Centro de Pesquisa Elétrica da Olivetti-General. Aqui, auxiliado por alguns técnicos internos e engenheiros, ele conseguiu compor e gravar alguns de seus primeiros trabalhos em música computacional. Eles foram, em sua maior parte, transcrições de música clássica ocidental. Contudo, houve algumas exceções, por exemplo, uma faixa chamada Mixed Paganini}{Grossi began to be interested in computer music during the first half of the 1960s, when he hosted a radio program centred around “innovative music” in general (Giomi1999). However, the first Grossi's experience with calculator took place in Milan, in the Olivetti-General Electric Research centre. Here, aided by some internal technicians and engineers, he managed to compose and record some of his first computer music works. They were, for the most part, transcriptions of Western classical music. However, there were some exceptions, for example a track called Mixed Paganini.}
\end{citacao}

Existe um exemplo na \emph{internet}\disponivelem{https://www.youtube.com/watch?v=ZQSP_wF7wSY}. Um disco realizado no Studio di Fonologia musicale di Firenze, entitulado ``GE-115 - Computer Concerto'', lançado pela Olivetti em 1967: \traducao{Do lado A existem algumas transcrições de música clássica, e do lado B existem três canções originais. (\ldots) Este 7''$[$polegadas$]$ foi distribuído como presente de natal e de ano novo pela companhia Olivetti.}{On side A there's transcribed classical music, on side B there are three original songs. (\ldots). This 7" was distributed as a christmas and new year gift by the Olivetti company.}. No entanto, é necessária uma correção sobre o lado A, e um detalhe do lado B\disponivelem{https://www.discogs.com/Studio-Di-Fonologia-Musicale-Di-Firenze-GE-115-Computer-Concerto/release/575632}. As transcrições realizadas foram da \emph{Oferenda Musical BWV 1079} de J.S.Bach e um dos 24 Caprichos de Nicolò Paganini : \emph{i})Canon a 2 \emph{Super Thema Regium}; \emph{ii})Canon Perpetuum a 2 \emph{Quaerando Invenietis}; \emph{iii}) Canon a 3 \emph{Super Thema Regium} e; \emph{iv}) Capriccio n$^o$ 5. As peças originais de Grossi foram três, sendo que uma dela é uma recomposição dos caprichos:  \emph{i})\emph{Mixed Paganini}; \emph{ii}) \emph{Permutations Of Five Sounds} e; \emph{iii}) \emph{Continuous}.


Mori explica que a peça \emph{Mixed Paganini} derivou da transcrição do quinto \emph{capriccio}: \traducao{Praticamente, Grossi modificou, auxiliado por alguns programas rudimentares, o material sonoro original. (\ldots) Uma coleção posterior dos Capricci de Paganini, gravado em Pisa, foi revista por Barry Truax na Computer Music Journal \cite{truax_barry_1984}}{Practically, Grossi modified, aided by some rudimental music programs, the original sound material. (\ldots) A later collection of Paganini’s Capricci, recorded in Pisa, was reviewed by Barry Truax on Computer Music Journal (Truax1984).} O tipo de material sonoro utilizado nestas peças utiliza um método tradicional, se comparada com o trabalho de Max Mathews. Por exemplo, as pesquisas desenvolvidas\footnote{\cfcite{mathews_digital_1963,mathews_technology_1969,roads_interview_1980,park_interview_2009,di_nunzio_genesi_2010}} estão debruçadas na resolução do problema timbrístico, contraposto à capacidade de processamento dos \emph{mainframes}. O caminho tomado nos EUA (e depois na Europa) seguiu a elaboração de algoritmos do timbre (Unidades Geradoras) para depois trabalhar sequência da programação-partitura. Para Grossi, com o problema da capacidade de processamento, os compositores deveriam esperar por melhores implementações técnicas dos engenheiros, e naquele momento, o computador foi capaz de maximizar o pensamento musical de um Período Comum (séc. XVII-XX, \emph{circa}). Isto é, operações como inversão, retrogradação, retrogradação da inversão, aceleração, diminuição, eram executadas rapidamente com comandandos. 

Grossi não fica satisfeito com o trabalho, e a Olivetti não se interessa mais por suas pesquisa. Ao procurar emprego e novos espaços criativos, é contratado pelo \traducao{Centro de pesquisa IBM, dentro do Instituto CNR (\emph{Centro Nazionale per la Ricerca}: Comitê Nacional para a Pesquisa)}{IBM Research Centre in Pisa, inside the CNR Institute (Centro Nazionale per la Ricerca: National Research Committee)}(\idemibdem). Ali desenvolveu, em linguagem FORTRAN, o DCMP (\emph{Digital Computer Music Program}), um programa integrado com um terminal de vídeo e um teclado alfanumérico, e segundo Mori, ao usar este terminal de áudio, o compositor escolheu deliberadamente abandonar o problema do timbre.  Esta abordagem parte de uma abordagem ``preguiçosa'' (\emph{prigo}). Grossi dizia sobre si mesmo, como ``uma pessoa que está consciente de que o seu tempo é limitado e não quer perder tempo em fazer coisas inúteis ou na espera de alguma coisa quando não é necessária.''\footnote{Tradução nossa de \emph{a person who is aware that his or her time is limited and do not want to waste time in doing useless things or in waiting for something when it is not necessary.}} (\idemibdem). Neste sentido, defendia que o desenvolvimento de novos timbres gerados por computador deveria esperar por melhores implementações de \emph{hardware}:

\begin{citacao}
\traducao{(\ldots) o intéprete era capaz de produzir e reproduzir música em tempo real, digitando alguns comandos específicos e os parâmetros composicionais desejados. O som resultante vinha imediatamente depois da operação de decisão, sem qualquer atraso causado por cálculos. Haviam muitas escolhas de reprodução no programa: era possível salvar na memória do computador peças de músicas pré-existentes, para elaborar qualquer material sonoro no disco rígido, para administrar arquivos musicais e iniciar um processo de composição automático, baseado em algoritmos que trabalham com procedimentos ``pseudo-casuais''. Existia também uma abundância de escolhas para mudanças na estrutura da peça. Um dos mais importantes aspectos do trabalho de Grossi foi que todas intervenções eram instantâneas: o operador não tinha que esperar pelo computador terminar todas operações requisitadas, e depois ouvir os resultados. Cálculos de dados e reprodução sonoras eram simultâneos. \textbf{Esta simultaneidade não era comum no campo da \emph{Computer Music} daquele tempo, e Grossi deliberadamente escolheu trabalhar desta forma, perdendo muito no lado da qualidade sonora. Seu desejo era poder escutar os sons resultantes imediatamente} (\idemibdem).}{(\ldots) the performer was able to produce and reproduce music in real time by typing some specific commands and the desired composition's parameters. The sound result came out immediately after the operator's decision, without any delay caused by calculations. There were many reproduction choices inscribed in this software: it was possible to save on the computer memory pieces of pre-existing music, to elaborate any sound material in the hard disk, to manage the music archive and to start an automated music composition process based on algorithms that worked with “pseudo-casual” procedures. There were also plenty of choices for piece structure modifications. One of the most important aspects of Grossi’s work was that all the interventions were instantaneous: the operator had not to wait for the computer to finish all the requested operations and then hear the results. Data calculation and sound reproduction were simultaneous. This simultaneity was not common in the computer music field of that time and Grossi deliberately chose to work in this way, losing much on the sound quality’s side. His will was to listen to the sound result immediately.}
\end{citacao}

Substituímos o termo ``preguiçoso'' por  \emph{reflexividade}, ou a \traducao{habilidade de um programa manipular como dados algo que representa o estado do programa durante sua própria execução, o mecanismo para codificação de estados de execução é chamado \emph{reificação}.\cite[p.~1]{malefant_reflection_1996}.}{the ability of a program to manipulate as data something representing the state of the program during its own execution, the mechanism for encoding execution states as data being called reification.}. Parece existir apenas um anseio em recuperar a reflexividade entre o dedo que toca a tecla e o som resultante. No entanto \citeonline[p.~127]{mori_pietro_2015} coloca a figura do compositor como consciente dos problemas técnicos, e de um descarte pelo pensamento timbrístico corrente na Europa:

\begin{citacao}
\traducao{O DCMP foi compilado na fase inicial do desenvolvimento de tecnologias computacionais. Naquele tempo, os recursos de cálculo eram escassos e, para obter a reprodução em tempo-real citada, era necessário pedir por pouca quantidade de dados. Contudo, o músico veneziano foi capaz escrever um programa muito leve, capaz de modificar somente os parâmetros necessários para um cálculo de recursos reduzidos: altura e duração. A síntese de timbres necessita de uma quantidade imensa de dados, e então a escolha foi descartá-la temporariamente, e todos os sons eram reproduzidos com o timbre de uma onda quadrada Esta forma de onda era gerada por extração do estado binário do \emph{pin} de saída da placa mãe que controla o programa. Essa saída tinha um único \emph{bit}, e então a onda sonora gerada era o resultado desta mudança do estado binário. Desta forma, o computador não emprega quaisquer recursos para calcular a síntese sonora, economizando-os para o processo de produção musical. Grossi não estava interessado na qualidade da saída sonora em sua primeira fase em Pisa. O que importava particularmente era a capacidade em travalhar em tempo eral, ou, em outras palavras, para ter a escolha de escutar imediatamente ao que ele escreveu no teclado do terminal de vídeo \apud{giomi_conversasioni_1999}{mori_pietro_2015}.}{The DCMP was compiled in the early phase of computer technology development. At that time, the calculation resources were low and, to obtain the just cited real time reproduction, it had to ask for very low quantity of data. Therefore, the Venetian musician chose to write very light software, able to modify only parameters that required a few calculation resources: pitch and duration. Timbre synthesis needed a big amount of data, so that choice was temporarily discarded and all the sounds were reproduced with square wave timbre. This waveform was generated by extracting the binary status of a motherboard's exit pin controlled by the software. This exit had only one bit, so the sound wave generated was the result of this bit status changing. In this way, the computer did not employ any resources for calculating the sound synthesis, saving them for music production process. Grossi was not very interested in the quality of sound output in this first phase in Pisa. What he cared particularly was to be able to work in real time, or, in other words, to have the choice to listen immediately to what he typed on the video terminal’s keyboard.}
\end{citacao}

É importante situar que a escolha deliberada para DCMP é justificada nos anos 70. Até a metade da década, Grossi foi capaz de implementar melhorias de timbre, \traducao{digitalmente controladas, mas com uma tecnologia de síntese analógica. Foi lançado em 1975 e foi chamado de TAU2 (\emph{Terminale Audio 2$^a$ versione -- Terminal de Áudio 2$^a$ versão}) (\idemibdem).}{digitally controlled but with analog sound synthesis technology. It was launched in 1975 and called TAU2 (Terminale Audio 2a versione – Audio Terminal 2nd version)}. Esta tecnologia tinha um programa, o TAUMUS, uma modificação do DCMP, que podia tocar:

\begin{citacao}
\traducao{(\ldots) até doze vozes simultâneas. Estas doze vozes eram divididas em três grupos, compostos de quatro canais cada. O operador poderia atribuir um timbre diferente para cada grupo, que era modulado usando síntese aditiva com sete sobretons. Cada sobretom era controlado individualmente pelo programa.}{(\ldots) twelve different voices simultaneously. These twelve voices were divided in three groups, composed of four channels each. The operator could choose to assign a different timbre to every single group, which was modulated using additive synthesis with seven overtones. Every overtone could be controlled individually by software.}
\end{citacao}

Segundo \citeonline[p.~128]{mori_pietro_2015}, Uma outra novidade do TAU2-TAUMUS, em relação às concepções do DCMP, era o conceito de modulação de modelos (\emph{modelli modulanti}), ou \traducao{uma espécie de remendos que agem em um parâmetro musical}{they were a sort of patches that acted on some musical parameter.}. É importante notar que, ao aplicar um remendo (\emph{patch}), através de comandos escritos com o teclado alfanumérico, o programa não interrompia o fluxo sonoro. \traducao{Esta era uma inovação crítica do ponto de vista performativo, porque então Grossi era capaz de tocar, e interagir, em tempo real com o programa, ao escrever instruções no teclado sem parar o fluxo sonoro.}{ This was a critical innovation under the performative point of view, because then Grossi was able to play and to interact in real time with the software, by typing instructions on the keyboard without stopping the sound flux.}

Por outro lado, Grossi foi além deste problema reflexivo. É importante lembrar que, no final da década de 1970, Grossi contribuiu para o desenvolvimento de tecnologias telemáticas \ver{sec:telepresenca}. O TAU2-TAUMUS sofreu uma considerável modificação, sendo que era possível controlar o sistema digital-analógico remotamente. O novo programa foi batizado de TELETAU \citeonline[p.~128--129]{mori_pietro_2015}. Uma descrição de 1986 aponta a possibilidade de acesso a um computador da CNR, em Pisa, com uma conexão da rede BITNET, que permitia de maneira virtual o acesso para qualquer programador. No entanto o TELETAU não vingou por diversos motivos: falhas e bugs que aumentavam de maneira dramática a manutenção e custos; o alto custo de transmissão e, por último mas não menos, a baixa qualidade da saída sonora devido à lentidão da conexão de dados.


\begin{citacao}
\traducao{$[$Pietro$]$ Grossi fez sua primeira experiência do tipo durante uma conferência de tecnologia em Rimini em 1970, onde o músico reproduzia algumas de suas composições, bem como sons randômicos, empregando um terminal de vídeo conectado pelo telefone para o computador da CNR em Pisa. A RAI, empresa de radiodifusão italiana, emprestou suas pontes de rádio $[$Comunicação entre duas antenas$]$ para enviar sinais sonoros entre Pisa e Rimini. É como se fosse o primeiro experimento de telemática musical no mundo.\cite[p.~129]{mori_pietro_2015}}{Grossi made his first experience of this kind during a conference on technology in Rimini in 1970, where the musician reproduced many of his compositions and random sounds as well, by employing a video terminal connected via telephone to the CNR's computer in Pisa. RAI, the Italian public broadcasting company, lent its powerful FM radio bridges to send back sound signals from Pisa to Rimini. It is likely to be the first official experiment of musical telematics in the world.
}
\end{citacao}



\section{Baía de São Franscisco}\label{sec:baiasaofranscisco}

A prática musical com o computador, realizada na Costa Oeste dos EUA durante os anos 1970 e 1980,  é bastante diversa daquela realizada em grandes centros europeus (como por exemplo Ircam ou o Conservatório de Haia). \citeauthoronline{brown_indigenous_2013} comentam que esta prática decorre de alguns fatores sociais. O primeiro fator seria uma diversidade musical existente na Costa Oeste dos EUA. O segundo fator é a transmissão de conhecimentos musicais propostos por Terry Riley, Pauline Oliveros e LaMonte Young, David Tudor e Gordon Mumma. Em especial, estes dois compositores propunham a utilização de  circuitos eletrônicos eles mesmos como atores musicais \cite[3$^o$ parágrafo]{brown_indigenous_2013}. Neste sentido, uma música computacional, colaborativa e livre de restrições de regras emerge em torno do \emph{Mills College} em Oakland. 

\traduzcitacao{Com o florescimento da indústria de computadores pessoais na Baía de São Franscisco, o acesso às novas tecnologias e pessoas que desenvolveram elas era talvez o melhor no mundo. Mas se para todos os jovens com fortunas como panos para suas mentes (e seus futuros) que perseguiam um excitamento aditivo na construção de máquinas eletrônicas, também existiam políticos utópicos que sonhavam com uma nova sociedade construída no livre e aberto acesso à informação, e na abrangente tecnologia baseada em sistemas inteligentes. Esta também é a cultura que deu ao mundo a música ``New Age'', uma versão aguada e comercializada das músicas com base em modos e drones que Terry Riley, Pauline Oliveros, e LaMonte Young inventaram durante os anos cinquenta e sessenta. Mas a música feita na Costa Oeste também incluiam improvisações barulhentas e livre de restrições, que sobraram das revoluções contra-culturais dos anos 60}{With the flowering personal computer industry in the Bay Area, access to the new digital technologies and to the people who developed them was perhaps the best in the world. But for all the young men with fortunes in the back of their minds (and in their futures) who pursued the addictive excitement of building electronic machines, there were also the political utopians whose dream was of a new society built on the free and open access to information, and on a comprehensively designed technology based on embedded intelligence. This was also the culture that gave the world "New Age" music, a watered-down and commercialized version of the musics based on modes and drones that Terry Riley, Pauline Oliveros, and LaMonte Young invented here during the late fifties and early sixties. But West Coast music-making also included a free-wheeling, noisy, improvisational edge left over from the counter-cultural revolutions of the sixties.}{1$^o$ parágrafo}{brown_indigenous_2013}



\subsection{The League of Automatic Composers}

Na segunda metade da década de setenta, Jim Horton começou a adquirir micro-controladores KIM-1\footnote{\emph{Keyboard Input Monitor}. Disponível em \url{http://www.6502.org/trainers/buildkim/kim.htm}.} com interesses musicais. Segundo \citeauthoronline{brown_indigenous_2013}, não demorou para que outros compositores interessados comprassem. Discussões informais posteriores, que incluiam, além de Horton, David Behrman e John Bischoff, Rich Gold, Cathy Morton, Paul Robinson, e Paul Kalbach. Em 1977 e 1978  Horton colaborou com duas peças, apresentadas no \emph{Mills College}, que interligavam sistemas musicais elaborados com os microcontroladores \ver{fig:siskim1}. A primeira peça era construída sobre algoritmos inspirados nas teorias matemáticas de Leonard Euler (séc. XVIII). A segunda peça também explorava a comunicação entre os microcontroladores, de forma que \traducao{notas ocasionais da minha $[$Bischof$]$ máquina faziam a máquina de Jim transpor atividades melódicas de acordo com minha nota base\cite[5$^o$ parágrafo]{brown_indigenous_2013}}{the occasional tones of my $[$Bischof$]$ machine caused Jim’s machine to transpose its melodic activity according to my "key" note.}. Em 1978, Bischof, Gold e Horton formaram uma banda nas proximidades de Berkley. Posteriormente Behrman se junta ao trio. No dia 26 de Novembro gravam um \emph{Extended Play} (EP)\footnote{Gravação muito longa para um \emph{demo} e insuficiente para um disco de vinil da época.} de quatro faixas no \emph{Blind Lemmon}, um ponto de encontro musical fundado em 1958\disponivelem{http://www.chickenonaunicycle.com/Berkeley\%20Art.htm}. O disco foi lançado pela Lovely Music (NY) em 1980 como \emph{The Hub: Computer Network Music}.  Durante este tempo, foi formado o grupo \emph{``The League of Automatic Music Composers''}\footnote{Segundo \citeonline[6$^o$parágrafo]{brown_indigenous_2013}, o nome é uma referência ao grupo ``The League of Composers'' formado por Aaron Copland nos anos 20.}, que além de  Bischof, Behrman, contava com Tim Perkis, Scot Gresham-Lancaster, Mark Trayle e Phil Stone. Mas nosso foco será a formação no trio formado por Horton, Bischof e Perkis.

\begin{figure}[!h]
  \centering
  \includegraphics[scale=0.7]{imagens/siskim1.jpg}
  \caption{Sistema de música computacional de John Bischof \emph{circa} 1980. Foto: Eva Shoshanny\protect\footnotemark. \textbf{Fonte}: \citeonline{brown_indigenous_2013}.}
  \label{fig:siskim1}
\end{figure}

\footnotetext{Tradução de \emph{John Bischoff's KIM-1 computer music system circa 1980 photo: Eva Shoshany}}

É interessante uma descrição de uma performance durante 1979. Propomos aqui realizar um paralelo com \emph{happenings} (acontecimentos), manifestações artísticas já consolidadas para a época:

\begin{citacao}
Na primavera de 1979, montamos uma série quinzenal regular de apresentações informais sob os auspícios da \emph{Bay Center for the Performing Arts}. Todos outros domingos à tarde passávamos algumas horas configurando nossa rede de KIMs na sala \emph{Finnish Hall}, na Berkeley, e deixávamos a rede tocando, com retoques aqui e ali, por uma ou duas horas. Os membros da audiência poderiam ir e vir como quisessem, fazer perguntas, ou simplesmente sentar e ouvir. Este foi um evento comunitário de tipos como outros compositores aparecendo, tocando ou compartilhando circuitos eletrônicos que tinham projetado e construído. Um interesse na construção de instrumentos eletrônicos de todos os tipos parecia estar "no ar". Os eventos da sala \emph{Finn Hall} foram feitos para uma cena com paisagens sonoras geradas por computador misturado com os sons de grupos de dança folclórica ensaiando no andar de cima e as reuniões ocasionais do Partido Comunista na sala de trás do edifício velho venerável. A série durou cerca de 5 meses que eu me lembre.\cite[online]{brown_indigenous_2013}\footnote{Tradução nossa de: \emph{In the spring of 1979, we set up a regular biweekly series of informal presentations under the auspices of the East Bay Center for the Performing Arts. Every other Sunday afternoon we spent a few hours setting up our network of KIMs at the Finnish Hall in Berkeley and let the network play, with tinkering here and there, for an hour or two. Audience members could come and go as they wished, ask questions, or just sit and listen. This was a community event of sorts as other composers would show up and play or share electronic circuits they had designed and built. An interest in electronic instrument building of all kinds seemed to be "in the air." The Finn Hall events made for quite a scene as computer-generated sonic landscapes mixed with the sounds of folk dancing troupes rehearsing upstairs and the occasional Communist Party meeting in the back room of the venerable old building. The series lasted about 5 months as I remember.}}
\end{citacao}

Em 1980, Gold e Behrman abandonam o grupo, sendo que Tim Perkis se junta. Este foi período em que o grupo solidifica suas atividades na região da Baía de São Franscisco. É interessante notar que uma metodologia modular começa a ser formalizada para permitir maior flexibilidade entre os sistemas de Horton, Bischof e Perkis. Isto é, ao invés de soldar componentes eletrônicos aos controladores, os membros conectavam os microcontroladores através de portas -- o que para a época era arriscado ao ponto de queimar componentes. Com as conexões feitas, tocavam até três horas, tempo em que ouviam e ajustavam (\emph{tuning}) os sistemas em interação\disponivelem{https://www.youtube.com/watch?v=HW0qax8M68A}\cite[7$^o$ parágrafo]{brown_indigenous_2013}. Outro evento de importância é a associação do grupo com a banda \emph{Rotary Club}, formada por alunos recém-formados da \emph{Mills College}: Sam Ashley, Kenneth Atchley, Ben Azarm, Barbara Golden, Jay Cloidt e Brian Reinbolt. O grupo \traducao{baseava seu estilo de performace em torno de uma caixa de comutação projetada por Brian Reinbolt}{based their performance style around an automatic switching box designed by member Brian Reinbolt.}\cite[8$^o$ parágrafo]{brown_indigenous_2013}. Em 1983 o grupo reduziu suas atividades, época em que Horton contraiu artrite degenerativa.

\begin{figure}[!h]
    \centering
    \includegraphics[scale=0.7]{imagens/perkis.jpg}
    \caption{Circuito do computador caseiro dedicado à síntese sonora de Tim Perkis, usado no começo dos anos 1980. Foto: Eva Shoshany\protect\footnotemark. \textbf{Fonte}: \citeonline{brown_indigenous_2013}}
    \label{fig:perkis}
  \end{figure}

\footnotetext{Tradução de \emph{Tim Perkis' homebuilt computer-driven sound synthesis circuitry used in early 1980s. photo: Eva Shoshany}.}

Seria possível discutir a elaboração de uma ``rede de composições''. No entanto, \citeonline[11$^o$ parágrafo]{brown_indigenous_2013} comentam que estas não eram composições específicas, mas sim concertos inteiros: \traducao{ocasiões públicas para escuta compartilhada}{public occasions for shared listening.}. Este conceito pode ser melhor compreendido através de uma descrição do processo criativo da banda, que lidavam com um sistema limitado, de \traducao{baixa velocidade 1 Mhz e poucos dados (8 bits)}{slow speed (1 Mhz) and data width (8 bits)} com uma ênfase do grupo em explorar uma luteria composicional \footnote{\cfcite{iazzetta_musica_2009,soares_luteria_2015}.} acompanhada de performance ao vivo; ou \traducao{A ênfase estava na exploração da tecnologia em mãos que poderia ser adquirida pessoalmente ou construída a partir do zero, em vez do desejo incessante de melhores ferramentas.}{The emphasis was on exploration of the technology at hand—technology that could be personally acquired or built from scratch—rather than the endless wish for better tools.}\cite[22$^o$ parágrafo]{brown_indigenous_2013}: 

\begin{citacao}
\traducao{Os membros da liga geralmente adaptavam composições solo para usar dentro da banda. Estes solos eram desenvolvidos independentemente por cada compositor, e eram tipicamente baseados em esquemas de algoritmos de um tipo ou outro. Existiam características de improvisação diferentes para muitas delas, como bem as músicas eram diferentes em detalhes. Teorias matemáticas, sistemas de afinação experimentais, algoritmos de inteligência artificial, projetos de instrumentos de improvisação, e performance interativa eram algumas das áreas exploradas nestes trabalhos (\ldots) Os solos tocavam simultaneamente no cenário de grupo, se tornando ``sub''-composições que interagem, cada uma enviando e recebendo dados pertinentes para o funcionamento musical. \cite[12$^o$ parágrafo]{brown_indigenous_2013}.}
{League members generally adapted solo compositions for use within the band. These solos were developed independently by each composer and were typically based on algorithmic schemes of one kind or another. There was a distinctly improvisational character to many of these as the music was always different in its detail. Mathematical theories of melody, experimental tuning systems, artificial intelligence algorithms, improvisational instrument design, and interactive performance were a few of the areas explored in these solo works. (\ldots) The solos, played simultaneously in the group setting, became interacting "sub"-compositions, each sending and receiving data pertinent to its musical functioning.}
\end{citacao}

\subsection{The Hub}

O grupo \emph{The Hub} era constituído, inicialmente pelo duo Bischoff e Perkis, após a saída de Horton. É interessante notar que um processo colaborativo entre os membros, e entre o duo formado por Chris Brown e Mark Trayle, em 1986, foi estimulado através de  uma série de concertos, em galerias e espaços musicais comunitários, resultando em um festival , \emph{THE NETWORK MUSE - Automatic Music Band Festival}. Neste festival outros grupos também realizaram suas apresentações, sendo um duo formado por Scott Greham-Lancaster/Richard Zvonar e um trio formado por Phil Burk/Larry Polansky/Phil Stone.

Bischoff pontua que o nome da banda era uma maneira simbólica de caracterizar o sistema musical centralizado, \traducao{(\ldots) um pequeno microcontrolador como caixa de correio, para postar dados usados no controle de seus sitemas individuais, que eram então acessados por outro intérprete, para usar de qualquer maneira e em qualquer tempo que escolher.}{(\ldots) a small microcomputer as a mailbox to post data used in controlling their individual music systems, which was then accessible to the other player to use in whatever way and at whatever time he chose.}. O computador centralizado original, \emph{Hub}, era um dos microcontroladores KIM-1 utilizados na época do \emph{The League}. O sistema, de certa forma, é uma sensibilidade computacional, no âmbito musical, do modelo da máquina de Turing, permitindo a performance de até quatro executantes simultâneos:  

\begin{citacao}
\traducao{\emph{The Hub} originalmente surgiu como uma maneira de limpar uma bagunça. (\ldots) Toda vez que nós ensaiamos, um conjunto complicado de conexões \emph{ad-hoc} entre computadores tinham de ser feitas. Isso criou um sistema com um comportamento rico e variado, mas sujeito a falhas, e trazer outros jogadores ficava difícil. Mais tarde, procuramos uma maneira de abrir o processo, para torná-lo mais fácil para os outros músicos tocarem no contexto de rede. O objetivo era criar uma nova maneira para pessoas fazerem música juntos. A solução bateu no ponto da facilidade de uso, e fornecimento de uma interface de usuário padrão, de modo que os jogadores poderiam conectar praticamente qualquer tipo de computador. \emph{The Hub} é um pequeno computador dedicado a passar mensagens entre os jogadores. Ele serve como uma memória comum, mantendo informações sobre a atividade de cada jogador que seja acessível para os computadores de outros jogadores \cite[seção 2.1]{brown_indigenous_2013}. }{The Hub originally came about as a way to clean up a mess. John Bischoff, (\ldots) Every time we rehearsed, a complicated set of ad-hoc connections between computers had to be made. This made for a system with rich and varied behavior, but it was prone to failure, and bringing in other players was difficult. Later we sought a way to open the process up, to make it easier for other musicians to play in the network situation. The goal was to create a new way for people to make music together. The solution hit upon had to be easy to use and provide a standard user interface, so that players could connect almost any type of computer. The Hub is a small computer dedicated to passing messages between players. It serves as a common memory, keeping information about each player's activity that is accessible to other players' computers.}
\end{citacao}

Em 1987, Nick Collins e Phil Niblock realizam uma curadoria para realizar uma performance telemática entre a \emph{Experimental Media} e \emph{The Clocktower} em Nova York. Para isso, chamam os membros do \emph{The Hub}, que estimulam uma performance de um grupo único, dividido em dois trios, formado por John Bischoff, Tim Perkis, Mark Trayle, Chris Brown, Scot Gresham-Lancaster, e Phil Stone. Estes dois trios se comunicam, entre os dois espaços, através de dois novos \emph{Hub} intercomunicáveis. Cada \emph{Hub} era um sistema centralizado para cada trio. As peças tocadas, ``Simple Degradation'', ``Borrowing and Stealing'' e ``Vague Notions'' permitiram a concepção de uma performance de um  \traducao{sexteto acusticamente divorciado mas informacionalmente ligado.}{acoustically divorced, but informationally joined sextet.} \cite[seção 2.2]{brown_indigenous_2013}.

\subsection{Ron Kuivila}\label{sec:kuivila}

\citeonline{mclean_patterns_2009} comentam  a performance \emph{Water Surfaces}, realizada na edição de 1985 da STEIM \footnote{\emph{STudio for Electro-Instrumental Music}, disponível em \url{http://steim.org/about/}.}, em Amsterdã, como significativa para a concepção de uma improvisação de códigos (excluindo a tecnologia de projeção visual) . A performance chamou a atenção, e foi incluída na primeira faixa do disco ``\emph{TOPLAP001 - A prehistory of live coding}'', como uma reconstrução da peça, 2007 \footnote{Disponível em \url{http://toplap.org/wiki/TOPLAP_CDs}.}; uma nota sobre a performance descreve o seguinte: \traducao{Esta obra usou programação FORTH ao vivo; Curtis \citeonline{roads_steim_1986} testemunhou e relatou a performance de Ron Kuivila feita na STEIM em Amsterdã, em 1985; a performance original termina com a quebra do sistema\ldots
}{This work used live FORTH programming; Curtis Roads witnessed and reported a performance by Ron Kuivila at STEIM in 1985; the original performance apparently closed with a system crash\ldots}


\traduzcitacao{Ronald Kuivila programou um computador Apple II no palco para cirar sons densos, rodopiantes e métricos, disposto em camadas e dobravam sobre si. Considerando o equipamento usado, os sons eram surpreendentemente grandes em escala. Kuivila teve problemas em controlar a peça devido q problemas sistêmicos. Ele finalmente entrou em dificuldades técnicas e finalizou a performance}{Ronald Kuivila programmed an Apple II computeronstage to create dense, whirling, metric sounds that layered in and folded over each other. Considering the equipment used, the sounds were often surprisingly gigantic in scale. Kuivila had trouble controlling the piece due to system problems. He finally gave in to technical difficulties and ended the performance}
{p.~47}
{roads_steim_1986}
%FORTH é uma linguagem de programação elaborada por Charles Moore (1938-). Entre seus paradigmas de programação, utiliza da \emph{reflexividade} como dispositivo de escrita e observação dos algoritmos elaborados.

Ge \citeonline{wang_historical_2005}, em uma comunicação pessoal com Curtis Roads, cita a seguinte declaração: \traducao{Eu vi o \emph{software} FORTH de Ron Kuivila quebrar e queimar no palco em Amsterdã em 1985, mas antes disso, não fez uma música muito interessante. A performance consistiu de digitação}{I saw Ron Kuivila's Forth software crash and burn onstage in Amsterdam in 1985, but not before making some quite interesting music. The performance consisted of typing.}

Nenhuma fonte sonora foi encontrada disponível online. 

\section{LAPTOP}\label{sec:laptoptoplap}

Por último, vamos discutir um recorte do documento-manifesto ``\emph{Live Algorithm Programming and Temporary Organization for its Promotion}'', de \citeauthoronline{ward_live_2004,mclean_patterns_2009}. Nossa discussão visa apontar espaços conceituais mais diretos da improvisação de códigos. Isto é, uma identidade cultural da organização TOPLAP \ver{sec:toplap}.  Dentre este manifesto, selecionamos dois pontos: \emph{i}) um comentário sobre a ideologia de projeção de telas \ver{sec:obscurantismo} e; \emph{ii}) ``Show us your screens'', como uma revisão de regras práticas do \emph{live coding} \ver{sec:showusyourscreens}.

``\emph{Live Algorithm Programming and Temporary Organization for its Promotion}'' \cite{ward_live_2004,blackwell_programming_2005} é um primeiro documento-manifesto sobre o \emph{live coding} como modalidade artística, e de suas regras práticas. O seu acrônimo LAPTOP representa o principal equipamento técnico utilizado. Este manifesto expõe o ambiente de performance característico do \emph{algorave} e um suporte ideológico para o \emph{Code DJing}. Ritos técnicos do improvisador, como por exemplo, a projeção do código, são justificados através do discurso de transparência e provável colaboração entre intérprete e público:

\begin{citacao}
O \emph{Livecoding} permite a exploração de espaços algorítmicos abstratos como uma improvisação intelectual. Como uma atividade intelectual, pode ser colaborativa. Codificação e teorização podem ser atos sociais. Se existe um público, revelar, provocar e desafiar eles com uma matemática complexa se faz com a esperança de que sigam, ou até mesmo participem da expedição. Estas questões são, de certa forma, independentes do computador, quando a valorização e exploração do algoritmo é o que importa. Outro experimento mental pode ser encarado com um DJ ao vivo codificando e escrevendo uma lista de instruções para o seu \emph{set} (feito com o iTunes, mas aparelhos reais funcionam igualmente bem). Eles passam ao HDJ $[$ \emph{Headphone Disk Jockey} $]$ de acordo com este conjunto de instruções, mas no meio do caminho modificam a lista. A lista está em um retroprojetor para que o público possa acompanhar a tomada de decisão e tentar obter um melhor acesso ao processo de pensamento do compositor. \cite[p.~245]{ward_live_2004} \footnote{Tradução nossa de: \emph{Live coding allows the exploration of abstract algorithm spaces as an intellectual improvisation. As an intellectual activity it may be collaborative. Coding and theorising may be a social act. If there is an audience, revealing, provoking and challenging them with the bare bone mathematics can hopefully make them follow along or even take part in the expedition. These issues are in some ways independent of the computer, when it is the appreciation and exploration of algorithm that matters.   Another thought experiment can be envisaged in which a live coding DJ writes down an instruction list for their set (performed with iTunes, but real decks would do equally well). They proceed to HDJ according to this instruction set, but halfway through they modify the list. The list is on an overhead projector so the audience can follow the decision making and try to get better access to the composer’s thought process.}}
\end{citacao}

Adiante podemos ver outros dois conceitos aglutinados: a Música de Processos, e a Música Generativa:


\begin{citacao}
Contudo, alguns músicos exploram suas idéias como processos de \emph{software}, muitas vezes ao ponto que o \emph{software} se torna a essência da música. Neste ponto, os músicos podem ser pensados como programadores explorando seu código manifestado como som. Isso não reduz seu papel principal como um músico, mas complementa, com a perspectiva única na composição de sua música. \textbf{Termos como ``música generativa'' e ``música de processos'' tem sido inventados e apropriados para descrever esta nova perspectiva de composição}. Muita coisa é feita das supostas propriedades da chamada ``música generativa'' que separa o compositor do resultado do seu trabalho. Brian Eno compara o fazer da música generativa com o semear de sementes que são deixadas para crescer, e sugere abrir mão do controle dos nossos processos, deixando eles ``brincarem ao vento''. \footnote{\opcit[p.~245-246]{ward_live_2004}. Tradução nossa de \emph{Indeed, some musicians explore their ideas as software processes, often to the point that a software becomes the essence of the music. At this point, the musicians may also be thought of as programmers exploring their code manifested as sound. This does not reduce their primary role as a musician, but complements it, with unique perspective on the composition of their music. Terms such as “generative music” and “processor music” have been invented and appropriated to describe this new perspective on composition. Much is made of the alleged properties of so called “generative music” that separate the composer from the resulting work. Brian Eno likens making generative music to sowing seeds that are left to grow, and suggests we give up control to our processes, leaving them to “play in the wind”.}}
\end{citacao}

Se por um lado, a Música como um Processo Gradual\footnote{\cfcite{reich_music_1968}} e a Música Generativa são referenciais possíveis na improvisação de códigos, essa não é a questão inicial. A ligação conceitual do \emph{live coding} com a Música de Processos e a Música Generativa é relativa ao uso de algoritmos, mas não ao resultado sonoro como processo de escuta. Por exemplo, uma abordagem sobre a Música de Processos é apresentada por \citeonline[p.~128]{mailman_agency_2013}, e descreve a Música Minimalista de Processos como uma Música de Algoritmos Simples,  um processo determinístico que age sobre focos de quadros temporais. Já a \traducao{Música Generativa é sensitiva às circuntâncias, isso quer dizer que irá reagir diferentemente dependendo das suas condições iniciais, onde ocorre e assim por diante.}{Generative music is sensitive to circumstances, that is to say it will react differently depending on its initial condition, on where it's happening and so on.}\cite{eno_generative_1996}. \citeonline[p.~130]{McLean2011} problematiza o processo na improvisação de códigos da seguinte forma:

\begin{citacao}
\traducao{Na codificação ao vivo a performance é o processo de desenvolvimento de \emph{software}, em vez de seu resultado. O trabalho não é gerado por um programa acabado, mas através de sua jornada de desenvolvimento do nada para um algoritmo complexo, gerando mudanças contínuas da forma musical ou visual ao longo do caminho. Isto contrasta com a arte generativa popularizada pela música geradora de Brian \citeonline{eno_generative_1996}. (\ldots) O resultado segue mais ou menos o mesmo estilo, com apenas algumas permutações, dando uma idéia das qualidades da peça. Isto é bem ilustrado pelo nosso estudo de caso de um artista-programador, que executa seu programa poucas vezes não para produzir novas obras, mas para obter diferentes perspectivas sobre o mesmo trabalho.}{In live coding the performanceis the \emph{process} of software development, rather than its outcome. The work is not generated by a finished program, but through its journey of development from nothing to a complex algorithm, generating continuously changing musical or visual form along the way. This is by contrast to \emph{generative} art popularised by the generative music of Brian \citeonline{eno_generative_1996} (\ldots)Output more or less follows the same style, with only a few permutations giving an idea of the qualities of the piece. This is well illustrated by our case study of an artist-programmer, who ran their program a few time not to produce new works, but to get different perspectives on the same work. }
\end{citacao}

\section{TOPLAP}\label{sec:toplap}

Uma permutação na ordem das letras do acrônimo LAPTOP dá origem ao acrônimo TOPLAP. \citeonline[p.~246]{ward_live_2004} e \citeonline{ramsay_algorithms_2010} apontam que este acrônimo dinâmico; isto quer dizer que as primeira, terceira  e quinta letras possuem diversos significados (ver \autoref{fig:TOPLAP}):

\begin{citacao}
\traducao{A organização TOPLAP (www.toplap.org), cuja sigla possui diversas interpretações, uma sendo \emph{Organização Temporária para a Proliferação da Programação de Algoritmos Ao Vivo}, foi criada para promover e explorar o \emph{live coding}. TOPLAP nasceu em um bar enfumaçada em Hamburgo à uma da manhã em 15 de Fevereiro de 2004.}{The organisation TOPLAP (www.toplap.org), whose acronym has a number of interpretations,  one being the Temporary Organisation for the Proliferation for Live Algorithm Programming, has been set up to promote and explore live coding. TOPLAP was born in a smoky Hamburg bar at 1am on Sunday 15th February 2004}
\end{citacao}

\begin{figure}[!h]
  \centering
  \includegraphics[scale=0.6]{imagens/TOPLAP.png}
  \caption{Definição do siginificado de TOPLAP. \textbf{Fonte}: \citeonline{ramsay_algorithms_2010}.}
  \label{fig:TOPLAP}
\end{figure}

O símbolo ``|'' é uma representação gráfica do operador lógico \emph{OR} (OU), bastante utilizado em algoritmos condicionais. Isto é, \emph{Temporary }| \emph{Trasnational} | \emph{Terrestrial} | \emph{Transdimensional} significa que as letras ímpares ``T'', e ``P'' e ``A'', podem significar um ou outro termo indicado pelo algoritmo.

Este comportamento, de permutar ordem das letras é praticado por Nick Collins (1975-); a permutação de suas letras é utilizada pelo pesquisador para gerar pseudônimos como Click Nilson, ou Sick Lincoln. Isso transparece uma técnica de uso frequente na improvisação de códigos, provavelmente pela facilidade de sua implementação computacional em amb. Por exemplo, o \emph{SuperCollider} oferece um método chamado \emph{scramble}, que embaralha a ordem de um conjunto (de caracteres). Mais especificamente, a permutação de letras transparece uma reorganização gramatica, mas que também reflete em técnicas de reorganização algorítmica da gramática musical.

\subsection{\emph{Show us your screens}}\label{sec:showusyourscreens}

Além das performances inaugurais nos festivais Europeus, o manifesto Lubeck04, \traducao{iniciado em um ônibus de trânsito Ryanair\disponivelem{https://www.ryanair.com/pt/pt/}, em  Hamburgo, para o aeroporto Lübeck\cite[p.~247]{ward_live_2004}}{begun on a Ryanair transit bus from Hamburg to Lubeck airport}, mais conhecido como ``\emph{Show us your screens}'', prescreve algumas regras práticas do \emph{live coding}. 

\begin{citacao}
Exigimos:

• Acesso à mente do intérprete, para todo o instrumento humano.

• Obscurantismo é perigoso. Mostre-nos suas telas.

• Programas são instrumentos que podem modificar eles mesmos.

• O programa será transcendido - Língua Artificial é o caminho.

• O código deve ser visto assim como ouvido, códigos subjacentes visualizados bem como seu resultado visual.

• Codificação ao vivo não é sobre ferramentas. Algoritmos são pensamentos. Motosserras são ferramentas. É por isso que às vezes algoritmos são mais difíceis de perceber do que motosserras.

Reconhecemos contínuos de interação e profundidade, mas preferimos:

• Introspecção dos algoritmos.

• A externalização hábil de algoritmo como exibição expressiva/impressiva de destreza mental.

• Sem \emph{backup} (minidisc, DVD, safety net computer).

Nós reconhecemos que:

• Não é necessário para uma audiência leiga compreender o código para apreciar, tal como não é necessário saber como tocar guitarra para apreciar uma performance de guitarra.

• Codificação ao vivo pode ser acompanhada por uma impressionante exibição de destreza manual e a glorificação da interface de digitação.

• Performance envolve contínuos de interação, cobrindo talvez o âmbito dos controles, no que diz respeito ao parâmetro espaço da obra de arte, ou conteúdo gestual, particularmente direcionado para o detalhe expressivo. Enquanto desvios na tradicional taxa de reflexos táteis da expressividade, na música instrumental, não são aproximadas no código, por que repetir o passado? Sem dúvida, a escrita de código e expressão do pensamento irá desenvolver suas próprias nuances e costumes. 
\footnote{\loccit{ward_live_2004}. Tradução nossa de:\emph{We demand: \begin{inparaenum}[•]
\item Give us access to the performer's mind, to the whole human instrument.
\item Obscurantism is dangerous. Show us your screens.
\item Programs are instruments that can change themselves.
\item The program is to be transcended - Artificial language is the way.
\item Code should be seen as well as heard, underlying algorithms viewed as well as their visual outcome.
\item Live coding is not about tools. Algorithms are thoughts. Chainsaws are tools. That's why algorithms are
sometimes harder to notice than chainsaws.
\end{inparaenum}. We recognise continuums of interaction and profundity, but prefer:  \begin{inparaenum}[•]
\item Insight into algorithms
\item The skillful extemporisation of algorithm as an expressive/impressive display of mental dexterity
\item No backup (minidisc, DVD, safety net computer)
\end{inparaenum}. We acknowledge that: \begin{inparaenum}[•]
\item It is not necessary for a lay audience to understand the code to appreciate it, much as it is not necessary
to know how to play guitar in order to appreciate watching a guitar performance.
\item Live coding may be accompanied by an impressive display of manual dexterity and the glorification of the
typing interface.
\item Performance involves continuums of interaction, covering perhaps the scope of controls with respect to
the parameter space of the artwork, or gestural content, particularly directness of expressive detail. Whilst
the traditional haptic rate timing deviations of expressivity in instrumental music are not approximated in
code, why repeat the past? No doubt the writing of code and expression of thought will develop its own
nuances and customs.
\end{inparaenum}}}
\end{citacao}

Escolhemos dois pontos de interesse ; as frase ``Obscurantismo é perigoso. Mostre-nos suas telas'' e  ``Algoritmos são pensamentos, motosserras são ferramentas'',  foi muito discutida no processo de qualificação desta tese. O primeiro questiona: é realmente necessário a projeção dos códigos para a questão da performance, do ponto de vista musical/cênico?

\subsubsection{Obscurantismo é perigoso, mostre-nos suas telas}\label{sec:obscurantismo}

O manifesto acima surgiu, entre outros motivos, como uma resposta ao artigo ``\emph{Using Contemporary Technology in Live Performance; the Dilemma of the Performer}'' \cite{schloss_dilemma_2003}. A crítica principal de \citeauthoronline{ward_live_2004} refere-se ao sétimo dos questionamentos sugeridos para uma performance de improvisação ao vivo com computadores. Isto é, em um contexto de embate acadêmico, o desafio colocado por \citeonline[p.~241]{schloss_dilemma_2003} foi um estímulo considerável para  emancipação da improvisação de códigos. É curioso notar que o problema e a intenção de Schloss eram opostas ao que foi proposto por \citeauthoronline{ward_live_2004}:


\begin{citacao}
\traducao{Para reiterar, agora que nós temos computadores rápidos o suficiente para execução ao vivo, nós temos novas possibilidades, e um novo problema. Do começo da evidência arqueológica da música até agora, música era tocada acusticamente, e sempre foi fisicamente evidente como o som era produzido; alí existia uma relação de proximiidade entre gesto e resultado. Agora nós não temos mais que seguir as leis da física (ultimamente temos, mas não nos termos do que o observador vê), uma vez que nós temos completo poder do computador como intérprete e intermediário entre nosso corpo físico e o som produzido. \textbf{Por esta causa, a ligação entre gesto e resultado foi completamente perdido, se é que existe ligação. Isto significa que nós podemos ir além da relação de causa-e-efeito entre executante e instrumento que faz a mágica.} Mágica é bom; muita mágica é fatal.}{
To reiterate, now that we have fast enough computers toperform live, we have new possibilities, and a new problem.From the beginning of the archeological evidence of musicuntil now, music was played acoustically, and thus it wasalways physically evident how the sound was produced; there was a nearly one-to-one relationship between gesture andresult. Now we don’t have to follow the laws of physicsanymore (ultimately we do, but not in terms of what theobserver observes), because we have the full power of computers as interpreter and intermediary between our physicalbody and the sound production. Because of this, the link between gesture and result can be completely lost, if indeed there is a link at all. This means that we can go so far beyond the usual cause-and-effect relationship between performer and instrument that it seems like magic. Magic is great; too much magic is fatal
} 
\end{citacao}

A crítica de \citeonline[p.~239]{schloss_dilemma_2003}: \traducao{considerar a visão do observador sobre os modos de performance das interações físicas e mapeamentos de gestos em som, para fazer uma performance convincente e efetiva}{Its now necessary, (\ldots) ;to consider the observer’s view of the performer’s modes of physical interactions and mappings from gesture to sound, in order to make the performance convincing and effective.} era especificamente direcionada aos compositores que improvisam música computacional no palco com foco apenas no aspecto sonoro ou tecnológico. Sua questão tange a ausência de gestos referenciais, esforço físico, no caso de performances com dispositivos extendidos, o problema do movimento exagerado, e a expectativa cênica na performance musical:


\begin{citacao}
\traducao{1. Causa-e-efeito é importante, pelo menos para o observador/audiência em uma sala de concerto. 
\ \\
2.Corolário: Mágica na performance é bom. Muita mágica é fatal! (chato).
\ \\
3. Um componente visual é essencial para a audiência, tal como existe um aparato visual de entrada para parâmetros e gestos.
\ \\
4. Sutileza é importante. Grandes gestos são facilmente visíveis de longe, o que é bom, mas eles são movimentos de desenho animado se comparados à execução de um instrumento musical.
\ \\
5. Esforço é importante. Neste sentido, nós estamos em desvantagem de desempenho na performance musical com o computador.
\ \\
6. Improvisação no palco é bom, mas ``mimar'' o aparato no palco não é improvisação, é edição. É provavelmente mais apropriado fazer isso no estúdio antes do concerto, ou se durante o concerto, com o console no meio ou atrás da sala de concerto.
\ \\
7. Pessoas que representam devem representar. Um concerto de música de computador não é uma desculpa/oportunidade para um programador(a) se sentar no palco. Sua presença melhora ou impede o desempenho da representação?
}{1. Cause-and-effect is important, at least for the observer/audience in a live concert venue. 2. Corollary: Magic in a performance is good. Too much magic is fatal! (Boring). 3. A visual component is essential to the audience, such that there is a visual display of input parameters/gestures. The gestural aspect of the sound becomes easier to experience. 4. Subtlety is important. Huge gestures are easily visible from far away, which is nice, but they are cartoon- movements compared to playi
ng a musical instrument. 5. Effort is important. In this regard, we are handicapped in computer music performance. 6. Improvisation on stage is good, but “baby-sitting” the apparatus on stage is not improvisation, it is editing. It is probably more appropriate to do this either in the studio before the concert, or if at the concert, then at the console in the middle or back of the concert hall. 7. People who perform should be performers. A computer music concert is not an excuse/opportunity for a computer programmer to finally be on stage. Does his/her presence enhance the performance or hinder it?} 
\end{citacao}

Duas opiniões divergentes resolvem seus problemas de maneiras divergentes sem considerarem como uma pode auxiliar a outra. No item 3, é apontado uma questão: para a audiência, e não para o improvisador, o componente visual é essencial (substantificação provável da prática). Ward et al. vão no caminho oposto ao de Schloss, e exageram este item, ao projetar códigos. Mas para Schloss, realizar isso é mimar o aparato (e o público), e tornar a apresentação pedante. É curioso notar que Schloss faz um apontamento importante, no item 5, sobre a ausência de esforço. Não que ela seja premissa para o resultado sonoro,  mas para o público, e Schloss trata exatamente deste ponto, ela é importante na performance musical (item 1). Mas a crítica mais ácida é o item 7, cujo pensamento não difere de uma lógica produtivista: as atividades de artista e de programador devem ser bem definidas, e separadas. Para Schloss, são duas atividades que não se complementam. Para McLean, são interdisciplinares.


\subsection{Algorithms are Thoughts, Chainsaws are Tools}

``Algorithms are Thoughts, Chainsaws are Tools'' é o nome dado ao vídeo de Stephen \citeonline{ramsay_algorithms_2010}, publicado no Vimeo, em  27 de fevereiro de 2010, como um \emph{Coffee-Table Movie}. É uma análise pessoal da performance de \emph{Strange Places} de Andrew Sorensen. O nome do vídeo é derivado de uma das regras práticas apresentadas na \autoref{sec:showusyourscreens}, p. \pageref{sec:showusyourscreens}; mais especificamente, o sexto item.

O algoritmo como pensamento é um espaço conceitual abstrato \ver{cap:metodologia}; pode conter qualquer fundamento teórico pertinente para uma improvisação específica. O dispositivo usado (motoserra, máquina de tecelagem ou o computador) é um meio pelo qual uma estratégia transversal \ver{sec:imagem_mental} toma sua forma sonora. É interessante aqui notar que este vídeo contém uma descrição e comentários que podem elucidar a frase-alvo sob o prisma da partitura musical. Abaixo realizei uma compilação de fragmentos de alguns dos comentários que considerei pertinentes. Ramsey apresenta a seguinte descrição do vídeo:

\begin{citacao}
Um curta sobre \emph{livecoding} apresentado como parte do Grupo de Estudos de Crítica de Códigos, em 2010, por Stephen Ramsay. Apresenta uma leitura ao vivo $[$\emph{live reading}$]$ de uma performance do compositor Andrew Sorensen. Também fala sobre J.D. Salinger, the Rockets, tocando instrumentos, Lisp, do clima em Brisbane e tímpanos \footnote{\loccit{ramsay_algorithms_2010} Tradução de \emph{A short film on livecoding presented as part of the Critical Code Studies Working Group, March 2010, by Stephen Ramsay. Presents a "live reading" of a performance by composer Andrew Sorensen. It also talks about J. D. Salinger, the Rockettes, playing musical instruments, Lisp, the weather in Brisbane, and kettle drums.}.}.
\end{citacao}

Sem entrar em méritos críticos do registro, limitamo-nos a descrever como um VLog, uma variante no formato audiovisual de \emph{weblogs}\footnote{\cfcite{baker_origins_2008}.}. Se caracteriza por ser um vídeo de curta duração, com opiniões pessoais de quem fez, geralmente no quarto da pessoa, com \emph{headsets} (microfone+headphones). A prática de inserir comentários em um é bastante útil para levantar outras opiniões. Realizamos a tradução de alguns comentários. Não são nossas opiniões, mas podem oferecer ao leitor uma abrengência sobre o que pensa um público entusiasta.

Amanda French nega a utilização do termo \emph{partitura} para explicitar diferenças no uso da programação-partitura, em uma performance de improvisação com o computador, para uma performance não-improvisada com partitura.

\begin{citacao}
A noção de partitura não se aplica aqui, é como não fosse possível aplicá-lo ao músico de \emph{jazz} ou tocador de \emph{bluegrass}. (\ldots). Levanta a questão, para mim, se, em uma sessão de \emph{livecoding} *feita*, constite simplismente no ato de digitar em um programa existente, seria tão convincente -- eu acho que isso pode definitivamente ter pontos de interesse. Ou qual seria o análogo do \emph{livecoding} para uma performance não-improvisada de música?\footnote{\loccit{ramsay_algorithms_2010} Tradução parcial de \emph{The notion of "sheet music" doesn't apply here, as it wouldn't apply to a jazz musician or a bluegrass picker. Even the name of his environment, Impromptu, makes that point. Raises the question for me precisely of whether a livecoding session that *did* consist of simply typing in an existing program would be as compelling -- I think it would definitely have its points of interest, actually. Or what would the livecoding analog be to a non-improvisational live performance of music?}}
\end{citacao}

Um segundo comentário de Matt King, coloca a pergunta de Amanda em outra perspectiva:

\begin{citacao}
O que torna o \emph{livecoding} diferente, e pode a performance de música tradicional imitar isso? Para responder esta questão, parece importante notar que as formas nas quais a música improvisada muitas vezes apela para alguma noção de autenticidade ou gênio. Enquanto o \emph{livecoding} ele mesmo à noção de virtuosismo de código, ``autenticidade'' parece fora de lugar aqui. Se música improvisada sugere expressão, o \emph{livecoding} sugere um conjunto de restrições na expressão, descrevendo os parâmetros através dos quais a máquina $[$midi$]$ ganha expressão \footnote{Tradução nossa de \emph{(\ldots) What makes livecoding different, and can a traditional music performance mimic it? To answer this question, it seems important to note the ways in which improvised music often appeals to some notion of authenticity or genius. While livecoding might lend itself to some notion of coding virtuosity, "authenticity" seems out of place here. If improvised music is expression, livecoding suggests a setting of constraints on expression, describing the parameters through which the machine (midi) gets expressed.}}
\end{citacao}

Michel Pasin defende que o ato de improvisação musical requer conhecimentos técnicos prévios, mas não necessariamente correlacionados ao conhecimento do que é uma partitura: \traducao{Em geral, é somente dominando um instrumento que você pode esquecer sobre a técnica e concentrar em 'dizer' coisas com o instrumento.}{In general, it is only by mastering an instrument that you can forget about the technique and concentrate on 'saying' things with the instrument.}. Este caso é bastante específico de performances com linguagens de baixo e alto-nível. Porém seria possível objetar que a prática de construção de linguagens artificiais, no topo de outras linguagens artificiais, possibilita um praticante não-familiarizado com a programação elaborar rotinas computacionais.  Porém aí caímos em um problema: estaria o praticante realizando uma improvisação de códigos? ou melhor, isso importa, se o objetivo é a criação musical? Para delinear estas questões buscamos definir no próximo capítulo o que consideramos por objetivo de um agenciamento sonoro improvisado.

\section{Discussão}

Oferecemos um cenário proto-histórico do ponto de vista, na Itália com o compositor Pietro Grossi, nos EUA com Jim Horton, John Bischoff, Tim Perkis, e na Holanda com Ron Kuivila (residente nos EUA), que deram suporte ao pensamento promovido na Inglaterra e Alemanha. Sugestões para essa proto-história foram colocados no \autoref{app:B}. Com um hiato na década de noventa, não que sejam ausentes informações, mas nossa pesquisa focou mais em citar eventos de improvisações de códigos de um nicho específico, aqueles músicos interessados em uma luteria composicional\footnote{\cfcite{iazzetta_musica_2009,soares_luteria_2015}}, onde a atividade de \emph{hackear} sistemas (conectar cabos, soldar componentes, programar códigos, etc.) se torna parte de uma sensibilidade computacional da atividade musical, próximo daquilo que Mumma e Tudor diziam como utilizar circuitos eletrônicos eles mesmos como atores musicais.
%\newpage
%\chapter{Metodologia de análise de uma Improvisação musical de códigos}\label{cap:metodologia}

Este capítulo contextualiza a \emph{criatividade} dentro do pensamento proposto por um improvisador programador \cite{McLean2011}. Apresentaremos uma definição de criatividade \ver{sec:criatividade} para contextualizar o \traducao{Quadro Conceitual de Sistemas Criativos}{Creative System Frameworks, \emph{ou CSF}.} \ver{sec:csf}. Em seguida especificamos este quadro para os propósitos específicos desta pesquisa \ver{sec:diagrama} e formalizamos o último capítulo \ver{sec:formaliza}.

\section{Criatividade}\label{sec:criatividade}

Para analisar uma improvisação de códigos, nos termos de um processo criativo com um instrumento musical (computador), definimos \emph{criatividadade} a partir de uma proposição do livro ``The Creative Mind: myths and mechanisms'' de Margaret \citeonline{boden_creative_1990},  \ver{sec:diferencas}. Em seguida discutimos a formação de \emph{imagens mentais}, durante um processo criativo idealizado \ver{sec:imagem_mental}, afim de discutir um mecanismo teórico para realização destas imagens mentais em performances de improvisadores-programadores \ver{sec:tidal}. 

\subsection{O paradoxo da criatividade}\label{sec:diferencas}

É possível discutir quais valores a palavra \emph{criatividade} carrega. Se por um lado faltam-nos documentos que comprovem qual é sua definição comum, sugerimos que uma definição cíclica (``a criatividade cria'') é uma maneira de pensar que, socialmente, a avaliação de um comportamento criativo está baseada em sua capacidade de \emph{gerar novidades}. Para \citeonline[p.~2]{thornton_quantitative_2007}, Boden discute que esta concepção de criatividade cria um paradoxo quando colocada sob a perspectiva da lógica mecanicista do computador:

\begin{citacao}
\traducao{O ponto inicial de Boden para o desenvolvimento de sua sua explicação, é a observação de que o conceito de criatividade contém um paradoxo. Por definição, criatividade cria, i.e., produz alguma coisa nova. Mas se estamos comprometidos com uma abordagem mecanicista do mundo -- nenhum milagre é permitido -- iremos acreditar que tudo o que ocorre é, em princípio, previsível. Iremos acreditar também que qualquer coisa nova deve ser construída de componentes existentes. Isso implica que nada pode ser intrinsicamente novo.}{Boden’s starting point for the development of her account is the observation that the concept of creativity contains a paradox. By definition, creativity creates, i.e., it produces something new. But if we are committed to a mechanistic account of the world — no miracles allowed — we believe that everything that occurs is predictable in principle. We also believe that any new thing must be constructed from existing components. This implies that nothing can ever be intrinsically new.}
\end{citacao}


 De outro ponto de vista, o mecanismo de julgamento do que é ou não é novidade parte de uma identificação do que é possível e do que não é, de forma que a impossibilidade é valorizada: \traducao{Para justificar a classificação de uma idéia criativa... alguém deve identificar os princípios generativos com respeito ao que é impossível}{To justify calling an idea creative... one must identify the generative principles with respect to which it is impossible.} \apud[p.~40;p.~3]{boden_creative_1990}{thornton_quantitative_2007}. Essa capacidade gerativa do aparentemente impossível pode estar conectada com uma inversão desta idéia comum de criatividade. Para Boden, \traducao{Qualquer ato criativo é fundado na conceitualização ou realização de um ponto dentro de um espaço conceitual particular (\emph{idem}, \emph{ibidem})}{Any creative act is thus founded on conceptualisation or the realisation of a point within a particular ‘conceptual space’.}. 

Para Thornton e \citeonline[p.~450--451]{wiggins_framework_2006}, na primeira edição de seu livro, Boden não explica como é o método de acesso aos espaços conceituais. Por outro lado, Boden oferece uma taxonomia de tipos de criativadade. 

As nomenclaturas derivadas são explicadas como uma divisão de duas classificações da criatividade \ver{fig:ortogonal}. A primeira classificação divide a criatividade em criatividade-psicológica, ou criatividade-pessoal (\emph{P-creativity}), e criatividade-histórica (\emph{H-creativity}). A segunda classificação separa criatividade em criatividade-exploradora e criatividade-transformacional.

\begin{figure}
\centering
\begin{tikzpicture}[scale=2.5]
\tikzstyle{every node}=[draw,shape=circle];
\path (0:0cm) node (v0) {\tiny $Criatividade$};
\path (0:1.25cm) node (v1) {\tiny $Criatividade-H$};
\path (2*90:1.25cm) node (v2) {\tiny $Criatividade-P$};
\path (3*90:1.25cm) node (v3) {\tiny $Criatividade Exp.$};
\path (90:1.25cm) node (v4) {\tiny $Criatividade Trans.$};
\draw (v0) -- (v1)
(v0) -- (v2)
(v0) -- (v3)
(v0) -- (v4);
\end{tikzpicture}
\caption{Classificação da criatividade : 1) criatividade-psicológica/criatividade-histórica; 2) critividade exploradora/criatividade transformacional. \textbf{Fonte}: autor com base em \citeonline{wiggins_framework_2006}.}
\label{fig:ortogonal}
\end{figure}

\begin{citacao}
\traducao{A distinção é entre o sentido de criar um conceito que nunca foi criado antes $[$criatividade-P$]$, e um conceito que nunca foi criado antes por um criador específico $[$criatividade-H$]$. Esta distinção será tangencialmente relevante para meu argumento aqui, mas antes de prosseguir, eu noto que esta não é uma simples escolha binária, mas ao invés, uma contextualização multidimensional: pode ser possível, por exemplo, para um comportamento criativo ser P-criativo em uma sociedade, mas H-criativo em outra; deste ponto de vista da segunda sociedade, apenas importam comportamentos H-criativos. (\ldots) no trabalho de Boden, existe uma distinção entre criatividade exploradora e transformacional, que será relevante aqui, e então merece alguma explicação. Boden concebe o processo de criatividade como uma identificação e/ou localização de novos objetos conceituais em um espaço conceitual.}{The distinction is between the sense of creating a concept which has never been created before at all, and a concept which has never been created before by a specific creator. This distinction will be only tangentially relevant to my argument here, but before proceeding, I note that this is not a simple binary choice, but rather multi-dimensional, context-based one: it would be possible, for example, for a creative behaviour to be only P-creative in one society, but H-creative in another; from the point of view of the second society, only the H-creativity matters. (\ldots) in Boden’s work, there is the distinction between exploratory and transformational creativity, which is directly relevant here, and so needs a little more explanation. \textbf{Boden conceives the process of creativity as the identification and/or location of new conceptual objects in a conceptual space}.}  
\end{citacao}

Para Wiggins, Boden chama de \emph{comportamento criativo explorador} a exploração de possibilidades completas ou parciais de um conceito. Já o \emph{comportamento criativo transformacional} pressupõe regras governadoras deste espaço conceitual em exploração, e busca transformar tais regras através de métodos. Para Boden,  são socialmente valorizados os conceitos-H e os processos transformacionais, enquanto o comportamento explorador e a criatividade-P são consideradas como característicos de crianças mais novas, ou pensamentos personalistas. Wiggins dá maior importância para a classificação exploradora/transformacional, e não desvaloriza o comportamento explorador em relação à transformação..  \citeonline[p.~3--4]{thornton_quantitative_2007} pontua que Boden admite que em algumas situações, uma criativade-exploradora não é menos criativa que uma criativade-transformadora. O comportamento explorador, como uma \emph{exploração guiada}, é muito útil em atividades onde se requer \traducao{(\ldots) a utilização de heurísticas e mapas para identificar conceitos valiosos dentre de um espaço conceitual existente}{the use of heuristics and maps to identify valuable concepts within an existing conceptual space} \ver{sec:showusyourscreens}. Neste sentido, os comportamentos criativos \emph{apenas transformadores} desenvolvem novos espaços conceituais que serão úteis para o comportamento explorador guiado. 

\begin{citacao}
\traducao{De fato, na primeira edição $[$\citeonline{boden_creative_1990}$]$, ela não oferece uma explicação do número de diferentes tipos de criatividade que ela identificou. Parece que sua intenção era distinguir os dois tipos notados, espaços conceituais devem ter uma característica generativa. E isso certamente é a interpretação comum. Ainda no sumário 'em uma casca de noz' de sua teoria, foi adicionado um prólogo à segunda edição (Boden, 2003), e em (Boden, 1998), ela coloca que sua explicação distingue três principais formas de criatividade, sendo exploração, transformação e \emph{combinação}. É somada à incerteza a observação que somente a definição forte $[$generalizadora$]$ da definição possue o poder de resolver o paradoxo da criatividade}{In fact, in the first edition, she offers no final count of the number of different types of creativity she has identified. It seems to be her intention to distinguish the two types noted, conceptual space must be generative in character. and this is certainly a common interpretation. Yet in the ‘nutshell’ summary of her theory, added as a prologue to the second edition (Boden, 2003), and in (Boden, 1998), she states that her account distinguishes three main forms of creativity, these being exploration, transformation and combination. Adding to the uncertainty is the observation that only the strong definition has the power to resolve the creativity paradox, arguably forcing us to recognise not two forms of creativity, or three, but one: transformation.}
\end{citacao}

Neste ponto \citeonline[p.~451]{wiggins_framework_2006} apresenta uma definição cíclica e generalizadora de criatividade, \traducao{A performance de tarefas que, quando executados por um humano, são consideradas criativas}{The performance of tasks which, if performed by a human, would be deemed creative}, e subdivide em quatro definições auxiliares: 

\begin{table}[!h]
\caption{Definições formais de criatividade por \citeonline[p.~451]{wiggins_framework_2006}}
\small
    \begin{tabular}{ | p{4cm} | p{11.25cm} |}
    \hline 
    \hline 

    \tiny{Criatividade} 
    & \tiny{``O estudo e suporte, através de meios e métodos computacionais, do comportamento exibido por sistemas naturais e artificiais, que podem ser considerados criativos se exibidos em humanos.''  \tablefootnote{Tradução de \emph{‘The study and support, through computational means and methods, of behaviour exhibited by natural and artificial systems, which would be deemed creative if exhibited by humans’’.}.}} \\
    \hline

    \tiny{Computação criativa} 
    & \tiny{``O estudo e suporte, através de meios e métodos computacionais, do comportamento exibido por sistemas naturais e artificiais, que são considerados criativos''. \tablefootnote{Tradução de \emph{The study and support, through computational means and methods, of behaviour exhibited by natural and artificial systems, which would be deemed creative if exhibited by humans.}.}} \\
    \hline

    \tiny{Sistemas criativos} 
    & \tiny{``Uma coleção de processos, naturais ou automáticos, que são capazes de alcançarem ou simularem comportamentos que em humanos seriam considerados criativos''} \\
    \hline

    \tiny{Comportamento Criativo} 
    & \tiny{``Um ou mais dos comportamentos exibidos por um sistema criativo''\tablefootnote{Tradução de \emph{One or more of the behaviours exhibited by a creative system.}}} \\
    \hline
    \hline
   
    \end{tabular}
\label{tab:criatividade}
\end{table}

\subsection{Criatividade, códigos e imagens mentais}\label{sec:imagem_mental}

Para \citeonline[p.~24--25]{McLean2011}, um comportamento criativo pode ser descrito como o processo de elaboração de uma \emph{imagem mental}. Mais especificamente, a imagem mental é um símbolo, e seu significado é governado por regras gramaticais e sociais (códigos). Por outro lado, McLean estabelece o conceito de \emph{imagem mental} como uma hierarquia de códigos simbólicos, visuais e gramaticais, como descrita pela teoria da Codificação Dual \apud[p.~25--29]{paivio_dual_1990}{McLean2011}:


\begin{citacao}
\traducao{Seu $[$Paivio$]$ argumento não é que existem dois códigos, mas sim que existe uma hierarquia de códigos, que se ramificam no topo em códigos lingüísticos discretos e códigos de percepção contínua, que Paivio nomeia como \emph {logogens} e \emph{imagens} respectivamente. (\ldots) \textbf{A explicação oferecida pela teoria da Codificação Dual é que existem sistemas de símbolos distindos, mas integrados, para linguagens e figuras}.}{His contention is not that there are two codes, but rather that there is a hierarchy of code, which branch at the top into discrete linguistic codes and contionuous perceptual codes, which Paivio names \emph{logogens} and \emph{imagens} respectively (\ldots) The explanation offered by Dual Coding Theory is that there are distinct, yet integrated symbol systems for imagery and language.}
\end{citacao}

Por exemplo, quando falamos em \emph{improvisação de códigos}, podemos imaginar uma pessoa escrevendo um texto em um computador. Dependendo do grau de intimidade com os símbolos aprendidos, pode ser que esta imagem seja mais ou menos específica, e , outras possibilidades de imagens mentais surgem. A própria atividade de descrição de uma imagem mental é uma \emph{estratégia} de conversão. No caso do improvisador-programador, que converte sua imagem mental em em código de computador, chamaremos de \emph{estratégia transversal}.

Neste processo, o improvisador-programador recorre aos padrões e estilos de escrita de uma linguagem. Entre padrões e estilos, são utilizados comentários, que explicam textualmente a imagem e seu resultado; da disposição espacial do código, ou  nomes de variáveis e funções que sugerem imagens mentais específicas. No caso de uma linguagem de programação de propósito geral, algumas regras definidas pela comunidade desenvolvedora da linguagem devem ser seguidas. No entanto, já vimos que o improvisador-programador é estimulado a recorrer às linguagens artificiais, ou linguagens de domínio específico (DSL) -- ``O programa será transcendido - Língua Artificial é o caminho'' \ver{sec:showusyourscreens}. Isto é, é parte da prática do improvisador-programador criar linguagens específicas para uma classe de imagens mentais específicas. 

Um exemplo de exploração de um processo transformacional será ilustrado a partir de uma descrição de \citeonline[p.~119]{McLean2011},  com o método da \emph{prática reflexiva} de um artista-plástico (no caso, o pintor Paul Klee). Supondo a discretização do processo transformacional, o artista cria uma imagem mental do que irá fazer, um esboço que delimita um campo de atividade (espaço conceitual), a atividade de pintura, e o resultado percebido.  Na fase de implementação do esboço, o pintor percebe que aquilo que foi considerado adequado no esboço, é inadequado para a situação prática. Neste momento, o artista reage ao resultado e re-elabora novas estratégias de conversão entre a imagem mental e o resultado. O processo continua até que o ofício seja considerado completo (\emph{obra}). Este não é o mesmo processo utilizado pelo(a) improvisador(a)-programador(a). O material, e o objetivo são diversos do(a) artista plástico(a). Seu material é o texto, mas não o texto discursivo, e sim o texto que descreve uma rotina de tarefas computacionais. Seu objetivo não é chegar a uma obra finalizada, mas sim não descontinuar o processo. McLean chama esse método de transformacional de programação por bricolagem \ver{fig:processo_criativo}. Em ambos os casos, na prática reflexiva e na bricolagem, são necessários conceitos que suportem a intenção (e vontade) do ofício. Conceitos elaborados através de restrições de idéias podem ser sistematizados, com o auxílio de esquemas visuais. Esta abordagem é muito semelhante à posição defendida por McLean, onde conceitos podem ser representados geometricamente. No caso do(a) improvisador(a)-programador(a), \emph{espaços conceituais} são reelaborados indefinidamente, começando por uma \emph{estratégia transversal}, passando para a observação, reação, e reformulação da imagem mental.

 \begin{figure}[h]
  \centering
  \includegraphics[scale=0.5]{imagens/processo_criativo.png}
  \caption{Modelo de bricolagem para o processo criativo realizado por um artista-programador. \textbf{Fonte}: \citeonline[p.~122]{McLean2011}. }
  \label{fig:processo_criativo}
\end{figure}

\begin{citacao}
\traducao{A Figura 6.2 $[$\autoref{fig:processo_criativo}$]$ caracteriza a programação por bricolagem como um laço retroalimentado envolvendo o algoritmo escrito, sua interpretação, e a percepção do programador e sua reação do resultado ou comportamento $[$do algoritmo$]$. (\ldots). No começo o programador tem um conceito meio-formado que só atinge consistência interna através do processo de ser expresso como um algoritmo. O laço interno é onde o programador elabora o objetivo de suas imaginações, e o laço externo é onde essa trajetória está fundamentada na pragmática do que elas realmente têm que fazer. Através deste processo ambos algoritmos e conceitos são desenvolvidos até que o programador sinta que um se aplica com o outro, ou de outra forma julga o processo criativo finalizado.
}{
Figure 6.2 characterises bricolage programming as a creative feedback loop encompassing the written algorithm, its interpretation, and the programmer’s perception and reaction to its output or behaviour. (\ldots). At the beginning, the programmer may have a half-formed concept, which only reaches internal consistency through the process of being expressed as an algorithm. The inner loop is where the programmer elaborates upon their imagination of what might be, and the outer where this trajectory is grounded in the pragmatics of what they have actually made. Through this process both algorithm and concept are developed, until the programmer feels they accord with one another, or otherwise judges the creative process to be finished.
}  
\end{citacao}

 Para McLean, DSLs ``provêm termos padronizados para descrever demandas particulares em um domínio de uma tarefa''. Isto é, uma linguagem customizada para atividades artísticas pode ser planejada, ou analizada, através de um \emph{Quadro de estruturação das Dimensões Cognitivas da Notação} \apud[p.~95--97]{church_cognitive_2008}{McLean2011}. Estas não são dimensões autônomas, o que diverge do foco de nosso trabalho. Uma situação comum na improvisação de códigos, é maleabilidade de escrita de um algoritmo. No caso, McLean define como a interdependência entre viscosidade e notação secundária: a possibilidade de diferentes soluções para o mesmo resultado. Por exemplo, em um \emph{patch} de PD, cuja imagem mental é uma onda quadrada. Podemos utilizar objetos nativos ou objetos extendidos \ver{fig:pd}. Uma outra interdependência de dimensões, entre as Operações Mentais Difíceis e a não-Invisibilidade de Dependências Escondidas, é notória do ponto de vista pedagógico. Podem afastar o compositor de seu objetivo musical, mais ou menos estruturado em uma teoria. DSLs como PureData, Max/MSP, CSound, SuperCollider, Tidal, praticam a invisibilidade de dependências em diferentes graus. Porém, entre os improvisadores de código, é considerado virtuosismo o praticante recorrer às Operações Mentais difíceis com o mínimo possível de Invisibilidade de Dependências. Por exemplo, utilizar linguagens de baixo nível como C \ver{sec:concerto}, ou de alto nível como Perl \cite{mclean_hacking_2004} e ainda assim, elaborar sons e imagens de maneira criativa. \ver{tab:dimensoes}:

\begin{table}[!h]
\caption{Dimensões cognitivas da Notação para linguagens de programação. \textbf{Fonte}: \apud{church_cognitive_2008}{McLean2011}.}
\small
    \begin{tabular}{ | p{7cm}| p{7cm} |}
    \hline 
    \hline 

    \tiny \textbf{Dimensão} & \textbf{Significado} \\
    \hline 
    \hline 

    \tiny \textbf{Abstração}  
    & \tiny \tabletraducao{Disponibilidade de mecanismos de abstração}{Avaliability of abstraction mechanisms} \\
    \hline

    \tiny \textbf{Dependências escondidas}

    & \tiny \tabletraducao{Invisibilidade de ligações importantes entre entidades.}{Invisibility of important links between entities.}\\
    \hline
    
    \tiny \textbf{Compromisso prematuro}  
    & \tiny \tabletraducao{Restrição na ordem de execução das coisas.}{Constraints on the order of doing things.} \\
  \hline

    \tiny \textbf{Notação secundária}  
    & \tiny \tabletraducao{Notação diversa da sintaxe formal.}{Notation other than formal syntax.} \\
    \hline

    \tiny \textbf{Viscosidade}  
    & \tiny \tabletraducao{Resistência à mudança.}{Resistance to change.} \\
    \hline

    \tiny \textbf{Proximidade de mapeamento}  
    & \tiny \tabletraducao{Proximidade de representação para o domínio-alvo.}{Closeness of representation to target domain.} \\
    \hline

    \tiny \textbf{Consistência}  
    & \tiny \tabletraducao{Semânticas similares são expressadas em formas sintáticas similares.}{Similar semantics are expressed in similar syntatic forms} \\
    \hline

    \tiny \textbf{Dispersividade}  
    & \tiny \tabletraducao{Prolixidade da linguagem.}{Verbosity of language.} \\
    \hline

    \tiny \textbf{Tendência ao erro}  
    & \tiny \tabletraducao{Probabilidade de erros.}{Likelihood of mistakes.} \\
    \hline

    \tiny \textbf{Operações mentais difíceis}  
    & \tiny \tabletraducao{Demanda de recursos cognitivos.}{Demand on cognitive resources.} \\
    \hline

    \tiny \textbf{Provisoriedade}  
    & \tiny \tabletraducao{Grau de compromisso com ações e marcos.}{Degree of commitment to actions or marks.} \\
    \hline
    
    \tiny \textbf{Função de expressividade}  
    & \tiny \tabletraducao{medida em que o efeito de um componente pode ser inferida.}{Extent to which the purpose of a component may be inferred.} \\
    \hline
    \hline
   
    \end{tabular}
\label{tab:dimensoes}
\end{table} 

\begin{figure}[!h]
  \centering
  \includegraphics[scale=0.7]{imagens/pd.png}
  \caption{Exemplo de uma caracteristica de viscosidade e notação secundária no PureData. \textbf{Fonte}: autor. }
  \label{fig:pd}
\end{figure}

 
\subsection{Comportamento Criativo: Bricolagem como estratégia transversal}\label{sec:tidal}

Vamos ilustrar um pequeno processo da \emph{estratégia transversal} com o \emph{software} Tidal.
Segundo \citeonline[p.~2]{mclean_tidal_2010}, \emph{Tidal} é uma linguagem de composição generativa, onde \traducao{padrões podem ser compostos de numerosos subpadrões em uma variedade de maneiras e para uma profundidade arbitrária, para produzir $[$partes$]$ inteiras complexas de partes simples}{patterns may be composed of numerous subpatterns in a variety of ways and to arbitrary depth, to produce complex wholes from simple parts}. Amostras sonoras representam imagens mentais de suas fontes (por exemplo ``sn'' para \emph{snare}, caixa-clara), com ritmos organizados com o auxílio de símbolos delimitadores de tempo (como espaço, `` ``, e colchetes, ``$[$'',``$]$'', \verb|{| e \verb|}|). Ritmos podem ser revertidos (\verb|rev|), diminuidos e aumentados (\verb|slow|, \verb|density|), iterados (\verb|every|) para recombinação permutação, padrões mais complexos (\verb|can|), como o algoritmode bjorklund que simula ritmos tradicionais\footnote{\cfcite{toussaint_euclidean_2005}}. Efeitos de panoramização, atraso (\emph{delay}), filtros e comunicação de rede. No Exemplo \ref{ex:tidal}. a imagem mental é a demanda da linguagem, que é produzir Música Eletrônica para Dançar.

\begin{citacao}
\traducao{Tidal é uma linguagem de padrões embebida em uma linguagem de programação Haskell, consistindo de representação de padrão, uma biblioteca de padrões geradores e combinadores, um $[$mecanismo$]$ de agendamento de eventos e uma interface para programar ao vivo. Esta é uma extensiva re-escrita de um trabalho anterior introduzido sobre o título \emph{Petrol} $[$\citeonline{mclean_petrol_2010}$]$. Extensões incluem melhoramentos de representação de padrão e um uma integração totalmente configurável do protocolo Open Sound Control $[$\citeonline{osc}$]$ \cite{mclean_tidal_2010}
}
{Tidal is a pattern language embedded in the Haskell programminglanguage, consisting of pattern representation, a library of pattern generators and combinators, an event scheduler and programmer’s live coding interface. This is an extensive re-write of earlier work introduced under the working title of Petrol [15]. Extensions include improved pattern representation and fully configurable integration with the Open Sound Control (OSC) protocol [16]
}
\end{citacao}

\begin{example}{Exemplo de Estratégia Transversal}\label{ex:tidal}

Imagem mental: um \emph{loop} sincopado, mas bastante regular, descrito em um compasso. Em uma ``partitura-mental'', estruturamos o primeiro tempo com um baixo, que volta a tocar na segunda semicolcheia do terceiro tempo. No Segundo tempo, silêncio. No quarto tempo uma caixa aberta:

{%
\parindent 0pt
\noindent
\ifx\preLilyPondExample \undefined
\else
  \expandafter\preLilyPondExample
\fi
\def\lilypondbook{}%
\includegraphics{53/lily-86c766d2-1}%
% eof

\ifx\postLilyPondExample \undefined
\else
  \expandafter\postLilyPondExample
\fi
}

O padrão acima pode ser elaborado em uma voz (\verb|d1|), que redireciona (\$) a função que toca amostras sonoras (\verb|sound|). Esta função lê uma corrente de caracteres (\textbf{string}) separados por um espaço em branco. Espaços em branco são delimitadores temporais. Cada subdivisão temporal é representada por delimitadores como $[$ e $]$. 

\begin{minted}{haskell}
-- Eletronic Dance Music, BPM = 120 
-- tempo 1 - baixo            (bass)
-- tempo 2 - silencio         (silence)
-- tempo 3 - silencio + baixo
-- tempo 4 - caixa            (sn e sn:4)
d1 \$ (sound "bass3 silence [silence bass3] sn:4")
\end{minted}

Sonoramente, é útil para começar. Mas uma Música Eletrônica para Dançar requer mais elementos. Seguiremos com mais dois passos. Podemos complementar os ritmos com uma caixa e um baixos mais secos no segundo e terceiro tempo.

\input{./tidal2}

\begin{minted}{haskell}
-- Eletronic Dance Music, BPM = 120 
-- tempo 1 - baixo            (bass)
-- tempo 2 - silencio         (silence)
-- tempo 3 - silencio + house
-- tempo 4 - caixa            (sn e sn:4)
d1 \$ (sound "bass3 sn [silence house] sn:4")
\end{minted}

É possível também fazer com que este padrão reduza seu tempo pela metade a cada quatro tempos, :

{%
\parindent 0pt
\noindent
\ifx\preLilyPondExample \undefined
\else
  \expandafter\preLilyPondExample
\fi
\def\lilypondbook{}%
\input{2b/lily-de0a8ae3-systems.tex}
\ifx\postLilyPondExample \undefined
\else
  \expandafter\postLilyPondExample
\fi
}

\begin{minted}{haskell}
-- Eletronic Dance Music, BPM = 120
-- com uma caixa seca no segundo tempo
-- e uma caixa aberta no quarto tempo
-- A cada 4 tempos, o ritmo diminui pela metade 
-- e depois volta ao normal.
d1 \$ every 4 (density 0.5) (sound "bass3 sn [silence house] sn:4")
\end{minted}
\end{example}

Para \citeonline[p.~130]{McLean2011}, esta estratégia criativa, de programar ``no momento'', a partir de um arquivo de texto em branco, com uma imagem mental do resultado sonoro (ou visual), é caracterizada pela  bricolagem. No início do exemplo acima, o programador elabora um meio-conceito do que quer fazer, cuja expressão apenas ganha existência através da codificação \ver{fig:processo_criativo}. As fases de observação, e reação levam o improvisador programador à reconceitualização, e um novo código é escrito. No entanto, ao invés de finalizar, o improvisador segue desenvolvendo.

\section{Quadro Conceitual de sistemas criativos}\label{sec:csf}

Uma maneira adequada de descrever um sistema criativo (ou parte dele) considera um \emph{Universo de Conceitos}:

\begin{citacao}
O universo, $\mathcal{U}$, é um espaço multidimensional, no qual dimensões são capazes de representar qualquer coisa, e todos os possíveis conceitos distintos correspondentes àqueles pontos em $\mathcal{U}$ (\ldots) Para tornar a proposta um espaço-tipo possível, permitirei que $\mathcal{U}$ contenha todos os conceitos abstratos, bem como os concretos, e que é possível representar os artefatos tanto completos e incompletos \cite[p.~451]{wiggins_framework_2006}.\footnote{Tradução de \emph{The universe, U, is a multidimensional space, whose dimensions are capable of representing anything, and all possible distinct concepts correspond with distinct points in U. (\ldots) To make the proposal as state-spacelike as possible, I allow that U contains all abstract concepts as well as all concrete ones, and that it is therefore possible to represent both complete and incomplete artefacts}}
\end{citacao}

Wiggins esclarece que Boden não reconhece de forma explícita $\mathcal{U}$, ``ela borra a distinção entre as regras que determinam a adesão do espaço (\ldots) e outras disposições que possam permitir a construção e/ou detecção de um conceito representado por um ponto no espaço'' (\emph{Idem, ibdem}). Espaços conceituais $\mathcal{C}$, finitos ou infinitos são definidos como restrições de um universo $\mathcal{U}$, caracterizando um conjunto não-determinístico de conhecimentos, representações, e conceitos:

\begin{citacao}
\traducao{A noção-chave na teoria de Boden é aquele do espaço conceitual. Enquanto nenhuma definição formal é provida, é comum interpretar esta frase literalmente, tomando o espaço conceitual sendo um espaço de conceitualizações, ou representações de conceitos \cite[p~.7]{thornton_quantitative_2007}.}{The key notion in Boden’s theory is that of the conceptual space. While no formal definition has been provided, it is common to interpret the phrase literally, taking the conceptual space to be a space of conceptualisations or concept representations.}
\end{citacao}

\citeonline[p.~452]{wiggins_framework_2006} considera  \traducao{(\ldots) um universo, $\mathcal{U}$, um espaço multidimensional, cujas dimensões são capazes de representar qualquer coisa, e todos possíveis conceitos distintos correspondentes com distintos pontos em $\mathcal{U}$.}{The universe, U, is a multidimensional space,whose dimensions are capable of representing anything, and all possible distinct concepts correspond with distinct points in U}. Uma incomensurabilidade é evitada através de uma restrição por meio de quatro axiomas. O primeiro axioma (Universalidade) estabelece que o universo $\mathcal{U}$ pode conter tantos conceitos bem definidos (completos), parcialmente definidos (incompletos), e  o mais incompleto dos conceitos. Os dois primeiros são representados pela letra $c$, enquanto o último é representado por \small{T}. O segundo axioma (Não-identidade dos conceitos), estabelece que dois conceitos em  $\mathcal{U}$ são mutuamente diferentes entre si ($c_1 \neq c_2$), e não são um Universo. Esta correção restringe a recursividade de conceitos, o que daria ênfase ao comportamento explorador e anularia a importância do comportamento transformacional. O terceiro axioma (Inclusão Universal 1) define que \emph{espaços conceituais} $\mathcal{C}$, que contêm instâncias de conceitos $c$, são subconjuntos não-estritos do conunto $\mathcal{U}$. O quarto axioma (Inclusão Universal 2) estabelece que espaços conceituais $\mathcal{C}$ também contem o conceito \small{T}.






 Wiggins define o Universo de Conceitos (\csf{U}{x}) como um conjunto não estrito dos \emph{Espaços Conceituais} (\csf{C}{x}) de Margaret \citeonline{boden_creative_1990}. Isto é, um Universo de Conceitos a respeito de alguma coisa, no nosso caso da improvisação de códigos \ver{eq:ul}. 

\begin{equation}
\mathcal{U}_\emph{livecoding} = [\mathcal{C}_\emph{Tecelagem}, \mathcal{C}_\emph{Audiovisual}, \mathcal{C}_\emph{Dança} \mathcal{C}_\emph{Música}, \ldots, ?]
\end{equation}\label{eq:ul}

\begin{table}[!h]
\caption{Definições formais do Universo de possibilidades de \citeonline{wiggins_framework_2006}, ou Universo de Conceitos por \citeonline{mclean_music_2006}.}
\small
    \begin{tabular}{ | p{4.25cm} | p{5.25cm} | p{5.25cm} |}
    \hline 
    \hline 

    Representação
    & \tiny{Nome}     
    & \tiny{Significado} \\
    \hline

    $c$
    & \tiny{Conceito} 
    & \tiny{Uma instância de um conceito, abstrato ou concreto \cite{wiggins_framework_2006}}. \\
    \hline

    $\mathcal{U}$
    & \tiny{Universo de Conceitos} 
    & \tiny{Superconjunto não restrito de conceitos. \cite{wiggins_framework_2006}. ``Um universo de todos conceitos possíveis'' \cite{mclean_music_2006} \tablefootnote{Tradução de \emph{A universe of all possible concepts}.}}\\
    \hline

    $\mathcal{L}$
    & \tiny{Linguagem} 
    & \tiny{Linguagem utilizada para expressar regras.} \\
    \hline

    $\mathcal{A}$
    & \tiny{Alfabeto} 
    & \tiny{Alfabeto da linguagen que contêm caracteres apropriadospara expressão das regras} \\
    \hline

    $\mathcal{R}$
    & \tiny{Regras de validação} 
    & \tiny{Validam os conceitos em um universo, se apropriados ou não para o espaço trabalhado.} \\
    \hline

    $[[.]]$
    & \tiny{Função de interpretação} 
    & \tiny{``Uma função parcial de $\mathcal{L}$ para funções que resultam em números reais entre [0, 1] (\ldots) 0.5 $[$ou maior$]$ significa uma verdade booleana e menos que 0.5 siginifica uma falsidade booleana; a necessidade disso para valores reais se tornará clara abaixo'' \cite[p.~452]{wiggins_framework_2006}\tablefootnote{Tradução de \emph{(\ldots) a partial function from $\mathcal{L}$ to functions yielding real numbers in [0, 1]. (\ldots) 0.5 to mean Boolean true and less than 0.5 to mean Boolean false; the need for the real values will become clear below}.}}\\
    \hline

     $[[\mathcal{R}]]$
    & \tiny{Regras de validação} 
    & \tiny{``Uma função que interpreta $\mathcal{R}$, resultando em uma função indicando aderência ao conceito em $\mathcal{R}$''\tablefootnote{Tradução de \emph{A function interpreting $\mathcal{R}$, resulting in a function indicating adherence of a concept to $\mathcal{R}$}}} \\
    \hline

     $\mathcal{C} = [[\mathcal{R}]](\mathcal{U}) $
    & \tiny{Espaço Conceitual} 
    & \tiny{``Todos espaços conceituais são um subconjunto não-estrito de $\mathcal{U}$''\tablefootnote{Tradução de \emph{All conceptual spaces are non-strict subset}.}. Um subconjunto contido em $\mathcal{U}$ \cite{wiggins_framework_2006}. Uma função que interpreta $\mathcal{R}$, resultando em uma função que indica aderência ao conceito em $\mathcal{R}$ \tablefootnote{Tradução de \emph{A function interpreting $\mathcal{R}$, resulting in a function indicating adherence of a concept to $\mathcal{R}$}.} } \\
    \hline

    $\mathcal{T}$
    & \tiny{Regras de detecção} 
    & \tiny{``Regras definidas dentro de $\mathcal{L}$ para definir estratégias transversais para localizar conceitos dentro de $\mathcal{U}$'' \cite{mclean_music_2006}\tablefootnote{Tradução de \emph{Rules defined within $\mathcal{L}$ to define a traversal strategy to locate concepts within $\mathcal{U}$ }}} \\
    \hline

    $\mathcal{E}$
    & \tiny{Regras de qualidade} 
    & \tiny{``(\ldots) conjunto de regras que permitem-nos avaliar qualquer conceito que nós encontramos em $\mathcal{C}$ e determinar sua qualidade, de acordo com critérios que nós considerarmos apropriados'' \cite[p.453]{wiggins_framework_2006}\tablefootnote{Tradução de \emph{(\ldots) set of rules which allows us to evaluate any concept we find in C and determine its quality, according to whatever criteria we may consider appropriate.}}``Regras definidas dentro de $\mathcal{L}$ para avaliar a qualidade ou a desejabilidade do conceito $c$'' \cite{mclean_music_2006}\tablefootnote{Tradução de \emph{Rules defined within $\mathcal{L}$ which evaluate the quality or desirability of a concept $c$.}}}\\
    \hline

    $<<<\mathcal{R}, \mathcal{T}, \mathcal{E}>>>$
    & \tiny{Função de interpretação} 
    & \tiny{Uma regra necessária para definir o espaço conceitual, ``independentemente da ordem, mas também, ficcionalmente, enumerá-los em uma ordem particular, sob o controle de $\mathcal{T}$ -- isto é cricial para a simulação de um comportamento criativo de um $\mathcal{T}$ particular \cite{wiggins_framework_2006} \tablefootnote{Tradução de \emph{We need a means not just of defining the conceptual space, irrespective of order, but also, at least notionally, of enumerating it, in a particular order, under the control of $\mathcal{T}$ -- this is crucial to the simulation of a particular creative behaviour by a particular $\mathcal{T}$.}}. ``Uma função que interpreta a estratégia transversal $\mathcal{T}$, informada por $\mathcal{R}$ e $\mathcal{E}$ . Opera sobre um subconjunto ordenado de $mathcal{U}$ (do qual tem acesso randômico) e resulta em outro subconjunto ordenado de $\mathcal{U}$.''\tablefootnote{Tradução de \emph{A function interpreting the traversal strategy $\mathcal{T}$, informed by $\mathcal{R}$ and $\mathcal{E}$ . It operates upon anordered subset of $mathcal{U}$ (of which it has random access) and results in another ordered subset of $\mathcal{U}$.}}} \\
    \hline
    \hline
   
    \end{tabular}
\label{tab:universodeconceitos}
\end{table}

\citeonline{mclean_music_2006} ainda descreve regras que validam concepções diferentes entre espaços conceituais $\mathcal{C}$ diversos em um Universo de Conceitos $\mathcal{U}$ (ver \autoref{tab:universodeconceitos}). McLean realiza uma comparação entre o \emph{Universo de possibilidades} de Wiggins com o \emph{Modelo de Improvisação} de Pressing \ver{sec:im}. No entanto, McLean argumenta que:

\begin{citacao}
Pressing discute comportamento criativo no contexto do Modelo de Improvisação, e de fato é parte do Quadro conceitual de Sistemas Criativos. (\ldots) Durante a transferência de notação do Modelo de Improvisação para a Ferramenta de Sistemas Criativos, nós consideramos improvisação musical de uma maneira clara e temos uma linguagem comum na qual comparar com outros modelos \footnote{Tradução de \emph{However Pressing does discuss creative behaviour in the context of the IM, and indeed the CSF is in part. (\ldots) In transferring the IM to the notation of the CSF we may consider music improvisation in a clearer manner and have a common language in which to compare it with other models.}}.
\end{citacao}


\subsection{O modelo de improvisação}\label{sec:im}

Segundo Pressing, o Modelo de Improvisação é ``um esboço para uma teoria geral da improvisação integrada com preceitos da Psicologia Cognitiva (\ldots) teoria do comportamento de improvisação na música'' \cite[p.~2]{pressing_improvisation_1987}. Este modelo será utilizado para especificar elementos de uma performance exemplar, como o caso investigado neste trabalho. Por exemplo, uma improvisação particionada em diferentes sequências pode ser parcialmente mapeada em categorias, como blocos sonoros, referentes conceituais e normas estilísticas, conjuntos de objetivos e processos. Este nos pareceu um modelo mais transparente para o compositor, músico e intérprete. O que não quer dizer que é possível readequar ambos para nosso interesse. Um sumário sobre o modelo de improvisação é apresentado na \autoref{tab:modelo_improvisacao}. Por seu caráter lógico, parece ser uma possibilidade interessante, e assumiremos como tal.

\begin{table}[!h]
\caption{Definições formais do Modelo de improvisação de Jeff \citeonline{pressing_improvisation_1987}, segundo \citeonline[p.~2]{mclean_music_2006}.}
\small
    \begin{tabular}{ | p{6cm} | p{9cm} |}
    \hline 
    \hline 

    \tiny{Representação}   
    & \tiny{Significado} \\
    \hline

    $E'$
    & \tiny{Um bloco de eventos sonoros}\tablefootnote{\emph{A cluster of sound events}.} \\
    \hline

    $K'$
    & \tiny{Uma seqüência de blocos de eventos E, onde um bloco de eventos não se sobrepõe com o seguinte}\tablefootnote{A sequence of E event clusters, where event cluster onsets do not overlap with those of a following one}\\
    \hline

    $I'$
    & \tiny{Uma improvisação, particionada por interrupções em um número de K sequências}\tablefootnote{An improvisation, partitioned by interrupts into a number of K sequences} \\
    \hline

    $R'$
    & \tiny{Um referente opcional, tal como uma partitura ou uma norma estilística}\tablefootnote{An optional referent, such as a score or stylistic norm} \\
    \hline

    $G'$
    & \tiny{Um conjunto de objetivos }\tablefootnote{A set of current goals.} \\
    \hline

    $M'$
    & \tiny{Uma memória de longo prazo}\tablefootnote{Long term memory.} \\
    \hline

    $O'$
    & \tiny{Um conjunto de objetos}\tablefootnote{An array of objects.} \\
    \hline

    $F'$
    & \tiny{Um conjunto de características dos objetos}\tablefootnote{An array of objects Features.} \\
    \hline

    $P'$
    & \tiny{Um conjunto de processos}\tablefootnote{An array of Process} \\
    \hline
    \hline
   
    \end{tabular}
\label{tab:modelo_improvisacao}
\end{table}

\begin{figure}[!h]
  \centering
  \includegraphics[scale=0.7]{imagens/contido.png}
  \caption{Representação da justaposição  entre dois epaços conceituais. A região em marrom representa um grupo de conceitos transitórios, bem como os limites desta transição. \textbf{Fonte}: autor. }
  \label{fig:contido}
\end{figure}

\section{Diagramação dos espaços conceituais}\label{sec:diagrama}

\newcommand{\csfeq}[2]{
\mathcal{#1}_\emph{#2}
}

\newcommand{\unionspaces}[6]{
\csfeq{#1}{#2} = \csfeq{#3}{#4} \bigcup \csfeq{#5}{#6}
}

\newcommand{\listspaces}[9]{
\csfeq{#1}{#2}~=~[\csfeq{#3}{#2},~\csfeq{#4}{#2},~\csfeq{#5}{#2},~\csfeq{#6}{#2},~\csfeq{#7}{#2},~\csfeq{#8}{#2},~\csfeq{#9}{#2}
}

Formalmente, a figura acima pode ser representada como na \autoref{eq:def} , se desconsiderarmos qualquer outros espaços conceituais.

\begin{example}{Representação formal da \autoref{fig:contido}}
\begin{equation}
\unionspaces{C}{Study in Keith}{C}{live coding}{C}{Sun Bears}
\label{eq:def}
\end{equation}
\end{example}

Este grupo também pode ser descrito como uma lista de propriedades como na \autoref{eq:def2}:

\begin{example}{Representação formal das propriedades da \autoref{fig:contido}}
\begin{equation}
\listspaces{C}{SK}{E'}{K'}{I'}{R'}{G'}{M'}{O'}{F'},~\csfeq{P'}{SK}]
\label{eq:def}
\end{equation}
\end{example}
  
Nos diagramas abaixo, $C_\emph{\ldots}$ representa qualquer espaço conceitual abstrato (que pode incluir outro previamente apresentado). Entre os elementos iniciais (raízes, vermelho) e transitórios (nós, azul), ocorrem as ramificações (ramos, linhas pretas), isto é, a exploração de conceitos dentro de outros conceitos. De um lado, a aplicação de regras de validação sobre o universo conceitual da pesquisa (tudo aquilo que foi produzido em dois anos de mestrado) gerou o espaço conceitual desta tese. Estas regras de validação foram, em sua maior parte, os processos de orientação e qualificação. Em outras palavras, \csf{C}{pesquisa}$=[[$\csf{R}{pesquisa}$]]($\csf{U}{pesquisa}$)$.

\begin{example}{Representação do universo conceitual da \emph{pesquisa}}

O Universo de Conceitos da pesquisa, \csf{U}{pesquisa}, é um recorte do universo conceitual da música, \csf{U}{música}:

\begin{tikzpicture}
  [
    grow                    = right,
    sibling distance        = 6em,
    level distance          = 10em,
    edge from parent/.style = {draw, -latex},
    every node/.style       = {font=\footnotesize},
    sloped
  ]
  \node [root] {\csf{U}{Música}}
    child { node [env] {\csf{U}{pesquisa}}
      child { node [env] {\csf{U}{livecoding}}}
    }
    child { node [env] {\csf{C}{\ldots}}};
\end{tikzpicture}

No primeiro capítulo, incluímos um subjconjunto neste Espaço Conceitual da Pesquisa \ver{app:A}). 

\begin{tikzpicture}
  [
    grow                    = right,
    sibling distance        = 6em,
    level distance          = 10em,
    edge from parent/.style = {draw, -latex},
    every node/.style       = {font=\footnotesize},
    sloped
  ]
  \node [root] {\csf{C}{pesquisa}}
    child { node [env] {\csf{C}{livecoding}}
      child { node [env] {\csf{C}{\ldots}}}
      child { node [env] {\csf{C}{ICLC}}}
    }
    child { node [env] {\csf{C}{\ldots}}}; 
\end{tikzpicture}

Podemos inclur elementos históricos, o período transitório entre 1970 e 2000 (\emph{circa}), onde emanciparam as práticas e as regras heurísticas.  

\begin{tikzpicture}
  [
    grow                    = right,
    sibling distance        = 6em,
    level distance          = 10em,
    edge from parent/.style = {draw, -latex},
    every node/.style       = {font=\footnotesize},
    sloped
  ]
  \node [root] {\csf{C}{livecoding}}
    child { node [env] {\csf{C}{Elementos Históricos}}
      child {node [env] {\csf{C}{Proto-História}}}
      child {node [env] {\csf{C}{Manifestos}}}
    }
    child { node [env] {\csf{C}{\ldots}}};
\end{tikzpicture}

Por último, \csf{C}{pesquisa} investiga o \emph{live coding} a partir de um caso específico:

\begin{tikzpicture}
  [
    grow                    = right,
    sibling distance        = 6em,
    level distance          = 10em,
    edge from parent/.style = {draw, -latex},
    every node/.style       = {font=\footnotesize},
    sloped
  ]
  \node [root] {\csf{C}{pesquisa}}
    child { node [env] {\csf{C}{livecoding}}
      child { node [env] {\csf{C}{\ldots}}}
      child { node [env] {\csf{C}{Sessão de Improvisação}}
        child { node [env] {\csf{C}{Study in Keith}}}
        child { node [env] {\csf{C}{\ldots}}}
      }
    }
    child { node [env] {\csf{C}{\ldots}}}; 
\end{tikzpicture}
\end{example}

Por outro lado \csf{C}{Study in Keith} pode ser definido pelo modelo de improvisação de Pressing (\autoref{tab:modelo_improvisacao}, \pageref{tab:modelo_improvisacao}).

\begin{example}{Representação do modelo de improvisação para \emph{Study in Keith}.}
\begin{tikzpicture}
  [
    grow                    = right,
    sibling distance        = 6em,
    level distance          = 10em,
    edge from parent/.style = {draw, -latex},
    every node/.style       = {font=\footnotesize},
    sloped
  ]
  \node [root] {\footnotesize \csf{C}{Study in Keith}}
    child { node [env] {\footnotesize \csf{E'}{Study in Keith}}}
    child { node [env] {\footnotesize \csf{K'}{Study in Keith}}}
    child { node [env] {\footnotesize \csf{I'}{Study in Keith}}}
    child { node [env] {\footnotesize \csf{R'}{Study in Keith}}}
    child { node [env] {\footnotesize \csf{G'}{Study in Keith}}}
    child { node [env] {\footnotesize \csf{O'}{Study in Keith}}}
    child { node [env] {\footnotesize \csf{F'}{Study in Keith}}}
    child { node [env] {\footnotesize \csf{P'}{Study in Keith}}}; 
\end{tikzpicture}
\end{example}

\section{Formalização}\label{sec:formaliza}

O espaço conceitual do \emph{livecoding} é definido como uma função de interpretação das regras de validação (o que pode ser ou não considerado como próprio de uma categorização musical), de gosto (questões de estilo) e de localização transversal de conceitos (conceitos internos que permitem o cruzamento com outros conceitos). As regras de validação foram estudadas neste trabalho como as regras heurísticas do \emph{live coding}. Isto é, que conjunto de métodos são utilizados para caracterizar uma performance de \emph{live coding} como tal? Elementos históricos, e ideológicos (divulgados em manifestos), são levantados para responder esta pergunta. Por outro lado, este estudo abandonou a investigação das regras de gosto, tema que pode ser melhor explorado em trabalhos posteriores, a partir de \citeonline{janotti_jr._a_2003,sa_musica_2006,sa_se_2009}. A tarefa de localização transversal de conceitos é trabalhada no último capítulo. O espaço conceitual de \emph{Study in Keith} está contido no espaço conceitual do \emph{live coding} através da união entre os conceitos deste último, com os espaços conceituais dos concertos \emph{Sun Bears}, de Keith Jarret, misturados. No entanto o espaço conceitual não será investigado em sua totalidade, e sim apenas uma sonoridade.
%\newpage
\chapter{Estudo de caso}\label{cap:estudos_de_caso}

A pesquisa desenvolvida nos capítulos anteriores sucitaram a seguinte pergunta: como investigar um caso musical de improvisação de códigos, um vídeo de \emph{A Study in Keith} de Andrew Sorensen (2012)?

Os trabalhos de \citeonline{Forth2010,McLean2011} possibilitaram investigar o vídeo dentro de um abordagem cognitivista, de forma que contextualizamos um método de análise sugerida por \citeonline[p.~117]{McLean2011}:

\begin{citacao}
\traducao{Aqui nós tomamos a perspectiva que uma propriedade conceitual é representada por um melhor \emph{exemplo simples} possível, ou \emph{protótipo}. (\ldots) Para fundamentar a discussão em música, considere uma peça de jazz, onde jazz é um conceito e uma composição particular é uma instância de um conceito. Um musicista, que explora os limites do jazz, encontrou uma peça para além das regras usuais do jazz. Através deste processo, os limites do gênero musical podem ser redefinidos em algum grau, ou se a peça está em um novo terreno particularmente fértil, um novo sub-gênero de jazz emerge. Contudo uma peça de música que não quebra limites, de alguma forma pode ser considerada não-criativa.}{Here we take the view that a conceptual property is represented by a single best possible example, or \emph{prototype}. In accordance with the theories reviewed in chapter 2, these prototypes arise through perceptual states, within the geometry of quality dimensions. To ground the discussion in music, consider a piece of jazz, where jazz is the concept and the particular composition is an instance of that concept. The musician, in exploring the boundaries of jazz, then finds a
piece beyond the usual rules of jazz. Through this process, the boundaries of a music genre
} 
\end{citacao}


\section{Metodologia de Análise}\label{sec:metodo}

Para análisar \emph{A Study in Keith} utilizamos um recorte do \emph{Quadro Conceitual de Sistemas Criativos}\footnote{Creative System Frameworks, \emph{ou CSF}, \cfcite{mclean_music_2006,Forth2010,McLean2011}.} \ver{tab:universodeconceitos}, e do modelo-baseem que este foi desenvolvido, o modelo de improvisação de Jeff \citeonline{pressing_improvisation_1987} \ver{sec:im}. É importante esclarecer que, para não confundirmos questões ontológicas,  substituímos \emph{conceito} por \emph{proposição da improvisação} ou simplismente \emph{proposição}\ver{sec:proposicao}. Renomeamos o termo \emph{Referente opcional} de Pressing por \emph{Referentes de A Study in Keith Zero, Um, Dois e Três}. A Linguagem é \emph{A Linguagem de Programação utilizada em A Study in Keith}.

%Buscamos desta forma discretizar a parte inicial do processo criativo de um artista-programador, de forma que trocamos conceito para proposição, universo de conceitos para universo de proposições, espaço conceitual para espaço de proposições, para discutir os três primeiros blocos de eventos sonoros, que formam a primeira sequência de eventos sonoros, formados a partir de um referente opcional

\newpage

\begin{table}[!h]
\caption{Definições formais do Universo de possibilidades de \citeonline{wiggins_framework_2006}, ou Universo de Conceitos por \citeonline{mclean_music_2006,Forth2010}. Neste trabalho, como quadro de proposições.}
\small
    \begin{tabular}{ | p{4.25cm} | p{5.25cm} | p{5.25cm} |}
    \hline 
    \hline 

    Representação
    & \tiny{Nome}     
    & \tiny{Significado} \\
    \hline

    $c$
    & \tiny{Conceito} 
    & \tiny{Uma instância de um conceito, abstrato ou concreto \cite{wiggins_framework_2006}}. \\
    \hline

    $\mathcal{U}$
    & \tiny{Universo de Conceitos} 
    & \tiny{Superconjunto não restrito de conceitos. \cite{wiggins_framework_2006}. ``Um universo de todos conceitos possíveis'' \cite{mclean_music_2006} \tablefootnote{Tradução de \emph{A universe of all possible concepts}.}}\\
    \hline

    $\mathcal{L}$
    & \tiny{Linguagem} 
    & \tiny{Linguagem utilizada para expressar regras.} \\
    \hline

    $\mathcal{A}$
    & \tiny{Alfabeto} 
    & \tiny{Alfabeto da linguagen que contêm caracteres apropriadospara expressão das regras} \\
    \hline

    $\mathcal{R}$
    & \tiny{Regras de validação} 
    & \tiny{Validam os conceitos em um universo, se apropriados ou não para o espaço trabalhado.} \\
    \hline

    $[[.]]$
    & \tiny{Função de interpretação} 
    & \tiny{``Uma função parcial de $\mathcal{L}$ para funções que resultam em números reais entre [0, 1] (\ldots) 0.5 $[$ou maior$]$ significa uma verdade booleana e menos que 0.5 siginifica uma falsidade booleana; a necessidade disso para valores reais se tornará clara abaixo'' \cite[p.~452]{wiggins_framework_2006}\tablefootnote{Tradução de \emph{(\ldots) a partial function from $\mathcal{L}$ to functions yielding real numbers in [0, 1]. (\ldots) 0.5 to mean Boolean true and less than 0.5 to mean Boolean false; the need for the real values will become clear below}.}}\\
    \hline

     $[[\mathcal{R}]]$
    & \tiny{Regras de validação} 
    & \tiny{``Uma função que interpreta $\mathcal{R}$, resultando em uma função indicando aderência ao conceito em $\mathcal{R}$''\tablefootnote{Tradução de \emph{A function interpreting $\mathcal{R}$, resulting in a function indicating adherence of a concept to $\mathcal{R}$}}} \\
    \hline

     $\mathcal{C} = [[\mathcal{R}]](\mathcal{U}) $
    & \tiny{Espaço Conceitual} 
    & \tiny{``Todos espaços conceituais são um subconjunto não-estrito de $\mathcal{U}$''\tablefootnote{Tradução de \emph{All conceptual spaces are non-strict subset}.}. Um subconjunto contido em $\mathcal{U}$ \cite{wiggins_framework_2006}. Uma função que interpreta $\mathcal{R}$, resultando em uma função que indica aderência ao conceito em $\mathcal{R}$ \tablefootnote{Tradução de \emph{A function interpreting $\mathcal{R}$, resulting in a function indicating adherence of a concept to $\mathcal{R}$}.} } \\
    \hline

    $\mathcal{T}$
    & \tiny{Regras de detecção} 
    & \tiny{``Regras definidas dentro de $\mathcal{L}$ para definir estratégias transversais para localizar conceitos dentro de $\mathcal{U}$'' \cite{mclean_music_2006}\tablefootnote{Tradução de \emph{Rules defined within $\mathcal{L}$ to define a traversal strategy to locate concepts within $\mathcal{U}$ }}} \\
    \hline

    $\mathcal{E}$
    & \tiny{Regras de qualidade} 
    & \tiny{``(\ldots) conjunto de regras que permitem-nos avaliar qualquer conceito que nós encontramos em $\mathcal{C}$ e determinar sua qualidade, de acordo com critérios que nós considerarmos apropriados'' \cite[p.453]{wiggins_framework_2006}\tablefootnote{Tradução de \emph{(\ldots) set of rules which allows us to evaluate any concept we find in C and determine its quality, according to whatever criteria we may consider appropriate.}}``Regras definidas dentro de $\mathcal{L}$ para avaliar a qualidade ou a desejabilidade do conceito $c$'' \cite{mclean_music_2006}\tablefootnote{Tradução de \emph{Rules defined within $\mathcal{L}$ which evaluate the quality or desirability of a concept $c$.}}}\\
    \hline

    $<<<\mathcal{R}, \mathcal{T}, \mathcal{E}>>>$
    & \tiny{Função de interpretação} 
    & \tiny{Uma regra necessária para definir o espaço conceitual, ``independentemente da ordem, mas também, ficcionalmente, enumerá-los em uma ordem particular, sob o controle de $\mathcal{T}$ -- isto é cricial para a simulação de um comportamento criativo de um $\mathcal{T}$ particular \cite{wiggins_framework_2006} \tablefootnote{Tradução de \emph{We need a means not just of defining the conceptual space, irrespective of order, but also, at least notionally, of enumerating it, in a particular order, under the control of $\mathcal{T}$ -- this is crucial to the simulation of a particular creative behaviour by a particular $\mathcal{T}$.}}. ``Uma função que interpreta a estratégia transversal $\mathcal{T}$, informada por $\mathcal{R}$ e $\mathcal{E}$ . Opera sobre um subconjunto ordenado de $mathcal{U}$ (do qual tem acesso randômico) e resulta em outro subconjunto ordenado de $\mathcal{U}$.''\tablefootnote{Tradução de \emph{A function interpreting the traversal strategy $\mathcal{T}$, informed by $\mathcal{R}$ and $\mathcal{E}$ . It operates upon anordered subset of $mathcal{U}$ (of which it has random access) and results in another ordered subset of $\mathcal{U}$.}}} \\
    \hline
    \hline
   
    \end{tabular}
\label{tab:universodeconceitos}
\end{table}


\subsection{O modelo de improvisação}\label{sec:im}

Segundo Pressing, o Modelo de Improvisação é ``um esboço para uma teoria geral da improvisação integrada com preceitos da Psicologia Cognitiva'' \cite[p.~2]{pressing_improvisation_1987}. Este modelo será utilizado para especificar elementos de uma performance exemplar, como o caso investigado neste trabalho. Por exemplo, uma improvisação particionada em diferentes sequências pode ser parcialmente mapeada em categorias, como blocos sonoros, referentes conceituais e normas estilísticas, conjuntos de objetivos e processos. Este nos pareceu um modelo mais transparente para o compositor, músico e intérprete. O que não quer dizer que é possível readequar ambos para nosso interesse. Um sumário sobre o modelo de improvisação é apresentado na \autoref{tab:modelo_improvisacao}. Por seu caráter lógico, parece ser uma possibilidade interessante, e assumiremos como tal.

\begin{table}[!h]
\caption{Definições formais do Modelo de improvisação de Jeff \citeonline{pressing_improvisation_1987}, segundo \citeonline[p.~2]{mclean_music_2006}.}
\small
    \begin{tabular}{ | p{6cm} | p{9cm} |}
    \hline 
    \hline 

    \tiny{Representação}   
    & \tiny{Significado} \\
    \hline

    $E'$
    & \tiny{Um bloco de eventos sonoros}\tablefootnote{\emph{A cluster of sound events}.} \\
    \hline

    $K'$
    & \tiny{Uma seqüência de blocos de eventos E, onde um bloco de eventos não se sobrepõe com o seguinte}\tablefootnote{A sequence of E event clusters, where event cluster onsets do not overlap with those of a following one}\\
    \hline

    $I'$
    & \tiny{Uma improvisação, particionada por interrupções em um número de K sequências}\tablefootnote{An improvisation, partitioned by interrupts into a number of K sequences} \\
    \hline

    $R'$
    & \tiny{Um referente opcional, tal como uma partitura ou uma norma estilística}\tablefootnote{An optional referent, such as a score or stylistic norm} \\
    \hline

    $G'$
    & \tiny{Um conjunto de objetivos }\tablefootnote{A set of current goals.} \\
    \hline

    $M'$
    & \tiny{Uma memória de longo prazo}\tablefootnote{Long term memory.} \\
    \hline

    $O'$
    & \tiny{Um conjunto de objetos}\tablefootnote{An array of objects.} \\
    \hline

    $F'$
    & \tiny{Um conjunto de características dos objetos}\tablefootnote{An array of objects Features.} \\
    \hline

    $P'$
    & \tiny{Um conjunto de processos}\tablefootnote{An array of Process} \\
    \hline
    \hline
   
    \end{tabular}
\label{tab:modelo_improvisacao}
\end{table}
  
%Nos diagramas abaixo, $C_\emph{\ldots}$ representamos qualquer proposição (que pode incluir outras). Entre os elementos iniciais (raízes, vermelho) e transitórios (nós, azul), ocorrem as ramificações (ramos, linhas pretas), isto é, a exploração da proposição dentro de outros conceitos.

%\begin{tikzpicture}
%  [
%    grow                    = right,
%    sibling distance        = 6em,
%    level distance          = 10em,
%    edge from parent/.style = {draw, -latex},
%    every node/.style       = {font=\footnotesize},
%    sloped
%  ]
%  \node [root] {\csf{U}{Música}}
%    child { node [env] {\csf{U}{pesquisa}}
%      child { node [env] {\csf{U}{livecoding}}}
%    }
%    child { node [env] {\csf{C}{\ldots}}};
%\end{tikzpicture}

%No primeiro capítulo, incluímos um subjconjunto neste Espaço Conceitual da Pesquisa \ver{app:A}). 

%\begin{tikzpicture}
%  [
%    grow                    = right,
%    sibling distance        = 6em,
%    level distance          = 10em,
%    edge from parent/.style = {draw, -latex},
%    every node/.style       = {font=\footnotesize},
%    sloped
%  ]
%  \node [root] {\csf{C}{pesquisa}}
%    child { node [env] {\csf{C}{livecoding}}
%      child { node [env] {\csf{C}{\ldots}}}
%      child { node [env] {\csf{C}{ICLC}}}
%    }
%    child { node [env] {\csf{C}{\ldots}}}; 
%\end{tikzpicture}

%Podemos inclur elementos históricos, o período transitório entre 1970 e 2000 (\emph{circa}), quando emanciparam as práticas e as regras heurísticas.  

%\begin{tikzpicture}
%  [
%    grow                    = right,
%    sibling distance        = 6em,
%    level distance          = 10em,
%    edge from parent/.style = {draw, -latex},
%    every node/.style       = {font=\footnotesize},
%    sloped
%  ]
%  \node [root] {\csf{C}{livecoding}}
%    child { node [env] {\csf{C}{Elementos Históricos}}
%      child {node [env] {\csf{C}{Proto-História}}}
%      child {node [env] {\csf{C}{Manifestos}}}
%    }
%    child { node [env] {\csf{C}{\ldots}}};
%\end{tikzpicture}

%Por último, \csf{C}{pesquisa} investiga o \emph{live coding} a partir de um caso específico:


%\begin{tikzpicture}
%  [
%    grow                    = right,
%    sibling distance        = 6em,
%    level distance          = 10em,
%    edge from parent/.style = {draw, -latex},
%    every node/.style       = {font=\footnotesize},
%    sloped
%  ]
%  \node [root] {\csf{C}{pesquisa}}
%    child { node [env] {\csf{C}{livecoding}}
%      child { node [env] {\csf{C}{\ldots}}}
%      child { node [env] {\csf{C}{Sessão de Improvisação}}
%        child { node [env] {\csf{C}{Study in Keith}}}
%        child { node [env] {\csf{C}{\ldots}}}
%      }
%    }
%    child { node [env] {\csf{C}{\ldots}}}; 
%\end{tikzpicture}

%\begin{example}{Representação do modelo de improvisação para \emph{Study in Keith}.}

%Da especificação \csf{C}{Study in Keith} derivamos blocos de eventos, sequências de blocos de eventos, interrupções, referentes opcionais, objetivos, um objeto que carrega uma memória de algo, objetos (musicais, sonoros, visuais, etc.) e processos

%\begin{tikzpicture}
%  [
%    grow                    = right,
%    sibling distance        = 4em,
%    level distance          = 12em,
%    edge from parent/.style = {draw, -latex},
%    every node/.style       = {font=\footnotesize},
%    sloped
%  ]
%  \node [root] {\footnotesize \csf{C}{Study in Keith}}
%    child { node [env] {\footnotesize \csf{E'}{Study in Keith}}}
%    child { node [env] {\footnotesize \csf{K'}{Study in Keith}}}
%    child { node [env] {\footnotesize \csf{I'}{Study in Keith}}}
%    child { node [env] {\footnotesize \csf{R'}{Study in Keith}}}
%    child { node [env] {\footnotesize \csf{G'}{Study in Keith}}}
%    child { node [env] {\footnotesize \csf{O'}{Study in Keith}}}
%    child { node [env] {\footnotesize \csf{F'}{Study in Keith}}}
%    child { node [env] {\footnotesize \csf{P'}{Study in Keith}}}; 

%\end{tikzpicture}
%\end{example}

%No entanto, exploramos apenas os conceitos envolvidos em um ciclo de bricolagem de um código, o que limita nossos resultados:

%\begin{example}{Especificação do modelo de improvisação para \emph{Study in Keith}.}

%Da representação derivamos uma sequências de blocos de eventos, uma interrupção, três referentes opcionais, um objetivo, e uma classe objetos (musicais, sonoros, visuais, etc.).

%\begin{tikzpicture}
% [
%    grow                    = right,
%    sibling distance        = 4em,
%    level distance          = 12em,
%    edge from parent/.style = {draw, -latex},
%    every node/.style       = {font=\footnotesize},
%    sloped
%  ]
%  \node [root] {\footnotesize \csf{C}{Study in Keith}}
%    child { node [env] {\footnotesize \pressingthree{K'}{Study in Keith}{0}}
%      child { node [env] {\footnotesize \pressingthree{E'}{Study in Keith}{0}}}
%      child { node [env] {\footnotesize \pressingthree{E'}{Study in Keith}{1}}}
%      child { node [env] {\footnotesize \pressingthree{E'}{Study in Keith}{2}}}
%    }
%    child { node [env] {\footnotesize \pressingthree{I'}{Study in Keith}{0}}}
%    child { node [env] {\footnotesize \pressingthree{G'}{Study in Keith}{0}}}
%    child { node [env] {\footnotesize \pressingthree{O'}{Study in Keith}{0}}}
%    child { node [env] {\footnotesize \csf{R'}{Study in Keith}}
%      child { node [env] {\footnotesize \pressingthree{R'}{Study in Keith}{0}}}
%      child { node [env] {\footnotesize \pressingthree{R'}{Study in Keith}{1}}}
%      child { node [env] {\footnotesize \pressingthree{R'}{Study in Keith}{2}}}
%    };

%\end{tikzpicture}
%\end{example}


%\section{Formalização}\label{sec:formaliza}

%O espaço conceitual do \emph{livecoding} é definido como uma função de interpretação das regras de validação (o que pode ser ou não considerado como próprio de uma categorização musical), de gosto (questões de estilo) e de localização transversal de conceitos (conceitos internos que permitem o cruzamento com outros conceitos). As regras de validação foram estudadas neste trabalho como as regras heurísticas do \emph{live coding}. Isto é, que conjunto de métodos são utilizados para caracterizar uma performance de \emph{live coding} como tal? Elementos históricos, e ideológicos (divulgados em manifestos), são levantados para responder esta pergunta. Por outro lado, este estudo abandonou a investigação das regras de gosto, tema que pode ser melhor explorado em trabalhos posteriores, a partir de Janotti \citeonline{janotti_jr._a_2003,sa_musica_2006,sa_se_2009}. A tarefa de localização transversal de conceitos é trabalhada no último capítulo. O espaço conceitual de \emph{Study in Keith} está contido no espaço conceitual do \emph{live coding} através da intersecção entre os conceitos deste último, com os espaços conceituais dos concertos \emph{Sun Bears}, de Keith Jarret, misturados. No entanto o espaço conceitual não será investigado em sua totalidade, e sim apenas uma sonoridade.

\section{\emph{A Study in Keith}: Proposição}\label{sec:proposicao}

Sorensen faz duas as descrições de uma mesma proposição, ou o \emph{Espaço conceitual de A Study in Keith}. Os Concertos \emph{Sun Bear} de Keith Jarret \ver{sec:sunbear} são citados como inspiradores da improvisação de códigos. Desta forma, existe um \emph{Referencial Zero de A Study in Keith}, ou \pressingthree{R}{ask}{0}.

\begin{citacao}
\traducao{\emph{A Study In Keith} é um trabalho para piano solo (NI's Akoustik Piano), inspirado nos concertos \emph{Sun Bear} de Keith Jarrett. Note que não existe som para os dois primeiros 2 minutos da performance, enquanto estruturas iniciais são construídas. \textbf{Não é bem Keith, mas inspirado por Keith}. \cite{sorensen_keith_2009}}{"A Study In Keith" is a work for solo piano (NI's Akoustik Piano) by Andrew Sorensen inspired by Keith Jarrett's Sun Bear concerts. Note that there is no sound for the first 2 minutes of the performance while initial structures are built. Not quite Keith, but inspired by Keith.}
\end{citacao}

\citeonline{sorensen_youtube_2014} indica outros referenciais, que chamamos de \emph{Referencial Um, Dois e Três de A Study in Keith}, \pressingthree{R}{ask}{1},\pressingthree{R}{ask}{2},\pressingthree{R}{ask}{3}, ou o piano virtual \emph{Akoustik Piano NI}, o ambiente de programação \emph{Impromptu} e a linguagem de programação \emph{Scheme}. Este último é a \emph{Linguagem de Programação de A Study in Keith}, ou \csf{L}{ask}:

\begin{citacao}
\traducao{\emph{A Study in Keith} é uma performance de programação ao vivo por Andrew Sorensen, inspirado nos concertos \emph{Sun Bear} de Keith Jarret. Toda a música que você ouve é gerada a partir do código do programa que é escrito e manipulado em \emph{tempo-real} durante a performance. O trabalho foi executado usando o ambiente de desenvolvimento $[$em linguagem$]$ Scheme $[$chamado$]$ Impromptu (\url{http://impŕomptu.moso.com.au}). Não é Keith, mas inspirado por Keith \cite{sorensen_youtube_2014}.
}
{
``A Study In Keith'' is a live programming performance by Andrew Sorensen inspired by Keith Jarrett's Sun Bear concerts. All of the music you hear is generated from the program code that is written and mani$[$p$]$ulated in real-time during the performance. The work was performed using the Impromptu Scheme software development environment (\url{http://impromptu.moso.com.au}). Not Keith, but inspired by Keith.
}
\end{citacao}


%\section{Objetivo}\label{sec:objetivo}

%Observação e análise de um comportamento criativo musical, de um improvisador-programador, que escreve uma programação-partitura, e realiza a manutenção de um pensamento musical tradicional. Mais especificamente, analisamos o contexto musical de uma simples sequência de blocos sonoros \pressingthree{K}{ask}{0}, gerada por uma função de interpretação $<<<$\csf{R}{ask}, \csf{T}{ask},\csf{E}{ask}$>>>$, cujo referente opcional direto, \pressingthree{R}{ask}{0}, são os Concertos \emph{Sun Bear} de Keith Jarret.

%\section{Justificativa}

%Parafraseamos \citeonline[p.~121]{McLean2011} ao afirmarmos que \traducao{Nosso estudo de caso é de alguma forma simplista e não é intenção ilustrar uma grande arte ou um grande código. Contudo delineia um processo criativo de classes, como efetuado pelo presente autor.}{Our case study is somewhat simplistic, and is not intended to illustrate either great art or great code. However it does trace a creative process of sorts, as carried out by the present author.}.


\section{Referentes Opcionais}\label{sec:sunbear}

Aqui foi possível elaborar uma solução possível ao problema enunciado como ``\emph{Study in Keith} não é \emph{Sun Bears}''. Em seguida tratamos do referencial um, \pressingthree{R}{ask}{1}, ou o timbre de piano utilizado \ver{sec:NI}, e de um ambiente de programação musical chamado \emph{Impromptu} \ver{sec:impromptu} como referencial dois \pressingthree{R}{ask}{2}. 

\subsection{Concertos Sun Bear}\label{sec:sunbearanal}

Os concertos \emph{Sun Bear} são originalmente dez LPs  de improvisações de Keith Jarret no Japão, produzidos pela \emph{ECM Records}\footnote{http://www.ecmrecords.com/} entre 1976 e 1978. Foram realizados e gravados como sessões de improvisação contínua, variando entre 31 a 43 minutos cada. Para cada dia, duas sessões de improvisação, em cidades diferentes. Kyoto, 5 de novembro\footnote{Disponível em \url{https://www.youtube.com/watch?v=T2TfIQNxhjc}.}; Osaka, 8 de novembro\disponivelem{https://www.youtube.com/watch?v=FC4iZ1wMoU8}; Nagoya, 12 de novembro\footnote{\url{https://www.youtube.com/watch?v=3a7ezm3D1jA}.}. Tokyo, 14 de novembro\disponivelem{https://www.youtube.com/watch?v=ZH8VIjjhPQ4}; Sapporo, 18 de Novembro\disponivelem{https://www.youtube.com/watch?v=BqYBT_HoG4M}.

%Um documento crítico impresso é mencionado na \emph{internet} como um antigo documento contendo notas discográficas \cite{rollingstone1985}. Seu acesso foi restrito durante a pesquisa, e não foi possível incluir alguma citação. Da mesma forma, não encontramos documentos análiticos específicos sobre a peça, mas uma tese de doutorado de Dariuz \citeonline{terefenko2004} auxiliou na compreensão harmônica do tema principal.

 Existem algumas notas discográficas compiladas por uma comunidade de fãs e críticos musicais estadounidenses. Duas notas sugerem uma descrição da forma musical aplicada por Keith Jarret: \traducao{``O tema de \emph{Kyoto Parte 1} é repetido por Keith Jarret no fim de \emph{Kyoto Parte 2}. Então podemos considerar o todo deste concerto como uma grande Suíte.''}{The theme of Kyoto Part 1 is repeated By Kj at the end of Kyoto Part 2. So we can consider the whole of this concert as one big Suite}\cite[p.~129]{jarret_discography_2014}. 

\traduzcitacao{Revisto por Richard S. Ginnel\footnote{Disponível em \url{http://www.mcana.org/formembersatlarge.html}.}: $[$--$]$ Este pacote gigantesco -- um conjunto de dez LPs agora comprimidos em uma caixa robusta de seis $[$embalagens de$]$ CDs -- foi ridicularizado uma vez como uma última viagem de ego, provavelmente por muitos que não tomaram um tempo para ouvir tudo. (\ldots) Ainda assim, o milagre é como esta caixa é consistentemente muito boa. \textbf{Na abertura de Kyoto, a meditação direcionada para o \emph{gospel}} está em plena atuação, ao nível de suas melhores performances solo em Bremen e Koln,\textbf{e os concertos Osaka e Nagoya possuem citações de primeira linha, geralmente do tipo \emph{folk}}, mesmo profundas, idéias líricas \cite[p.~130]{jarret_discography_2014}
}{
Review by Richard S. Ginell: $[$--$]$ This gargantuan package -- a ten-LP set now compressed into a chunky six-CD box -- once was derided as the ultimate ego trip, probably by many who didn't take the time to hear it all. You have to go back to Art Tatum's solo records for Norman Granz in the '50s to find another large single outpouring of solo jazz piano like this, all of it improvised on the wing before five Japanese audiences in Kyoto, Osaka, Nagoya, Tokyo, and Sapporo. Yet the miracle is how consistently good much of this giant box is.  In the opening Kyoto concert, Jarrett's gospel-driven muse is in full play, up to the level of his peak solo performances in Bremen and Koln, and the Osaka and Nagoya concerts have pockets of first-rate, often folk-like, even profound, lyrical ideas.
}

O \emph{gospel} e o \emph{folk} são categorizados como gêneros musicais nesta suíte que não possui pausas entre as partes (o improviso é contínuo, mas seccionado por transições). Seu motivo gerador é, segundo Uwe \citeonline{karcher2009}, \traducao{``Na verdade, a abertura não é realmente improvisada - ela é baseada em uma canção chamada \emph{Song Of The Heart}''}{Actually, the opening was not really improvised - it is based on a tune named \emph{Song Of The Heart}.}\disponivelem{https://www.youtube.com/watch?v=JgyRoQPDwM8}.

%\begin{figure}[!h]
%  \centering
  %\includegraphics[scale=0.5]{imagens/Jarret_intro.png}
%  {%
\parindent 0pt
\noindent
\ifx\preLilyPondExample \undefined
\else
  \expandafter\preLilyPondExample
\fi
\def\lilypondbook{}%
\input{b1/lily-fc418522-systems.tex}
\ifx\postLilyPondExample \undefined
\else
  \expandafter\postLilyPondExample
\fi
}
%  \caption{Transcrição do motivo gerador do disco Kyoto, parte 1. \textbf{Fonte}: autor.}
%  \label{fig:Jarret_intro}
%\end{figure}

%É importante também mencionar que a fonte original de uma transcrição, utilizada para escuta e anotação, foi retirada do ar por violação de direitos autorais, o que desviou o foco de uma transcrição mais detalhada e correta. No final desta pesquisa, encontramos um outro vídeo, uma áudio-transcrição executada por Uwe \citeonline{karcher2009} que aponta erros na primeira transcrição acima, e também indica que este tema não é improvisado:

%\begin{citacao}
% Eu tenho interesse especial em transcrever os primeiros 10, 11 minutos (que são simplesmente fenomenais). Por fim, eu usei a Reprise que Keith jogou no final da Parte II.}{My transcription of the famous opening of the concert played by Keith in Kyoto on November 5th, 1976. Released in a 6-CD-Box-Set called "Sun Bear Concerts" (ECM). A must have for all (Jazz)piano enthusiasts! Actually,  I have been interested particularly in transcribing the first 10, 11 minutes (which are simply phenomenal). To come to an end, I used the Reprise which Keith played at the end of Part II. To memorize these 34 pages was pretty ambitious, but it worked :) Hope you enjoy!}
%\end{citacao}

\begin{figure}[!h]
  \centering
%  %\includegraphics[scale=0.5]{imagens/Jarret_intro.png}
  {%
\parindent 0pt
\noindent
\ifx\preLilyPondExample \undefined
\else
  \expandafter\preLilyPondExample
\fi
\def\lilypondbook{}%
\includegraphics{5c/lily-03a7ab98-1}%
\ifx\betweenLilyPondSystem \undefined
  \linebreak
\else
  \expandafter\betweenLilyPondSystem{1}%
\fi
\includegraphics{5c/lily-03a7ab98-2}%
\ifx\betweenLilyPondSystem \undefined
  \linebreak
\else
  \expandafter\betweenLilyPondSystem{2}%
\fi
\includegraphics{5c/lily-03a7ab98-3}%
% eof

\ifx\postLilyPondExample \undefined
\else
  \expandafter\postLilyPondExample
\fi
}
  \caption{Transcrição do motivo gerador do disco Kyoto, parte 1. textbf{Fonte}: Uwe \citeonline{karcher2009}.}
  \label{fig:Jarret_intro2}
\end{figure}

A maneira como Jarret improvisou este concerto possibilita questionar como o código é improvisado por \citeonline{sorensen_keith_2009}: o resultado é de fato uma improvisação de códigos, ou existe um agenciamento onde o improvisador prepara um código?

\begin{example}{Redução da primeira sonoridade dos concertos \emph{Sun Bear}}\label{ex:schenker}

%O padrão cromático \^1\^b2-\^1-\^7b forma uma figura alternada com um ostinato no baixo.Em seguida, o padrão \^{b5}-\^4-\^3 cria uma cadência progressiva de contra-polo da dominante, subdominante da dominante e dominante. No entanto, é importante destacar que a análise foi feita para uma sonoridade muito específica, que não considera outras sequências da improvisação.

\emph{Song of the heart} apresenta três blocos de eventos iniciais: uma figura que alterna, a partir de um baixo, os intervalos nona menor, terça menor, segunda maior, e oitava, formando um ostinato. Nos compassos 3 a 5 aparecem uma nota que forma uma relação de trítono com o baixo. O acorde formado, um Sol bemol Maior (transcrito assim para facilitar a leitura), é expandido nos compassos 6 a 10, gerando uma figura cromática cuja transcrição apresenta a seguinte cadência: Sol Bemol Maior (com décima primeira aumentada adicionada, em terceira inversão), Fá Maior (que alterna o \^5 com o \^13\footnote{\cfcite[p.~]{terefenko2004}}) e Dó Maior com sétima menor (posição fundamental). Limitamo-nos a considerar a progressão do ponto de vista da exploração de um contra-polo seguido de uma subdominante da dominante e dominante.  

Tomando um \emph{blues} tradicional de 12 compassos, seguimos uma fórmula prática $C:~I^7~$ $\Rightarrow~IV^7~\Rightarrow~I^7~\Rightarrow~I^7$ $\Rightarrow~IV^7~\Rightarrow~IV^7~\Rightarrow~I^7~\Rightarrow~I^7~\Rightarrow~V^7~\Rightarrow~IV^7~\Rightarrow~I^7~$. É possível explorar ``seção plagal'' do padrão, ao separarmos os acordes 4 a 7 ($C:~I^7~\Rightarrow~IV^7~\Rightarrow~IV^7~\Rightarrow~I^7$), e transformarmos, por substituição de trítono, a primeira subdominante da sequência, ou $C: subV/IV$ (substituição por trítono da subdominante em Dó Maior). Isto é, a substituição-padrão, $C:~bV/V^7$ (acorde de quinto grau bemol da dominante com sétima), ou $C:~bII^7$ (acorde de segundo grau bemol com sétima), passa a ser operacionalizada como um contra-polo à tônica \footnote{\cfcite{soares_luteria_2015}.}.  O que pode ser notado como $C:~(bV^7/V)/IV~\Rightarrow~IV^7$, ou $C:~bII^7/IV~\Rightarrow~IV^7$ (sequência do segundo grau bemol da subdominante para a subdominante do tom), pode ser simplificado como $C:~bV^7~\Rightarrow~IV^7$, ou uma sequência do quinto grau bemol para o quarto grau.  No entanto, a última transcrição de Jarret (considerando sua futura correção), suprime e transforma as sétimas no quinto grau bemol, o que caracteriza uma sonoridade de tensão progressiva (e ambígua) com um baixo pedal. Uma tríade do quinto grau bemol, uma tétrade do quarto grau com sexta e quinta(com sua sétima no baixo), e primeiro grau com sétima:  $C:~bV~\Rightarrow~IV^6_{4}~\Rightarrow~I^7$. O baixo pedal pode sugerir um 11 grau aumentado no primeiro acorde da sequência ($C:~_{11\#add}bV$). Uma notação sintética desta sequência sugere o padrão  $C:~_{11\#add}bV~\Rightarrow~_{5}IV^{4-5}~\Rightarrow~I^{7-8}$.
\centering{{%
\parindent 0pt
\noindent
\ifx\preLilyPondExample \undefined
\else
  \expandafter\preLilyPondExample
\fi
\def\lilypondbook{}%
\includegraphics{60/lily-c8fe7d04-1}%
% eof

\ifx\postLilyPondExample \undefined
\else
  \expandafter\postLilyPondExample
\fi
}}
\end{example}


\subsection{NI-Akoustik Piano}\label{sec:NI}

\emph{A Study in Keith} pode ser observado como uma simulação de \emph{Disklavier Sessions} \citeonline{sorensen_disklavier_2013}. Este último caso não é citado por  \citeonline{sorensen_keith_2009} e \citeonline{sorensen_youtube_2014} mas a improvisação de códigos é semelhante àquela de \emph{A Study in Keith}.

\emph{A Study in Keith} utiliza o piano virtual \emph{Akoustic Piano NI} (\emph{Native Instruments})\footnote{Disponível em \url{http://www.native-instruments.com/en/company/}.}. Como um \emph{plugin} VST, os sons do instrumento virtual são gravações nota-a-nota de um piano acústico; a tomada de som é realizada em diversos pontos do tampo harmônico, de forma que detalha as ressonâncias do instrumento. São então amostrados digitalmente\disponivelem{http://www.native-instruments.com/en/products/komplete/keys/definitive-piano-collection/}.

\emph{Disklavier Sessions} utiliza um piano \emph{Disklavier} da Yamaha, um modelo que internalizou um computador registrador de eventos, que podem ser codificados e então são convertidos como movimentos do martelo do Piano:

%\begin{figure}[h]
% \centering
%  \includegraphics[scale=0.5]{imagens/disklavier.jpg}
%  \caption{Piano Disklavier de armário, com a parte interna exposta para exibir a placa-mãe. \textbf{Fonte}: wikimedia.org}
%  \label{fig:disklavier}
%\end{figure}

\begin{citacao}
\traducao{Em \emph{Disklavier Sessions} os programas escritos em tempo-real por Ben e Andrew geram um fluxo de dados de notas que é enviado para ser executado em um piano disklavier mecanizado. Assim como as alturas das notas, toda a performance do piano deve ser codificada na informação gerada pelo programa e enviada para o piano disklavier.}{In the Disklavier Sessions the programs beign written in real-time by Ben and Andrew are generating a live stream of note data which is sent to a mechanized disklavier piano to be performed. As well the individual note pitches all of the piano performance must be encoded into the information being generated by the program and sent to disklavier piano}
\end{citacao}

%Sorensen e Swift controlam os eventos MIDI e osconvertidos em ações do martelo do piano \ver{sec:eventos}. No momento, podemos dizer que são programados, em tempo-real, em um \emph{software}/Ambiente de programação nomeado como \emph{Impromptu}, cuja base de desenvolvimento é o \emph{Extempore}. 

\subsection{Ambiente e Linguagem: Impromptu}\label{sec:impromptu}

\begin{citacao}
\traducao{
Impromptu é uma linguagem e um ambiente de programação OSX\footnote{Sistema Operacional Mac OSX.} para compositores, artistas sonoros, VJ's e artistas gráficos  com um interesse em programação ao vivo ou $[$programação$]$ interativa. Impromptu  é um ambiente de linguagem Scheme, um membro da família das linguages Lisp. Impromptu é usado por artistas-programadores em performances de \emph{livecoding} em torno do mundo.
}{
Impromptu is an OSX programming language and environment for composers, sound artists, VJ's and graphic artists with an interest in live or interactive programming. Impromptu is a Scheme language environment, a member of the Lisp family of languages. Impromptu is used by artist-programmers in livecoding performances around the globe.\emph{Disponível em \url{http://impromptu.moso.com.au/}}}
\end{citacao}

Segundo \citeonline[p.~823]{sorensen_impromptu_2010}, o Impromptu é um ambiente de programação ciberfísico, análogo à \emph{partitura} tradicional. O ambiente suporta a compilação de pequenos trechos de códigos executáveis em linguagem \emph{Scheme} (\csf{L}{ask}). Nos termos de \citeonline{magnusson_algorithms_2011}, os algoritmos codificados nesta linguagem \emph{são o instrumento}:

\begin{citacao}
\traducao{Considere a analogia da partitura musical tradicional. A partitura provê uma especificação estática da intenção -- um programa de domínio estático. Musicistas, representam o domínio do processo, executam ações requeridas para realizar ou reificar a partitura. Finalmente, as ações no domínio do processo resultam em ondas sonoroas que são percebidas por uma audiência humana como música. Este estágio final é o nosso domínio real de trabalho. Agora considere um domínio de programação dinâmica no qual o compositor concebe e descreve uma partitura em \emph{tempo-real}. Nós geralmente chamamos este tipo de composição de improvisação. \textbf{Na improvisação o(a) musicista é envolvido em um circuito-fechado retroalimentado que envolve premeditação, movendo para ação casual e finalmente para reação, refinamento e reflexão.}}{
Consider the analogy of a traditional musical score. The score provides a static specification of intention – a static program domain. Musicians, representing the process domain, perform the actions required to realise or reify the score. Finally, the actions in the process domain result in sound waves which are perceived by a human audience as music. This final stage is our real-world task domain. Now consider a dynamic program domain in which a composer conceives of and describes a musical score in real-time. We commonly call this type of composition improvisation. In it, the improvising musician is involved in a feedback loop involving forethought, moving to causal action and finally toreaction, refinement and reflection.}
\end{citacao}

Existe uma restrição quanto ao nicho de usuários do \emph{software}, com suporte para usuários de computadores Apple. 

Para lidar com outros sistemas (como por exemplo, sistemas operacionais Linux) e outras arquiteturas de processamento (32bit e 64 bit), o projeto foi liberado como código-aberto, com o nome \emph{Extempore}.

\subsection{Extempore}

O \emph{Extempore} possui um sistema de \emph{programação ciberfísica} reflexiva \ver{sec:grossi}:

\begin{citacao}
\traducao{\emph{Extempore} é projetado para suportar um estilo de programação apelidado de $[$''$]$programação ciberfísica''. Programação ciberfísica suporta a noção de um programador humano operando como um agente ativo em uma rede distribuída em tempo-real de sistemas ambientalmente conscientes.} {Extempore is designed to support a style of programming dubbed 'cyberphysical' programming. Cyberphysical programming supports the notion of a human programmer operating as an active agent in a real-time distributed network of environmentally aware systems. \disponivelem{https://github.com/digego/extempore}. }
\end{citacao}

Entre suas características incluímos \disponivelem{http://benswift.me/2012/08/07/extempore-philosophy/}:

- Codificação  de alto-nível em linguagem Scheme;

- Processamento de Sinais Digitais (DSP) \footnote{Sobre DSP, \cfcite{smith_dsp_2012}.} em  tempo-real;

- Sequenciamento de áudio, baseado em notas, como o disparo de sons parametrizados em altura, intensidade e duração \disponivelem{http://benswift.me/2012/10/15/playing-an-instrument-part-i/};

As primeira e terceiras características serão explorada neste capítulo como base do processo de codificação na improvisação \emph{Study in Keith}.

\subsection{Scheme}\label{sec:scheme}

Esta subseção demonstra a \emph{Linguagem de A Study in Keith} \ver{tab:universodeconceitos}, ou o dialeto \emph{Scheme} da linguagem de programação \emph{Lisp}. Abaixo descrevemos uma característica pertinente à análise de \emph{Study in Keith}, a saber, sua expressão textual através de uma gramática generativa:

%\begin{citacao}
%\traducao{Um sistema chamado LISP (para Processador de LISta) foi desenvolvido para um computador IBM 704 pelo grupo de Inteligência Artificial no M.I.T. O sistema foi projetado para facilitar experimentos com um sistema proposto chamado ``Recebedor de conselhos'' $[$Advice Taker$]$, onde uma máquina pode ser instruída para lidar com sentenças declarativas, bem como imperativas, e poderia exibir um ``senso comum'' no desempenho de suas instruções. A proposta original para o \emph{Advice Taker} foi feita em novembro de 1958. O principal requerimento foi um sistema de programação para manipular expressões que representam sentenças formais, declarativas e imperativas, de modo que o sistema \emph{Advice Taker} pode fazer deduções. No curso do desenvolvimento, o sistema LISP passou por diversas simplificações e, eventualmente, se baseou em um esquema para representar funções recursivas parciais de certas classes de expressões simbólicas. Esta representação é independente do computador IBM 704, ou qualquer outro computador eletrônico, e agora parece útil expor o sistema, começando com a classe de expressões chamadas expressões-S e as chamadas funções-S \cite[seção 1]{mccarthy_recursive_1960}.}{A programming system called LISP (for LISt Processor) has been developed for the IBM 704 computer by the Artificial Intelligence group at M.I.T. The system was designed to facilitate experiments with a proposed system called the Advice Taker, whereby a machine could be instructed to handle declarative as well as imperative sentences and could exhibit ``common sense'' in carrying out its instructions. The original proposal [1] for the Advice Taker was made in November 1958. The main requirement was a programming system for manipulating expressions representing formalized declarative and imperative sentences so that the Advice Taker system could make deductions.In the course of its development the LISP system went through several stages of simplification and eventually came to be based on a scheme for representing the partial recursive functions of a certain class of symbolic expressions. This representation is independent of the IBM 704 computer, or of any other electronic computer, and it now seems expedient to expound the system by starting with the class of expressions called S-expressions and the functions called S-functions.}
%\end{citacao}

%A definição de funções-S foge do escopo de nossa pesquisa, mas ela pode ser compreendida de maneira intuitiva, a partir das expressões-S. \citeonline[seção~3]{mccarthy_recursive_1960} define expressões- S como ``átomos'' e listas de átomos, onde um átomo também pode ser uma lista de átomos. Existe uma classe de expressões simbólicas definida por parênteses. Dentro desta expressão simbólica são inseridos átomos (ver exemplo \ref{ex:s-expression}, \ref{ex:s-expression2} e \ref{ex:s-expression3}). 

%%%%%%%%%%%%%%%%%%%%%%%%%%%%%%%%%%%%%%%%%%%%%%%%%
\begin{example}{Expressão simbólica vazia}\label{ex:s-expression}
\begin{minted}[fontsize=\scriptsize]{cl}
( )
\end{minted}
\end{example}
%%%%%%%%%%%%%%%%%%%%%%%%%%%%%%%%%%%%%%%%%%%%%%%%%

Segundo \citeonline[seção~3]{mccarthy_recursive_1960}, listas são sentenças abstratas de átomos, que são os elementos constituintes de uma lista:

%%%%%%%%%%%%%%%%%%%%%%%%%%%%%%%%%%%%%%%%%%%%%%%%%
\begin{example}{Expressão simbólica com átomos}\label{ex:s-expression2}
\begin{minted}[fontsize=\scriptsize]{cl}
;;A     -> 
( A )

;;AB    -> 
( A B )

;;ABA   ->
( A B A )

;;ABAC  ->
( A B A C )
( A B A C A )
\end{minted}
\end{example}
%%%%%%%%%%%%%%%%%%%%%%%%%%%%%%%%%%%%%%%%%%%%%%%%%

Átomos também podem ser outras listas:

%%%%%%%%%%%%%%%%%%%%%%%%%%%%%%%%%%%%%%%%%%%%%%%%%
\begin{example}{Expressão simbólica com átomos}\label{ex:s-expression3}
\begin{minted}[fontsize=\scriptsize]{cl}
;;A = A
;;B = AB
;;C = BAB

;; ABA  -> 
( A ( A B ) A )
;; ABAC ->
( A ( A B ) A ( B A B ) )
( A ( A B ) A (( A B ) A ( A B )))
\end{minted}
\end{example}
%%%%%%%%%%%%%%%%%%%%%%%%%%%%%%%%%%%%%%%%%%%%%%%%%

Uma característica da \emph{Linguagem de A Study in Keith} (\csf{L}{ask}) é sua notação para expressar uma proposição como  ``some as unidades de uma lista'':

%%%%%%%%%%%%%%%%%%%%%%%%%%%%%%%%%%%%%%%%%%%%%%%%%
\begin{example}{Notação prefixada}
\begin{minted}[fontsize=\scriptsize]{cl}
;;A = 1
;;B = +
;;C = 2
;; ABA  -> BAA
( + 1 1 )  ;; = 2

;;ABAC -> BAAC -> 
( + 1 1 2 );; = 4

;;ABACA -> BAACA -> 
( + 1 1 2 1);; = 5
\end{minted}
\end{example}
%%%%%%%%%%%%%%%%%%%%%%%%%%%%%%%%%%%%%%%%%%%%%%%%%

Existe um vocabulário pré-definido para criação de sentenças, de forma que o significado do código possa ser legível para os propósitos desejados

%%%%%%%%%%%%%%%%%%%%%%%%%%%%%%%%%%%%%%%%%%%%%%%%%
\begin{example}{Notação Scheme}
\begin{minted}[fontsize=\scriptsize]{cl}
;; define A = 1
(define A 1)

;; define B = 2
(define B 2)

;; divisao na forma (lambda argumentos operacao) 
(define divide            ;; define nome da funcao
        (lambda (a b)     ;; argumentos da funcao (calculo lambda)
                (/ a b))  ;; o que faz a funcao
)

;; execucao descritiva
(divide A B)
\end{minted}
\end{example}
%%%%%%%%%%%%%%%%%%%%%%%%%%%%%%%%%%%%%%%%%%%%%%%%%

Para os propósitos deste trabalho, será útil apresentar um código musical fictício, como um protótipo de \emph{jazz} tonal. O exemplo abaixo alterna citações e códigos para contextualizarmos um pseudo-código descrito por \citeonline[p.~823-824]{sorensen_impromptu_2010}:

%%%%%%%%%%%%%%%%%%%%%%%%%%%%%%%%%%%%%%%%%%%%%%%%%
\begin{example}{Elaboração/Codificação Musical em Scheme}
Este exemplo é semelhante ao primeiro algoritmo de \emph{A Study in Keith} \ver{sec:eventos}:

\begin{citacao}
\traducao{
\small{Dois performers se apresentam no palco. Um violinista, em pé e parado, com seu arco preparado. Outro senta-se atrás do brilho da tela do \emph{laptop}. Uma projeção da tela do \emph{laptop} é projetada acima do palco, e mostra uma página em branco, com um simples cursor piscando. O musicista-programador começa a digitar \ldots}
}{
Two performers are present on stage. One, a violinist, stands paused, bow at the ready. Another sits behind the glow of a laptop screen. A projection of the laptop screen is cast above the stage showing a blank page with a single blinking cursor. The laptop musician begins to type ...
}
\end{citacao}

\begin{minted}{cl}
( play-sound ( now ) synth c3 soft minute)
\end{minted}

\begin{citacao}
\traducao{
\small{\ldots a expressao é avaliada, e lampeja no retroprojetor, para exibir a ação do executante. Um som etéreo sintetizado entra imediatamente no espaço e o violinista começa a improvisar em simpatia com a novidade da textura. O músico-programador, ouve o material temático fornecido pelo violinista e começa a delinear um processo generativo Markoviano para acompanhar o violino:}
}
{
(\ldots) the expression is evaluated and blinks on the overhead projection to display the performer’s action. An ethereal synthetic sound immediately enters the space and the violinist begins to improvise in sympathy with the newly evolving synthetic texture. The laptop performer, listens to the thematic material provided by the violinist and begins to outline a generative Markov process to accompany the violin ...
}
\end{citacao}


\begin{minted}[fontsize=\scriptsize]{cl}
( define chords
  ( lambda ( beat chord duration )
    ( for-each ( lambda ( pitch )
                   ( play synthj pitch soft duration ))
               chord )
    ( schedule (* metro * ( + beat duration )) chords
               (+ beat duration )
               ( random ( assoc chord (( Cmin7 Dmin7 )
                                       ( Dmin7 Cmin7 ))))
               duration )))

( chords (* metro * get-beat 4) Cmin7 4)
\end{minted}
%%%%%%%%%%%%%%%%%%%%%%%%%%%%%%%%%%%%%%%%%%%%%%%%%

\begin{citacao}
\traducao{\small{\ldots A função \emph{chords} é chamada no primeiro tempo e um nova barra de tempo, e uma simples progressão recursiva de acordes começa a suportar a performance melódica do violino. A função \emph{chords} cria um laço temporal, gerando uma sequência interminável de acordes de quatro tempos. Depois de poucos momentos de reflexão, o musicista-programador começa a modificar a função \emph{chords} para suportar uma progressão de acordes mais variada, com uma razão aleatória $[$em função$]$ da recursão temporal\ldots}}
{\ldots the “chords” function is called on the first beat of a new common time bar and a simple recursive chord progression begins supporting the melodic performance of the violin. The chord function loops through time, creating an endless generative sequence of four beat chords. After a few moments of reflection the laptop performer begins to modify the “chords” function to support a more varied chord progression with a randomised rate of temporal recursion\ldots}
\end{citacao}

%%%%%%%%%%%%%%%%%%%%%%%%%%%%%%%%%%%%%%%%%%%%%%%%%
\begin{minted}[fontsize=\scriptsize]{cl}
( define chords
  ( lambda ( beat chord duration )
    ( for-each ( lambda ( pitch )
                   ( play dls (+ 60 pitch) soft duration ))
               chord )
    ( schedule (* metro * ( + beat duration )) chords
               (+ beat duration )
               ( random ( assoc chord (( Cmin7 Dmin7 Bbmaj )
                                       ( Bbmaj Cmin7 )
                                       ( Dmin7 Cmin7 )))
               ( random (3 6))))))
               
( chords (* metro * get-beat 4) Cmin7 4)
\end{minted}
\end{example}
%%%%%%%%%%%%%%%%%%%%%%%%%%%%%%%%%%%%%%%%%%%%%%%%%

Este código é a \emph{estratégia transversal de A Study in Keith}, ou \csf{T}{ask}:

%%%%%%%%%%%%%%%%%%%%%%%%%%%%%%%%%%%%%%%%%%%%%%%%%
\begin{example}{Nome da estratégia transversal}
\verb|chords| é o nome da estratégia.

\begin{minted}[fontsize=\scriptsize]{cl}
;; Definicao de acordes
( define chords
...
)
\end{minted}
\end{example}
%%%%%%%%%%%%%%%%%%%%%%%%%%%%%%%%%%%%%%%%%%%%%%%%%

A função \verb|chords| é executada como um impulso musical, com um único acorde com os seguintes parâmetros: momento de execução, grau e qualidade do acorde, e duração do acorde:

\begin{example}{Estímulo inicial para a estratégia}
\begin{minted}[fontsize=\scriptsize]{cl}
; Execucao da funcao
( chords (* metro * get-beat 4) Cmin7 4)
\end{minted}
\end{example}


Adiante são definidas propriedades com termos do vocabulário da música tonal, ou, coloquialmente, batida (no sentido da posição de uma unidade de tempo em um pulso, \emph{tactus}), acorde (tríades, tétrades, formadas por relações de intervalos de terças maiores e menores), e duração (o quanto, em relação à unidade de tempo, este acorde irá durar):

%%%%%%%%%%%%%%%%%%%%%%%%%%%%%%%%%%%%%%%%%%%%%%%%%
\begin{example}{O que operacionaliza a estratégia}
\begin{minted}[fontsize=\scriptsize]{cl}
( ...
    ( lambda ( beat chord duration )
  ...
)    
\end{minted}
\end{example}
%%%%%%%%%%%%%%%%%%%%%%%%%%%%%%%%%%%%%%%%%%%%%%%%%%%

Existem duas estratégias internas na estratégia principal, cuja execução é realizada atavés de outras palavras-chaves. A palavra-chave \verb|for-each| realiza um laço iterativo para cada altura do acorde:

%%%%%%%%%%%%%%%%%%%%%%%%%%%%%%%%%%%%%%%%%%%%%%%%%%%%%%%%
\begin{example}{Laço iterativo para cada altura do acorde}
\begin{minted}[fontsize=\scriptsize]{cl}
;; Primeira estrategia interna 
;; Para cada acorde operacionalize cada altura
( for-each ( lambda ( pitch )
               ( play dls (+ 60 pitch) soft duration ))
           chord )
\end{minted}
\end{example}
%%%%%%%%%%%%%%%%%%%%%%%%%%%%%%%%%%%%%%%%%%%%%%%%%%%%%%%%%%

Para cada acorde \verb|chord|, é tocada uma nota (\verb|pitch|), com um centro em Dó 3 (MIDI 60), em piano (\verb|soft|) e uma duração padrão (\verb|duration|):

%%%%%%%%%%%%%%%%%%%%%%%%%%%%%%%%%
\begin{example}{Execução da nota}
\begin{minted}[fontsize=\scriptsize]{cl}
( play dls (+ 60 pitch) soft duration ))
\end{minted}
\end{example}
%%%%%%%%%%%%%%%%%%%%%%%%%%%%%%%%%%

A palavra-chave \verb|schedule| executa, recursivamente, um fluxo de acordes associados (\verb|random (assoc chord|), em resposta ao estímulo (\verb|( chords (* metro * get-beat 4) Cmin7 4)|). 

%%%%%%%%%%%%%%%%%%%%%%%%%%%%%%%%%%%%%%
\begin{example}{Fluxo de novos acordes}
\begin{minted}[fontsize=\scriptsize]{cl}
( schedule (* metro * ( + beat duration )) chords
               (+ beat duration )
               ( random ( assoc chord (( Cmin7 Dmin7 Bbmaj )
                                       ( Bbmaj Cmin7 )
                                       ( Dmin7 Cmin7 )))
               ( random (3 6))))
\end{minted}
\end{example}
%%%%%%%%%%%%%%%%%%%%%%%%%%%%%%%%%%%%%%%%

O momento de execução deste acorde depende da execução do acorde anterior

%%%%%%%%%%%%%%%%%%%%%%%%%%%%%%%%%%%%%%%%%%%%%%%%%%%%%%%
\begin{example}{Quando novos acordes serão computados}
\begin{minted}[fontsize=\scriptsize]{cl}
( schedule (* metro * ( + beat duration )) chords
                  ...
                 )
\end{minted}
\end{example}
%%%%%%%%%%%%%%%%%%%%%%%%%%%%%%%%%%%%%%%%%%%%%%%%%%%%%%%

Sendo que o acorde será executado logo em seguida que anterior terminar, com uma cadência harmônica escolhida dentre uma lista de cadências, com uma duração randômica entre três e seis unidades de tempo:

%%%%%%%%%%%%%%%%%%%%%%%%%%%%%%%%%%%%%%%%%%
\begin{example}{Propriedades de novos acordes}
\begin{minted}[fontsize=\scriptsize]{cl}
( schedule ... chords
               (+ beat duration )
               ( random ( ... ))
               ( random (3 6)))
\end{minted}
\end{example}
%%%%%%%%%%%%%%%%%%%%%%%%%%%%%%%%%%%%%%%%%%

A escolha de acordes é feita de maneira randômica, segundo uma lista de cadências predeterminadas. Neste ponto, podemos indicar de maneira mais explícita uma regra de qualidade \ver{tab:universodeconceitos}:

%%%%%%%%%%%%%%%%%%%%%%%%%%%%%%%%%%%%%%%%%%%%
\begin{example}{Propriedades de novos acordes}
\begin{minted}[fontsize=\scriptsize]{cl}
( random ( assoc chord (( Cmin7 Dmin7 Bbmaj )
                        ( Bbmaj Cmin7 )
                        ( Dmin7 Cmin7 )))
\end{minted}
\end{example}
%%%%%%%%%%%%%%%%%%%%%%%%%%%%%%%%%%%%%%%%%%%

Como definido pela função \verb|chords|, o acorde será tocado em um momento que depende do cronograma, cuja duração pode variar de 3 a 6 unidades de tempo. No caso, é prototipado um fluxo recursivo de acordes.

\begin{figure}
  \centering
  \includegraphics[scale=0.3]{imagens/markov.png}
  \caption{Distribuição, aproximada, de probabilidades de acontecimento com um conjunto de possíveis cadências tonais organizados como uma cadeia de Markov. \textbf{Fonte}: \citeonline{swift_playingII_2012}.}
   \label{fig:markov}
\end{figure}

No caso do bloco de código de \citeonline[p.~823-824]{sorensen_impromptu_2010}, são utilizadas os seguintes movimentos harmônicos: $I^{7+}~\Rightarrow ii^{7}~\Rightarrow~IV^{7}/IV$, e $IV^{7}/IV~\Rightarrow~I^{7}$ e $ii^{7}~\Rightarrow~I^{7}$.

%A partir de uma exploração destes referentes opcionais -- $[$\csf{R}{ask}{0},~\csf{R}{ask}{1},~\csf{R}{ask}{2}~$]$ -- foi possível classificar, além de uma linguagem \pressingthree{L}{ask}{0}, o algoritmo gerador da uma sonoridade tonal em \emph{Study in Keith}, ou \csf{T}{ask} \ver{sec:scheme}. Uma análise do algoritmo permite verificar algumas regras de qualidade \csf{E}{ask}. A interpretação $<<<$~\csf{R}{ask},~\csf{T}{ask},~\csf{E}{ask}~$>>>$ produziu uma sequência de blocos de eventos \pressingthree{K}{ask}{0} \ver{sec:eventos}. Outros blocos também são produzidos, porém nossa análise busca investigar o espaço conceitual que possibilitou os primeiros resultados em um ciclo de bricolagem.


\section{Blocos de Eventos}\label{sec:eventos}

Na seção anterior definimos qual é a proposição de \emph{A Study in Keith} e seus referenciais.

Seguiremos com a fase de codificação da estratégia transversal \ver{tab:universodeconceitos} de Sorensen, como regra de detecção \csf{T}{ask}, que possui uma regra de qualidade \csf{E}{ask} \ver{sec:define_instr}. Este espaço conceitual gera uma sequência de blocos de eventos \pressingthree{K}{ask}{0} \ver{tab:modelo_improvisacao}, como um contraponto de primeira espécie \ver{sec:define_instr}, sem relação alguma com o plano harmônico de \emph{Sun Bears} \ver{sec:proposicao}.

Uma nota sobre esta improvisação é feita pelo próprio Sorensen: nos primeiros dois minutos do vídeo (aproximadamente 1$'$53$''$), existe um silêncio característico do momento em que os primeiros códigos são escritos. Este comportamento, do tempo de codificação, ao tempo de ação musical, é similar em outros vídeos de Sorensen: \sorensen{An evening of livecoding at 53 Rusden Street}{https://vimeo.com/2433303}, \sorensen{Just for Fun}{https://vimeo.com/2433971}, \sorensen{A Study in Part}{https://vimeo.com/2434054}, \sorensen{Stained}{https://vimeo.com/2502546}, \sorensen{Transmissions in Sound}{Transmissions in Sound}, \sorensen{Antiphony}{https://vimeo.com/2503188},  \sorensen{Strange Places}{https://vimeo.com/2503257}, \sorensen{Orchestral}{https://vimeo.com/2579694}, \sorensen{UMDT}{https://vimeo.com/2579880}, \sorensen{Day of Triffords}{https://vimeo.com/2735394}, \sorensen{Face to Face}{https://vimeo.com/5690854}, \sorensen{BM\&E}{https://vimeo.com/7339135}, \sorensen{A Christimas Carol}{https://vimeo.com/8364077} \sorensen{Dancing Phalanges}{https://vimeo.com/8732631}, \sorensen{Livecoding Audio DSP}{https://vimeo.com/15585520}, \sorensen{Jazz Ensenble Study}{https://vimeo.com/15679078}, \sorensen{Variations on a Christmas Theme}{https://vimeo.com/18008372}.

\subsection{Definição do instrumento e do tempo}\label{sec:define_instr}

Seu início é um pequeno comentário que contem o nome do executante e seu email para contato (primeiros sete segundos), bem como a escrita de um código que inicializa o NI-Akoustik (até 0$'$43$''$): 

%%%%%%%%%%%%%%%%%%%%%%%%%%%%%%%%%%%%%%%%%
\begin{example}{Definição de instrumento}
  \begin{minted}[fontsize=\footnotesize]{cl}
    ;;;;;;;;;;;;;;;;;;;;;;;;;;;;;;;;;;;;;;;;;;;;;;;;;
    ;; Andrew Sorensen andrew@moso.com.au
    (define piano (au:make-node "aumu" "NaDd" "-NI-"))
    (au:connect-node piano 0 *au:output-node* 0)
    (au:update-graph)

    (au:load-preset piano "/tmp/convert_grand.aupreset")
  \end{minted}
  \label{fig:SIK_piano}
\end{example}
%%%%%%%%%%%%%%%%%%%%%%%%%%%%%%%%%%%%%%%%%


Em \tempo{0}{52} Sorensen define um tempo base. Em seguida, Sorensen apaga o código para então iniciar definições de notas (\tempo{0}{54}).

%%%%%%%%%%%%%%%%%%%%%%%%%%%%%%%%%%%%%%%%%
\begin{example}{Definição de tempo}\label{ex:def_tempo}
  \centering
  \begin{minted}[fontsize=\footnotesize]{cl}
    (define *metro* (make-metro 110))
  \end{minted}
  
\end{example}
%%%%%%%%%%%%%%%%%%%%%%%%%%%%%%%%%%%%%%%%%

\subsubsection{Definição de uma sequência de blocos}

Até \tempo{1}{07}, uma rotina auxiliar é definida como um laço iterativo. Porém não encontramos sua especificação no código-fonte do \emph{Extempore}.

%%%%%%%%%%%%%%%%%%%%%%%%%%%%%%%%%%%%%%%%%
\begin{example}{Definição de uma função auxiliar}
  \begin{minted}[fontsize=\footnotesize]{cl}
    (pc:cb-for-each-p chords piano
                      (pc:make-chord 50 70 2 (pc:diatonic 0 '- degree))
                      dur)
  \end{minted}
\end{example}
%%%%%%%%%%%%%%%%%%%%%%%%%%%%%%%%%%%%%%%%%

Internamente, existe uma rotina que será o cerne de execução de uma nota, acompanhada de uma lista de 4 parâmetros (50, 70, 2):

%%%%%%%%%%%%%%%%%%%%%%%%%%%%%%%%%%%%%%%%%
\begin{example}{Definição de uma nota}\label{fig:SIK_acorde}
  \begin{minted}[fontsize=\footnotesize]{cl}
(pc:make-chord 50 70 2 (pc:diatonic 0 '- degree))
  \end{minted}
\end{example}
%%%%%%%%%%%%%%%%%%%%%%%%%%%%%%%%%%%%%%%%%

A abreviação \verb|pc| significa \emph{pitch class}, e a função \verb|pc:make-chord| significa que a função cria um acorde segundo parâmetros definidos no código-fonte do \emph{Extempore}\disponivelem{https://github.com/digego/extempore/blob/master/libs/core/pc_ivl.xtm}:

\begin{citacao}
\traducao{Cria uma lista do ``número'' $[$com$]$ alturas entre limites ``menor'' e ``maior'' do \emph{pc} dado. Uma divisão dos limites, pelo número de elementos requisitados, decompõem a seleção em extensões diferentes, do qual cada altura é selecionada. \emph{make-chord} tenta selecionar alturas para todos os graus do \emph{pc}. É possível, para  os elementos de um acorde resultarem em -1, se não existir nenhum \emph{pc} para a extensão dada. $[$É$]$ não-determinístico (i.e., resultados variam com o tempo). Argumento 1: limite menor (inclusivo). Argumento 2: Limite maior (exclusivo). Argumento 3: Número de alturas no acorde. Argumento 4: \emph{pitch class} \cite{swift_playingII_2012}.}{Creates a list of "number" pitches between "lower" and "upper" bounds from the given "pc". A division of the bounds by the number of elements requested breaks down the selection into equal ranges from which each pitch is selected.  \emph{make-chord} attempts to select pitches of all degrees of the pc.  It is possible for elements of the returned chord to be -1 if no possible pc is available for the given range. Non-deterministic (i.e. result can vary each time). arg1: lower bound (inclusive).  arg2: upper bound (exclusive). arg3: number of pitches in chord.  arg4: pitch class}
\end{citacao}

Este bloco de códigos cria uma díade, no âmbito de um Ré 2 (MIDI 50) e Si bemol 3 (MIDI 70), dentro de um campo harmônico diatônico (\verb|pc:diatonic|). Por sua vez, este último cria ``um acorde seguindo regras básicas de harmonia diatônca: baseado em uma raiz (0 para C, etc.), maior/menor (\verb|'-| ou \verb|'^|) e graus (i-vii)''\footnote{Tradução nossa de: \emph{(\ldots) a chord following basic diatonic harmony rules: based on root (0 for C etc.) maj/min ('- or '\^) and degree (i-vii)}.}. O resultado não é previsível, e depende de regras específicas de qualidade, que apresentaremos adiante, para classificar os \emph{pitch class} dentro de um grau de um campo harmônico.

\subsubsection {Definição de blocos}\label{sec:define_chords}

Em \tempo{1}{08}, a função \emph{chords} surge no fluxo audiovisual, sem nenhum processo de escrita. Este comportamento caracteriza a utilização de, ou uma cópia/cola de texto, ou de uma execução de um macro do editor de texto usado. Macros são pequenos programas no editor que auxiliam o processo de produção do código. De qualquer forma é importante salientar que o código é preparado por \citeonline{sorensen_youtube_2014}.

%%%%%%%%%%%%%%%%%%%%%%%%%%%%%%%%%%%%%%%%%%%%%%%%
\begin{example}{Algoritmo que define os acordes}

O algoritmo apresenta apenas uma propriedade, tempo (\verb|time|).

\begin{minted}[fontsize=\footnotesize]{cl}
    (define chords
       (lambda (time)
          (for-each (lambda (p)
                       (play-note (*metro* time) piano p 80 (*metro* 'dur dur)))                                 
                    (pc:make-chord 50 70 2 (pc:diatonic 0 (quote -) degree)))
          (callback (*metro* (+ time (* .5 dur))) chords (+ time dur))))

    (chords (*metro* 'get-beat 4.0) 'i 3.0)
\end{minted}
\end{example}
%%%%%%%%%%%%%%%%%%%%%%%%%%%%%%%%%%%%%%%%%%%%%%%%

Primeiro é definida a estratégia transversal, \csf{T}{ask}, com um parâmetro, \verb|time|

%%%%%%%%%%%%%%%%%%%%%%%%%%%%%%%%%%%%%%%%%%%%%%%%
\begin{example}{Estratégia transveral}
\begin{minted}[fontsize=\footnotesize]{cl}
(define chords
   (lambda (time) ... ))
\end{minted}
\end{example}

Seguido de um ``impulso'', ou um estímulo sonoro:

%%%%%%%%%%%%%%%%%%%%%%%%%%%%%%%%%%%%%%%%%%%%%%%%
\begin{example}{Impulso, ou acorde inical}
\begin{minted}[fontsize=\footnotesize]{cl}
     (chords (*metro* 'get-beat 4.0) 'i 3.0)
\end{minted}
\end{example}
%%%%%%%%%%%%%%%%%%%%%%%%%%%%%%%%%%%%%%%%%%%%%%%%

Dentro de \csf{T}{ask}, é executado um laço iterativo, \verb|for-each|, para cada nota de uma díade.

%%%%%%%%%%%%%%%%%%%%%%%%%%%%%%%%%%%%%%%%%%%%%%%%
\begin{example}{Laço iterativo}\label{sec:iterativo}
\begin{minted}[fontsize=\footnotesize]{cl}
(for-each (lambda (p)
             (play-note (*metro* time) piano p 80 (*metro* 'dur dur)))                                 
          (pc:make-chord 50 70 2 (pc:diatonic 0 (quote -) degree)))
\end{minted}
\end{example}
%%%%%%%%%%%%%%%%%%%%%%%%%%%%%%%%%%%%%%%%%%%%%%%%

Cada nota é executada com uma altura \verb|p|, para cada díade definida em \verb|pc:make-chord|, em um momento definido por \verb|time| em relação ao pulso rítmico, com uma duração ainda a ser definida. 

%%%%%%%%%%%%%%%%%%%%%%%%%%%%%%%%%%%%%%%%%%%%%%%%
\begin{example}{Execução da nota}
\begin{minted}[fontsize=\footnotesize]{cl}
(play-note (*metro* time) piano p 80 (*metro* 'dur dur))
\end{minted}
\end{example}
%%%%%%%%%%%%%%%%%%%%%%%%%%%%%%%%%%%%%%%%%%%%%%%%

\verb|play-note| é definido com os seguintes argumentos, momento de execução ($time~\Rightarrow~(*metro* time)$), o instrumento tocado, ($instr~\Rightarrow~piano$), a altura ($pitch~\Rightarrow~p$), o volume ($vol~\Rightarrow~80$) e a duração do acorde ($dur~\Rightarrow~(*metro* 'dur dur)$)\disponivelem{https://github.com/digego/extempore/blob/5aec8b35c6b3058d1c66de7abf752dc667ab61e4/libs/core/instruments-scm.xtm}. 

\subsection{Primeira sonoridade tonal}\label{sec:1aSonoridade}

Este código inicial é então modificado, e finalizado em \tempo{1}{57}, momento em que é possível ouvir uma figura musical (uma classe de objeto \pressingthree{O}{ask}{0}), duas díades, um intervalo de quarta justa entre Sol 2 (MIDI 55) e Dó 3 (MIDI 60). entre Mi bemol 2 (MIDI 51) e Dó 3 (MIDI 60).

%%%%%%%%%%%%%%%%%%%%%%%%%%%%%%%%%%%%%%%%%%%%%%%%
\begin{example}{Estratégia transversal}
\begin{minted}[fontsize=\footnotesize]{cl}
    (define chords
       (lambda (time degree dur)
          (if (member degree '(i)) (set! dur 3.0))
          (for-each (lambda (p)
                       (play-note (*metro* time) piano p
                                  (+ 50 (* 20 (cos (* pi time))))
                                  (*metro* 'dur dur)))
                    (pc:make-chord 50 70 2 (pc:diatonic 0 (quote -) degree)))
          (callback (*metro*) (+ time (* .5 dur))) chords (+ time dur)
                    (random (assoc degree '((i vii)
                                            (vii i))))
                    dur))
    
     (chords (*metro* 'get-beat 4.0) 'i 3.0)
\end{minted}
\end{example}
%%%%%%%%%%%%%%%%%%%%%%%%%%%%%%%%%%%%%%%%%%%%%%%%

Duas transcrições desta primeira figura seguem uma estrutura literal do código, e uma perceptiva. Os primeiros eventos sonoros que ocorrem após o momento de silêncio foram transcritos antes da análise do código. Enquanto Sorensen define um tempo regular de 110 BPM, transcrevemos este trecho com um andamento entre 35--40 BPM \ver{fig:ask1}. %É interessante notar que tais figuras simbolizam neumas, no caso, um \emph{bipunctum}, ou duas notas repetidas, na mão direita, e na mão esquerda um \emph{clivis}, ou um \traducao{acento agudo com um grave}{\cfcite[\emph{idem}]{gasperini_semiografia_1905}. Unione dell'accento acuto col grave}. No caso específico desta primeira figura, na mão direita, um \emph{bipunctum} , e na mão esquerda, uma \emph{clivis}.

\begin{figure}[!h]
  \centering
  \centering 
  \input{./ask1}
  \input{./ask2}
  %\input{./gregorian}
  \caption{Transcrição literal e perceptiva do primeiro evento em \emph{A Study in Keith}. \textbf{Fonte}: autor.}
  \label{fig:ask1}
\end{figure}

 É importante notar que algumas alterações são feitas. A primeira delas é definir outros argumentos para \verb|chords|, como um acorde localizado em um grau de um campo harmônico abstrato, e a duração do acorde executado:

%%%%%%%%%%%%%%%%%%%%%%%%%%%%%%%%%%%%%%%%%%%%%%%%
\begin{example}{Modificação do código original}
\begin{minted}[fontsize=\footnotesize]{cl}
    (define chords
       (lambda (time degree dur) ...))
\end{minted}


A segunda alteração é a indicação de uma situação condicional na primeira transformação da estratégia transversal \csf{T}{ask}. Se o grau a ser executado for uma tônica, no caso, menor, a duração deste acorde será configurada para uma duração de três unidades de tempo -- no caso da nossa transcrição, uma unidade de pulso.

\begin{minted}[fontsize=\footnotesize]{cl}
(define chords
   (lambda (time degree dur)
      (if (member degree '(i)) (set! dur 3.0)) ... ))
\end{minted}

A terceira alteração modifica a intensidade das notas:

\begin{minted}[fontsize=\footnotesize]{cl}
(play-note (*metro* time) piano p
           (+ 50 (* 20 (cos (* pi time))))
           (*metro* 'dur dur))
\end{minted}

Onde a a dinâmica específica ocorre como um comportamento periódico de volumes máximos (fortes), e mínimos (pianos), em, proporcional ao cosseno do tempo instantâneo (\verb|cos (* pi time)|), escalonado para valores MIDI:

\begin{minted}[fontsize=\footnotesize]{cl}
(+ 50 (* 20 (cos (* pi time))))
\end{minted}
\end{example}
%%%%%%%%%%%%%%%%%%%%%%%%%%%%%%%%%%%%%%%%%%%%%%%%

\subsubsection{Regras de qualidade}\label{sec:regras_qualidade}

A estrutura interna da estratégia \verb|chords| explicita algumas regras de qualidade, bem como permite apresentar uma primieira sequência de blocos de eventos \pressingthree{K}{ask}{0}, um conjunto de características \pressingthree{F}{ask}{0} e um pequeno grupo de objetos \pressingthree{O}{ask}{0}. Um conjunto de características é definido pelo momento de execução do evento,\pressingthree{F}{ask}{0}, o grau, \pressingthree{F}{ask}{1}, e a duração deste evento, \pressingthree{F}{ask}{2}. É importante destacar que o momento de execução é relativo ao tempo base, definido dentro do padrão \verb|* metro *| (que será explicado a seguir) de um campo harmônico, onde i representa uma tônica menor, e vii, um acorde de sétimo grau, e a duração deste acorde.

\begin{example}{Regra de qualidade \csf{R}{ask}.}
\begin{minted}[fontsize=\scriptsize]{cl}
( ... (... (callback (*metro*) (+ time (* .5 dur))) chords (+ time dur)
                    (random (assoc degree '((i vii)
                                            (vii i))))
                    dur))
\end{minted}

Cujas características irão gerar blocos de eventos, e sequências de blocos de eventos:

\begin{minted}[fontsize=\scriptsize]{cl}
( ...
  (lambda (time degree dur) ... ))
\end{minted}

O que permite executar como:
\begin{minted}[fontsize=\scriptsize]{cl}
(chords (*metro* 'get-beat 4.0) 'i 3.0)
\end{minted}
\end{example}

\section{Primeira sequência de blocos de eventos}\label{sec:primeiro_evento}

A \autoref{fig:ask3} indica uma primeira sequência de notas geradas pelo algoritmo, um padrão que é repetido por dois ciclos (blocos de eventos \pressingthree{E}{ask}{0} e \pressingthree{E}{ask}{1}). Durante este tempo, Sorensen realiza uma mudança \ver{fig:bricolagem}. Esta mudança transita entre o segundo bloco \pressingthree{E}{ask}{1} e terceiro bloco \pressingthree{E}{ask}{2}, e sua exeucção resulta em uma transformação da acentuação, o que termina por colocar, no último compasso deste ciclo, o sétimo grau no tempo forte e o primeiro grau no tempo fraco. 

\subsection{Primeiro Bloco}

O primeiro bloco de eventos \pressingthree{E}{ask}{0} aprensenta um contraponto de primeira espécie, aticulado em tempos fortes e fracos, de acordo com um movimento cadencial $i~\Rightarrow~vii$ \ver{fig:ask3}. 

\begin{figure}[!h]
  \centering
  \input{./ask3}
  \caption{Primeiros eventos musicais gerados a partir das primeiras estruturas válidas de código. \textbf{Fonte}: autor.}
  \label{fig:ask3}
\end{figure}

Esta cadência descaracteriza \emph{A Study in Keith} ser uma peça, do ponto de vista harmônico, semelhante aos Concertos \emph{Sun Bear} de Keith Jarret. Em outras palavras, esta é uma razão pela qual \emph{A Study in Keith} é apenas inspirado por Keith Jarret, mas não é Keith Jarret, no sentido harmônico que a expressão carrega \ver{sec:proposicao}.

Uma pequena comparação com o plano harmônico do concerto de Kyoto \ver{sec:sunbearanal} com primeiro bloco de \emph{A Study in Keith} pode elucidar a questão. Enquanto em Jarret temos uma estrutura cromática oblíqua a um baixo pedal, em Sorensen temos uma estrutura diatônica e com movimentos contrários:

%A aparente repetição de um mesma classe de eventos sonoros, este mesmo um objeto \pressingthree{O}{ask}{0}, pode ser diferenciada através de figuras neumáticas na mão direita e na mão esquerda \exref{fig:neumaMD1}:

\begin{example}{Redução do primeiro bloco}\label{fig:neumaMD1}

%  Notação neumática para a um \emph{bipunctum}, dois \emph{podatus} e uma \emph{clivis} na mão direita. E na mão esquerda, três \emph{clivis} e um bipunctum. 

%  \centering{\input{./ask6}}

Movimentos da voz superior (\^1-\^4-\^1) e inferior (\^5-\^2-\^3).

  \centering{\input{./ask12}}

\end{example}

\subsection{Segundo Bloco}

O Segundo bloco dá continuidade ao contraponto inicial, de forma que o âmbito melódico da voz superior é mantida e da voz inferior é restringida:

\begin{figure}[!h]
  \centering
  \input{./ask4}
  \caption{Segundo bloco de eventos musicais. \textbf{Fonte}: autor.}
  \label{fig:ask4}
\end{figure}

%Que pode ser reescrito como neumas na mão direita:

\begin{example}{Redução do segundo bloco}\label{fig:neumaMD2}

%  Notação neumática para cinco \emph{podatus} e uma \emph{clivis} na mão direita. E na mão esquerda uma \emph{clivis}, um \emph{podatus}, um \emph{bipunctus}, três \emph{podatus}.

%  \centering{\input{./ask8}}

Movimento da voz superior (\^1-\^5-\^1) e inferior (\^5-\^2-\^3).

  \centering{\input{./ask13}}
\end{example}

\subsection{Terceiro Bloco}

Enquanto nos blocos \pressingthree{E}{ask}{0} e \pressingthree{E}{ask}{1} existem eventos significativos do ponto de vista figurativo, o aspecto rítmico é único (um tempo forte no $i$ grau, um tempo fraco na $vii$ grau). É importante destacar que, entre estes blocos, Sorensen realiza uma transformação na estratégia transversal:

%%%%%%%%%%%%%%%%%%%%%%%%%%%%%%%%%%%%%%%%%%%%%%%%
\begin{example}{Primeira transformação da estratégia transversal}
\begin{minted}[fontsize=\footnotesize]{cl}
    (define chords
       (lambda (time degree dur)
          (if (member degree '(i)) (set! dur 3.0))
          (for-each (lambda (p)
                       (let* (dur1 (* dur (random '(0.5 1))))
                             (dur2 (- dur dur1)))
                       (play-note (*metro* time) piano p
                                  (+ 50 (* 20 (cos (* pi time))))
                                  (*metro* 'dur dur1))
                       (if (> dur2 0)
                           (play-note (*metro* (+ time dur1)) piano
                                      (pc:relative p (random '(-2 -1 1 2))
                                                   (pc:scale 0 'aeolian))
                                      (+ 50 (* 20 (cos (* pi (+ time dur1)))))
                                      (*metro* 'dur dur2))))
                       (pc:make-chord 50 70 2 (pc:diatonic 0 (quote -) degree)))
          (callback (*metro*) (+ time (* .5 dur)) chords (+ time dur)
                    (random (assoc degree '((i vii)
                                            (v i))))
                    (random (list 1 2 3)))))
    
     (chords (*metro* 'get-beat 4.0) 'i 3.0)
\end{minted}
\end{example}
%%%%%%%%%%%%%%%%%%%%%%%%%%%%%%%%%%%%%%%%%%%%%%%%

O que, durante esta transição, gera uma transformação na acentuação \ver{fig:ask4}.

\begin{figure}{Transcrição do terceiro bloco}
  \centering
  \input{./ask5}
  \caption{Terceiro bloco de eventos musicais. \textbf{Fonte}: autor.}
  \label{fig:ask4}
\end{figure}

\begin{example}{Redução do terceiro bloco}\label{fig:neumaMD3}

%Notação neumática para: \emph{bipunctus}, um \emph{clivis}, um \emph{porrectus}, um \emph{clivis}, um \emph{torculus}, um \emph{clivis}, um \emph{torculus}, dois \emph{clivis} e um \emph{clivis subpunctum} na mão direita. E na mão esquerda um \emph{bipunctus}, um \emph{clivis}, um \emph{podatus}, um \emph{porrectus}, um \emph{torculus}, um \emph{podatus}, um \emph{torculus}, um \emph{climatus}, e um \emph{clivis}.

%  \centering{\input{./ask10}}

Movimento da voz superior (\^1-\^2-\^1) e da voz inferior (\^5-\^7$b$-\^3).
  
  \centering{\input{./ask14}}
\end{example}

Esta estratégia modifica o o laço iterativo interno de cada altura da díade:

%%%%%%%%%%%%%%%%%%%%%%%%%%%%%%%%%%%%%%%%%%%%%%%%
\begin{example}{Laço iterativo modificado}
\begin{minted}[fontsize=\footnotesize]{cl}
(for-each (lambda (p)
             (let* (dur1 (* dur (random '(0.5 1))))
                   (dur2 (- dur dur1)))
             (play-note (*metro* time) piano p
                        (+ 50 (* 20 (cos (* pi time))))
                        (*metro* 'dur dur1))
             (if (> dur2 0)
                 (play-note (*metro* (+ time dur1)) piano
                            (pc:relative p (random '(-2 -1 1 2))
                                         (pc:scale 0 'aeolian))
                            (+ 50 (* 20 (cos (* pi (+ time dur1)))))
                            (*metro* 'dur dur2))))
             (pc:make-chord 50 70 2 (pc:diatonic 0 (quote -) degree)))
\end{minted}
\end{example}
%%%%%%%%%%%%%%%%%%%%%%%%%%%%%%%%%%%%%%%%%%%%%%%%

A primeira grande mudaça é a definição de duas variáveis internas, através do comando \verb|let| (seja), chamadas \verb|dur1| e\verb|dur2|:

%%%%%%%%%%%%%%%%%%%%%%%%%%%%%%%%%%%%%%%%%%%%%%%%
\begin{example}{Laço iterativo modificado}
\begin{minted}[fontsize=\footnotesize]{cl}
(let* (dur1 (* dur (random '(0.5 1))))
                   (dur2 (- dur dur1)))
\end{minted}
\end{example}
%%%%%%%%%%%%%%%%%%%%%%%%%%%%%%%%%%%%%%%%%%%%%%%%

Estas variáveis irão tornar os ritmos de ambas as mãos independentes. O ritmo da mão direita pode ser mantido ou diminuido (\verb|(* dur (random '(0.5 1))|), enquanto o ritmo da mão esquerda é uma diferença entre uma duração geral, e o ritmo da mão direita. No caso desta nova duração da mão esquerda, é aplicado uma verificação condicional:

%%%%%%%%%%%%%%%%%%%%%%%%%%%%%%%%%%%%%%%%%%%%%%%%
\begin{example}{Laço iterativo modificado}
\begin{minted}[fontsize=\footnotesize]{cl}
(if (> dur2 0)
    (play-note (*metro* (+ time dur1)) piano
               (pc:relative p (random '(-2 -1 1 2))
                            (pc:scale 0 'aeolian))
               (+ 50 (* 20 (cos (* pi (+ time dur1)))))
               (*metro* 'dur dur2)))
\end{minted}
\end{example}
%%%%%%%%%%%%%%%%%%%%%%%%%%%%%%%%%%%%%%%%%%%%%%%%

Se a diferença entre a duração total e a nova duração for inválida (igual a $0$), a nota tocada dependerá do resultado de \verb|pc:relative|. A função \verb|pc:relative| é definida\disponivelem{https://github.com/digego/extempore/blob/master/libs/core/pc_ivl.xtm} como \traducao{seleção de uma altura, de uma classe de alturas relativa à uma dada altura}{select pitch from pitch class relative to a given pitch}. Sua altura serão dadas em passos de segundas menores ou maiores ascendentes/descendentes, relativas ao modo eólico da escala (que no caso transforma a sonoridade tonal em sonoridade modal). 

%%%%%%%%%%%%%%%%%%%%%%%%%%%%%%%%%%%%%%%%%%%%%%%%
\begin{example}{Laço iterativo modificado}
\begin{minted}[fontsize=\footnotesize]{cl}
(pc:relative p (random '(-2 -1 1 2))
             (pc:scale 0 'aeolian))
(+ 50 (* 20 (cos (* pi (+ time dur1)))))
\end{minted}
\end{example}
%%%%%%%%%%%%%%%%%%%%%%%%%%%%%%%%%%%%%%%%%%%%%%%%

O ritmo da mão esquerda será semelhante ao da mão direita. 

%%%%%%%%%%%%%%%%%%%%%%%%%%%%%%%%%%%%%%%%%%%%%%%%
\begin{example}{Laço iterativo modificado}
\begin{minted}[fontsize=\footnotesize]{cl}
(+ 50 (* 20 (cos (* pi (+ time dur1)))))
\end{minted}
\end{example}
%%%%%%%%%%%%%%%%%%%%%%%%%%%%%%%%%%%%%%%%%%%%%%%%

No entanto esta característica é um dos fios condutores de uma seção \pressingthree{K}{ask}{1}, o que excede um objetivo deste documento. Nosso interesse nesta análise foi investigar, através do estudo de contextos, e de diferente notações musicais (código, partitura, neuma e esquema analítico), de uma mesma música, a sonoridade que irá gerar outras sonoridades, no caso desta pesquisa, uma sonoridade com raízes em esquemas tonais.

\section{Discussão}

Este capítulo buscou analisar uma improvisação publicada em 2009 por Andrew Sorensen. Levantamos um conjunto de informações que estimularam o improvisador-programador a realizar \emph{A Study in Keith}. A partir de uma outra improivisação,  definida como \emph{referencial zero}, ou os \emph{Concertos Sun Bear}, Sorensen buscou simular um estilo de \emph{jazz} do pianista e compositor Keith Jarret. No entanto, esta relação é pouco significativa do ponto de vista harmônico. Na nossa transcrição de uma cadência da abertura do disco Kyoto I, encontramos uma exploração da cadência plagal do \emph{blues} por substituição de trítono. Já em Sorensen o procedimento é uma simples cadência \emph{i} -- \emph{vii}. A partir desta regra, foi possível apontar figuras musicias separadas em três blocos, o que delineou um método de análise para outras improvisações de código. Por outro lado, nossa análise não contemplou sequências seguintes, o que impediu observar detalhes sobre o processo geral da improvisação. Nosso interesse residiu em investigar o ciclo de elaboração/codificação apresentado na \autoref{fig:bricolagem} (ver p.~\pageref{fig:bricolagem}). Por fim, descrevemos esta análise como uma experiência preliminar em análise de códigos de um ponto de vista cognitivista. 

%\newpage

% ---
% Conclusão
% ---
\chapter*[Conclusão]{Conclusão}\addcontentsline{toc}{chapter}{Conclusão}\label{conclusao}

A pesquisa tomou um rumo diferente da proposta inicial, : um estudo que discutiria a questão da aparente intimidação experenciada por compositores no uso de linguagens de programação para composição musical, especificamente em ambientes de redes de computadores, bem como a elaboração de um \emph{software} original com base nessa reflexão. 

Para chegar neste aplicativo, tomei conhecimento de uma cena emergente no que é conhecido hoje como \emph{live coding} em navegadores de internet\footnote{\emph{Apple Safari}, \emph{Google Chrome}, \emph{Mozzilla Firefox}.}, tais como \emph{Gibber} \cite{roberts_gibber:_2012}, \emph{Vivace} \cite{vieira_vivace:_2015}, \emph{Wavepot}\footnote{\url{http://www.wavepot.com}, acessado em \today}, \emph{Html5Bytebeat}\footnote{\url{https://github.com/greggman/html5bytebeat}, acessado em \today.}. Uma reflexão a respeito das linguagens nestes aplicativos possibilitou o desenvolvimento de um aplicativo \emph{web} em conjunto com Luíz Schiavonni, professor da UFSJ (Universidade Federal de São João del-Rei), que foi chamado de \emph{Termpot}\footnote{\url{http://jahpd.githb.io/termpot}, acessado em \today.}. 

No entanto, o Programa de Pós-graduação em Artes, Cultura e Linguagens  (PPG-ACL/UFJF) proporcionou o contato com obras  de autores como \citeonline{kuhn_structure_1970}, \citeonline{feyerabend_against_1975} e \citeonline{santos_filosofia_2008} que foram significativas para aprimorar a metodologia; arrisquei-me a uma pesquisa orientada a paradigmas, aqui, paradigmas do \emph{livecoding}, considerando a existência cooperativa/competitiva de um conjunto deles em ambientes de produção de conhecimento (incluímos aqui a Universidade). Adicionalmente, a vivência com uma botânica dedicada em problemas de plano de manejo de uma espécie vegetal em Ubatuba/SP e um comunicador debruçado em problemas de gênero em Juiz de Fora/JF, me colocou na posição de refletir sobre uma questão, pessoalmente mais fundamental que a proposta inicial, de uma ecologia de gêneros musicais utilizando um fragmento de produções do \emph{livecoding}\footnote{Em comunicação pessoal com Tiago Rubini, ``a primeira menção à palavra 'gênero' se refere à questão de identidade de gênero. A segunda, sobre dinâmicas de gêneros musicais''. Infiro que uma dinâmica de gêneros musicais pressupõe um conjunto limitado de conhecimentos, que exige ser discutido para oferecer uma idéia das próprias limitações do conhecimento discutido.}. Autores citados acima não foram utilizados neste trabalho para discutir o \emph{livecoding}, mas foram importantes para a a concepção do seguinte triangulamento metodológico realizado: levantamento qualitativo do \emph{livecoding}, a partir de uma bibliografia básica disponível\footnote{\url{http://toplap.org/wiki/Videos,_Articles_and_Papers}. Acessado em \today.} e verificação quantitativa, através de levantamento de dados seletivo na internet que confirmasse o levantamento qualitativo em um nicho específico\footnote{Utilizando um SDK disponibilizado pelo \emph{Soundcloud}.}.


A partir de diferentes textos, derivei palavras-chave que carregam significados do objeto de pesquisa: \emph{live coding}, \emph{live-coding}, \emph{livecoding}, \emph{live code}, \emph{conversational programming}, \emph{on-the-fly programming}, \emph{live algorithm programming}; estas palavras-chaves apontam para uma prática de improvisação utilizando o computador, articulada de maneira colaborativa ou competitiva entre \emph{performers}, mediada por projetores visuais e acústicos (projetores e alto-falantes), e direcionadas para um público parcialmente passivo em ato de ouvir música e ver imagens (codificadas durante a improvisação). Este aspecto de improvisação com \emph{scripts} criados e editados no computador, segundo \citeonline{cox_coding_2004}, reside em uma capacidade de predizer (de forma aproximada) resultados complexos antes de codificar, isto é, o  executante deve ser capaz de realizar em um curto espaço de tempo uma formalização lógica de um comportamento audiovisual antes de sua programação em um \emph{script}: 

\begin{citacao}
Um programador é, portanto, capaz de predizer e especular sobre como o seu código irá se comportar em circunstâncias mais usuais. Como com qualquer coisa que é de autoria, a questão da subjetividade é inevitável, uma vez que qualquer resultado particular pode ser alcançado em diferentes (e muitas vezes concorrentes) maneiras. Nesse sentido, qualquer senso de improvisação depende de um entendimento preditivo de sistemas complexos e geradores. \cite[p.~169]{cox_coding_2004}\footnote{Tradução nossa de: \emph{A programmer is therefore able to predict and speculate upon how their code will behave in most usual circumstances. As with anything that is authored, the issue of subjectivity is unavoidable, since any particular result can be achieved in different (and often competing) ways. In this, any sense of improvisation relies on a preditive understanding of complex and generative systems}.} 
\end{citacao}

Colocado de outra forma por \citeonline{ruthmann_teaching_2010}

\begin{citacao}
Executar de maneira efetiva uma manipulação em tempo-real do código (live coding musical) para criar e formalizar música gerada requer ambos entendimentos musicais e computacionais. De uma perspectiva musical, é necessário entender o fluxo de como uma música generativa deve soar. Da perspectiva computacional, é preciso entender como o código deve ser ajustado e manipulado em tempo real para atingir mudanças aurais e musicais desejadas. \cite[p.~3-4]{ruthmann_teaching_2010}. \footnote{Tradução nossa de \emph{Performing effective real-time manipulation of code (musical live coding) to create and shape generated music requires both musical and computational understanding. From a musical perspective, one needs to understand how the ongoing, generative music should sound. From a computational perspective, one needs to understand how the code can be adjusted and manipulated in real time to achieve the aural and musical changes and outcomes one desires}.}
\end{citacao}

Esses conhecimentos pré-concebidos antes de codificar podem ter origem no conhecimento de mundo da Música daquele que realiza a improvisação; de diferentes maneiras, músicos-programadores adequam programação-partituras para determinados contextos do fazer musical. Expressa-se uma teoria musical que regula os algoritmos, esta por sua vez regulada por uma ecologia de gostos musicais adequadas para cada contexto. Interessado na dinâmica dos gêneros musicais que emergem de diferentes contextos de improvisação, coloquei-me a tarefa de discutir o que diferentes autores debruçados no assunto \emph{livecoding} entendem por música, como realizam, e seus discursos da respeito de realizações em público. 

Decidi recortar o assunto a partir de comparações limitadas ao âmbito musical por julgar-me incapaz de abordar diferentes linguagens artísticas como a literatura generativa e práticas derivadas de exibição simultanea da imagem renderizada e seu respectivo código; algumas menções foram feitas apenas para delimitar o que esté fora e o que está dentro de uma janela de discussão. Uma bibliografia básica foi levantada como forma de esclarecer origens e práticas musicais do \emph{livecoding}: \begin{inparaenum}[]
\item \citeonline{cox_aesthetics_2000},
\item \citeonline{cox_coding_2004},
\item \citeonline{collins_live_2003},
\item \citeonline{mclean_hacking_2004},
\item \citeonline{wang_--fly_2004},
\item \citeonline{ward_live_2004},
\item \citeonline{collins_live_2007},
\item \citeonline{rohrhuber_improvising_2009},
\item \citeonline{ruthmann_teaching_2010},
\item \citeonline{mclean_visualisation_2010},
\item \citeonline{magnusson_algorithms_2011},
\item \citeonline{magnusson_herding_2014},
\item \citeonline{magnusson_scoring_2014}
\end{inparaenum}.


O \emph{livecoding} pode existir em contextos tradicionalmente reservados para concertos ou em contextos informais que estimulam a sociabilização através da dança (os \emph{Night Clubs} de \citeonline{mclean_hacking_2004}); noto que, em um período de aproximadamente dez anos, ocorre um movimento de cooperação entre espaços acadêmicos e de entretenimento, onde emerge o termo \emph{algorave}\footnote{``Nenhuma conferência acadêmica está completa sem uma \emph{algorave}, uma chance de dançar algoritmos com velhos ou novos amigos. Teremos pelo menos um clube de noite, no excelente co-op Wharf Chambers $[$\url{http://www.wharfchambers.org/}$]$. em \url{http://iclc.livecodenetwork.org/cfp.html\#performance}. Tradução nossa de: \emph{No academic conference is complete without an algorave, a chance to dance to algorithms with friends new and old. We will have at least one club night, at the excellent Wharf Chambers co-op. More details to follow.}}. Por outro lado, estes mesmos espaços estimularam o desenvolvimento de novas áreas de pesquisa, entre elas, o \emph{livecoding} de ambientes virtuais, isto é, a prática sendo aplicada em pequenas ou grandes redes de computadores; ao mesmo tempo, o advento da biblioteca \emph{WebAudio API} possibilitou a experimentação do \emph{livecoding} na internet\footnote{Como os aplicativos \emph{Gibber}, \emph{Wavepot} e \emph{Html5Bytebeat}, \emph{Vivace}. Para mais informações a respeito, sugiro a leitura de "The Web Browser As Synthesizer And Interface" de \citeonline{roberts_web_2013} e "The Viability of the Web Browser as a Computer Music Platform"\citeonline{wyse_viability_2014}.}.

%No primeiro modo, existe uma pretensa sociabilização entre agentes. Por agentes me refiro aos membros do grupo de executantes e membros do grupo do público. Por sociabilização, entendo a interação entre membros. Esta separação ilustra uma característica observada (e praticada pelo autor deste trabalho) até o momento: em algumas performances ao vivo e vídeos da internet, ou em apresentações, a sociabilização entre membros executantes se caracteriza como ativa (modificação  da improvisação através da edição do código compartilhado ou por interferências sonoras), e a sociabilização entre executantes/público tem uma tendência a ser passiva (os membros do público são apenas convidados a observar e a escutar o que está sendo realizado). Tais agrupamentos ainda podem ser organizados em: \begin{inparaenum}[\itshape 1)\upshape] \item agrupamentos institucionalizados (como as \emph{Laptop orchestras}); e \item agrupamentos informais (\emph{solos}, \emph{duos}, \emph{trios}) \end{inparaenum}. 

Destes dois modos busquei fazer um mapeamento de pŕaticas musicais mencionadas por autores de manifestos e artigos mencionados no segundo parágrafo desta conclusão, bem como a localização geográfica e que práticas musicais são colocadas no plano de discussão em relação ao \emph{livecoding}: \begin{inparaenum}[\itshape 1)\upshape]
\item Música Algorítmica (MA),
\item Música Processual (MP), 
\item Música Generativa (MG),
\item e Música de Pista (ou o que que denominamos a partir do termo simplificado \emph{Disk Jockey}, (DJ)
\end{inparaenum}. Busquei averiguar até que ponto o uso dos termos referidos estão de acordo com definições compartilhadas, respaldados em um embate com autores como \citeonline{reich_music_1968}, \citeonline{eno_music_1978}, \citeonline{kramer_sonification_1999}, \citeonline{roads_times_2001}, \citeonline{wooler_framework_2005} \citeonline{malt_concepts_2006}, \citeonline{walker_auditory_2006}, \citeonline{essl_algorithmic_2007},  \citeonline{cope_prefacio_2008}, \citeonline{iazzetta_musica_2009}, \citeonline{mailman_agency_2013} e \citeonline{collins_algorave:_2014} \citeonline{casteloes_conversores_2015}. 

Em um segundo momento, levantamos dados pertinentes ao tema na rede social \emph{Soundcloud} como maneira de justificar o primeiro mapeamento das comunidades de gosto.

Realizando uma retrospectiva geopolítica dos autores, isto é, observando as localizações do globo terrestre em que foram escritos os textos supracitados no primeiro parágrafo deste capítulo, outros em \emph{passim}, mais um conjunto de dados que podem ser observados no \autoref{dados_sclivecoding} -- que descrevem essas regiões de maneira quantitativa em um período de tempo entre 2008 e 2015--, é possível confirmar uma tese que foi sendo construída no decorrer da pesquisa, mas que somente pode ser verbalizada no segundo parágrafo de ``A genealogia da moral'' de Nietzsche, ``o privilégio senhorial de dar nomes permitem-nos conceber a origem da linguagem ela mesma como uma manifestação de poder dos governantes''\footnote{Cf. \emph{On the Genealogy of Morality} Edited by Keith Ansell-Pearson. Translated by Carol Diethe. Cambridge. 2006. Tradução nossa do segundo parágrafo do primeiro ensaio \emph{The seigneurial pribilege of giving names even allow us to conceive of the origin of a language itself as a manifestation of the power of the rulers}.}, no sentido de que o \emph{live coding} possue esse nome, como \emph{programa de pesquisa}, que integra todo uma Epistemologia do Norte que que \citeauthoronline{santos_filosofia_2008} tanto comenta.

Na inglaterra identifico os autores como Alex McLean, Nick Collins Adrian Ward, e Dave Griftths. Nos EUA identifico as \emph{Laptop Orchestras}, em Stanford e Princeton, bem como compositores como James Harkins e Joshua Parmenter e Ge Wang;  na Austrália, tem sido notável o papel de Andrew Sorensen na \emph{Queensland University of Technology} utilizando o piano; no Brasil identifico os trabalhos de Bernardo Barros, André Damião, Antônio Goulart, Vilson Vieira, Geraldo Magela de Castro Rocha Junior, Caleb Mascarenhas Luporini, Daniel Penalva, Ricardo Fabbri, Renato Fabbri, Ricardo Brasileiro e Daniel Penalva e Flávio Luiz Schiavonni. 

No levantamento de dados do Soundcloud pude confirmar alguns dos pontos levantados no levantamento bibliográfico e questionar algumas afirmações feitas: por exemplo, na questão de qual país produz grande quantidade de livecoding confirmei posições de países falantes da língua inglesa, como Inglaterra e EUA, mas ao mesmo tempo, notei uma grande quantidade de produções anônimas, isto é, com localizações não identificadas, desacreditando em uma centralização de produção; ademais foi possível perceber uma miríade de produções em países como Alemanha, México e Japão. Na questão de gênero musical, articulado através de \emph{hashtags}\footnote{Para mais informações, sugiro a leitura de \url{https://en.wikipedia.org/wiki/Hashtag} e \url{https://en.wikipedia.org/wiki/Tag_(metadata)}, acessados em \today.}, pude confirmar uma hibridização de práticas musicais consideradas anteriormente distintas, e ao mesmo tempo, questionar uma centralização conceitual em um ou outro termo, como \begin{inparaenum}[\itshape a)\upshape]
\item \emph{algorithmic music}
\item \emph{algorave},
\item \emph{algopop},
\item \emph{bytebeat},
\item \emph{drone}
\item \emph{electronic music},
\item \emph{electroacoustic},
\item \emph{glitch},
\item \emph{noise},
\item \emph{whistling}
\end{inparaenum}.

O seguinte ponto-de-vista foi desenvolvido: cada termo utilizado no \emph{livecoding} carrega uma teoria musical e, por outro lado, costumes distintos do fazer musical no \emph{livecoding} estão relacionados com gêneros musicais, a partir  daquilo que  discute comunidades de gosto; tais comunidades de gosto foram investigadas em um nicho musical virtual específico, a saber, a rede social Soundcloud. 

Percebi o \emph{livecoding} como um campo de estudos em transição entre Música e Ciências da Computação; esse campo tem sido estimulado em países considerados como centros de referência, por exemplo, Inglaterra e EUA. Esse estímulo tem sido acompanhado por colaborações interdisciplinares formalizadas (isto é, reconhecidas institucionalmente); não coincidentemente, isso foi pressuposto por \citeonline{mathews_digital_1963} no desenvolvimento da família MUSIC N como única maneira de avançar neste campo de estudo\footnote{Lembro que o desenvolvimento de estudos musicais com CSIRAC (na Austrália),  antes mesmo do advento do MUSIC N, tem como um dos fatores de seu fracasso, segundo \citeonline{di_nunzio_genesi_2010}, devido a uma falta de cooperação entre músicos e cientistas da computação.}. No Brasil, no entanto, existem barreiras institucionais que impedem a colaboração interdisciplinar formal, porém estas mesmas barreiras forçam músicos interessados na área, a aprender a programar, mesmo com algumas barreiras do jargão técnico (muitas vezes intimidadoras); de maneira semelhante, cientistas (da computação e muitas vezes físicos) e não-acadêmicos brasileiros, com seu conhecimento de programação, exploraram comportamentos musicais em \emph{softwares} como forma de ultrapassar uma barreira impostas pelo jargão técnico dos músicos.



% ---
% Bibliografia
% ---
\bibliography{main}

% ----------------------------------------------------------
% Glossário
% ----------------------------------------------------------
%
% Consulte o manual da classe abntex2 para orientações sobre o glossário.
%
%\glossary

% ----------------------------------------------------------
% Apêndices
% ----------------------------------------------------------

% Imprime uma página indicando o início dos apêndices
%\partapendices

% ---
% Inicia os apêndices
% ---
%\begin{apendicesenv}
%\chapter{Código fonte de um Universo Conceitual como nuvem de palavras sobre o improviso de códigos}\label{app:A}

O tema da improvisação de códigos, como um universo de conceitos, surgiu de uma experiência com um código em linguagem \emph{python}\disponivelem{https://www.python.org/}, útil para gerar um mapa de termos, como ilustrado na figura \autoref{fig:nuvemlivecoding}

\begin{figure}[!h]
\includegraphics[scale=0.8]{imagens/nuvem.png}
\caption{Nuvem de palavras do \citeonline{ICLC2015},  1$^o$ Congresso Internacional de Live Coding. \textbf{Fonte}: autor.}
\label{fig:nuvemlivecoding}
\end{figure}

A imagem acima foi gerada com um código nomeado como \emph{cloupdf.py}, e considera seguinte situação, subdividida em três passos: 

1) Converter um arquivo de texto, ou um conjunto de textos cientítificos, em formato \emph{.pdf} para formato \emph{.txt}; 

2) Feita a conversão, plotar uma imagem com as palavras mais relevantes, do ponto de vista quantitativo; 

3) Desta plotagem, organizar as palavras qualitativamente, de acordo com suas funções gramaticais no texto, em classes quantitativas (ver \autoref{tab:gen1}, p.~\pageref{tab:gen1}; \autoref{tab:gen2}, p.~\pageref{tab:gen2}; \autoref{tab:gen3}, p.~\pageref{tab:gen3}; \autoref{tab:gen4}, p.~\pageref{tab:gen4}; \autoref{tab:gen5}, p.~\pageref{tab:gen5}).

\input{./cloupdf.tex}

\section{Utilização}

Em um terminal Linux (3.16.0-49-generic, Ubuntu 14.04.1, i686), executamos o código como um comando  com opções de: 1) arquivo de entrada (\verb|--entrada| ou \verb|-e|), número de páginas rastreadas (\verb|--paginas| ou \verb|-p|), classes (\verb|--qualidades| ou \verb|-q|), codificação do pdf (\verb|--codec| ou \verb|-c|), criação de uma nuvem de palavras (\verb|--foto| ou \verb|-f|) e organização de uma tabela (\verb|--table| ou \verb|-t|). 

Este código utilizou três bibliotecas auxiliares \emph{pdf2text.py}\disponivelem{https://pypi.python.org/pypi/pdf2text}, \emph{Wordcloud}\disponivelem{https://github.com/amueller/word_cloud} e \emph{NLTK}\disponivelem{http://nltk.org/}. 

A primeira biblioteca permite extrair do pdf caracteres válidos para análise. A segunda biblioteca realiza o levantamento de dados. E a terceira biblioteca auxilia na organização das funções gramaticais. Essas operações serão exemplificadas na próxima seção.

\section{Experiências}

Durante uma primeira experiência, abrimos um arquivo pdf, mais especificamente os anais de um congresso internacional \cite{ICLC2015}, e organizamos algumas páginas pertinentes, neste caso, todas a páginas com o corpo de texto (p. 4 -- 230),  supostamente codificado em um padrão ISO8895-1 \ver{fig:nuvemlivecoding}.

\begin{example}{Código-fonte do \emph{cloud.py}}
\begin{minted}[fontsize=\scriptsize]{bash}
./cloud.py -e ./iclc2015-proceedings.pdf -p 4..231 -f
\end{minted}
\end{example} 

Uma segunda experiência permitiu organizar a mesma nuvem de palavras em um uma tabela de funções gramaticais. As duas experiências dispararam a intenção de realizar uma pesquisa onde fosse possível verificar os mesmos conceitos em uma bibliografia geral, como feita nos \autoref{cap:introducao} e \autoref{sec:protohistoria}.

\begin{example}{Código-fonte do \emph{cloud.py}}
\begin{minted}[fontsize=\scriptsize]{bash}
./cloud.py -e ./iclc2015-proceedings.pdf -q 10 -p 4..231 -c iso8859-1 -t tex
\end{minted}
\end{example} 

A execução do comando acima gera a seguinte saída de texto:

\begin{example}{Saída de texto do \emph{cloud.py}}
\begin{minted}[fontsize=\scriptsize]{bash}
A converter ./iclc2015-proceedings.pdf para ./iclc2015-proceedings.txt  ... 

=> pdf2txt.py -o /home/guilherme/bitbucket/mestrado/iclc2015-proceedings.txt 
-p 4,5,6,7,8,9,10,11,12,13,14,15,16,17,18,19,20,21,22,23,24,25,26,27,28,29,30,31,
32,33,34,35,36,37,38,39,40,41,42,43,44,45,46,47,48,49,50,51,52,53,54,55,56,57,58,
59,60,61,62,63,64,65,66,67,68,69,70,71,72,73,74,75,76,77,78,79,80,81,82,83,84,85,
86,87,88,89,90,91,92,93,94,95,96,97,98,99,100,101,102,103,104,105,106,107,108,109,
110,111,112,113,114,115,116,117,118,119,120,121,122,123,124,125,126,127,128,129,130,
131,132,133,134,135,136,137,138,139,140,141,142,143,144,145,146,147,148,149,150,151,
152,153,154,155,156,157,158,159,160,161,162,163,164,165,166,167,168,169,170,171,172,
173,174,175,176,177,178,179,180,181,182,183,184,185,186,187,188,189,190,191,192,193,
194,195,196,197,198,199,200,201,202,203,204,205,206,207,208,209,210,211,212,213,214,
215,216,217,218,219,220,221,222,223,224,225,226,227,228,229,230 
-c iso8859-1 /home/guilherme/Dropbox/Mestrado/livecoding/iclc2015-proceedings.pdf
=> ... checking ascii characters
=> Feito

A gerar nuvem ...
=> Feito

A classificar e organizar palavras em uma tabela
=> Feito
\end{minted}
\end{example}

\newpage

\begin{table}
\centering
\caption{VB -- Verbo, forma básica. VBZ -- presente na terceira pessoa do singular.}
\label{tab:gen1}
\small
\begin{tabular}{ | p{2.6cm} | p{2.1cm} | p{2.1cm} | p{1.5cm} | p{0.5cm} | p{0.5cm} | p{0.25cm} | p{0.25cm} | p{0.25cm} | p{0.25cm} | p{0.25cm} | p{0.75cm} |}
\hline
\hline
\tiny \textbf{Qualidade/Função}
 & \textbf{0}
 & \textbf{1}
 & \textbf{2}
 & \textbf{3}
 & \textbf{4}
 & \textbf{5}
 & \textbf{6}
 & \textbf{7}
 & \textbf{8}
 & \textbf{9}
 & \textbf{10} \\ 
\hline
\hline
\tiny \textbf{VB} & \tiny See, Take, Allow, Make, Explore, Provide, Change, Support, Result, Become, Play, Create, Set, Laptop, Show, Project, Different, Type, Output, Object, Present, Point, Parameter, Structure, Memory, Need, Feature, Cognitive, Open, Interface, End, Text, C, Working, Control, Musician, Form, Line, Technique, Ensemble, Networked  & \tiny Use, Design, Machine, Work, State, Problem, Experience, Audio  & \tiny Sound, User, Time, Practice  & \tiny E  & \tiny Code  & \tiny --  & \tiny --  & \tiny --  & \tiny --  & \tiny --  & \tiny Live \\ 
\hline
\tiny \textbf{VBZ}
 & \tiny Collins  & \tiny --  & \tiny --  & \tiny --  & \tiny --  & \tiny --  & \tiny --  & \tiny --  & \tiny --  & \tiny --  & \tiny -- \\
 \hline
 \hline
\end{tabular}
\end{table}

\begin{table}
\centering
\caption{VBG -- \emph{present participle}. VBD -- \emph{past tense}. VBN -- \emph{past participle}}
\label{tab:gen2}
\small
\begin{tabular}{ | p{2.6cm} | p{2.6cm} | p{1.75cm} | p{1.75cm} | p{0.25cm} | p{0.25cm} | p{0.25cm} | p{1cm} | p{0.25cm} | p{0.25cm} | p{0.25cm} | p{0.25cm} |}
\hline
\hline
\tiny \textbf{Qualidade/Funcao}
 & \textbf{0}
 & \textbf{1}
 & \textbf{2}
 & \textbf{3}
 & \textbf{4}
 & \textbf{5}
 & \textbf{6}
 & \textbf{7}
 & \textbf{8}
 & \textbf{9}
 & \textbf{10} \\ 
\hline
\hline
\tiny \textbf{VBG}
 & \tiny Working, Making, Livecoding, Solving  & \tiny Using, Writing  & \tiny Programming  & \tiny --  & \tiny --  & \tiny --  & \tiny Coding  & \tiny --  & \tiny --  & \tiny --  & \tiny -- \\
\hline
\tiny \textbf{VBD}
 & \tiny Set, Developed, Made, Concept, Networked, Shared, Output, Become, Dierent  & \tiny Used, Instrument, Based  & \tiny Sound  & \tiny --  & \tiny --  & \tiny --  & \tiny --  & \tiny --  & \tiny --  & \tiny --  & \tiny -- \\
\hline
\tiny \textbf{VBN}
 & \tiny Developed, Made, Become, Shared, Networked, Method, Set, Need, Output  & \tiny Used, Based  & \tiny --  & \tiny --  & \tiny --  & \tiny --  & \tiny --  & \tiny --  & \tiny --  & \tiny --  & \tiny -- \\
 \hline
  \hline
\end{tabular}
\end{table}

\begin{table}
\centering
\caption{VBP -- \emph{past tense}, sem ser 3$^a$ pessoa do singular. JJ -- \emph{adjetivo, numeral ou ordinal}.}
\label{tab:gen3}
\small
\begin{tabular}{ | p{2.6cm} | p{2cm} | p{1.5cm} | p{1cm} | p{0.25cm} | p{0.75cm} | p{1.25cm} | p{0.25cm} | p{0.25cm} | p{0.25cm} | p{0.25cm} | p{0.25cm} |}
\hline
\hline
\tiny \textbf{Qualidade/Funcao}
 & \textbf{0}
 & \textbf{1}
 & \textbf{2}
 & \textbf{3}
 & \textbf{4}
 & \textbf{5}
 & \textbf{6}
 & \textbf{7}
 & \textbf{8}
 & \textbf{9}
 & \textbf{10} \\ 
\hline
\hline
\tiny \textbf{VBP}
 & \tiny Pattern, Framework, See, Create, Show, Need, Mean, Take, Support, Become, Make, Object, Present, Play, Explore, Project, Change, Type, Point, Allow, Digital, Result, Provide, Et, Knowledge, End, Approach, Video, Cognitive, Collaborative, Server, Screen, Free, Algorithm, Dierent  & \tiny Use, Experience, Work, Process, Network, Audio  & \tiny Sound, Practice  & \tiny E  & \tiny Code  & \tiny --  & \tiny --  & \tiny --  & \tiny --  & \tiny --  & \tiny Live \\
 \hline
\tiny \textbf{JJ}
 & \tiny Current, Electronic, Human, First, Possible, Particular, Free, Open, Mean, Virtual, Potential, Present, Visual, Future, Different, Digital, Collaborative, Important, Cognitive, Similar, Real, Musician, Ensemble, Mclean, Dierent, Livecoding, Working, Networked, Sonic, International, Material, Text, Al, Object, Create, Context, Allow, Laptop, Shared, Developed  & \tiny Musical, Many, Used  & \tiny New, Sound  & \tiny E  & \tiny --  & \tiny Music  & \tiny --  & \tiny --  & \tiny --  & \tiny --  & \tiny Live \\
 \hline
 \hline
\end{tabular}
\end{table}

\begin{table}
\centering
\caption{DT -- \emph{determinant}, sem ser 3$^a$ pessoa do singular. NN -- \emph{noun} (substantivo), comum, singular ou de massa.}
\label{tab:gen3}
\small
\begin{tabular}{ | p{1cm} | p{1cm} | p{1cm} | p{1cm} | p{1cm} | p{1cm} | p{1cm} | p{1cm} | p{1cm} | p{1cm} | p{1cm} | p{1cm} |}
\hline
\hline
\tiny \textbf{Qualidade/Funcao}
 & \textbf{0}
 & \textbf{1}
 & \textbf{2}
 & \textbf{3}
 & \textbf{4}
 & \textbf{5}
 & \textbf{6}
 & \textbf{7}
 & \textbf{8}
 & \textbf{9} \\ 
\hline
\hline
\tiny \textbf{DT}
 & \tiny Another  & \tiny --  & \tiny --  & \tiny --  & \tiny --  & \tiny --  & \tiny --  & \tiny --  & \tiny --  & \tiny --  & \tiny -- \\
 \hline
\tiny \textbf{RP}
 & \tiny --  & \tiny --  & \tiny Sound  & \tiny --  & \tiny --  & \tiny --  & \tiny --  & \tiny --  & \tiny --  & \tiny --  & \tiny -- \\
 \hline
\tiny \textbf{NN}
 & \tiny Control, Art, Number, Point, Method, Et, Al, Interface, Paper, Interaction, Analysis, Feature, Part, Information, Present, Output, Video, Collaboration, Result, Case, Source, Algorithm, Session, Piece, Dierent, Line, Supercollider, Provide, Memory, Program, Web, Browser, Function, Node, End, Form, Set, Parameter, Value, Laptop, Allow, Sonic, Text, Environment, Action, Future, Human, Material, Pattern, Framework, Kind, Body, Potential, Concept, Software, Term, Approach, Context, Order, Application, Programmer, Community, Relation, Audience, Screen, Conference, Show, Development, Structure, Technology, Digital, Tool, Aspect, Activity, Change, Support, Play, Group, Space, Idea, Knowledge, Cell, Server, Figure, Object, World, Need, Type, Composition, Orchestra, Musician, Level, Livecoding, Expression, Within, Project, See, Take, Working, B, Less, Become, Make, C, G, J, Member, Technique, Rule, Cognitive, Explore, Current, Making, Different, University, Solving, Ensemble, Well, Rather, Mean, Particular  & \tiny Machine, Research, State, Example, Work, Use, Audio, Problem, Coder, Process, Writing, Experience, Performer, Network, Design, Way, Improvisation, Instrument, Using, Data, Will  & \tiny Computer, Programming, Language, Time, System, Sound, User, Practice, One  & \tiny E  & \tiny Performance, Code  & \tiny Music  & \tiny Coding  & \tiny --  & \tiny --  & \tiny --  & \tiny Live \\
 \hline
 \hline
\end{tabular}
\end{table}
 
\begin{table}
\centering
\caption{NNS -- \emph{noun}, substantivo próprio no plural. NNP -- \emph{noun} substantivo próprio, singular ou de massa.}
\label{tab:gen4}
\small
\begin{tabular}{ | p{1cm} | p{1cm} | p{1cm} | p{1cm} | p{1cm} | p{1cm} | p{1cm} | p{1cm} | p{1cm} | p{1cm} | p{1cm} | p{1cm} |}
\hline
\hline
\tiny \textbf{Qualidade/Funcao}
 & \textbf{0}
 & \textbf{1}
 & \textbf{2}
 & \textbf{3}
 & \textbf{4}
 & \textbf{5}
 & \textbf{6}
 & \textbf{7}
 & \textbf{8}
 & \textbf{9} \\ 
\hline
\hline
\tiny \textbf{NNS}
 & \tiny Processes, Proceedings, People, Davies, Collins  & \tiny Data  & \tiny --  & \tiny --  & \tiny --  & \tiny --  & \tiny --  & \tiny --  & \tiny --  & \tiny --  & \tiny -- \\
 \hline
\tiny \textbf{NNP}
 & \tiny Collins, Davies, University, Blackwell, Mclean, Supercollider, Figure, Line, Web, Set, Well, Value, Group, Information, Future, Electronic, Expression, Laptop, Body, Proceedings, Conference, Level, International, Making, Analysis, Control, Open, Interface, Interaction, Digital, Algorithm, Art, Cognitive, However, Audience, Visual, Activity, Collaboration, Structure, Within, Sonic, Pi, Although, Take, Collaborative, Framework, Software, Community, J, People, Development, Knowledge, Browser, Technology, Gibber, Free, Orchestra, Number, Livecoding, Composition, Acm, Human, Material, Idea, First, Paper, B, Program, Relation, Davies, Context, World, Show, Working, Approach, Case, Space, Video, C, Action, Networked, Real, Text, Screen, Environment, Processes, Rule, Create, Form, Memory, Application, Method, Two, Solving, G, Without, Org, Source, Object, Shared, See, Grossi, Make, Current, Play, Ensemble, Dierent, Rather, Virtual, Parameter, Support, Type  & \tiny Machine, Musical, Audio, Instrument, Research, Experience, Data, Design, Use, Process, Many, Using, Network, Improvisation, Coder, Work, May, Example, Writing, Based, State, Will, Problem, Performer  & \tiny New, Programming, Language, Sound, System, Computer, User, Practice, Time  & \tiny E  & \tiny Performance, Code  & \tiny Music  & \tiny Coding  & \tiny --  & \tiny --  & \tiny --  & \tiny Live \\
\hline
\hline
\end{tabular}
\end{table}
 
\begin{table}
\centering
\caption{RB -- \emph{adverb}. RBR -- advérbio comparativo. CD -- Numeral ou cardinal. MD -- Auxiliar modal. JJR -- Adjetivo comparativo. }
\label{tab:gen5}
\small
\begin{tabular}{ | p{1cm} | p{1cm} | p{1cm} | p{1cm} | p{1cm} | p{1cm} | p{1cm} | p{1cm} | p{1cm} | p{1cm} | p{1cm} | p{1cm} |}
\hline
\hline
\tiny \textbf{Qualidade/Funcao}
 & \textbf{0}
 & \textbf{1}
 & \textbf{2}
 & \textbf{3}
 & \textbf{4}
 & \textbf{5}
 & \textbf{6}
 & \textbf{7}
 & \textbf{8}
 & \textbf{9} \\ 
\hline
\hline
 \tiny \textbf{RB} & \tiny Well, Even, However, Rather, First, Show, Server, Explore, Video  & \tiny Also  & \tiny --  & \tiny --  & \tiny --  & \tiny --  & \tiny --  & \tiny --  & \tiny --  & \tiny --  & \tiny -- \\
 \hline
\tiny \textbf{RBR}
 & \tiny Less  & \tiny --  & \tiny --  & \tiny --  & \tiny --  & \tiny --  & \tiny --  & \tiny --  & \tiny --  & \tiny --  & \tiny -- \\
 \hline
\tiny \textbf{CD}
 & \tiny Two  & \tiny --  & \tiny One  & \tiny --  & \tiny --  & \tiny --  & \tiny --  & \tiny --  & \tiny --  & \tiny --  & \tiny -- \\
 \hline
\tiny \textbf{IN}
 & \tiny Within, Although, Without, Context, Text, Screen, Explore, Output  & \tiny --  & \tiny Sound  & \tiny --  & \tiny --  & \tiny --  & \tiny --  & \tiny --  & \tiny --  & \tiny --  & \tiny -- \\
 \hline
\tiny \textbf{MD}
 & \tiny Might  & \tiny May, Will  & \tiny --  & \tiny --  & \tiny --  & \tiny --  & \tiny --  & \tiny --  & \tiny --  & \tiny --  & \tiny -- \\
 \hline
\tiny \textbf{JJR}
 & \tiny Less, Programmer, Parameter  & \tiny --  & \tiny User  & \tiny --  & \tiny --  & \tiny --  & \tiny --  & \tiny --  & \tiny --  & \tiny --  & \tiny -- \\
 \hline

\hline
\end{tabular}
\end{table}

%\chapter{Sugestões Proto-históricas}\label{app:B}

Descrevemos neste capítulo um trabalho de \citeonline{mathews_groove_1970}, GROOVE, ainda pouco observado por improvisadores de códigos e a atuação de uma tecnologia, a compilação JIT \cite{aycock_brief_2003} como um sujeito sócio-técnico fundamental para que o \emph{live coding} fosse possível.

\section{GROOVE}

GROOVE, ou \emph{Generated Real-time Operations On Voltage-controlled Equipment} foi um computador desenvolvido na Bell Labs por \cite{mathews_groove_1970}. Alex \citeonline{di_nunzio_genesi_2010} discute como um precedente direto do família de \emph{softwares} MUSIC N\footnote{Desenvolvidos a partir de 1957. As versões \emph{softwares} MUSIC I, II, III, IV, IV-B, IV-BF, V (que passou por modificações no IRCAM), MUSIC 360, MUSIC 11 acarretaram no desenvolvimento do \emph{software} CSound, disponível em \url{https://csound.github.io/}.}. É o primeiro de trabalho de Mathews com reflexões nos aspectos performáticos. Não foi usado para ambientes de performance, mas a peça \emph{The expanding universe} da compositora Laurie \citeonline{spiegel_expanding_1975} foi considerada como exemplar por sua execução instrumental e disponibilidade \emph{online} (ver a seguir). Seu desenvolvimento iniciou em 1968 na \emph{Bell Labs}. Segundo o próprio Mathews, o funcionamento do sistema oferece algumas possibilidades a partir de três conceitos: \emph{criação}, \emph{retroalimentação} e \emph{ciberficação}. O primeiro conceito foi implementado com um sistema de arquivos, onde as funções criadas no processo criativo são memorizadas, e podem ser editadas. O segundo conceito se relaciona com o terceiro:

\traduzcitacao{
O GROOVE provê oportunidades para uma retroalimentação imediata de observações dos efeitos das funções temporais para as entradas do computador, que compõem a função. No modo de composição do sistema GROOVE, um ser humano está em um ciclo de retroalimentação, como mostrado na figura 1 $[$\autoref{fig:groove_sistema}$]$. Assim ele é capaz de modificar as funções instantâneamente como um resultado de suas observações daqueles efeitos
}
{
GROOVE provides opportunity for immediate feedback from observations of the effects of time functions to computer inputs which compose the function. In the compose mode of the GROOVE system, a human beign is in the feedback loop (\ldots) Thus he is able to modify the functions instanteneously as a result of his observations of their effects.
}
{p.~715}{mathews_groove_1970}

\begin{figure}
\begin{center}
\includegraphics[scale=0.618]{./imagens/GROOVE.png}
\caption{Esquema de concepção do projeto GROOVE. \textbf{Fonte}: \cite{mathews_groove_1970}.}
\label{fig:groove_sistema}
\end{center}
\end{figure} 

O terceiro conceito observa a existência de uma relação entre um humano e uma máquina. Mathews descreve-o como uma \emph{engenharia humana}. Esta engenharia consistiu na observação de um tempo diferencial entre o que o(a) musicista cria e o que edita:

\traduzcitacao{
O conceito final é mais nebuloso. Desde que o GROOVE é um sistema homem-máquina, a engenharia humana do sistema foi a mais importante. Por exemplo, nós descobrimos que o controle do programa de tempo necessita ser bastante diferente para a composição do que para a edição, e o programa foi modificado de acordo. (\ldots) O intérprete de computador não deve tentar definir todo o som em tempo real. Ao invés, o computador deve ter uma partitura e o intérprete deve influenciar a forma como a partitura é tocada. Seus modos de influência podem ser mais variados do que aqueles que um regente convencional, que pode principalmente controloar o tempo, intensidade, e estilo
}
{
The final concept is more nebulous. Since GROOVE is a man-computer system, the human engeneering of the system is most important. For example, we discovered that the control of the program time needs to be quite different for composing than for editing, and the program was modiffied accordingly. (\ldots) The computer performer should not attempt to define the entire sound in real-time. Instead, the computer should have a score and the performer should influence the way in which the score is played. His modes of influence can be much more varied than that a conventional conductor who primarily controls tempo, loudness, and style.
}
{p.~715-716}{mathews_groove_1970}


Como exemplo , selecionamos uma descrição da compositora Laurie \citeonline{spiegel_expanding_1975} (ver \autoref{fig:groove}) para sumarizar as características do GROOVE, durante a produção de \emph{The Expanding Universe} \footnote{Disponível em \url{https://www.youtube.com/watch?v=dYUZmsfm4Ww}.}, entre as salas 2D-506 da Bell Labs (contendo o computador DDP-224) e a sala analógica 2D-562 (laboratório de Mathews). A ``performance'' da obra era realizada, com a programação de funções temporais e a manipulação de parâmetros dessas funções através de dispositivos físicos:

\begin{citacao}
Todas as músicas no GROOVE eram representadas na memória digital como funções abstratas do tempo, séries paralelas de dois pontos, cada ponto sendo um instante no tempo e um valor instantâneo. A taxa de amostragem para essas funções, usada principalmente como controle de voltagem, era cronometrada por um grande e antiquado oscilador analógico que era normalmente fixado em 100 Hertz, cada ciclo do oscilador pulsando à frente do código, o computador lia, em cada uma das funções, naquele ponto do tempo, todos dispositivos de entrada e executava todas amostras. (\ldots) Tínhamos uma pequena caixa com 4 potenciômetros e quatro chaves (alternadores fixados onde você os colocava) e dois botões de disparo.\footnote{Tradução de \emph{All music in GROOVE was represented in digital memory as abstract functions of time, parallel series of point pairs, each point being an instant in time and an instantaneous value. The sampling rate for these functions, which would be used mostly as control voltages, was clocked by a big old-fashioned analog oscillator that was usually set to 100 Hertz, each cycle of the oscillator pulsing one run through the code, the computer reading all of the real time input devices and playing of all of the samples at that time point in each of the time functions. (\ldots)  We had a small box with 4 knobs, 4 set switches (toggles that stay where you put them) and 2 momentary-contact push buttons on it.}}
\end{citacao}

\begin{figure}[!h]
  \begin{center}
  \includegraphics[scale=0.618]{./imagens/spiegel.jpg}
  \caption{\small Laurie Spiegel configurando a saída analógica do GROOVE, durante a produção de \emph{The Expanding Universe}. \textbf{Fonte}: \cite{spiegel_expanding_1975}.}
  \label{fig:groove}
  \end{center}
\end{figure}

Embora não declare ser uma peça minimalista, a descrição de \emph{The Expanding Universe} considera, de maneira asséptica, os fenômenos psicoacústicos como elementos composicionais. Por exemplo, a utilização da continuidade progressiva de sons (ou \emph{drones} transitórios) como elemento criativo permite, segundo a compositora, à sensibilização do ouvido, o que não seria possível na música minimalista instrumental:

\begin{citacao}
 A violência da perturbação sonora, disjunção, descontinuidade e mudanças súbitas desanitizam o ouvinte e nos afastam, de forma que não estamos mais abertos aos sons mais sutis. Mas com continuidade e gentileza, o ouvido se torna re-sensibilizado para mais e mais fenômenos auditoriais sutis dentro do som que estamos imersos. Em vez de sermos arrastados, como nas cascatas de muitas notas executadas em blocos de tempo que mudam repentinamente, tal como tantas vezes consite a música "minimalista", abrimos nossos ouvidos mais e mais para os fenômenos que nos envolvem. Isto também não é música ambiente, um termo que veio a ser usado alguns anos depois. Esta é música para atenção concentrada, uma experiência musical do através, pensando que, lógico, existe também um pano de fundo. \footnote{Tradução de\emph{The violence of sonic disruption, disjunction, discontinuity and sudden change desensitizes the listener and pushes us away so we are no longer open to the subtlest sounds. But with continuity and gentleness, the ear becomes increasingly re-sensitized to more and more subtle auditory phenomena within the sound that immerses us. Instead of being swept along, as with cascades of many running notes in suddenly-changing blocks of time, such as “minimalist” music so often consists of, we open up our ears more and more to the more minute phenomena that envelope us. This is also not “ambient music”, a term that came into use some years later. This is music for concentrated attention, a through-composed musical experience, though of course it also can be background.}}
\end{citacao}

Nesta citação podemos sumarizar um conceito para o \emph{live coding}: Música como um Processo Gradual \cfcite{reich_music_1968}. Porém, o significado de processo pode ser desenvolvido infinitamente. Isso não será realizado. O Processo para Spiegel é diverso daquele considerado no \emph{live coding}, e uma digressão desta pode afastar demais o foco do trabalho principal. Para compreesão deste termo, será necessário explorar outros aspectos correlacionados no decorrer deste trabalho.

Para finalizar esta sessão, a figura \autoref{fig:groove} sugere um conceito rotineiro para o \emph{live coder}. Esta rotina é uma atividade constante de improvisação códigos para aquisição de destreza para uma performance. \citeonline{iazzetta_musica_2009,soares_luteria_2015} lembram que esta atividade, de codificar como se construísse um instrumento musical, se caracteriza por sua conexão com a esfera composicional, nomeado \emph{luteria composicional}.

\section{SuperCollider}

O SuperCollider\disponivelem{https://supercollider.github.io/} é um ambiente de programação desenvolvido por James McCartney, lançado em 1996. Segundo \citeonline{mccartney_supercollider_1996}, 

\begin{citacao}
\traducao{SuperCollider começou como dois programas separados que eu escrevi. O primeiro foi um programa chamado \emph{Synth-O-Matic} que era um sintetizador de tempo diferido, escrito de forma semelhante à linguagem C, para Machintosh em 1990 e foi abandonado. O segundo era um objeto \emph{MAX} chamado \emph{Pyrite} que continha um interpretador para a linguagem que se extendeu e foi usado no SuperCollider. Escrever o SuperCollider envolveu integrar uma linguagem interpretada, um coletor delixo $[$administração automática da memória$]$, e uma biblioteca de funções do Pyrite com uma máquina de síntese e funções do Synth-O-Matic. Eu gostaria de agradecer ao Curtis Roads por encorajar-me a reviver o programa do Synth-O-Matic, que levou ao SuperCollider.}{SuperCollider began as two separate programs that I wrote. The first was a program called Synth-O-Matic which was a non-real-time C-like synthesis programming language for the Macintosh written in 1990 and abandoned. The second was a MAX object called Pyrite which contained the interpreter for the language which was extended and used in SuperCollider. Writing SuperCollider involved integrating the language interpreter, garbage collector and function library of Pyrite with the synthesis engine and functions of Synth-O-Matic. I'd like to thank Curtis Roads for encouraging me to revive the Synth-O-Matic program which ultimately led to SuperCollider.}
\end{citacao}

Uma outra perspectiva é oferecida como uma possibilidade lógica e enxuta de outros paradigmas de programação musical.\citeonline[p.~1]{mccartney_supercollider_1996}, descreve alguns problemas com o paradigma de programação musical desenvolvido a partir do Music N \cite{mathews_digital_1963}, como por exemplo, a uma estrutura estática inerente à concepção de objetos conectados por cabos:

\begin{citacao}
\traducao{As abstrações fornecidas pelas linguagens MUSIC N, incluindo o CSound, são abstrações de unidades geradoras, o laço de computação para a amostra de áudio, a representação de conexões entre unidades geradoras, e instanciamento e desalocação de instrumentos. Estas abstrações tornaram a escrita de algoritmos de processamento de sinais mais fácil, porquê eles abstraem um número de detalhes incômodos. Contudo, a família Music N provê poucas estruturas de controle, nenhuma estrutura de dados reais, e nenhuma função de usuário. SAOL melhora o paradigma do Music N provendo tipos de abstrações encontradas na linguagem C, tais como estruturas de controle, funções e algumas estruturas de dados. Max, que é um tipo diferente de linguagem de programação, provê um conjunto interessante de abstrações que permite muitas pessoas usá-lo sem perceberem que estão programando acima de tudo. (\ldots) A linguagem Max també é limitada em sua habilidade de tratar seus objetos como dados, o que torna uma estrutura de objetos estáticos. Evoluções posteriores do Max, como o jMax, e o Pd, fazem várias coisas para expandir as limitações de estrutura de dados do Max, mas ainda assim possuem uma estrutura estática de objeto.
}
{
The abstractions provided by the Music N languages, including Csound (www.csounds.com), are the abstraction of a unit generator, the audio sample computation loop, the representation of the connections between unit generators, and instrument instantiation and de-allocation. These abstractions make writing signal-processing algorithms easier, because they abstract a number of cumbersome details. However, the Music N family provides few control structures, no real data structures, and no user functions. SAOL (www.saol.net) improves the Music N paradigm by providing the kinds of abstractions found in the C language, such as control structures, functions, and some data structures. Max (www.cycling74.com/products/maxmsp.html), which is quite a different kind of programming language, provides an interesting set of abstractions that enable many people to use it
without realizing they are programming at all. (\ldots). The Max language is also limited in its ability to treat its own objects as data, which makes for a static object structure. Later evolutions of Max, such as jMax (www.ircam.fr/produits/logiciels/log-forum/jmax-e.html) and Pd (www.pure-data.org), do various things to expand the data structure limitations of Max but still have a generally static object structure.
}
\end{citacao}

McCartney discute adiante um modelo alternativo de notação musical, adaptado aos padrões de uma linguagem que expresse comportamentos musicais, ao invés da determinação de pontos fixos de parâmetros musicais:

\begin{citacao}
\traducao{Uma lingagem musical de computador deve prover um conjunto de abstralções que expressam idéias composicionais e de processamento de sinais da maneira mais fácil e direta possível. Os tipos de idéias que alguém pode expressar, contudo, podem ser diferentes e levar para diferentes ferramentas. Se alguém interessado em realizar uma partitura que represente uma peça musical como um artefato fixado, então o modelo de partitura/orquestra será suficiente. Motivações que levaram a projetar o SuperCollider estavam na habilidade de perceber processos sonoros que são diferentes, a cada vez que eles são tocados, para escrever peças que de alguma forma descrevem um campo de possibilidades ao invés de uma entidade fixa, e o que facilita a improvisação ao vivo por um compositor/executante.}{
{A computer music language should provide a set of abstractions that makes expressing compositional and signal processing ideas as easy and direct as possible. The kinds of ideas one wishes to express, however, can be quite different and lead to very different tools. If one is interested in realizing a score that represents a piece of music as a fixed artifact, then a traditional orchestra/score model will suffice. Motivations for the design of SuperCollider were the ability to realize sound processes that were different every time they are played, to write pieces in a way that describes a range of possibilities rather than a fixed entity, and to facilitate live improvisation by a composer/performer.
}
\end{citacao}

Um exemplo de uso pode ilustrar o discurso do McCartney. O exemplo abaixo (p.~\pageref{ex:artificial}) é um código de Fredrik Olofson, outro personagem importante para a improvisação de códigos \cite{ward_live_2004}. Este exemplo é peculiar, uma vez que expõe não somente estruturas dinâmicas e deterministas, mas possibilita criar uma conexão com dois assuntos discutidos anteriormente, algoraves \ver{sec:algorave} e microchips \ver{sec:baiadesaofransisco} .  Um sintetizador recria o timbre do videogame \emph{Atari2600} (laçado no EUA em 1977); mais especificamente é um simulador do \emph{chip} TIA (\emph{Television Interface Adapter}), responsável pela geração de gráficos e imagens no videogame Atari\disponivelem{https://www.atariage.com/2600/archives/schematics_tia/index.html}. O exemplo é relativamente simples. Um sintetizador (\verb|SynthDef|) e um sequenciador (\verb|Pbind|)são definidos. O senquenciador controla parâmetros como as tons, frequências de modulação e panoramização do sintetizador. É interessante notar que padrões fixos (\verb|Pseq| e \verb|Pn|) se misturam com padrões variváveis (\verb|Pbrown|) e segue padrões de movimentos brownianos diferentes que variam entre 28 e 31 $Hz$, alternados com 23 e 26 $Hz$. Enquanto isso a frequência moduladora segue um padrão repetitivo que alterna 10 e 16 $Hz$ com 11 e 16 $Hz$.


\begin{example}{Notação do SuperCollider}
\textbf{Fonte}: \url{http://supercollider.sourceforge.net/audiocode-examples/}

\begin{lstlisting}[style=SuperCollider-IDE]
// Simple synth definition using the Atari2600 UGen:
(
SynthDef(\atari2600, {|out= 0, gate= 1, tone0= 5,
tone1= 8, freq0= 10, freq1= 20, amp= 1, pan= 0|
  var e, z;
  e= EnvGen.kr(Env.asr(0.01, amp, 0.05), gate, doneAction:2);
  z= Atari2600.ar(tone0, tone1, freq0, freq1, 15, 15);
  Out.ar(out, Pan2.ar(z*e, pan));
}).store
)

// And a pattern to play it:
(
Pbind(
  \instrument, \atari2600,
  \dur, Pseq([0.25, 0.25, 0.25, 0.45], inf),
  \amp, 0.8,
  \tone0, Pseq([Pseq([2, 5], 32), Pseq([3, 5], 32)], inf),
  \tone1, 14,
  \freq0, Pseq([Pbrown(28, 31, 1, 32), Pbrown(23, 26, 3, 32)], inf),
  \freq1, Pseq([Pn(10, 16), Pn(11, 16)], inf)
).play
)
\end{lstlisting}
\end{example}\label{ex:artificial}

\subsection{Just In Time Library (JITLib)}\label{sec:jit}

A reflexividade \ver{sec:grossi} é uma característica de diversos ambientes de \emph{live coding}. Segundo \citeonline{aycock_brief_2003}, o primeiros programas JIT foram Genesis (com base no LISP, 1960), LC$^2$ (\emph{Language for Conversational Computing}, 1968) e APL (1970). Este último deu origem ao conceito \emph{lazy evaluation} (avaliação preguiçosa). 

O \emph{SuperCollider} foi o primeiro dos ambientes de programação musical a implementar esta caracterísitca. Com a divulgação da biblioteca JITLib\footnote{Disponível em \url{http://doc.sccode.org/Overviews/JITLib.html}.}, os primeiros Espaços Conceituais do \emph{live coding} se estruturam de maneira bastante formal na comunidade de músicos-programadores. Isto é, durante o ato de codificação podemos codificar a execução de um som antes mesmo de definí-lo (ver \autoref{cod:proxy}).

\begin{example}{Exemplo de avaliação preguiçosa no \emph{Supercollider}.}

Um tipo de variável específica começa com o caractere $~$ para indicar um ambiente propício para a avaliação preguiçosa. Mesmo antes de sua definição, podemos tocar um sintetizador:

  \begin{minted}{c}
    // play some output to the hardware busses,
    // this could be any audio rate key.
    ~out.play;
    ~out = { SinOsc.ar([400, 408] * 0.8, 0, 0.2) };
  \end{minted}

Depois que o código acima é escrito e executado, podemos escrever outros códigos para substituir o sintetizador durante sua execução (\emph{runtime}):
  
  \begin{minted}{c}
    // replacing the node. 
    // the crossfade envelope is created internally.
    ~out = { SinOsc.ar([443, 600 - Rand(0,200)], 0, 0.2) };
    ~out = { Resonz.ar(Saw.ar(40 + [0,0.2], 1), [1200, 1600], 0.1) 
           + SinOsc.ar(60 * [1,1.1],0,0.2) };
    ~out = { Pan2.ar(PinkNoise.ar(0.1), LFClipNoise.kr(2)) };
  \end{minted}
    
  \textbf{Fonte}: \url{http://doc.sccode.org/Tutorials/JITLib/proxyspace_examples.html}
\label{cod:proxy}
\end{example}

Outros \emph{softwares} e ambientes também merecem menção: \emph{ixiLang}\footnote{Disponível em \url{http://www.ixi-audio.net/ixilang/}}\emph{ChucK}\footnote{Disponível em \url{http://chuck.cs.princeton.edu/}.}, \emph{Extempore}\footnote{Disponível em \url{http://benswift.me/extempore-docs/}.}, \emph{Impromptu}\footnote{Disponível em \url{http://impromptu.moso.com.au/}}, \emph{SonicPi}\footnote{Disponível em \url{http://sonic-pi.net/}}

Esta técnica têm sido largamente implementada em navegadores de internet \cite{roberts_web_2013}, ou remotamente \cite{junior_supercopair_2015}. Isto é, entre duas pessoas distantes uma da outra, mas concetadas através da \emph{internet} ou de redes privadas.

Trabalhos neste caminho incluem o Gibber\footnote{Disponível em \url{http://gibber.mat.ucsb.edu/}. \cfcite{roberts_gibber:_2012}} e \emph{Wavepot}\footnote{Disponível em \url{https://www.wavepot.com}.}. Com base neste último, impĺementamos um ambiente chamado \emph{Termpot}\footnote{Disponível em \url{https://jahpd.github.io/termpot}. \cfcite{lunhani_termpot_2015}.}.
%\end{apendicesenv}

\phantompart
\printindex
%---------------------------------------------------------------------
\end{document}