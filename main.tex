%% abtex2-modelo-trabalho-academico.tex, v-1.9.2 laurocesar

%% Copyright 2012-2014 by abnTeX2 group at http://abntex2.googlecode.com/ 
%%
%% This work may be distributed and/or modified under the
%% conditions of the LaTeX Project Public License, either version 1.3
%% of this license or (at your option) any later version.
%% The latest version of this license is in
%%   http://www.latex-project.org/lppl.txt
%% and version 1.3 or later is part of all distributions of LaTeX
%% version 2005/12/01 or later.
%%
%% This work has the LPPL maintenance status `maintained'.
%% 
%% The Current Maintainer of this work is the abnTeX2 team, led
%% by Lauro César Araujo. Further information are available on 
%% http://abntex2.googlecode.com/
%%
%% This work consists of the files abntex2-modelo-trabalho-academico.tex,
%% abntex2-modelo-include-comandos and abntex2-modelo-references.bib
%%
%------------------------------------------------------------------------
% ------------------------------------------------------------------------ 
% abnTeX2: Modelo de Trabalho Academico (tese de doutorado, dissertacao de
% mestrado e trabalhos monograficos em geral) em conformidade com 
% ABNT NBR 14724:2011: Informacao e documentacao - Trabalhos academicos -
% Apresentacao
% ------------------------------------------------------------------------
% ------------------------------------------------------------------------

\documentclass[
	% -- opções da classe memoir --
	12pt,				% tamanho da fonte
	openright,			% capítulos começam em pág ímpar (insere página vazia caso preciso)
	twoside,			% para impressão em verso e anverso. Oposto a oneside
	a4paper,			% tamanho do papel. 
	% -- opções da classe abntex2 --
	%chapter=TITLE,		        % títulos de capítulos convertidos em letras maiúsculas
	%section=TITLE,		        % títulos de seções convertidos em letras maiúsculas
	%subsection=TITLE,	        % títulos de subseções convertidos em letras maiúsculas
	%subsubsection=TITLE,           % títulos de subsubseções convertidos em letras maiúsculas
	% -- opções do pacote babel --
	english,			% idioma adicional para hifenização
	french,				% idioma adicional para hifenização
	spanish,			% idioma adicional para hifenização
        italian,                        % idioma adicional para hifenização
	brazil				% o último idioma é o principal do documento
	]{abntex2}
%https://code.google.com/p/abntex2/wiki/Texmaker
% ---
% Pacotes
% ---
% ---
% PACOTES
% ---
\usepackage{lmodern}			% Usa a fonte Latin Modern
\usepackage[T1]{fontenc}		% Selecao de codigos de fonte.
\usepackage[utf8]{inputenc}		% Codificacao do documento (conversão automática dos acentos)
\usepackage{indentfirst}		% Indenta o primeiro parágrafo de cada seção.
\usepackage{nomencl} 			% Lista de simbolos
\usepackage{color}		               	% Controle das cores
\usepackage{fancyvrb}
\usepackage{graphicx}			% Inclusão de gráficos
\usepackage{txfonts}                	% Fontes virtuais 
\usepackage{listings} 
\usepackage{minted}


% ---
% ABNT
% ---
\usepackage[brazilian,hyperpageref]{backref}    % Paginas com as citações na bibl
\usepackage[alf]{abntex2cite}                  	% Citações padrão ABNT
\usepackage[brazil]{babel}	               	% Idioma do documento


%TODO
\usepackage[colorinlistoftodos]{todonotes}

\makeatletter
\hypersetup{
  pdftitle={\@title},
  pdfauthor={\@author},
  pdfsubject={Seminário para curso de metodologia},
  pdfkeywords={Música}{Metodologia}{Paradigma},
  %pdfcreator={\@LaTeX with \@abnTeX},
  colorlinks=true,
  linkcolor=blue,
  citecolor=blue, 
  urlcolor=blue}
\makeatother

% --- 
% CONFIGURAÇÕES DE PACOTES
% --- 

% ---
% Configurações do pacote backref
% Usado sem a opção hyperpageref de backref
\renewcommand{\backrefpagesname}{Citado na(s) página(s):~}
% Texto padrão antes do número das páginas
\renewcommand{\backref}{}
% Define os textos da citação
\renewcommand*{\backrefalt}[4]{
	\ifcase #1 %
		Nenhuma citação no texto.%
	\or
		Citado na página #2.%
	\else
		Citado #1 vezes nas páginas #2.%
	\fi}%
% ---

\usepackage{tikz}
\tikzset{
  treenode/.style = {shape=rectangle, rounded corners,
                     draw, align=center,
                     top color=white, bottom color=blue!20},
  root/.style     = {treenode, font=\Large, bottom color=red!30},
  env/.style      = {treenode, font=\ttfamily\normalsize},
  dummy/.style    = {circle,draw}
}

% ---
% Informações de dados para CAPA e FOLHA DE ROSTO
% ---
\titulo{\emph{Live Coding}: um algoritmo de sonoridade tonal em \emph{A Study in Keith} (2009) de Andrew Sorensen}
\autor{Guilherme Martins Lunhani}

\instituicao{Universidade Federal de Juiz De Fora -- UFJF
  \par
  Instituto de Artes e Design -- IAD
  \par
  Programa de Pós-Graduação em Artes Visuais, Música e Tecnologia}

\orientador[Orientador: ]{Prof. Dr. Luiz Eduardo Castelões}

% \changes{Versão inicial }{2013/07/22 }{v0.0.3}
\tipotrabalho{Dissertação (Mestrado)}

% O preambulo deve conter o tipo do trabalho, o objetivo, 
% o nome da instituição e a área de concentração 
\preambulo{Dissertação corrigida segundo orientações da banca de defesa no Programa de Mestrado em Artes, Cultura e Linguagens do Instituto de Artes e Design da Universidade Federal de Juiz de Fora (UFJF), linha de Artes Visuais, Musica e Tecnologia.}
%\EnableCrossrefs
%\CodelineIndex
%\RecordChanges

% ---
% Configurações de aparência do PDF final

% alterando o aspecto da cor azul
\definecolor{blue}{RGB}{41,5,195}

% informações do PDF
\makeatletter
\hypersetup{
     	%pagebackref=true,
		pdftitle={\@title}, 
		pdfauthor={\@author},
    	pdfsubject={\imprimirpreambulo},
	    pdfcreator={LaTeX with abnTeX2},
		pdfkeywords={abnt}{latex}{abntex}{abntex2}{trabalho acadêmico}, 
		colorlinks=true,       		% false: boxed links; true: colored links
    	linkcolor=blue,          	% color of internal links
    	citecolor=blue,        		% color of links to bibliography
    	filecolor=magenta,      		% color of file links
		urlcolor=blue,
		bookmarksdepth=4
}
\makeatother

%\newcommand{\todosautoresdelivecoding}{\begin{inparaenum}[]\item \citeonline{collins_live_2003},\item \citeonline{collins_generative_2003},\item \citeonline{collins_live_2003-1},\item \citeonline{wang_--fly_2004},\item \citeonline{ward_live_2004},\item \citeonline{blackwell_programming_2005},\item \citeonline{collins_live_2007},\item \citeonline{griffiths_fluxus:_2008},\item \citeonline{mclean_patterns_2009},\item \citeonline{rohrhuber_improvising_2009},\item \citeonline{mclean_visualisation_2010},\item \citeonline{magnusson_algorithms_2011},\item \citeonline{mccallum_show_2011},\item \citeonline{magnusson_herding_2014},\item \citeonline{magnusson_scoring_2014},\item \citeonline{collins_algorave:_2014},\item \citeonline{sorensen_livecodings_2014}\end{inparaenum}}
% --- 
% Espaçamentos entre linhas e parágrafos 
% --- 

% O tamanho do parágrafo é dado por:
\setlength{\parindent}{1.3cm}

% Controle do espaçamento entre um parágrafo e outro:
\setlength{\parskip}{0.2cm}  % tente também \onelineskip

% ---
% compila o indice
% ---
\makeindex
\makeindex
% ----
% Início do documento
% ----
\begin{document}
% Retira espaço extra obsoleto entre as frases.
\frenchspacing 

% ----------------------------------------------------------
% ELEMENTOS PRÉ-TEXTUAIS
% ----------------------------------------------------------
\pretextual

% ---
% Capa
% ---
\imprimircapa
% ---

% ---
% Folha de rosto
% (o * indica que haverá a ficha bibliográfica)
% ---
\imprimirfolhaderosto*
% ---

% ---
% Inserir a ficha bibliografica
% ---
%\input{./ficha_catalografica}

% ---
% Inserir errata
% ---
%\begin{errata}
%\end{errata}

% \includepdf{folhadeaprovacao_final.pdf}
%\input{./folha_aprovacao}

% ---
% Dedicatória
'% ---
%\begin{dedicatoria}
%   \vspace*{\fill}
%   \centering
%   \noindent
%   \textit{ Este trabalho é dedicado às crianças adultas que,\\
%   quando pequenas, sonharam em se tornar cientistas.} \vspace*{\fill}
%\end{dedicatoria}
% ---
% Agradecimentos
% ---
%\begin{agradecimentos}
%A você...\footnote{...principalmente pela atenção até nas notas de rodapé.}
%\end{agradecimentos}
% ---

% ---
% Epígrafe do livecoding
% ---
\epigraph{\emph{Criei todas as festas, todos os triunfos, todos os dramas. Experimentei inventar novas flores, novos astros, novas carnes, novas línguas. Acreditei adquirir poderes sobrenaturais. Ora bem! eis que devo enterrar minha imaginação e minhas lembranças!  Que bela glória de artista e narrador arrebatada!}}{Arthur Rimbaud }
%Alguém poderia imiaginar uma interface musical na qual um músico especifica o som resultante desejado em uma linguagem descritiva na qual poderia ser então traduzido em parâmetros de partículas e renderizados em som. Uma alternativa poderia especificar um exemplo: "Faça um , but with less vibrato" (Curtis Roads, 2001)\footnote{Tradução de: \emph{One can imagine a musical interface in which a musician specifies the desired sonic result in a musically descriptive language which would then be translated into particle parameters and rendered into sound. An alternative would be to specify an example: "Make me a sound like this (soundfile), but with less vibrato"}}
\newpage

% ---
% RESUMOS
% ---
% resumo em português
\setlength{\absparsep}{18pt} % ajusta o espaçamento dos parágrafos do resumo
\begin{resumo}
Esta pesquisa foi construída a partir da seguinte pergunta: ao discutir uma prática conhecida como \textit{live coding}, quais categorizações sonoras (ou gêneros musicais) são contextualizadas? 

Contudo, a variedade dos diferentes exemplos da pergunta forçou a redução do \emph{universo de conceitos} inerente às categorizações sonoras estudadas. Neste sentido, o texto deste trabalho busca partir de um suposto espaço conceitual generalizado (\autoref{cap:introducao}); um método de pesquisa que contemple as multiplicidades deste espaço conceitual (\autoref{cap:metodologia}); uma contextualização histórica, anterior à concepção deste espaço conceitual (\autoref{cap:trabalhos_relacionados}); apresentação de alguns casos do compositor Australiano Andrew Sorensen (\autoref{cap:estudos_de_caso}).

Estes casos foram escolhidos por representarem, em um mesma pessoa, uma plaralidade de práticas musicais. Não é nossa intenção a estéticas musical, mais sim alguns meios técnicos pelos quais a variedade delas emerge. Mais especificamente, com uma mesma técnica de improvisação (ou \emph{live coding}), Sorensen improvisa desde \emph{Jazz}, \emph{Minimalismo} de um Steve Reich, \emph{Músicas-populares massivas}, e sínteses sonoras. Dois destes casos são de interesse por envolver uma busca pelo controle do Piano MIDI e o Piano Acústico. O terceiro é interessante por lembrar uma mistura das estéticas da \emph{Computer Music} dos anos setenta e 80, e a música de dança para ambientes sociais noturnos, algo que denominado na cena européia por \emph{algorave}. Por outro lado, os três são paradigmáticos por envolver replicação de estilos.

\textbf{Palavras-chaves}: \textit{Live coding}. Categorizações Sonoras. Sorensen
\end{resumo}

%%%%%%%%%% traduçoes resumo
\begin{comment}
% resumo em inglês
\begin{resumo}[Abstract]
 \begin{otherlanguage*}{english}
   This is the english abstract.

   \vspace{\onelineskip}
 
   \noindent 
   \textbf{Key-words}: latex. abntex. text editoration.
 \end{otherlanguage*}
\end{resumo}

% resumo em francês 
\begin{resumo}[Résumé]
 \begi'n{otherlanguage*}{french}
    Il s'agit d'un résumé en français.
 
   \textbf{Mots-clés}: latex. abntex. publication de textes.
 \end{otherlanguage*}
\end{resumo}

% resumo em espanhol
\begin{resumo}[Resumen]
 \begin{otherlanguage*}{spanish}
   Este es el resumen en español.
  
   \textbf{Palabras clave}: latex. abntex. publicación de textos.
 \end{otherlanguage*}
\end{resumo}
% ---
\end{comment}
% ---
% inserir lista de ilustrações
% ---
\pdfbookmark[0]{\listfigurename}{lof}
%\listoffigures*
%\cleardoublepage
% ---
% inserir lista de tabelas
% ---
%\pdfbookmark[0]{\listtablename}{lot}
%\listoftables*
%\cleardoublepage
% ---
% inserir lista de abreviaturas e siglas
% ---
%\begin{siglas}\label{siglas}
  \item[AMC] \textit{Composição Musical Automática/Algorítmica}
  \item[CAC] \textit{Composição Assistida por Computador}
  \item[CGA] \textit{Assistência Gerada por Computador}
  \item[CGC] \textit{Composição Gerada por Computador}
  \item[CGM] \textit{Composição Musical Generativa}
  \item[CSG] \textit{Sons Gerados por Computador}
  \item[DJ] \textit{code DJing}
  \item[MA] \textit{Música Algorítmica}
  \item[MG] \textit{Música Generativa}
  \item[MP] \textit{Música Processual}
\end{siglas}\label{siglas}
% ---
% inserir o sumario
% ---
\pdfbookmark[0]{\contentsname}{toc}
\tableofcontents*
\cleardoublepage
% ----------------------------------------------------------
% ELEMENTOS TEXTUAIS
% ----------------------------------------------------------
\textual

% ----------------------------------------------------------
% Introdução (exemplo de capítulo sem numeração, mas presente no Sumário)
% ----------------------------------------------------------
\chapter*{Introdução}\addcontentsline{toc}{chapter}{Introdução} 

%``O marquês de X tem, como se sabe, um belo gabinete de física, mas a Eletricidade é sua paixão e, se o paganismo ainda vigorasse, ele decerto ergueria altares elétricos. Ele sabia quais são minhas preferências e não ignorava que também sou fã da \emph{Eletromania}. Convidou-me, portanto, para um jantar onde estariam presentes, segundo ele, os medalhões da ordem dos eletrizantes e das eletrizadas''. Conviria conhecer essa eletricidade falada que, sem dúvida, revelaria muito mais sobre a psicologia da época do que sobre sua ciência\footnote{Bachelard, G. \textit{A formação do espírito científico}. p.~41. Ed. Contraponto. Trad. Estela dos Santos de Abreu 1938 -- 1996}.

Em seu artigo ``A filosofia à venda, a douta ignorância e a aposta de pascal'', \citeonline[p.~11--12]{santos_filosofia_2008} elabora a imagem mental de uma feira do conhecimento, onde teorias  são antropomorfizadas, escravizadas e vendidas:\ ``determinismo, livre arbítrio,universalismo, relativismo, realismo, construtivismo, marxismo, liberalismo, neoliberalismo, estruturalismo, pós-estruturalismo, modernismo, pós-modernismo, colonialismo, pós-colonialismo, etc.''. As idéias perderam a utilidade para os ex-adeptos, que não estão mais interessados em comprá-las. E vendem aos que supõe algum valor. Para efetuar a venda, é necessário estabelecer uma relação de custo-benefício, negociadas através de respostas às perguntas: ``qual a utilidade que esta ou aquela teoria poderá ter para mim? Qual o preço?''. A valorização ocorre quando esta teoria se torna mais apelativa que aquela. Com a concorrência, a livre-associação dos vendedores regulamentará compras e vendas de conhecimentos conforme seu interesse mais fundamental: se todas teorias forem vendidas não existirá teoria para se vender amanhã (o que não exclúi monopolizações). 

É possível pensar que Santos realiza uma  metáfora de um Mercado contemporâneo do conhecimento. Mas Santos esclarece que este tema é anterior à formação do espírito científico moderno: no texto satírico \emph{A venda de filosofias} (165), Luciano de Samósata (125 -- 181?),  escreve sobre um mercado estimulado por Zeus e gerenciado por Hermes:

\begin{citacao}
Hermes atrai os potenciais compradores, todos comerciantes, gritando alto e bom som “À venda! Uma variedade sortida de filosofias vivas! Posições de todo o tipo! Pagamento à vista ou mediante garantia!” (1905: 190). A “mercadoria” vai sendo exposta, os comerciantes vão chegando e têm o direito de interrogar cada uma das filosofias à venda, começando invariavelmente com a pergunta pela utilidade para o comprador e a sua família ou grupo. O preço é estabelecido por Zeus que, por vezes, se limita a aceitar ofertas feitas pelos comerciantes compradores. A venda tem pleno êxito e Hermes termina, ordenando às teorias que deixem de oferecer resistência e sigam com os seus compradores, ao mesmo tempo que avisa o público: “Senhores, esperamos vê-los amanhã. Estaremos oferecendo novos lotes úteis para homens comuns, artistas e comerciantes”   
\end{citacao}

Recolhendo esta imagem mental do Mercado de conhecimentos escravizados, sugerimos espelhar a metáfora para o assunto específico deste documento. As filosofias vendidas estão em uma feira chamada \emph{live coding}, que traduzimos por  improvisação de códigos. Ali vendem o tonalismo, o pós-tonalismo, o \emph{jazz}, a música algorítmica, o minimalismo, \emph{live computer music}, música ambiental, música \emph{rave},  música-ruído. Além disso são vendidas teorias da tecelagem, do audiovisual, da dança, e do lado científico, as ciências históricas e ciências cognitivas. Compramos uma amostra, cujo exemplares foram divididos em três grupos (ver Objetivos, p.~\pageref{sec:objetivos}).


\section*{Conceito de Pensamento Ortopédico}

Alex McLean defende que a metáfora é de importância central para a linguagem, e o faz através da \emph{teoria da Metáfora Conceitual}\apud[p.~32]{lakoff_methaphors_1980}{McLean2011}. Utilizamos a metáfora de Santos para exemplificar o tipo de pensamento analítico empregado por improvisadores-programadores, quando estes se posicionam em um ambiente acadêmico. Para explicar o sentido de uma metáfora, McLean sugere utilizar texto entre aspas, seguido de outro em caixa alta. 

Santos utiliza três metáforas. A primeira metáfora é o \metafora{pensamento ortopédico}{o mesmo processo especialista utilizado pelo médico responsável em corrigir deformidades do corpo}\disponivelem{http://www.priberam.pt/dlpo/ortopedia}. A segunda é a \metafora{a razão indolente}{insensibilidade com respeito às consequências do processo de correção}: ``A carência a respeito da finitude transforma-se num problema técnico-científico, enquanto a carência a respeito da diversidade infinita é ignorada como um não-problema.'' \cite[p.~15]{santos_filosofia_2008}. O terceiro é \metafora{o pensamento abissal}{a percepção de uma distância que delimita conhecimentos} \cite[p.~1--4]{santos_abissal_2007}:

\begin{citacao}
Consiste num sistema de distinções visíveis e invisíveis, sendo que as invisíveis fundamentam as visíveis. As distinções invisíveis são estabelecidas através de linhas radicais que dividem a realidade social em dois universos distintos: o universo  'deste lado da linha' e o universo 'do outro lado da linha'. A divisão é tal que 'o outro lado da linha' desaparece enquanto realidade, torna-se inexistente, e é mesmo produzido como inexistente. (\ldots) \textbf{O pensamento abissal moderno salienta-se pela sua capacidade de produzir e radicalizar distinções.} Contudo, por mais radicais que sejam estas distinções e por mais dramáticas que possam ser as consequências de estar de um ou do outro dos lados destas distinções, elas têm em comum o facto de pertencerem a este lado da linha e de se combinarem para tornar invisível a linha abissal na qual estão fundadas.\footnote{Grifo nosso.} 
\end{citacao}

\section*{Objetivos}\label{sec:objetivos}

• Investigar um Universo de Conceitos sobre a improvisação de códigos (\emph{live coding});

• Investigar um método de análise/criação, ortopédico, para uma improvisação de códigos;

• Investigar um Espaço Conceitual de uma sonoridade de um algoritmo musical de uma improvisação de códigos;

\section*{Estrutura dos Capítulos}

No \autoref{cap:introducao} selecionamos três abordagens, escolhidas por manterem alguma conexão com a improvisação de códigos no contexto musical.  No \autoref{cap:metodologia} apresentamos um modelo de formalização da criatividade, do ponto de vista do Modelo de Improvisação discutido por Alex \citeonline{McLean2011}. No \autoref{cap:estudos_de_caso}, organizamos conceitos para analisar o contexto de uma sonoridade  de \emph{A Study in Keith} (2009) de Andrew Sorensen.  O \autoref{app:A} foi adicionado para expor o material que estimulou o interesse pelo tema discutido. O \autoref{app:B} sugere a inclusão de um trabalho de \citeonline{mathews_groove_1970} no âmbito proto-histórico da improvisação de códigos.

%------------------------------------------------------
% PARTE
% ----------------------------------------------------------
%\part{Ecologia de saberes no \emph{livecoding}}\label{parte1}

%\begingroup
%\let\clearpage\relax
\chapter{Pergunta para o método: \emph{live coding} é música?}\label{cap:introducao}
1Giovanni \citeonline{mori_analysing_2015} perspetivou a importância da Música para os \emph{live coders} de um ponto-de-vista etnongráfico, ao apontar locais e costumes onde o \emph{live coding} é praticado:

\begin{citacao}
\emph{Live coding} é uma técnica artística de improvisação. Pode ser empregada em muitos contextos performativos diferentes: dança, música, imagens em movimento e mesmo tecelagem. Concentrei minha atenção no lado musical, que parece ser o mais proeminente. \cite[p.~117]{mori_analysing_2015}\footnote{Tradução de \emph{Live coding is an improvisatory artistic technique. It can be employed in many different performative contexts: dance, music, moving images and even weaving. I have concentrated my attention on the music side, which seems to be the most prominent.}}
\end{citacao}

O fragmento acima sugere que a Música, usada sozinha ou com outra linguagem (artes do corpo e audiovisual), articula os afetos entre intérpretes e público.

Separando a primeira personagem,  intérpretes (\emph{live coders}) idealizam uma comunidade participativa \cite[p.~71]{prospero_social_2015}, através de diversas apresentações, publicações de artigos, e registros em mídias audio/visuais. A permissividade pode ser um carro-chefe que, por sua vez, possibilita  hibridizações entre os locais e os modos de produção.

\section{O universo de conceitos durante uma improvisação}\label{sec:universo}

Uma sessão de \emph{live coding} é uma improvisação. \citeonline{mclean_music_2006}, um dos principais pesquisadores ingleses, no campo musical, articula a improvisação musical como algo computável, organizado por representações lógicas. 

O \emph{conceito} se torna uma instância, uma variável. É representado pela letra $c$. O  \emph{universo de conceitos} se torna um conjunto, representado pela letra $U$, como um agrupamento de instâncias de $c$. Por outro lado, $c$ é definido por conjuntos de ``objetos'' ($O'$), características ($F'$), e processos ($P'$). 

O primeiro contem a premissa para uma improvisação. O segundo é um conjunto de várias premissas, no sentido de uma rede de conceitos que definem o conceito principal. Os três últimos são classes de informações que definem como o conceito deve ser tocado para caracterizar uma linguagem musical específica.

Nesse sentido, diferentes performances de \emph{live coding} possuem diferentes configurações destas classes lógicas. Os objetos, características e processos de uma improvisação de \emph{Jazz} são diferentes daqueles de uma improvisação de \emph{Música eletroacústica} ou de uma improvisação de \emph{Músicas-populares massivas}\footnote{Sobre Músicas-populares massivas e regras de gêneros musicais, \cfcite{sa_se_2009}.}

Esta contraposição, específica entre o \emph{Jazz} e \emph{Músicas-populares massivas}, será trabalhado no \emph{cap:estudos_de_caso}, ao descrever \emph{Study in Keith} (2009)\footnote{Disponível em \url{https://vimeo.com/2433947}.}, \emph{Day Of The Triffords} (2009)\footnote{Disponível em \url{https://www.youtube.com/watch?v=DGcE8P5F29A}.}.

\section{A questão da tradução}\label{sec:traducao}

Giovanni Mori denomina a prática com dois termos separados, isto é, \emph{live coding}. Tomando esta separação, o prefixo \emph{live} é taduzido como ``ao vivo'', e o sufixo \emph{coding} como ``codificar''. Uma performance cujo ato é improvisar leis ou fórmulas dispersas.
-
Porém a tradução falha em identificar questões-satélites, como por exemplo, a própria Música. Para identificar algumas palavras-chaves, uma compilação dos dos anais do ICLC 2015 \cite{ICLC2015} pode ser esclarecedor.

A \autoref{fig:nuvemlivecoding} é uma \emph{nuvem de palavras} \footnote{Disponível em \url{https://github.com/amueller/word_cloud}.}, uma representação visual de dados textuais. Longe de ser uma análise linguística, utilizei esta ferramenta para me auto-induzir a encontrar as questões-satélites do \emph{live coding}.

\begin{figure}[!h]
\begin{center}
\centering
\includegraphics[scale=0.71]{./imagens/livecoding_cloud1.png}
\caption{Nuvem de palavras dos anais ICLC2015 \textbf{Fonte}: autor.}
\label{fig:nuvemlivecoding}
\end{center}
\end{figure}

\begin{table}
\caption{Tabela de classes qualitativas de termos utilizados nos anais do ICLC2015, agrupados por funções textuais.}
\small
    \begin{tabular}{ p{1.6cm} | p{1.4cm} | p{2cm} | p{1.45cm} | p{1.45cm} | p{1.45cm} | p{1.45cm} | p{1.45cm} | p{1.2cm}}
    \hline 
    \hline 

    \tiny \textbf{Número Qualitativo/Função} & \textbf{0} & \textbf{1}  & \textbf{2} & \textbf{3} & \textbf{4}  & \textbf{5} & \textbf{8} & \textbf{9}\\
    \hline 
    \hline 

    \tiny \textbf{Pessoas}  
    & - 
    & \tiny Collins, Blackwell, McLean, Grossi 
    & - 
    & - 
    & - 
    & -  
    & - 
    & - \\
    \hline

    \tiny \textbf{Aplicativos}
    & - 
    & \tiny SuperCollider, Gibber, SonicPi  
    & - 
    & - 
    & - 
    & -  
    & - 
    & \\
    \hline
    
    \tiny \textbf{Verbos}  
    & \tiny take, see, shared, networked, explore, made
    & \tiny make, provide, writing, solving, making
    & \tiny used
    & \tiny using, coding  
    & \tiny performer
    & - 
    & - 
    & -  \\
  \hline

     \tiny \textbf{Adjetivo ou numeral, ordinal}  
    & \tiny less, open, potential, similar, important, cognitive, virtual
    & \tiny first, real, electronic, visual, ensemble, possible, free, livecoding, aspect  
    & \tiny musical, many
    & \tiny new, one
    & - 
    & -  
    & \tiny live 
    & - \\
    \hline

    \tiny \textbf{Substantivo}  
    & \tiny Browser, point, approach, order, node, collaborative, number, source, present, community, server, framework, orchestra, digital, level, kind, type, memory, analysis, line, body, concept, technology, working, org, current, show, mean, end, processes, people, international
    & \tiny University, conference, proceedings, network, interface, environment, text, form, context, musician, space, paper, program, audience, function, change, control, human, laptop, interaction, structure, part, session, tool, result, create, object, case, algorithm, value, development, material, set, technique, parameter, idea, screen, video, application, support, composition, piece, knowledge, feature, cell, activity, art, action, information, method, web, rule, group, need, particular, project, allow, collaboration, programmer, member, play, output 
    & \tiny use, coder, process, state, example, way, software, research, problem, experience, design, improvisation, different, machine, pattern, audio
    & \tiny work, instrument
    & \tiny system, computer, user, language, time, practice, sound
    & \tiny programming
    & \tiny performance, code
    & \tiny ``live coding'', music  \\
    \hline
    \hline
   
    \end{tabular}
\label{tab:comparacao}
\end{table}

Uma breve análise da nuvem de palavras pode elucidar parte das questões-satélites. Na \autoref{tab:comparacao} filtrei parte dos resultados na nuvem de palavras por conjuntos de funções textuais -- sujeitos-humanos, sujeitos-ferramentas, verbos, adjetivos e substantivos -- e quantas vezes foram utilizados, em categorias qualitativas (0, menos usado e 9 o mais usado, sendo que 6 e 7 não apresentaram resultados). \footnote{O método de extração será explicado em um apêndice oportuno.}. 

No caso dos sujeitos-humanos, podemos ver nomes de Nick Collins e Alex McLean, praticantes responsáveis pela criação de um manifesto, cujo um fragmento será discutido no capítulo 2. Pietro Grossi, é um personagem recentemente estudado por \citeonline{mori_pietro_2015} como um caso prematuro de \emph{live coding}, a partir do final da década de sessenta.

No caso dos sujeitos-ferramentas, destacamos o papel do \emph{SuperCollider}, já citado anteiormente, e do \emph{Gibber}\footnote{Disponível em \url{http://gibber.mat.ucsb.edu}}. Ambos são ambiente de programação para de síntese sonora e composição algorítmica. Uma semelhança paradigmática para estes ambientes, é o procedimento de compilação de códigos conhecido como \emph{Just In Time} \cite{aycock_brief_2003}. Enquanto no primeiro \emph{software} a questão está posta em uma máquina -- \emph{laptop} -- local, o \emph{Gibber} representa a viabilidade do navegador de \emph{internet} como plataforma musical \citeonline{roberts_gibber:_2012,wyse_viability_2014}.

Os verbos fornecem informação sobre o comportamento dos improvisadores de códigos. Além da atividades como \emph{performatizar} e \emph{codificar}, é notável atividades sociais ligadas à visão, à escrita, à técnica, à lógica. Embora a Música seja a atividade proeminente do \emph{live coding}, não obtivemos resultados que retornassem, por exemplo, a palavra \emph{hearing}. Isso é significativo, e no \autoref{cap:trabalhos_relacionados}, exploro estas ações sob a ótica da Música de Processos de Steve Reich.

Já os adjetivos destacam características da prática, onde \emph{live} é a palavra-chave. Como será observado no \autoref{cap:trabalhos_relacionados}, a ação pode ocorrer em uma sala de concerto, um espaço público ou em uma casa noturna. Palavras como  \emph{electronic}, podem sugerir tanto uma música ``eletroacústica'', quanto gêneros de música para dançar. \emph{Visual} remete a uma característica tão fundamental quanto a Música. Um sem número de performances utilizam a projeção de telas de computadores como dispositivo de ``transparência''; isto é, uma ideologia de justificação do ato de improvisação. \emph{Ensemble} destaca uma a natureza de grupos. Poucas performances \emph{solo} são realizadas se comparadas às performances de \emph{duos}, \emph{trios}. 

Substantivos relacionam a atividade como processo (\emph{process}). Como será discutido no Capítulo 2, o \emph{live coding} se apropria de conceitos da Música como Processo de Steve Reich e da Música Generativa de Brian Eno para justificar um processo de codificação incessante. Por outro lado, palavras como \emph{university}, \emph{research} e \emph{technology}, e \emph{laptop} acusam não apenas uma prática artística, mas um programa de pesquisa tipo de performance, que utiliza um computador para resultados audiovisuais, realizados por universitários. A esfera de pesquisa acadêmica permitiu ramos de desenvolvimento com linguagens de programação, cognição, inteligência artificial, semiologia, performance musical (improvisação), e mais recentemente, etnologia, conferindo à produção de \emph{live coding} uma aura de legitimidade escolar.
%Os dados abaixo são parciais de um levantamento de músicas no \textit{Soundcloud} utilizando palavras-chave\footnote{No sentido de um parâmetro de procura em \emph{Query-strings}; para mais informações, ver \url{https://en.wikipedia.org/wiki/Query_string}} relativas à produção da prática discutida neste trabalho organizadas em categorias de informação.

\emph{Palavras-chave} são os termos que apresentaram maior consistência de dados (considerados pelo ponto de vista abstrato do autor do trabalho): \begin{inparaenum}[\itshape a)\upshape]
\item \emph{algorave}/\emph{algopop}: parte considerável da produção do \emph{livecoding} realizada em ambientes noturnos, informais ou de entretenimento (possue relação com o elemento dança);
\item \emph{livecoding}/\emph{live-coding}: apresentaram dados coerentes com a bibliografia pesquisada\footnote{A exclusão do termo \emph{live code/live coding} foi feita pois a separação criava uma ambiguidade de busca no \emph{Soundcloud}, isto é, \emph{live} poderia não se referir ao que pesquisamos por \emph{livecoding}.};
\item \emph{bytebeat}: parte considerável de uma técnica de programação musical descrita pela primeira vez por \citeonline{heikkila_discovering_2011} e aplicada no \emph{livecoding}, isto é, apenas um fragmento dessa prodção pode se encaixar como \emph{livecoding} (um desses programas é o \emph{Wavepot});
\item \emph{algorithmic music}: utilizei o termo para apontar o quanto da música algorítmica publicada no \emph{Soundcloud} poderia ser classificado como \emph{livecoding}, em conjunto com outros termos que não foram abordados nesta pesquisa;
\item \emph{wavepot}: performances feitas com o aplicativo \emph{Wavepot}, são por definição \emph{livecoding};
\end{inparaenum}

Categorias de informação são:\begin{inparaenum}[\itshape a)\upshape]
\item ano de publicação (\emph{created\_at});
\item países (\emph{country});
\item cidade (\emph{city});
\item gênero musical(\emph{genre});e 
\item licensa de uso (\emph{license})
\end{inparaenum}

\section{Plotagem em forma de torta}

Os dados foram levantados entre janeiro e fevereiro de 2015; não realizamos desde então levantamentos por utilizarmos um outro sistema de visualização. Estão disponíveis na \autoref{apend:torta}

\section{Plotagem em forma de círculos empacotados}\label{dados_pacotao}

Os dados utilizados são os mesmos. A diferença é o tratamento dos dados: correlação de dados por duas categorias diferentes. Nos exemplos abaixo, salientei comparações de ano e gênero (\autoref{pacotao} e \autoref{pacotao2}) e país e gênero (\autoref{pacotao3} e \autoref{pacotao4}).

\begin{figure}
\begin{center}
\includegraphics[scale=0.6]{./imagens/zoomable_circle_packing_genre_year_livecoding.png}
\caption{Empacotamento de informações a respeito de gênero musical a partir de anos de produção}
\label{pacotao}
\end{center}
\end{figure}

\begin{figure}
\begin{center}
\includegraphics[scale=0.6]{./imagens/zoomable_circle_packing_genre_year_livecoding2.png}
\caption{Detalhamento de informações a respeito de gênero musical a partir de anos de produção}
\label{pacotao2}
\end{center}
\end{figure}

\begin{figure}
\begin{center}
\includegraphics[scale=0.6]{./imagens/zoomable_circle_packing_genre_year_livecoding3.png}
\caption{Empacotamento de informações a respeito de gênero musical a partir de países onde ocorreram as produções}
\label{pacotao3}
\end{center}
\end{figure}

\begin{figure}
\begin{center}
\includegraphics[scale=0.6]{./imagens/zoomable_circle_packing_genre_year_livecoding4.png}
\caption{Detalhamento de informações a respeito de gênero musical a partir de países onde ocorreram as produções.}
\label{pacotao4}
\end{center}
\end{figure}





\chapter{Metodologia}\label{cap:metodologia}
\chapter{Metodologia de análise de uma Improvisação musical de códigos}\label{cap:metodologia}

Este capítulo contextualiza a \emph{criatividade} dentro do pensamento proposto por um improvisador programador \cite{McLean2011}. Apresentaremos uma definição de criatividade \ver{sec:criatividade} para contextualizar o \traducao{Quadro Conceitual de Sistemas Criativos}{Creative System Frameworks, \emph{ou CSF}.} \ver{sec:csf}. Em seguida especificamos este quadro para os propósitos específicos desta pesquisa \ver{sec:diagrama} e formalizamos o último capítulo \ver{sec:formaliza}.

\section{Criatividade}\label{sec:criatividade}

Para analisar uma improvisação de códigos, nos termos de um processo criativo com um instrumento musical (computador), definimos \emph{criatividadade} a partir de uma proposição do livro ``The Creative Mind: myths and mechanisms'' de Margaret \citeonline{boden_creative_1990},  \ver{sec:diferencas}. Em seguida discutimos a formação de \emph{imagens mentais}, durante um processo criativo idealizado \ver{sec:imagem_mental}, afim de discutir um mecanismo teórico para realização destas imagens mentais em performances de improvisadores-programadores \ver{sec:tidal}. 

\subsection{O paradoxo da criatividade}\label{sec:diferencas}

É possível discutir quais valores a palavra \emph{criatividade} carrega. Se por um lado faltam-nos documentos que comprovem qual é sua definição comum, sugerimos que uma definição cíclica (``a criatividade cria'') é uma maneira de pensar que, socialmente, a avaliação de um comportamento criativo está baseada em sua capacidade de \emph{gerar novidades}. Para \citeonline[p.~2]{thornton_quantitative_2007}, Boden discute que esta concepção de criatividade cria um paradoxo quando colocada sob a perspectiva da lógica mecanicista do computador:

\begin{citacao}
\traducao{O ponto inicial de Boden para o desenvolvimento de sua sua explicação, é a observação de que o conceito de criatividade contém um paradoxo. Por definição, criatividade cria, i.e., produz alguma coisa nova. Mas se estamos comprometidos com uma abordagem mecanicista do mundo -- nenhum milagre é permitido -- iremos acreditar que tudo o que ocorre é, em princípio, previsível. Iremos acreditar também que qualquer coisa nova deve ser construída de componentes existentes. Isso implica que nada pode ser intrinsicamente novo.}{Boden’s starting point for the development of her account is the observation that the concept of creativity contains a paradox. By definition, creativity creates, i.e., it produces something new. But if we are committed to a mechanistic account of the world — no miracles allowed — we believe that everything that occurs is predictable in principle. We also believe that any new thing must be constructed from existing components. This implies that nothing can ever be intrinsically new.}
\end{citacao}


 De outro ponto de vista, o mecanismo de julgamento do que é ou não é novidade parte de uma identificação do que é possível e do que não é, de forma que a impossibilidade é valorizada: \traducao{Para justificar a classificação de uma idéia criativa... alguém deve identificar os princípios generativos com respeito ao que é impossível}{To justify calling an idea creative... one must identify the generative principles with respect to which it is impossible.} \apud[p.~40;p.~3]{boden_creative_1990}{thornton_quantitative_2007}. Essa capacidade gerativa do aparentemente impossível pode estar conectada com uma inversão desta idéia comum de criatividade. Para Boden, \traducao{Qualquer ato criativo é fundado na conceitualização ou realização de um ponto dentro de um espaço conceitual particular (\emph{idem}, \emph{ibidem})}{Any creative act is thus founded on conceptualisation or the realisation of a point within a particular ‘conceptual space’.}. 

Para Thornton e \citeonline[p.~450--451]{wiggins_framework_2006}, na primeira edição de seu livro, Boden não explica como é o método de acesso aos espaços conceituais. Por outro lado, Boden oferece uma taxonomia de tipos de criativadade. 

As nomenclaturas derivadas são explicadas como uma divisão de duas classificações da criatividade \ver{fig:ortogonal}. A primeira classificação divide a criatividade em criatividade-psicológica, ou criatividade-pessoal (\emph{P-creativity}), e criatividade-histórica (\emph{H-creativity}). A segunda classificação separa criatividade em criatividade-exploradora e criatividade-transformacional.

\begin{figure}
\centering
\begin{tikzpicture}[scale=2.5]
\tikzstyle{every node}=[draw,shape=circle];
\path (0:0cm) node (v0) {\tiny $Criatividade$};
\path (0:1.25cm) node (v1) {\tiny $Criatividade-H$};
\path (2*90:1.25cm) node (v2) {\tiny $Criatividade-P$};
\path (3*90:1.25cm) node (v3) {\tiny $Criatividade Exp.$};
\path (90:1.25cm) node (v4) {\tiny $Criatividade Trans.$};
\draw (v0) -- (v1)
(v0) -- (v2)
(v0) -- (v3)
(v0) -- (v4);
\end{tikzpicture}
\caption{Classificação da criatividade : 1) criatividade-psicológica/criatividade-histórica; 2) critividade exploradora/criatividade transformacional. \textbf{Fonte}: autor com base em \citeonline{wiggins_framework_2006}.}
\label{fig:ortogonal}
\end{figure}

\begin{citacao}
\traducao{A distinção é entre o sentido de criar um conceito que nunca foi criado antes $[$criatividade-P$]$, e um conceito que nunca foi criado antes por um criador específico $[$criatividade-H$]$. Esta distinção será tangencialmente relevante para meu argumento aqui, mas antes de prosseguir, eu noto que esta não é uma simples escolha binária, mas ao invés, uma contextualização multidimensional: pode ser possível, por exemplo, para um comportamento criativo ser P-criativo em uma sociedade, mas H-criativo em outra; deste ponto de vista da segunda sociedade, apenas importam comportamentos H-criativos. (\ldots) no trabalho de Boden, existe uma distinção entre criatividade exploradora e transformacional, que será relevante aqui, e então merece alguma explicação. Boden concebe o processo de criatividade como uma identificação e/ou localização de novos objetos conceituais em um espaço conceitual.}{The distinction is between the sense of creating a concept which has never been created before at all, and a concept which has never been created before by a specific creator. This distinction will be only tangentially relevant to my argument here, but before proceeding, I note that this is not a simple binary choice, but rather multi-dimensional, context-based one: it would be possible, for example, for a creative behaviour to be only P-creative in one society, but H-creative in another; from the point of view of the second society, only the H-creativity matters. (\ldots) in Boden’s work, there is the distinction between exploratory and transformational creativity, which is directly relevant here, and so needs a little more explanation. \textbf{Boden conceives the process of creativity as the identification and/or location of new conceptual objects in a conceptual space}.}  
\end{citacao}

Para Wiggins, Boden chama de \emph{comportamento criativo explorador} a exploração de possibilidades completas ou parciais de um conceito. Já o \emph{comportamento criativo transformacional} pressupõe regras governadoras deste espaço conceitual em exploração, e busca transformar tais regras através de métodos. Para Boden,  são socialmente valorizados os conceitos-H e os processos transformacionais, enquanto o comportamento explorador e a criatividade-P são consideradas como característicos de crianças mais novas, ou pensamentos personalistas. Wiggins dá maior importância para a classificação exploradora/transformacional, e não desvaloriza o comportamento explorador em relação à transformação..  \citeonline[p.~3--4]{thornton_quantitative_2007} pontua que Boden admite que em algumas situações, uma criativade-exploradora não é menos criativa que uma criativade-transformadora. O comportamento explorador, como uma \emph{exploração guiada}, é muito útil em atividades onde se requer \traducao{(\ldots) a utilização de heurísticas e mapas para identificar conceitos valiosos dentre de um espaço conceitual existente}{the use of heuristics and maps to identify valuable concepts within an existing conceptual space} \ver{sec:showusyourscreens}. Neste sentido, os comportamentos criativos \emph{apenas transformadores} desenvolvem novos espaços conceituais que serão úteis para o comportamento explorador guiado. 

\begin{citacao}
\traducao{De fato, na primeira edição $[$\citeonline{boden_creative_1990}$]$, ela não oferece uma explicação do número de diferentes tipos de criatividade que ela identificou. Parece que sua intenção era distinguir os dois tipos notados, espaços conceituais devem ter uma característica generativa. E isso certamente é a interpretação comum. Ainda no sumário 'em uma casca de noz' de sua teoria, foi adicionado um prólogo à segunda edição (Boden, 2003), e em (Boden, 1998), ela coloca que sua explicação distingue três principais formas de criatividade, sendo exploração, transformação e \emph{combinação}. É somada à incerteza a observação que somente a definição forte $[$generalizadora$]$ da definição possue o poder de resolver o paradoxo da criatividade}{In fact, in the first edition, she offers no final count of the number of different types of creativity she has identified. It seems to be her intention to distinguish the two types noted, conceptual space must be generative in character. and this is certainly a common interpretation. Yet in the ‘nutshell’ summary of her theory, added as a prologue to the second edition (Boden, 2003), and in (Boden, 1998), she states that her account distinguishes three main forms of creativity, these being exploration, transformation and combination. Adding to the uncertainty is the observation that only the strong definition has the power to resolve the creativity paradox, arguably forcing us to recognise not two forms of creativity, or three, but one: transformation.}
\end{citacao}

Neste ponto \citeonline[p.~451]{wiggins_framework_2006} apresenta uma definição cíclica e generalizadora de criatividade, \traducao{A performance de tarefas que, quando executados por um humano, são consideradas criativas}{The performance of tasks which, if performed by a human, would be deemed creative}, e subdivide em quatro definições auxiliares: 

\begin{table}[!h]
\caption{Definições formais de criatividade por \citeonline[p.~451]{wiggins_framework_2006}}
\small
    \begin{tabular}{ | p{4cm} | p{11.25cm} |}
    \hline 
    \hline 

    \tiny{Criatividade} 
    & \tiny{``O estudo e suporte, através de meios e métodos computacionais, do comportamento exibido por sistemas naturais e artificiais, que podem ser considerados criativos se exibidos em humanos.''  \tablefootnote{Tradução de \emph{‘The study and support, through computational means and methods, of behaviour exhibited by natural and artificial systems, which would be deemed creative if exhibited by humans’’.}.}} \\
    \hline

    \tiny{Computação criativa} 
    & \tiny{``O estudo e suporte, através de meios e métodos computacionais, do comportamento exibido por sistemas naturais e artificiais, que são considerados criativos''. \tablefootnote{Tradução de \emph{The study and support, through computational means and methods, of behaviour exhibited by natural and artificial systems, which would be deemed creative if exhibited by humans.}.}} \\
    \hline

    \tiny{Sistemas criativos} 
    & \tiny{``Uma coleção de processos, naturais ou automáticos, que são capazes de alcançarem ou simularem comportamentos que em humanos seriam considerados criativos''} \\
    \hline

    \tiny{Comportamento Criativo} 
    & \tiny{``Um ou mais dos comportamentos exibidos por um sistema criativo''\tablefootnote{Tradução de \emph{One or more of the behaviours exhibited by a creative system.}}} \\
    \hline
    \hline
   
    \end{tabular}
\label{tab:criatividade}
\end{table}

\subsection{Criatividade, códigos e imagens mentais}\label{sec:imagem_mental}

Para \citeonline[p.~24--25]{McLean2011}, um comportamento criativo pode ser descrito como o processo de elaboração de uma \emph{imagem mental}. Mais especificamente, a imagem mental é um símbolo, e seu significado é governado por regras gramaticais e sociais (códigos). Por outro lado, McLean estabelece o conceito de \emph{imagem mental} como uma hierarquia de códigos simbólicos, visuais e gramaticais, como descrita pela teoria da Codificação Dual \apud[p.~25--29]{paivio_dual_1990}{McLean2011}:


\begin{citacao}
\traducao{Seu $[$Paivio$]$ argumento não é que existem dois códigos, mas sim que existe uma hierarquia de códigos, que se ramificam no topo em códigos lingüísticos discretos e códigos de percepção contínua, que Paivio nomeia como \emph {logogens} e \emph{imagens} respectivamente. (\ldots) \textbf{A explicação oferecida pela teoria da Codificação Dual é que existem sistemas de símbolos distindos, mas integrados, para linguagens e figuras}.}{His contention is not that there are two codes, but rather that there is a hierarchy of code, which branch at the top into discrete linguistic codes and contionuous perceptual codes, which Paivio names \emph{logogens} and \emph{imagens} respectively (\ldots) The explanation offered by Dual Coding Theory is that there are distinct, yet integrated symbol systems for imagery and language.}
\end{citacao}

Por exemplo, quando falamos em \emph{improvisação de códigos}, podemos imaginar uma pessoa escrevendo um texto em um computador. Dependendo do grau de intimidade com os símbolos aprendidos, pode ser que esta imagem seja mais ou menos específica, e , outras possibilidades de imagens mentais surgem. A própria atividade de descrição de uma imagem mental é uma \emph{estratégia} de conversão. No caso do improvisador-programador, que converte sua imagem mental em em código de computador, chamaremos de \emph{estratégia transversal}.

Neste processo, o improvisador-programador recorre aos padrões e estilos de escrita de uma linguagem. Entre padrões e estilos, são utilizados comentários, que explicam textualmente a imagem e seu resultado; da disposição espacial do código, ou  nomes de variáveis e funções que sugerem imagens mentais específicas. No caso de uma linguagem de programação de propósito geral, algumas regras definidas pela comunidade desenvolvedora da linguagem devem ser seguidas. No entanto, já vimos que o improvisador-programador é estimulado a recorrer às linguagens artificiais, ou linguagens de domínio específico (DSL) -- ``O programa será transcendido - Língua Artificial é o caminho'' \ver{sec:showusyourscreens}. Isto é, é parte da prática do improvisador-programador criar linguagens específicas para uma classe de imagens mentais específicas. 

Um exemplo de exploração de um processo transformacional será ilustrado a partir de uma descrição de \citeonline[p.~119]{McLean2011},  com o método da \emph{prática reflexiva} de um artista-plástico (no caso, o pintor Paul Klee). Supondo a discretização do processo transformacional, o artista cria uma imagem mental do que irá fazer, um esboço que delimita um campo de atividade (espaço conceitual), a atividade de pintura, e o resultado percebido.  Na fase de implementação do esboço, o pintor percebe que aquilo que foi considerado adequado no esboço, é inadequado para a situação prática. Neste momento, o artista reage ao resultado e re-elabora novas estratégias de conversão entre a imagem mental e o resultado. O processo continua até que o ofício seja considerado completo (\emph{obra}). Este não é o mesmo processo utilizado pelo(a) improvisador(a)-programador(a). O material, e o objetivo são diversos do(a) artista plástico(a). Seu material é o texto, mas não o texto discursivo, e sim o texto que descreve uma rotina de tarefas computacionais. Seu objetivo não é chegar a uma obra finalizada, mas sim não descontinuar o processo. McLean chama esse método de transformacional de programação por bricolagem \ver{fig:processo_criativo}. Em ambos os casos, na prática reflexiva e na bricolagem, são necessários conceitos que suportem a intenção (e vontade) do ofício. Conceitos elaborados através de restrições de idéias podem ser sistematizados, com o auxílio de esquemas visuais. Esta abordagem é muito semelhante à posição defendida por McLean, onde conceitos podem ser representados geometricamente. No caso do(a) improvisador(a)-programador(a), \emph{espaços conceituais} são reelaborados indefinidamente, começando por uma \emph{estratégia transversal}, passando para a observação, reação, e reformulação da imagem mental.

 \begin{figure}[h]
  \centering
  \includegraphics[scale=0.5]{imagens/processo_criativo.png}
  \caption{Modelo de bricolagem para o processo criativo realizado por um artista-programador. \textbf{Fonte}: \citeonline[p.~122]{McLean2011}. }
  \label{fig:processo_criativo}
\end{figure}

\begin{citacao}
\traducao{A Figura 6.2 $[$\autoref{fig:processo_criativo}$]$ caracteriza a programação por bricolagem como um laço retroalimentado envolvendo o algoritmo escrito, sua interpretação, e a percepção do programador e sua reação do resultado ou comportamento $[$do algoritmo$]$. (\ldots). No começo o programador tem um conceito meio-formado que só atinge consistência interna através do processo de ser expresso como um algoritmo. O laço interno é onde o programador elabora o objetivo de suas imaginações, e o laço externo é onde essa trajetória está fundamentada na pragmática do que elas realmente têm que fazer. Através deste processo ambos algoritmos e conceitos são desenvolvidos até que o programador sinta que um se aplica com o outro, ou de outra forma julga o processo criativo finalizado.
}{
Figure 6.2 characterises bricolage programming as a creative feedback loop encompassing the written algorithm, its interpretation, and the programmer’s perception and reaction to its output or behaviour. (\ldots). At the beginning, the programmer may have a half-formed concept, which only reaches internal consistency through the process of being expressed as an algorithm. The inner loop is where the programmer elaborates upon their imagination of what might be, and the outer where this trajectory is grounded in the pragmatics of what they have actually made. Through this process both algorithm and concept are developed, until the programmer feels they accord with one another, or otherwise judges the creative process to be finished.
}  
\end{citacao}

 Para McLean, DSLs ``provêm termos padronizados para descrever demandas particulares em um domínio de uma tarefa''. Isto é, uma linguagem customizada para atividades artísticas pode ser planejada, ou analizada, através de um \emph{Quadro de estruturação das Dimensões Cognitivas da Notação} \apud[p.~95--97]{church_cognitive_2008}{McLean2011}. Estas não são dimensões autônomas, o que diverge do foco de nosso trabalho. Uma situação comum na improvisação de códigos, é maleabilidade de escrita de um algoritmo. No caso, McLean define como a interdependência entre viscosidade e notação secundária: a possibilidade de diferentes soluções para o mesmo resultado. Por exemplo, em um \emph{patch} de PD, cuja imagem mental é uma onda quadrada. Podemos utilizar objetos nativos ou objetos extendidos \ver{fig:pd}. Uma outra interdependência de dimensões, entre as Operações Mentais Difíceis e a não-Invisibilidade de Dependências Escondidas, é notória do ponto de vista pedagógico. Podem afastar o compositor de seu objetivo musical, mais ou menos estruturado em uma teoria. DSLs como PureData, Max/MSP, CSound, SuperCollider, Tidal, praticam a invisibilidade de dependências em diferentes graus. Porém, entre os improvisadores de código, é considerado virtuosismo o praticante recorrer às Operações Mentais difíceis com o mínimo possível de Invisibilidade de Dependências. Por exemplo, utilizar linguagens de baixo nível como C \ver{sec:concerto}, ou de alto nível como Perl \cite{mclean_hacking_2004} e ainda assim, elaborar sons e imagens de maneira criativa. \ver{tab:dimensoes}:

\begin{table}[!h]
\caption{Dimensões cognitivas da Notação para linguagens de programação. \textbf{Fonte}: \apud{church_cognitive_2008}{McLean2011}.}
\small
    \begin{tabular}{ | p{7cm}| p{7cm} |}
    \hline 
    \hline 

    \tiny \textbf{Dimensão} & \textbf{Significado} \\
    \hline 
    \hline 

    \tiny \textbf{Abstração}  
    & \tiny \tabletraducao{Disponibilidade de mecanismos de abstração}{Avaliability of abstraction mechanisms} \\
    \hline

    \tiny \textbf{Dependências escondidas}

    & \tiny \tabletraducao{Invisibilidade de ligações importantes entre entidades.}{Invisibility of important links between entities.}\\
    \hline
    
    \tiny \textbf{Compromisso prematuro}  
    & \tiny \tabletraducao{Restrição na ordem de execução das coisas.}{Constraints on the order of doing things.} \\
  \hline

    \tiny \textbf{Notação secundária}  
    & \tiny \tabletraducao{Notação diversa da sintaxe formal.}{Notation other than formal syntax.} \\
    \hline

    \tiny \textbf{Viscosidade}  
    & \tiny \tabletraducao{Resistência à mudança.}{Resistance to change.} \\
    \hline

    \tiny \textbf{Proximidade de mapeamento}  
    & \tiny \tabletraducao{Proximidade de representação para o domínio-alvo.}{Closeness of representation to target domain.} \\
    \hline

    \tiny \textbf{Consistência}  
    & \tiny \tabletraducao{Semânticas similares são expressadas em formas sintáticas similares.}{Similar semantics are expressed in similar syntatic forms} \\
    \hline

    \tiny \textbf{Dispersividade}  
    & \tiny \tabletraducao{Prolixidade da linguagem.}{Verbosity of language.} \\
    \hline

    \tiny \textbf{Tendência ao erro}  
    & \tiny \tabletraducao{Probabilidade de erros.}{Likelihood of mistakes.} \\
    \hline

    \tiny \textbf{Operações mentais difíceis}  
    & \tiny \tabletraducao{Demanda de recursos cognitivos.}{Demand on cognitive resources.} \\
    \hline

    \tiny \textbf{Provisoriedade}  
    & \tiny \tabletraducao{Grau de compromisso com ações e marcos.}{Degree of commitment to actions or marks.} \\
    \hline
    
    \tiny \textbf{Função de expressividade}  
    & \tiny \tabletraducao{medida em que o efeito de um componente pode ser inferida.}{Extent to which the purpose of a component may be inferred.} \\
    \hline
    \hline
   
    \end{tabular}
\label{tab:dimensoes}
\end{table} 

\begin{figure}[!h]
  \centering
  \includegraphics[scale=0.7]{imagens/pd.png}
  \caption{Exemplo de uma caracteristica de viscosidade e notação secundária no PureData. \textbf{Fonte}: autor. }
  \label{fig:pd}
\end{figure}

 
\subsection{Comportamento Criativo: Bricolagem como estratégia transversal}\label{sec:tidal}

Vamos ilustrar um pequeno processo da \emph{estratégia transversal} com o \emph{software} Tidal.
Segundo \citeonline[p.~2]{mclean_tidal_2010}, \emph{Tidal} é uma linguagem de composição generativa, onde \traducao{padrões podem ser compostos de numerosos subpadrões em uma variedade de maneiras e para uma profundidade arbitrária, para produzir $[$partes$]$ inteiras complexas de partes simples}{patterns may be composed of numerous subpatterns in a variety of ways and to arbitrary depth, to produce complex wholes from simple parts}. Amostras sonoras representam imagens mentais de suas fontes (por exemplo ``sn'' para \emph{snare}, caixa-clara), com ritmos organizados com o auxílio de símbolos delimitadores de tempo (como espaço, `` ``, e colchetes, ``$[$'',``$]$'', \verb|{| e \verb|}|). Ritmos podem ser revertidos (\verb|rev|), diminuidos e aumentados (\verb|slow|, \verb|density|), iterados (\verb|every|) para recombinação permutação, padrões mais complexos (\verb|can|), como o algoritmode bjorklund que simula ritmos tradicionais\footnote{\cfcite{toussaint_euclidean_2005}}. Efeitos de panoramização, atraso (\emph{delay}), filtros e comunicação de rede. No Exemplo \ref{ex:tidal}. a imagem mental é a demanda da linguagem, que é produzir Música Eletrônica para Dançar.

\begin{citacao}
\traducao{Tidal é uma linguagem de padrões embebida em uma linguagem de programação Haskell, consistindo de representação de padrão, uma biblioteca de padrões geradores e combinadores, um $[$mecanismo$]$ de agendamento de eventos e uma interface para programar ao vivo. Esta é uma extensiva re-escrita de um trabalho anterior introduzido sobre o título \emph{Petrol} $[$\citeonline{mclean_petrol_2010}$]$. Extensões incluem melhoramentos de representação de padrão e um uma integração totalmente configurável do protocolo Open Sound Control $[$\citeonline{osc}$]$ \cite{mclean_tidal_2010}
}
{Tidal is a pattern language embedded in the Haskell programminglanguage, consisting of pattern representation, a library of pattern generators and combinators, an event scheduler and programmer’s live coding interface. This is an extensive re-write of earlier work introduced under the working title of Petrol [15]. Extensions include improved pattern representation and fully configurable integration with the Open Sound Control (OSC) protocol [16]
}
\end{citacao}

\begin{example}{Exemplo de Estratégia Transversal}\label{ex:tidal}

Imagem mental: um \emph{loop} sincopado, mas bastante regular, descrito em um compasso. Em uma ``partitura-mental'', estruturamos o primeiro tempo com um baixo, que volta a tocar na segunda semicolcheia do terceiro tempo. No Segundo tempo, silêncio. No quarto tempo uma caixa aberta:

{%
\parindent 0pt
\noindent
\ifx\preLilyPondExample \undefined
\else
  \expandafter\preLilyPondExample
\fi
\def\lilypondbook{}%
\includegraphics{53/lily-86c766d2-1}%
% eof

\ifx\postLilyPondExample \undefined
\else
  \expandafter\postLilyPondExample
\fi
}

O padrão acima pode ser elaborado em uma voz (\verb|d1|), que redireciona (\$) a função que toca amostras sonoras (\verb|sound|). Esta função lê uma corrente de caracteres (\textbf{string}) separados por um espaço em branco. Espaços em branco são delimitadores temporais. Cada subdivisão temporal é representada por delimitadores como $[$ e $]$. 

\begin{minted}{haskell}
-- Eletronic Dance Music, BPM = 120 
-- tempo 1 - baixo            (bass)
-- tempo 2 - silencio         (silence)
-- tempo 3 - silencio + baixo
-- tempo 4 - caixa            (sn e sn:4)
d1 \$ (sound "bass3 silence [silence bass3] sn:4")
\end{minted}

Sonoramente, é útil para começar. Mas uma Música Eletrônica para Dançar requer mais elementos. Seguiremos com mais dois passos. Podemos complementar os ritmos com uma caixa e um baixos mais secos no segundo e terceiro tempo.

\input{./tidal2}

\begin{minted}{haskell}
-- Eletronic Dance Music, BPM = 120 
-- tempo 1 - baixo            (bass)
-- tempo 2 - silencio         (silence)
-- tempo 3 - silencio + house
-- tempo 4 - caixa            (sn e sn:4)
d1 \$ (sound "bass3 sn [silence house] sn:4")
\end{minted}

É possível também fazer com que este padrão reduza seu tempo pela metade a cada quatro tempos, :

{%
\parindent 0pt
\noindent
\ifx\preLilyPondExample \undefined
\else
  \expandafter\preLilyPondExample
\fi
\def\lilypondbook{}%
\input{2b/lily-de0a8ae3-systems.tex}
\ifx\postLilyPondExample \undefined
\else
  \expandafter\postLilyPondExample
\fi
}

\begin{minted}{haskell}
-- Eletronic Dance Music, BPM = 120
-- com uma caixa seca no segundo tempo
-- e uma caixa aberta no quarto tempo
-- A cada 4 tempos, o ritmo diminui pela metade 
-- e depois volta ao normal.
d1 \$ every 4 (density 0.5) (sound "bass3 sn [silence house] sn:4")
\end{minted}
\end{example}

Para \citeonline[p.~130]{McLean2011}, esta estratégia criativa, de programar ``no momento'', a partir de um arquivo de texto em branco, com uma imagem mental do resultado sonoro (ou visual), é caracterizada pela  bricolagem. No início do exemplo acima, o programador elabora um meio-conceito do que quer fazer, cuja expressão apenas ganha existência através da codificação \ver{fig:processo_criativo}. As fases de observação, e reação levam o improvisador programador à reconceitualização, e um novo código é escrito. No entanto, ao invés de finalizar, o improvisador segue desenvolvendo.

\section{Quadro Conceitual de sistemas criativos}\label{sec:csf}

Uma maneira adequada de descrever um sistema criativo (ou parte dele) considera um \emph{Universo de Conceitos}:

\begin{citacao}
O universo, $\mathcal{U}$, é um espaço multidimensional, no qual dimensões são capazes de representar qualquer coisa, e todos os possíveis conceitos distintos correspondentes àqueles pontos em $\mathcal{U}$ (\ldots) Para tornar a proposta um espaço-tipo possível, permitirei que $\mathcal{U}$ contenha todos os conceitos abstratos, bem como os concretos, e que é possível representar os artefatos tanto completos e incompletos \cite[p.~451]{wiggins_framework_2006}.\footnote{Tradução de \emph{The universe, U, is a multidimensional space, whose dimensions are capable of representing anything, and all possible distinct concepts correspond with distinct points in U. (\ldots) To make the proposal as state-spacelike as possible, I allow that U contains all abstract concepts as well as all concrete ones, and that it is therefore possible to represent both complete and incomplete artefacts}}
\end{citacao}

Wiggins esclarece que Boden não reconhece de forma explícita $\mathcal{U}$, ``ela borra a distinção entre as regras que determinam a adesão do espaço (\ldots) e outras disposições que possam permitir a construção e/ou detecção de um conceito representado por um ponto no espaço'' (\emph{Idem, ibdem}). Espaços conceituais $\mathcal{C}$, finitos ou infinitos são definidos como restrições de um universo $\mathcal{U}$, caracterizando um conjunto não-determinístico de conhecimentos, representações, e conceitos:

\begin{citacao}
\traducao{A noção-chave na teoria de Boden é aquele do espaço conceitual. Enquanto nenhuma definição formal é provida, é comum interpretar esta frase literalmente, tomando o espaço conceitual sendo um espaço de conceitualizações, ou representações de conceitos \cite[p~.7]{thornton_quantitative_2007}.}{The key notion in Boden’s theory is that of the conceptual space. While no formal definition has been provided, it is common to interpret the phrase literally, taking the conceptual space to be a space of conceptualisations or concept representations.}
\end{citacao}

\citeonline[p.~452]{wiggins_framework_2006} considera  \traducao{(\ldots) um universo, $\mathcal{U}$, um espaço multidimensional, cujas dimensões são capazes de representar qualquer coisa, e todos possíveis conceitos distintos correspondentes com distintos pontos em $\mathcal{U}$.}{The universe, U, is a multidimensional space,whose dimensions are capable of representing anything, and all possible distinct concepts correspond with distinct points in U}. Uma incomensurabilidade é evitada através de uma restrição por meio de quatro axiomas. O primeiro axioma (Universalidade) estabelece que o universo $\mathcal{U}$ pode conter tantos conceitos bem definidos (completos), parcialmente definidos (incompletos), e  o mais incompleto dos conceitos. Os dois primeiros são representados pela letra $c$, enquanto o último é representado por \small{T}. O segundo axioma (Não-identidade dos conceitos), estabelece que dois conceitos em  $\mathcal{U}$ são mutuamente diferentes entre si ($c_1 \neq c_2$), e não são um Universo. Esta correção restringe a recursividade de conceitos, o que daria ênfase ao comportamento explorador e anularia a importância do comportamento transformacional. O terceiro axioma (Inclusão Universal 1) define que \emph{espaços conceituais} $\mathcal{C}$, que contêm instâncias de conceitos $c$, são subconjuntos não-estritos do conunto $\mathcal{U}$. O quarto axioma (Inclusão Universal 2) estabelece que espaços conceituais $\mathcal{C}$ também contem o conceito \small{T}.






 Wiggins define o Universo de Conceitos (\csf{U}{x}) como um conjunto não estrito dos \emph{Espaços Conceituais} (\csf{C}{x}) de Margaret \citeonline{boden_creative_1990}. Isto é, um Universo de Conceitos a respeito de alguma coisa, no nosso caso da improvisação de códigos \ver{eq:ul}. 

\begin{equation}
\mathcal{U}_\emph{livecoding} = [\mathcal{C}_\emph{Tecelagem}, \mathcal{C}_\emph{Audiovisual}, \mathcal{C}_\emph{Dança} \mathcal{C}_\emph{Música}, \ldots, ?]
\end{equation}\label{eq:ul}

\begin{table}[!h]
\caption{Definições formais do Universo de possibilidades de \citeonline{wiggins_framework_2006}, ou Universo de Conceitos por \citeonline{mclean_music_2006}.}
\small
    \begin{tabular}{ | p{4.25cm} | p{5.25cm} | p{5.25cm} |}
    \hline 
    \hline 

    Representação
    & \tiny{Nome}     
    & \tiny{Significado} \\
    \hline

    $c$
    & \tiny{Conceito} 
    & \tiny{Uma instância de um conceito, abstrato ou concreto \cite{wiggins_framework_2006}}. \\
    \hline

    $\mathcal{U}$
    & \tiny{Universo de Conceitos} 
    & \tiny{Superconjunto não restrito de conceitos. \cite{wiggins_framework_2006}. ``Um universo de todos conceitos possíveis'' \cite{mclean_music_2006} \tablefootnote{Tradução de \emph{A universe of all possible concepts}.}}\\
    \hline

    $\mathcal{L}$
    & \tiny{Linguagem} 
    & \tiny{Linguagem utilizada para expressar regras.} \\
    \hline

    $\mathcal{A}$
    & \tiny{Alfabeto} 
    & \tiny{Alfabeto da linguagen que contêm caracteres apropriadospara expressão das regras} \\
    \hline

    $\mathcal{R}$
    & \tiny{Regras de validação} 
    & \tiny{Validam os conceitos em um universo, se apropriados ou não para o espaço trabalhado.} \\
    \hline

    $[[.]]$
    & \tiny{Função de interpretação} 
    & \tiny{``Uma função parcial de $\mathcal{L}$ para funções que resultam em números reais entre [0, 1] (\ldots) 0.5 $[$ou maior$]$ significa uma verdade booleana e menos que 0.5 siginifica uma falsidade booleana; a necessidade disso para valores reais se tornará clara abaixo'' \cite[p.~452]{wiggins_framework_2006}\tablefootnote{Tradução de \emph{(\ldots) a partial function from $\mathcal{L}$ to functions yielding real numbers in [0, 1]. (\ldots) 0.5 to mean Boolean true and less than 0.5 to mean Boolean false; the need for the real values will become clear below}.}}\\
    \hline

     $[[\mathcal{R}]]$
    & \tiny{Regras de validação} 
    & \tiny{``Uma função que interpreta $\mathcal{R}$, resultando em uma função indicando aderência ao conceito em $\mathcal{R}$''\tablefootnote{Tradução de \emph{A function interpreting $\mathcal{R}$, resulting in a function indicating adherence of a concept to $\mathcal{R}$}}} \\
    \hline

     $\mathcal{C} = [[\mathcal{R}]](\mathcal{U}) $
    & \tiny{Espaço Conceitual} 
    & \tiny{``Todos espaços conceituais são um subconjunto não-estrito de $\mathcal{U}$''\tablefootnote{Tradução de \emph{All conceptual spaces are non-strict subset}.}. Um subconjunto contido em $\mathcal{U}$ \cite{wiggins_framework_2006}. Uma função que interpreta $\mathcal{R}$, resultando em uma função que indica aderência ao conceito em $\mathcal{R}$ \tablefootnote{Tradução de \emph{A function interpreting $\mathcal{R}$, resulting in a function indicating adherence of a concept to $\mathcal{R}$}.} } \\
    \hline

    $\mathcal{T}$
    & \tiny{Regras de detecção} 
    & \tiny{``Regras definidas dentro de $\mathcal{L}$ para definir estratégias transversais para localizar conceitos dentro de $\mathcal{U}$'' \cite{mclean_music_2006}\tablefootnote{Tradução de \emph{Rules defined within $\mathcal{L}$ to define a traversal strategy to locate concepts within $\mathcal{U}$ }}} \\
    \hline

    $\mathcal{E}$
    & \tiny{Regras de qualidade} 
    & \tiny{``(\ldots) conjunto de regras que permitem-nos avaliar qualquer conceito que nós encontramos em $\mathcal{C}$ e determinar sua qualidade, de acordo com critérios que nós considerarmos apropriados'' \cite[p.453]{wiggins_framework_2006}\tablefootnote{Tradução de \emph{(\ldots) set of rules which allows us to evaluate any concept we find in C and determine its quality, according to whatever criteria we may consider appropriate.}}``Regras definidas dentro de $\mathcal{L}$ para avaliar a qualidade ou a desejabilidade do conceito $c$'' \cite{mclean_music_2006}\tablefootnote{Tradução de \emph{Rules defined within $\mathcal{L}$ which evaluate the quality or desirability of a concept $c$.}}}\\
    \hline

    $<<<\mathcal{R}, \mathcal{T}, \mathcal{E}>>>$
    & \tiny{Função de interpretação} 
    & \tiny{Uma regra necessária para definir o espaço conceitual, ``independentemente da ordem, mas também, ficcionalmente, enumerá-los em uma ordem particular, sob o controle de $\mathcal{T}$ -- isto é cricial para a simulação de um comportamento criativo de um $\mathcal{T}$ particular \cite{wiggins_framework_2006} \tablefootnote{Tradução de \emph{We need a means not just of defining the conceptual space, irrespective of order, but also, at least notionally, of enumerating it, in a particular order, under the control of $\mathcal{T}$ -- this is crucial to the simulation of a particular creative behaviour by a particular $\mathcal{T}$.}}. ``Uma função que interpreta a estratégia transversal $\mathcal{T}$, informada por $\mathcal{R}$ e $\mathcal{E}$ . Opera sobre um subconjunto ordenado de $mathcal{U}$ (do qual tem acesso randômico) e resulta em outro subconjunto ordenado de $\mathcal{U}$.''\tablefootnote{Tradução de \emph{A function interpreting the traversal strategy $\mathcal{T}$, informed by $\mathcal{R}$ and $\mathcal{E}$ . It operates upon anordered subset of $mathcal{U}$ (of which it has random access) and results in another ordered subset of $\mathcal{U}$.}}} \\
    \hline
    \hline
   
    \end{tabular}
\label{tab:universodeconceitos}
\end{table}

\citeonline{mclean_music_2006} ainda descreve regras que validam concepções diferentes entre espaços conceituais $\mathcal{C}$ diversos em um Universo de Conceitos $\mathcal{U}$ (ver \autoref{tab:universodeconceitos}). McLean realiza uma comparação entre o \emph{Universo de possibilidades} de Wiggins com o \emph{Modelo de Improvisação} de Pressing \ver{sec:im}. No entanto, McLean argumenta que:

\begin{citacao}
Pressing discute comportamento criativo no contexto do Modelo de Improvisação, e de fato é parte do Quadro conceitual de Sistemas Criativos. (\ldots) Durante a transferência de notação do Modelo de Improvisação para a Ferramenta de Sistemas Criativos, nós consideramos improvisação musical de uma maneira clara e temos uma linguagem comum na qual comparar com outros modelos \footnote{Tradução de \emph{However Pressing does discuss creative behaviour in the context of the IM, and indeed the CSF is in part. (\ldots) In transferring the IM to the notation of the CSF we may consider music improvisation in a clearer manner and have a common language in which to compare it with other models.}}.
\end{citacao}


\subsection{O modelo de improvisação}\label{sec:im}

Segundo Pressing, o Modelo de Improvisação é ``um esboço para uma teoria geral da improvisação integrada com preceitos da Psicologia Cognitiva (\ldots) teoria do comportamento de improvisação na música'' \cite[p.~2]{pressing_improvisation_1987}. Este modelo será utilizado para especificar elementos de uma performance exemplar, como o caso investigado neste trabalho. Por exemplo, uma improvisação particionada em diferentes sequências pode ser parcialmente mapeada em categorias, como blocos sonoros, referentes conceituais e normas estilísticas, conjuntos de objetivos e processos. Este nos pareceu um modelo mais transparente para o compositor, músico e intérprete. O que não quer dizer que é possível readequar ambos para nosso interesse. Um sumário sobre o modelo de improvisação é apresentado na \autoref{tab:modelo_improvisacao}. Por seu caráter lógico, parece ser uma possibilidade interessante, e assumiremos como tal.

\begin{table}[!h]
\caption{Definições formais do Modelo de improvisação de Jeff \citeonline{pressing_improvisation_1987}, segundo \citeonline[p.~2]{mclean_music_2006}.}
\small
    \begin{tabular}{ | p{6cm} | p{9cm} |}
    \hline 
    \hline 

    \tiny{Representação}   
    & \tiny{Significado} \\
    \hline

    $E'$
    & \tiny{Um bloco de eventos sonoros}\tablefootnote{\emph{A cluster of sound events}.} \\
    \hline

    $K'$
    & \tiny{Uma seqüência de blocos de eventos E, onde um bloco de eventos não se sobrepõe com o seguinte}\tablefootnote{A sequence of E event clusters, where event cluster onsets do not overlap with those of a following one}\\
    \hline

    $I'$
    & \tiny{Uma improvisação, particionada por interrupções em um número de K sequências}\tablefootnote{An improvisation, partitioned by interrupts into a number of K sequences} \\
    \hline

    $R'$
    & \tiny{Um referente opcional, tal como uma partitura ou uma norma estilística}\tablefootnote{An optional referent, such as a score or stylistic norm} \\
    \hline

    $G'$
    & \tiny{Um conjunto de objetivos }\tablefootnote{A set of current goals.} \\
    \hline

    $M'$
    & \tiny{Uma memória de longo prazo}\tablefootnote{Long term memory.} \\
    \hline

    $O'$
    & \tiny{Um conjunto de objetos}\tablefootnote{An array of objects.} \\
    \hline

    $F'$
    & \tiny{Um conjunto de características dos objetos}\tablefootnote{An array of objects Features.} \\
    \hline

    $P'$
    & \tiny{Um conjunto de processos}\tablefootnote{An array of Process} \\
    \hline
    \hline
   
    \end{tabular}
\label{tab:modelo_improvisacao}
\end{table}

\begin{figure}[!h]
  \centering
  \includegraphics[scale=0.7]{imagens/contido.png}
  \caption{Representação da justaposição  entre dois epaços conceituais. A região em marrom representa um grupo de conceitos transitórios, bem como os limites desta transição. \textbf{Fonte}: autor. }
  \label{fig:contido}
\end{figure}

\section{Diagramação dos espaços conceituais}\label{sec:diagrama}

\newcommand{\csfeq}[2]{
\mathcal{#1}_\emph{#2}
}

\newcommand{\unionspaces}[6]{
\csfeq{#1}{#2} = \csfeq{#3}{#4} \bigcup \csfeq{#5}{#6}
}

\newcommand{\listspaces}[9]{
\csfeq{#1}{#2}~=~[\csfeq{#3}{#2},~\csfeq{#4}{#2},~\csfeq{#5}{#2},~\csfeq{#6}{#2},~\csfeq{#7}{#2},~\csfeq{#8}{#2},~\csfeq{#9}{#2}
}

Formalmente, a figura acima pode ser representada como na \autoref{eq:def} , se desconsiderarmos qualquer outros espaços conceituais.

\begin{example}{Representação formal da \autoref{fig:contido}}
\begin{equation}
\unionspaces{C}{Study in Keith}{C}{live coding}{C}{Sun Bears}
\label{eq:def}
\end{equation}
\end{example}

Este grupo também pode ser descrito como uma lista de propriedades como na \autoref{eq:def2}:

\begin{example}{Representação formal das propriedades da \autoref{fig:contido}}
\begin{equation}
\listspaces{C}{SK}{E'}{K'}{I'}{R'}{G'}{M'}{O'}{F'},~\csfeq{P'}{SK}]
\label{eq:def}
\end{equation}
\end{example}
  
Nos diagramas abaixo, $C_\emph{\ldots}$ representa qualquer espaço conceitual abstrato (que pode incluir outro previamente apresentado). Entre os elementos iniciais (raízes, vermelho) e transitórios (nós, azul), ocorrem as ramificações (ramos, linhas pretas), isto é, a exploração de conceitos dentro de outros conceitos. De um lado, a aplicação de regras de validação sobre o universo conceitual da pesquisa (tudo aquilo que foi produzido em dois anos de mestrado) gerou o espaço conceitual desta tese. Estas regras de validação foram, em sua maior parte, os processos de orientação e qualificação. Em outras palavras, \csf{C}{pesquisa}$=[[$\csf{R}{pesquisa}$]]($\csf{U}{pesquisa}$)$.

\begin{example}{Representação do universo conceitual da \emph{pesquisa}}

O Universo de Conceitos da pesquisa, \csf{U}{pesquisa}, é um recorte do universo conceitual da música, \csf{U}{música}:

\begin{tikzpicture}
  [
    grow                    = right,
    sibling distance        = 6em,
    level distance          = 10em,
    edge from parent/.style = {draw, -latex},
    every node/.style       = {font=\footnotesize},
    sloped
  ]
  \node [root] {\csf{U}{Música}}
    child { node [env] {\csf{U}{pesquisa}}
      child { node [env] {\csf{U}{livecoding}}}
    }
    child { node [env] {\csf{C}{\ldots}}};
\end{tikzpicture}

No primeiro capítulo, incluímos um subjconjunto neste Espaço Conceitual da Pesquisa \ver{app:A}). 

\begin{tikzpicture}
  [
    grow                    = right,
    sibling distance        = 6em,
    level distance          = 10em,
    edge from parent/.style = {draw, -latex},
    every node/.style       = {font=\footnotesize},
    sloped
  ]
  \node [root] {\csf{C}{pesquisa}}
    child { node [env] {\csf{C}{livecoding}}
      child { node [env] {\csf{C}{\ldots}}}
      child { node [env] {\csf{C}{ICLC}}}
    }
    child { node [env] {\csf{C}{\ldots}}}; 
\end{tikzpicture}

Podemos inclur elementos históricos, o período transitório entre 1970 e 2000 (\emph{circa}), onde emanciparam as práticas e as regras heurísticas.  

\begin{tikzpicture}
  [
    grow                    = right,
    sibling distance        = 6em,
    level distance          = 10em,
    edge from parent/.style = {draw, -latex},
    every node/.style       = {font=\footnotesize},
    sloped
  ]
  \node [root] {\csf{C}{livecoding}}
    child { node [env] {\csf{C}{Elementos Históricos}}
      child {node [env] {\csf{C}{Proto-História}}}
      child {node [env] {\csf{C}{Manifestos}}}
    }
    child { node [env] {\csf{C}{\ldots}}};
\end{tikzpicture}

Por último, \csf{C}{pesquisa} investiga o \emph{live coding} a partir de um caso específico:

\begin{tikzpicture}
  [
    grow                    = right,
    sibling distance        = 6em,
    level distance          = 10em,
    edge from parent/.style = {draw, -latex},
    every node/.style       = {font=\footnotesize},
    sloped
  ]
  \node [root] {\csf{C}{pesquisa}}
    child { node [env] {\csf{C}{livecoding}}
      child { node [env] {\csf{C}{\ldots}}}
      child { node [env] {\csf{C}{Sessão de Improvisação}}
        child { node [env] {\csf{C}{Study in Keith}}}
        child { node [env] {\csf{C}{\ldots}}}
      }
    }
    child { node [env] {\csf{C}{\ldots}}}; 
\end{tikzpicture}
\end{example}

Por outro lado \csf{C}{Study in Keith} pode ser definido pelo modelo de improvisação de Pressing (\autoref{tab:modelo_improvisacao}, \pageref{tab:modelo_improvisacao}).

\begin{example}{Representação do modelo de improvisação para \emph{Study in Keith}.}
\begin{tikzpicture}
  [
    grow                    = right,
    sibling distance        = 6em,
    level distance          = 10em,
    edge from parent/.style = {draw, -latex},
    every node/.style       = {font=\footnotesize},
    sloped
  ]
  \node [root] {\footnotesize \csf{C}{Study in Keith}}
    child { node [env] {\footnotesize \csf{E'}{Study in Keith}}}
    child { node [env] {\footnotesize \csf{K'}{Study in Keith}}}
    child { node [env] {\footnotesize \csf{I'}{Study in Keith}}}
    child { node [env] {\footnotesize \csf{R'}{Study in Keith}}}
    child { node [env] {\footnotesize \csf{G'}{Study in Keith}}}
    child { node [env] {\footnotesize \csf{O'}{Study in Keith}}}
    child { node [env] {\footnotesize \csf{F'}{Study in Keith}}}
    child { node [env] {\footnotesize \csf{P'}{Study in Keith}}}; 
\end{tikzpicture}
\end{example}

\section{Formalização}\label{sec:formaliza}

O espaço conceitual do \emph{livecoding} é definido como uma função de interpretação das regras de validação (o que pode ser ou não considerado como próprio de uma categorização musical), de gosto (questões de estilo) e de localização transversal de conceitos (conceitos internos que permitem o cruzamento com outros conceitos). As regras de validação foram estudadas neste trabalho como as regras heurísticas do \emph{live coding}. Isto é, que conjunto de métodos são utilizados para caracterizar uma performance de \emph{live coding} como tal? Elementos históricos, e ideológicos (divulgados em manifestos), são levantados para responder esta pergunta. Por outro lado, este estudo abandonou a investigação das regras de gosto, tema que pode ser melhor explorado em trabalhos posteriores, a partir de \citeonline{janotti_jr._a_2003,sa_musica_2006,sa_se_2009}. A tarefa de localização transversal de conceitos é trabalhada no último capítulo. O espaço conceitual de \emph{Study in Keith} está contido no espaço conceitual do \emph{live coding} através da união entre os conceitos deste último, com os espaços conceituais dos concertos \emph{Sun Bears}, de Keith Jarret, misturados. No entanto o espaço conceitual não será investigado em sua totalidade, e sim apenas uma sonoridade.
%\endgroup

% ----------------------------------------------------------
% PARTE
% ----------------------------------------------------------
%\part{Trabalhos }\label{parte2}
\chapter{Trabalhos Relacionados}\label{cap:trabalhos_relacionados}
%\section{Uma breve história do \emph{live coding}}\label{duas_ondas}


\section{Predecessores}

Na \autoref{sec:groove} descrevo um trabalho de \citeonline{mathews_groove_1970}, GROOVE, ainda pouco observado por \emph{live coders}. Seu paradigma composicional é diverso do MUSIC N, e o primeiro de Mathews com reflexões nos aspectos performáticos, semelhantes aos pesquisados neste trabalho.

\citeonline{mori_pietro_2015}  descreve um caso prematuro de \emph{live coding} na Itália, com o compositor Pietro Grossi (1917-2002).Divergente em algumas das propostas de Max Mathews, sacrificou a questão timbrística para trabalhar na questão performática. Esta abordagem será trabalhada na seção \autoref{sec:grossi}.

\citeonline{mclean_patterns_2009} descrevem, no final dos anos 70 e dos anos 80, atividades dos grupos \emph{The Hub} e \emph{The League of Automatic Composers} como fundamentais para o entendimento histórico do \emph{live coding}. Será explicado na seção \autoref{sec:baiasaofranscisco}

Como um breve parêntese, sugiro falar sobre a compilação JIT \cite{aycock_brief_2003}. Este é um personagem sócio-técnico fundamental para que o \emph{live coding} fosse possível. Será descrito na \autoref{sec:jit}.

Para \citeauthoronline{mclean_patterns_2009}, o \emph{live coding} não possui sua identidade cultural até a emergência da organização TOPLAP. Na \autoref{sec:laptoptoplap} proponho a revisão de um trecho da publicação ``\emph{Live Algorithm Programming and Temporary Organization for its Promotion}''.

Além deste primeiro manifesto, existe outro de grande importância, ``Show us your screens'', que define alguns cânones do \emph{live coding}. Na \autoref{sec:showusyourscreens} tratarei do caso específico.

Nestes manifestos podemos observar uma heurística do \emph{livecoding}. Isto é, experiências práticas de improvisadores \cite{ward_live_2004} possibilitaram a formulação de um conjunto de premissas, fielmente seguida por parte substancial dos exemplos de vídeos observados udrante a pesquisa. Esta heurística envolve diretamente a duas frentes composicionais bastante próximas.

Neste trabalho realizarei um comentário, na \autoref{sec:algoritmos}, sobre três apropriações ideológicas , a primeira, indireta, com a Música de Processos de Steve \citeonline{reich_music_1968}, ou nas palavras de \citeonline{mailman_agency_2013}; e a a segunda com a Música Generativa de \citeonline{eno_music_1978}. A terceira, por uma aparente substancialidade (envolve boa parte de vídeos exemplares na internet), será se discutida na \autoref{sec:musica_vanguarda_pista}. Um comentário sobre DJs, separado, se faz necessário pois, embora não sendo o foco desta pesquisa, procede de um modo de operação bastante aceito na comunidade, ganhando bastante espaço em bares ingleses.

\subsection{GROOVE}\label{sec:groove}

Conectando as práticas da imediaticidade, foi possível concluir que o GROOVE \cite{mathews_groove_1970,di_nunzio_genesi_2010} é um precedente mais antigo do \emph{live coding} (até que outro programa seja descoberto). Seu desenvolvimento iniciou em 1968 na \emph{Bell Labs}. Segundo o próprio Mathews, o funcionamento do sistema oferece algumas possibilidades a partir de três conceitos: criação, \emph{retroalimentação} e \emph{ciberficação}. 

O primeiro conceito foi implementado com um sistema de arquivos, onde as funções criadas no processo criativo são memorizadas, e podem ser editadas.

O segundo conceito se relaciona com o terceiro. Tange a imediatidade do fazer musical, e potencialmente, de uma necessidade de improvisação. Esta abordagem é contemporânea às técnicas de composição baseadas por \emph{retroalimentação}, como as descritas por Pauline \citeonline{oliveros_tape_1969}. Não sabemos se a primeira foi inspirada na segunda, mas o paralelo não deixa de ser significativo e importante para a concepção do que é o \emph{live coding}:

\begin{citacao}
O GROOVE provê oportunidades para uma retroalimentação imediata de observações dos efeitos das funções temporais para as entradas do computador, que compõem a função. No modo de composição do sistema GROOVE, um ser humano está em um ciclo de retroalimentação, como mostrado na figura 1 $[$\autoref{fig:groove_sistema}$]$. Assim ele é capas de modificar as funções instantâneamente como um resultado de suas observações daqueles efeitos.\cite[p.~715]{mathews_groove_1970}
\footnote{Tradução nossa de \emph{GROOVE provides opportunity for immediate feedback from observations of the effects of time functions to computer inputs which compose the function. In the compose mode of the GROOVE system, a human beign is in the feedback loop (\ldots) Thus he is able to modify the functions instanteneously as a result of his observations of their effects.}}
\end{citacao}

\begin{figure}
\begin{center}
\includegraphics[scale=0.5]{./imagens/GROOVE.png}
\caption{Esquema de concepção do projeto GROOVE descrito no artigo homônimo por Max Mathews \textbf{Fonte}: \cite{mathews_groove_1970}.}
\label{fig:groove_sistema}
\end{center}
\end{figure} 

O terceiro conceito observa a existência de uma relação entre um humano e uma máquina. De certa forma, o sistema passa pr um  processo de ciberficação, considerado por Mathews como um conceito nebuloso, e chamado de \emph{engenharia humana}. Consistiu na observação de um tempo diferencial entre o que o(a) musicista cria e o que edita. De certa forma, é o conceito central que permite que os dois anteriores fossem possíveis, tange a corporificação de um regente:

\begin{citacao} 
O conceito final é mais nebuloso. Desde que o GROOVE é um sistema homem-máquina, a engenharia humana do sistema foi a mais importante. Por exemplo, nós descobrimos que o controle do programa de tempo necessita ser bastante diferente para a composição do que para a edição, e o programa foi modificado de acordo. A engenharia humana adetou toda estrutura do GROOVE de forma que pontuaremos subsequentemente. (\ldots) O intérprete de computador não deve tentar definir todo o som em tempo real. Ao invés, o computador deve ter ua partitura e o intérprete deve influenciar a forma como a partitura é tocada. Seus modoes de influência pode ser mais variados do que aqueles que um regente convencional, que pode principalmente controloar o tempo, \emph{loudness} e estilo.\cite[p.~715-716]{mathews_groove_1970}
\footnote{Tradução nossa de \emph{The final concept is more nebulous. Since GROOVE is a man-computer system, the human engeneering of the system is most important. For example, we discovered that the control of the program time needs to be quite different for composing than for editing, and the program was modiffied accordingly. Human engineering has affected the entire structure of GROOVE in ways which will be pointed out subsequently. (\ldots) The computer performer should not attempt to define the entire sound in real-time. Instead, the computer should have a score and the performer should influence the way in which the score is played. His modes of influence can be much more varied than that a conventional conductor who primarily controls tempo, loudness, and style.}.}
\end{citacao}

%O segundo ponto abordado por \citeonline{mclean_patterns_2009} no início desta subseção (1.1.1), existe uma cena de \emph{live coding} no início dos anos 2000 com as experimentações do compositor Julian Rohruber utilizando o SuperCollider\footnote{\url{http://supercollider.github.io/}, acessado em \today.}, bem como a cena musical noturna em torno do \emph{Slub}, formado por Adrian Ward, Alex McLean e Dave Griffiths \cite[p.~3]{collins_live_2003}. O \emph{SuperCollider}\footnote{\url{http://github.io.supercollider}, acessado em \today.}, inicialmente desenvolvido por James McCartney (este também um ativo praticante do \emph{live coding}), possuia apenas funcionalidades de composição algorítmica e síntese sonora em tempo-real, isto é, a demanda temporal entre o que é programado e seus resultados é consideravelmente reduzida; mas ainda assim, após uma edição, era necessário uma  reinterpretação do código\footnote{SuperCollider é uma linguagem interpretada, isto é, têm por base no código fonte linguagens compiladas. Diferentemente de linguagens como C e C++, que necessitam, dentre vários processos a compilação}. O trabalho de Rohruber, \emph{JITlib} a biblioteca \emph{Dewdrop lib} \footnote{\emph{Dewdrop} é o nome artístico de James McCartney. Para mais informações a respeito, \url{http://www.dewdrop-world.net/bio/index.html}} contribui para a conceitualização antes mesmo do \emph{Slub}. É a partir destes adventos, uma sequência de apresentações musicais e publicações de artigos que a configuração cultural do que chamarei aqui de \emph{live coding anglófono}\footnote{Inglaterra, EUA, parte inglesa do Canadá e Austrália.} começou a tomar corpo.\todo{\tiny inserir mais coisas}

\subsection{Pietro Grossi}\label{sec:grossi}

Embora pouco conhecido no contexto geral da música européia, o compositor Pietro Grossi foi  um dos pioneiros da \emph{Computer Music} Italiana. Diverge do \emph{paradigma} do MUSIC III e suas preocupações timbrísticas. Mas concorda, parcialmente, com a abordagem do GROOVE. Sacrificou o desenvolvimento do timbre desde o início, e para concentrar na performance, utilizou apenas uma forma de onda (pulsos). O primeiro \emph{software} desenvolvido foi o DCMP (\emph{Digital Computer Music Program}) e, segundo \cite{mori_pietro_2015}, ao usar este programa,

\begin{citacao}
(\ldots) o intéprete era capaz de produzir e reproduzir música em tempo real, digitando alguns comandos específicos e os parâmetros composicionais desejados. O som resultante vinha imediatamente depois da operação de decisão, sem qualquer atraso causado por cálculos. Haviam muitas escolhas de reprodução no programa: era possível salvar na memória do computador peças de músicas pré-existentes, para elaborar qualquermaterial sonoro no disco rígido, para administrar arquivos musicais e iniciar um processo de composição automático, baseado em algoritmos que trabalham com procedimentos ``pseudo-casuais''. Exsitia também uma abundância de escolhas para mudanças na estrutura da peça. Um dos mais importantes aspectos do trabalho de Grossi foi que todas intervenções eram instantâneas: o operado não tinha que esperar pelo computador terminar todas operações requisitadas, e depois ouvir os resultados. Cálculos de dados e reprodução sonoras eram simultâneos. Esta simultaneidade não era comum no campo da \emph{Computer Music} daquele tempo, e Grossi deliberadamente escolheu trabalhar desta forma, perdendo muito no lado da qualidade sonora. Seu desejo era poder escutar os sons resultantes imediatamente. \cite[p.~126]{mori_pietro_2015} \footnote{Tradução nossa de \emph{(\ldots) the performer was able to produce and reproduce music in real time by typing some specific commands and the desired composition's parameters. The sound result came out immediately after the operator's decision, without any delay caused by calculations. There were many reproduction choices inscribed in this software: it was possible to save on the computer memory pieces of pre-existing music, to elaborate any sound material in the hard disk, to manage the music archive and to start an automated music composition process based on algorithms that worked with “pseudo-casual” procedures. There were also plenty of choices for piece structure modifications. One of the most important aspects of Grossi’s work was that all the interventions were instantaneous: the operator had not to wait for the computer to finish all the requested operations and then hear the results. Data calculation and sound reproduction were simultaneous. This simultaneity was not common in the computer music field of that time and Grossi deliberately chose to work in this way, losing much on the sound quality’s side. His will was to listen to the sound result immediately.}}
\end{citacao}

Esta abordagem parte de uma abordagem ``preguiçosa'' (\emph{lazy}). Grossi dizia sobre si mesmo, como ``uma pessoa que está consciente de que o seu tempo é limitado e não quer perder tempo em fazer coisas inúteis ou na espera de alguma coisa quando não é necessário.''\footnote{Tradução nossa de \emph{a person who is aware that his or her time is limited and do not want to waste time in doing useless things or in waiting for something when it is not necessary.}}. Neste sentido, defendia que o desenvolvimento de novos timbres deveria esperar por melhores implementações. .

O sacrifício do timbre reflete a utilização o computador como uma paródia do piano, ou até mesmo de um violino. Grava em 1967 ``Mixed Paganini''\footnote{Disponível em \url{https://www.youtube.com/watch?v=ZQSP_wF7wSY}.}, uma execução ultra-virtuosística dos \emph{Capricci op.1} de Niccolò Paganini. Aparentemente pode soar como um pastiche musical. Porém uma escuta mais atenta permite perceber  que, a utilização de operações canônicas (inversão, aceleração e retrogradação),  executadas no computador Olivetti GE-115 \citeonline[p.~126]{mori_pietro_2015}, lembra dicotomias entre o tempo e o ritmo já discutidas por \citeonline{stockhausen_how_1957}.


\subsection{Baía de São Franscisco}\label{sec:baiasaofranscisco}

No final dos anos 70, na cena musical da Baía de São Fransisco, uma das atividades de John Bischoff (1949-, aluno de composição de James Tenney and Robert Ashley.) e Tim Perkis \todo{\tiny qual sua data de nascimento?} era passar horas ajustando uma rede de microcontroladores programáveis KIM-1\footnote{Disponível em \url{http://www.6502.org/trainers/buildkim/kim.htm}.}. Aquele era um momento onde os \emph{happenings} já eram manifestações artísticas consolidadas. Não demorou muito para que o público participasse da atividade:

\begin{citacao}
Na primavera de 1979, montamos uma série quinzenal regular de apresentações informais sob os auspícios da \emph{Bay Center for the Performing Arts}. Todos outros domingos à tarde passávamos algumas horas configurando nossa rede de KIMs na sala \emph{Finnish Hall}, na Berkeley, e deixávamos a rede tocando, com retoques aqui e ali, por uma ou duas horas. Os membros da audiência poderiam ir e vir como quisessem, fazer perguntas, ou simplesmente sentar e ouvir. Este foi um evento comunitário de tipos como outros compositores aparecendo, tocando ou compartilhando circuitos eletrônicos que tinham projetado e construído. Um interesse na construção de instrumentos eletrônicos de todos os tipos parecia estar "no ar". Os eventos da sala \emph{Finn Hall} foram feitos para uma cena com paisagens sonoras geradas por computador misturado com os sons de grupos de dança folclórica ensaiando no andar de cima e as reuniões ocasionais do Partido Comunista na sala de trás do edifício velho venerável. A série durou cerca de 5 meses que eu me lembre.\cite[online]{brown_indigenous_2013}\footnote{Tradução nossa de: \emph{In the spring of 1979, we set up a regular biweekly series of informal presentations under the auspices of the East Bay Center for the Performing Arts. Every other Sunday afternoon we spent a few hours setting up our network of KIMs at the Finnish Hall in Berkeley and let the network play, with tinkering here and there, for an hour or two. Audience members could come and go as they wished, ask questions, or just sit and listen. This was a community event of sorts as other composers would show up and play or share electronic circuits they had designed and built. An interest in electronic instrument building of all kinds seemed to be "in the air." The Finn Hall events made for quite a scene as computer-generated sonic landscapes mixed with the sounds of folk dancing troupes rehearsing upstairs and the occasional Communist Party meeting in the back room of the venerable old building. The series lasted about 5 months as I remember.}}
\end{citacao}

Esta experiência foi levada a cabo por Bischof, Perkis, Chris Brown (1953-), Scot Gresham-Lancaster (1954-)\footnote{Aluno de Darius Milhaud, John Chowning, Robert Ashley e Terry Riley.}, Mark Trayle (1955-)\footnote{Aluno de  Robert Ashley.} e Phil Stone\todo{\tiny ajustar informações}, que formaram nos anos 80 o grupo \emph{The Hub}, com um primeiro disco lançado em 1989 entitulado \emph{The Hub: Computer Network Music}.

Outro artista , Ron Kuivila, segundo \citeonline{mclean_patterns_2009} realiza as primeiras performances de \emph{live coding} em 1985. Um pequeno panorama de suas produções pode ajudar a entender um contexto sonoro. Em 1982 é lançada ``Going out with slow smoking'', co-produzida com Nicolas \citeonline{collins_brains_2002}\footnote{Áudio disponível em \url{http://www.nicolascollins.com/slowsmoketracks.htm}.}. Em 1985 são lançadas duas faixas em um disco peças pela \emph{TELLUS \#9 -- The Audio Casset Magazine}, ``Cannon Y for C.N.'' e ``Parodicals''. Em 1988 Kuivila grava ``Linear Predictive Zoo''\footnote{Disponível em \url{https://www.youtube.com/watch?v=DZ5pUUXqkMc}}, parte da \emph{TELLUS \#22}\footnote{Disponível em \url{http://www.discogs.com/Various-False-Phonemes/release/785540}.}.

\begin{citacao}
A primeira performance conhecida de \emph{live coding} foi em 1985, por Run Kuivila na STEIM em Amsterdã \cite{blackwell_programming_2005}. O \emph{live coding} não possui sua própria identidade cultural até o TOPLAP, a Organização Temporária para a Promoção da Programação Ao Vivo de Algoritmos, formada na \emph{Changing Grammars Workshop} em 2004 \cite{ward_live_2004}. Mesmo sendo possível pensar em fazer \emph{live coding} sem computadores, através da auto-modificação de composições baseadas em regras, não existe evidência que isto foi feito antes da invenção dos computadores. Parece que foi necessária a invenção da interpretação dinâmica de códigos para o \emph{live coding} aparecer como possível ou talvez desejável \cite[p.~1-2]{mclean_patterns_2009}\footnote{Tradução nossa de: \emph{The earliest known live coding performance was in 1985, by Ron Kuivila at STEIM in Amsterdam (Blackwell and Collins, 2005). Live coding did not get its own cultural identity until TOPLAP, the Temporary Organisation for the Promotion of Live Algorithm Programming was formed at the Changing Grammars workshop in Hamburg in 2004 (Ward et al., 2004). Although it is possible to do live coding without computers, through self-modifying rule-based composition, there is no evidence that this was done before the invention of computers. It would seem that it required the invention of dynamic code interpretation for live coding to appear possible or perhaps desirable.}}
\end{citacao}\todo{\tiny inserir mais coisas}

Embora uma breve menção de Kuivila seja realizada, é enfatizado o fato de que os construtos sócio-técnicos dos anos 80 ainda não poderiam articular o \emph{live coding} como conhecemos. Isso só foi possível com alguns desenvolvimentos técnicos e organizações de sujeitos. Como a compilação JIT \cite{aycock_brief_2003} em \emph{softwares} musicais; e a emancipação da organização TOPLAP. Antes de descrever o JIT e o TOPLAP, sugiro pensar em outras proto-histórias.

\subsection{Just In Time (JIT)}\label{sec:jit}

\begin{citacao}
Passageiro para o motorista: leve-me ao número 37. Eu te digo o nome da rua quando chegarmos lá.\footnote{Tradução nossa de \emph{Passenger to taxtidriver: take me to number 37. I'll give you the street name when we are there.}. Disponível em \url{http://doc.sccode.org/Overviews/JITLib.html}.}
\end{citacao}

A sentença acima é uma ``piada de um professor austríaco'' (\emph{idem, ibdem}), e descreve como este paradigma de programação em tempo-real funciona. Segundo \citeonline{aycock_brief_2003}, o primeiros programas JIT foram Genesis (com base no LISP, 1960), LC$^2$ (\emph{Language for Conversational Computing}, 1968) e APL (1970). Este último tinha duas novidades técnicas, a partir dos termos \emph{drag-along} e \emph{beating}; estes são hoje chamados de \emph{lazy evaluation} (avaliação preguiçosa). 

O \emph{SuperCollider} foi o primeiro dos \emph{softwares} descritos na introdução destre trabalho que implementou a avaliação preguiçosa.
A docuentação oferece uma descrição de como isso pode funcionar durante uma performance de \emph{live coding}:

\begin{citacao}
Para programação interativa, pode ser útil ser capaz de usar algo antes de estar ali -- isso faz o pedido de avaliação ser mais flexível e permite postergar decisões para um outro momento. Algumas preparações geralmenet tem que ser feitas (\ldots) Em outras situações este tipo de preparação não é suficiente, por exemplo se alguém quer aplicar operações matemáticas em sinais de processos sendo executados no servidor $[$do \emph{SuperCollider}$]$\footnote{Tradução nossa de \emph{For interactive programming it can be useful to be able to use something before it is there - it makes evaluation order more flexible and allows to postpone decisions to a later moment. Some preparations have to be done usually - like above, a reference has to be created. In other situations this sort of preparation is not enough, for example if one wants to apply mathematical operations on signals of running processes on the server.}. Disponível em \url{http://doc.sccode.org/Tutorials/JITLib/jitlib_basic_concepts_01.html}}
\end{citacao}

Atualmente, esta técnica têm sido largamente implementada para navegadores de internet \cite{roberts_web_2013}. Programas como Gibber\footnote{Disponível em \url{http://gibber.mat.ucsb.edu/}.} \cite{roberts_gibber:_2012} e \emph{wavepot}\footnote{Disponível em \url{https://www.wavepot.com}.} são exemplares. Durante a pesquisa, foi desenvolvido em parceria com o pesquisador Flávio Schiavonni um ambiente JIT, inspirado no GROOVE. Enquanto não foi publicado um artigo explicativo, anexamos ele no \autoref{sec:sbcm}.

\subsection{Live Algorithm Programming and Temporary Organization for its Promotion}\label{sec:laptoptoplap}

Na \autoref{sec:perguntametodo}, \citeonline{blackwell_programming_2005} comenta a emancipação de um grupo conhecido como TOPLAP. Este acrônimo é de difícil definição. Deriva  do manifesto ``\emph{Live Algorithm Programming and Temporary Organization for its Promotion}'' \citeonline{ward_live_2004}. No \emph{Wiki} do site oficial\footnote{Disponível em \url{http://toplap.org/wiki/Main_Page}.}, a cada visita, cada letra é substituída por palavras randômicas, criando diferentes significados\footnote{Deparamo-nos, por exemplo como \emph{Transdimensional Organisation for the Pragmatics of Live Algorithm Programming}, \emph{Terrestrial Organisation for the Proliferation of Live Artistic Programming}, \emph{Temporal Organisation for the Proliferation of Live AudioVisual Programming} e outros.}.

Este manifesto expõe os ambientes de performance, bem como alguns ritos técnicos do improvisador.  Espaços de Música Eletrônica de Pista se misturam com a Música algorítmica. e Música de processos. Em outras palavras, um fenômeno onde produtores e DJs se misturam aos universitários.

\begin{citacao}
O \emph{Livecoding} permite a exploração de espaços algorítmicos abstratos como uma improvisação intelectual. Como uma atividade intelectual, pode ser colaborativa. Codificação e teorização podem ser atos sociais. Se existe um público, revelar, provocar e desafiar eles com uma matemática complexa se faz com a esperança de que sigam, ou até mesmo participem da expedição. Estas questões são, de certa forma, independentes do computador, quando a valorização e exploração de algoritmo é que importa. Outro experimento mental pode ser encarado com um DJ ao vivo codificando e escrevendo uma lista de instruções para o seu \emph{set} (realizada com o iTunes, mas aparelhos reais funcionam igualmente bem). Eles passam ao HDJ $[$ \emph{Headphone Disk Jockey} $]$ de acordo com este conjunto de instruções, mas no meio do caminho modificam a lista. A lista está em um retroprojetor para que o público possa acompanhar a tomada de decisão e tentar obter um melhor acesso ao processo de pensamento do compositor. \cite[p.~245]{ward_live_2004} \footnote{Tradução nossa de: \emph{Live coding allows the exploration of abstract algorithm spaces as an intellectual improvisation. As an intellectual activity it may be collaborative. Coding and theorising may be a social act. If there is an audience, revealing, provoking and challenging them with the bare bone mathematics can hopefully make them follow along or even take part in the expedition. These issues are in some ways independent of the computer, when it is the appreciation and exploration of algorithm that matters.   Another thought experiment can be envisaged in which a live coding DJ writes down an instruction list for their set (performed with iTunes, but real decks would do equally well). They proceed to HDJ according to this instruction set, but halfway through they modify the list. The list is on an overhead projector so the audience can follow the decision making and try to get better access to the composer’s thought process.}}
\end{citacao}\todo{\tiny inserir mais informações ou discussões}

O manifesto ``\emph{Live Algorithm Programmin and Temporary Organization for its Promotion}'' \cite{ward_live_2004} fornece informações a respeito de comportamentos sociais e gostos musicais:

\begin{citacao}
Contudo, alguns músicos exploram suas idéias como processos de \emph{software}, muitas vezes ao ponto que o \emph{software} se torna a essência da música. Neste ponto, os músicos podem ser pensados como programadores explorando seu código manifestado como som. Isso não reduz seu papel principal como um músico, mas complementa, com a perspectiva única na composição de sua música. \textbf{Termos como ``música generativa'' e ``música de processos'' tem sido inventados e apropriados para descrever esta nova perspectiva de composição}. Muita coisa é feita das supostas propriedades da chamada ``música generativa'' que separa o compositor do resultado do seu trabalho. Brian Eno compara o fazer da música generativa com o semear de sementes que são deixadas para crescer, e sugere abrir mão do controle dos nossos processos, deixando eles ``brincarem ao vento''. \footnote{\opcit[p.~245-246]{ward_live_2004}. Tradução nossa de \emph{Indeed, some musicians explore their ideas as software processes, often to the point that a software becomes the essence of the music. At this point, the musicians may also be thought of as programmers exploring their code manifested as sound. This does not reduce their primary role as a musician, but complements it, with unique perspective on the composition of their music. Terms such as “generative music” and “processor music” have been invented and appropriated to describe this new perspective on composition. Much is made of the alleged properties of so called “generative music” that separate the composer from the resulting work. Brian Eno likens making generative music to sowing seeds that are left to grow, and suggests we give up control to our processes, leaving them to “play in the wind”.}}
\end{citacao}

Isso sumariza a prática musical do \emph{live coding} como \emph{1)} Música de Processos (\autoref{sec:alg_simples}), ou Música de algoritmos simples, \emph{2)} Música Generatva (\autoref{sec:alg_complexo}), e \emph{3)} práticas de Disk Jockey (\autoref{sec:musica_vanguarda_pista}).

\subsection{\emph{Show us your screens}}\label{sec:showusyourscreens}

Além das performances inaugurais nos festivais Europeus, e do manifesto ``Live Algorithm Programming and Temporary Organization for its Promotion'', um texto possui uma importância fundamental para \emph{live coding}. Premissas de comportamentos e técnicas são delineadas no texto ``\emph{Show Us Your Screens}''\cite[p.~22; online]{griffiths_fluxus:_2008,mccallum_show_2011}:\todo{\tiny Aqui as correções ainda não foram aplicadas, apenas sessões foram ajustadas, suas ordens}

\begin{citacao}
Exigimos:

• Acesso à mente do intérprete, para todo o instrumento humano.

• Obscurantismo é perigoso. Mostre-nos suas telas.

• Programas são instrumentos que podem modificar eles mesmos.

• O programa será transcendido - Língua Artificial é o caminho.

• O código deve ser visto assim como ouvido, códigos subjacentes visualizados bem como seu resultado visual.

• Codificação ao vivo não é sobre ferramentas. Algoritmos são pensamentos. Motosserras são ferramentas. É por isso que às vezes algoritmos são mais difíceis de perceber do que motosserras.

Reconhecemos contínuos de interação e profundidade, mas preferimos:

• Introspecção dos algoritmos.

• A externalização hábil de algoritmo como exibição expressiva/impressiva de destreza mental.

• Sem \emph{backup} (minidisc, DVD, safety net computer).

Nós reconhecemos que:

• Não é necessário para uma audiência leiga compreender o código para apreciar, tal como não é necessário saber como tocar guitarra para apreciar uma performance de guitarra.

• Codificação ao vivo pode ser acompanhada por uma impressionante exibição de destreza manual e a glorificação da interface de digitação.

• Performance envolve contínuos de interação, cobrindo talvez o âmbito dos controles, no que diz respeito ao parâmetro espaço da obra de arte, ou conteúdo gestual, particularmente direcionado para o detalhe expressivo. Enquanto desvios na tradicional taxa de reflexos táteis da expressividade, na música instrumental, não são aproximadas no código, por que repetir o passado? Sem dúvida, a escrita de código e expressão do pensamento irá desenvolver suas próprias nuances e costumes.
\footnote{Tradução nossa de:\emph{We demand: \begin{inparaenum}[•]
\item Give us access to the performer's mind, to the whole human instrument.
\item Obscurantism is dangerous. Show us your screens.
\item Programs are instruments that can change themselves.
\item The program is to be transcended - Artificial language is the way.
\item Code should be seen as well as heard, underlying algorithms viewed as well as their visual outcome.
\item Live coding is not about tools. Algorithms are thoughts. Chainsaws are tools. That's why algorithms are
sometimes harder to notice than chainsaws.
\end{inparaenum}. We recognise continuums of interaction and profundity, but prefer:  \begin{inparaenum}[•]
\item Insight into algorithms
\item The skillful extemporisation of algorithm as an expressive/impressive display of mental dexterity
\item No backup (minidisc, DVD, safety net computer)
\end{inparaenum}. We acknowledge that: \begin{inparaenum}[•]
\item It is not necessary for a lay audience to understand the code to appreciate it, much as it is not necessary
to know how to play guitar in order to appreciate watching a guitar performance.
\item Live coding may be accompanied by an impressive display of manual dexterity and the glorification of the
typing interface.
\item Performance involves continuums of interaction, covering perhaps the scope of controls with respect to
the parameter space of the artwork, or gestural content, particularly directness of expressive detail. Whilst
the traditional haptic rate timing deviations of expressivity in instrumental music are not approximated in
code, why repeat the past? No doubt the writing of code and expression of thought will develop its own
nuances and customs.
\end{inparaenum}}}
\end{citacao}

``Dar acesso à mente do intérprete'' e ``obscurantismo é perigoso'' descrevem um meio de evitar qualquer código mal intencionado; isto é uma hipótese: \begin{inparaenum}[\itshape i)\upshape]
\item programas de \emph{live coding} geralmente são programas em fase de desenvolvimento;
\item programas em desenvolvimento possuem, inevitavelmente, \emph{bugs}\footnote{Segundo James S. Huggins, historicamente ``O termo \emph{bug} é usado de forma limitada para designar qualquer falha ou problema em conexões ou no trabalho com aparatos elétricos'' Tradução nossa de \emph{The term "bug" is used to a limited extent to designate any fault or trouble in the connections or working of electric apparatus.} (ver \url{http://www.jamesshuggins.com/h/tek1/first_computer_bug.htm}). Nesse sentido, um \emph{bug} em um programa é uma falha de operação, geralmente causada por algum erro de lógica, por parte do programador.}
\item \emph{bugs} podem ser explorados e levar à corrupção do sistema.
\end{inparaenum} Se esta hipótese estiver correta, justificaria a atitude de exposição da \emph{imagem-texto}. No entanto não encontrei algum estudo crítico descrevendo se isso é verdade a partir do ponto de vista do público, isto é, será que o público pode estar realmente interessado na exposição da \emph{imagem-texto}? Será que essa exposição não pode ser perigosa para o processo artístico e para a experiência do público? Embora sejam questões que fogem do escopo do trabalho, são importantes, necessitando verificar algumas performances para averiguar.

``Programas são instrumentos'' e ``O programa será transcendido - Língua Artificial é o caminho'', são frases que fazem menção direta à experiência de usuário (\emph{live coder}), isto é, um sistema que é programável de maneira facilitada. A seguinte hipótese pode ser feita: quanto mais simplificada a linguagem de programação, mais expressão visual ou musical um espetáculo poderá ter (o que pode não ser verdade, e sim que a expressão musical estaria no nível sensível). Por ``Língua Artifical'' entendo que o \emph{live coder} pode criar \emph{mini-linguagens} ou Linguagens de Domínio Específico (DSL)\footnote{Sobre esse tema recomendo o texto ``Minilanguages, finding a notation that sings'' de \citeonline{raymond_minilanguages_2003}: ``Historicamente, linguagens de dominio especifico sao do tipo que sao chamadas de 'pequenas linguagens' ou 'minilinguagens' no mundo do Unix, porque os primeiros exemplos eram pequenos e de pouca complexidade, em relaçao às linguagens de propósito geral (\ldots) Nós manteremos o termo tradicional 'minilinguagem'para engatizar que no decorrer do curso é geralmente utilizado para projetar e mantê-las o menor e simples possível'' \cite[$3^o$ parágrafo]{raymond_minilanguages_2003}. Tradução nossa de \emph{Historically, domain-specific languages of this kind have been called ‘little languages’ or ‘minilanguages’ in the Unix world, because early examples were small and low in complexity relative to general-purpose languages (\ldots) We'll keep the traditional term ‘minilanguage’ to emphasize that the wise course is usually to keep these designs as small and simple as possible.}} que possibilitam criar programas para criar um espetáculo audiovisual ou musical. Segundo \citeonline{collins_algorave:_2014}, tais DSLs estariam no formato de ``mini-linguagens'' bem desenvolvidas para a tarefa específica de codificar música ao vivo, operando técnicas composicionais como a transformação de um padrão musical (como por exemplo, técnicas barrocas como inversão e retrogradação ou técnicas aleatórias, como emabaralhamento de um conjunto de eventos sonoros), facilitando a espontaneidade no processo criativo:

\begin{citacao}
Existe um número crescente de sistemas de \emph{livecoding} com ``mini-linguagens'' amigáveis, que facilitam o \emph{loop} e contruções de camadas centrais típicas para dançar música. \emph{Ixilang} é um exemplo primário, e possui um editor de código estruturado que, enquanto baseado em texto, suporta correspondências visuais. \emph{Tidals} é outro, e, embora com foco na rapidez de utilização ao invés da facilidade de aprendizagem, está começando a ter mais ampla aceitação. Ambos \emph{ixilang} e \emph{Tidal} promovem padrões em termos de funções transformadoras como embaralhamento, inversão e extrapolação de formas diferentes.  \cite[p~.357]{collins_algorave:_2014}\footnote{Tradução nossa de: \emph{There are increasingly user friendly “mini-language” livecoding systems which facilitate loop and layer-centric con-structions typical to dance music. ixilang is a primary example, and features a structured code editor which while text-based, supports visual correspondences. Tidal is another, and although its focus is on speed of use rather than ease of learning, is beginning to see wider take-up. Both ixilang and Tidal promote pattern in terms of transformative functions as scrambling, reversal and extrapolation in different ways}.}
\end{citacao}


``O código deve ser visto assim como ouvido'' entraria em um problema próprio de programas de pesquisa em notação musical, sendo que o processo de correlação entre o que está escrito e o que está sendo ouvido leva um tempo ou pode mesmo nem existir. Mesmo com o convite expressado pelo manifesto ``\emph{Live Algorithm Programming and Temporary Organization for its Promotion}'' no início do capítulo, a questão não está nem no uso do computador nem em alguma abordagem musical, e conforme a performance avança, a imagem-texto vai se tornando tão poluída que poderia causar um desinteresse.

\begin{citacao}
Codificação e teorização podem ser atos sociais. Se existe um público, revelar, provocar e desafiar eles com uma matemática complexa se faz com a esperança de que sigam, ou até mesmo participem da expedição. Estas questões são, de certa forma, independentes do computador, quando a valorização e exploração de algoritmo é que importa.\cite[p.~204]{ward_live_2004}
\end{citacao}

``Algoritmos são pensamentos. Motosserras são ferramentas.'', ``Introspecção dos algoritmos.'' e ``A externalização hábil de algoritmo'' descrevem uma atividade constante de formalizações lógicas, do processo febril de explorar uma complexidade própria do que se criou, de ficar digitando sem parar um teclado de computador. Alguns colegas e amigos não programadores, músicos e não músicos, utilizam a expressão ``gostar de apertar botão'' para se referir à caricatura do programador em um espaço reservado, no qual controla dispositivos diversos.

``Sem \emph{backup}'' indica o comportamento do \emph{live coder} após uma improvisação, que não memoriza em discos rígidos, cd's ou \emph{pendrives} o documento criado (isto é, o código textual, em alguma extensão apropriada para a linguagem utilizada, por exemplo, \emph{.pl}, Perl, \emph{.scheme}, Scheme, \emph{.js}, JavaScript), da mesma forma que um músico de improvisação dificilmente transcreveria o que tocou em uma partitura, no máximo gravando o áudio da performance.

A respeito de ``Não é necessário para uma audiência leiga compreender'' e ``A codificação ao vivo pode ser acompanhada por uma impressionante exibição de destreza'' pode indicar uma proximidade com aquele modelo de prática musical virtuosística (como um espetáculo de habilidades técnicas); mais especificamente, este modelo poderia partir daquilo que  \citeonline{magnusson_herding_2014} chama de ``adoção de um método pré-romântico de compor através da performance em tempo real, onde tudo fica aberto a mudar -- o processo composicional o design do instrumento e a inteligência do sistema tocando a peça.'' \cite[p.~4]{magnusson_herding_2014}\footnote{Tradução nossa de: \emph{live coding adopts a pre-Romantic method of composing through performance in real time, where everything remains open to change -- the compositional process, the instrument design, and the inteligence of the system performing the piece.}}

\section{Três apropriações ideológicas}\label{sec:algoritmos}

Não é nossa intenção discutir a Música Algorítmica de maneira pormenorizada, mas sim relacioná-la com o \emph{livecoding}. Mais especificamente, nos manifestos apresentados nas \autoref{sec:laptoptoplap} e \autoref{sec:showusyourscreens}, observamos os \emph{livecoders} se apropriarem de dois conhecimentos musicais históricos, e de uma prática contemporânea.

O primeiro conhecimento, a Música de Processos, é incluída de maneira indireta. Será discutido na \autoref{sec:alg_simples}. O segundo conhecimento, é incluído de maneira um pouco mais específica, na \autoref{sec:alg_complexo}. 

A apropriação de uma prática contemporânea, \emph{Disk Jockey} (DJ), será comentado na \autoref{sec:musica_vanguarda_pista}.


\subsection{Música de Processos}\label{sec:alg_simples}

Segundo \citeauthoronline{wooler_framework_2005}, a Música de Processos é uma ``Música resultante de um conjunto de processos colocados em movimento pelo compositor, tais como  'In C' de Terry Rilley e 'It's gonna rain' de Steve Reich'' \apud[p.~1]{wooler_framework_2005}{eno_generative_1996}\footnote{Tradução nossa de \emph{Music resulting from processes set in motion by the composer “In C” by Terry Riley and “Its gonna rain” by Steve Reich}}; o que é chamado de ``procedural'' por \citeauthoronline{wooler_framework_2005}, é chamado por Joshua \citeonline{mailman_agency_2013} em seu artigo ``\emph{Agency, Determinism, Focal Time Frames and Processive Minimalist Music}'', música de processos mínimos, ou música minimalista de processos determinísticos. O autor faz menção aqui ao compositor Alvin Lucier (1931-) e sua peça \emph{Crossings} (1984). Segundo o autor,

\begin{citacao}
A longa forma de \emph{Crossings} (1984) de Alvin Lucier é especialmente clara; ela surge processivamente, neste caso de um glissando de som senoidal puro que lentamente (mais de 16 minutos) ascende de infra-sons para a ultra-sons. Um processo discreto combina com este glissando, criando uma série de processos contínuos de curto alcance. Uma orquestra dividida alterna no jogo de alturas consecutivas, em uma escala cromática ascendente ascendente que interage com o glissando. Um grupo inicia e sustenta um intervalo acima daquela do glissando de tal modo que a uma dissonância mútua cria batimentos; conforme o glissando aumenta, a velocidade do batimento diminui; quando se alcança um uníssono com os instrumentos, os batimentos momentaneamente cessam; como o glissando ascende acima da nota dos instrumentos, batimentos gradualmente aceleram. Eventualmente, o outro conjunto entra e sustenta um intervalo de um semitom acima, começando esse processo de curto alcance novamente.\cite[p.~128]{mailman_agency_2013}\footnote{Tradução nossa de \emph{The long-range form of Alvin Lucier’s Crossings (1984) is especially clear; it arises processively, in this case from a glissando of a pure sine tone that slowly (over sixteen minutes) ascends from infrasonic to ultrasonic. A discrete process combines with this glissando, creating a series of short-range continuous processes. A divided orchestra alternates in playing the consecutive pitches of a rising chromatic scale that interacts with the glissando. One ensemble initiates and sustains a pitch just higher than that of the glissando such that their mutual dissonance creates beats; as the glissando rises, the speed of the beats slows; when it reaches a unison with the instruments, the beats momentarily cease; as the glissando ascends above the pitch of the instruments, the beats gradually accelerate. Eventually, the other ensemble enters and sustains a pitch a semitone higher, starting this short-range process again.}}
\end{citacao}

Essa descrição é um breve comentário da peça, mas descreve um simples processo de ação musical, bem como relata de forma clara os resultados sonoros. É possível presumir que existe, na execução de \emph{Crossings}, um simples algoritmo que permeia toda a peça: uma alternância de um ritmo lento de dois grupos instrumentais defasados temporalmente em relação ao som emitido pelo alto-falante\footnote{Esse tipo de algoritmo pode ser facilmente implementado em uma linguagem de programação musical, como o SuperCollider (ver \autoref{crossings}).}.

Uma vez que na MP também é possível notar a influência de um algoritmo, é necessário evitar confusão com o uso original da MP, tal qual em compositores como Reich, Lucier e Riley; já o seu uso apropriado é realizado por Adrian Ward, Julian Rohruber, Frederick Olofsson, Alex McLean, Dave Griffiths, Nick Collins e Amy Alexander. Sugiro portanto comparar o fragmento do manifesto citado na página 53 com um fragmento do manifesto \emph{Music as a Gradual Process} do compositor Steve \citeonline{reich_music_1968}: 

\begin{citacao}
O que é distintivo sobre processos musicais é que eles determinam todos detalhes de nota-a-nota (som-a-som) e toda a forma simultaneamente (pense em um \emph{Round} ou um \emph{Canon}). \textbf{Eu estou interessado em processos perceptivos. Eu quero ser capaz de ouvir o processo acontecendo através da música que soa.} Para facilitar a descrição detalhada de um processo musical, a escuta  deve acontecer muito gradualmente. \cite[p.~1]{reich_music_1968}.\footnote{Tradução nossa de: \emph{The distinctive thing about musical processes is that they determine all the note-to-note (sound-to-sound) details and the over all form simultaneously. (Think of a round or infinite canon.) I am interested in perceptible processes. I want to be able to hear the  process happening throughout the sounding music. To facilitate closely detailed listening a musical process should happen extremely gradually. }}
\end{citacao}

Quanto ao manifesto de Adrian Ward, Julian Rohruber, Frederick Olofsson, Alex McLean, Dave Griffiths, Nick Collins e Amy Alexande, tenho uma primeira impressão que essa transformação gradual do som não está presente da mesma forma que em Reich, Lucier ou Rilley; vejo que o uso a reminiscência do termo ``processo'' a partir de ``procedural'' e ``processador'' se dá mais pela atividade de reprogramação em uma CGS ou CGC, mais especificamente, para fins de entretenimento. Aqui o \emph{processo} é formalizado como algoritmo, colocado na situação de constante modificação.

Já no manifesto de Reich o som não é programado, mas se prevê um comportamento musical em constante modificação a partir de uma ação mínima sobre aquilo que produz o som. Um exemplo interessante é seu \emph{Pendulum Music} (1968), onde deixa oscilar microfones posicionados acima de alto-falantes, com os cabos presos no teto; através de um único tipo de ação, restrita apenas pelas distâncias tomadas por intérpretes antes de colocar os microfones em movimento e pela comprimento do cabo preso ao teto\footnote{Seria possível incluir aqui também a resistência do ar ao movimento do objeto.}, ocorrerá um \emph{processo físico} (desaceleração das oscilações que influenciam quando o microfone passará pelo alto-falante) que desencadeará um \emph{processo perceptivo} (escuta de diferentes momentos de quando estes microfones passam pelo alto-falante até um momento estacionário). 

Tal técnica faz uso deliberado de um fenômeno conhecido como microfonia, um \emph{processo recursivo} onde os sons projetados pelo alto-falante são retro-alimentaados pelo microfone. A retroalimentação já era usada por Pauline \citeonline{oliveros_tape_1969}, como por exemplo em \emph{Beaultifoul Soup} (1967), peça para fita de dois canais, elaborada através de um circuito em retroalimentação (a compositora indica uma atividade musical econômica que ao longo do tempo causa "interessantes mudanças timbrísticas em sons sustentados" \footnote{\cfcite[p.~43]{oliveros_tape_1969} Tradução nossa de: \emph{interesting timbre changes on sustained sounds}}). Uma proposta posterior também foi realizada por Alvin Lucier (1931-) em \emph{I'm sitting in a room} (1969) quando retroalimenta seu próprio discurso em uma sala, incrementando sinais ao gravador, criando um \emph{drone}, ou um som de duração indefinida\footnote{É interessante comentar, nesse contexto, o uso deliberado de \emph{drones} no \emph{livecoding} através de uma apropriação da \emph{Drone Music}.}. Esta técnica de retroalimentação também pode ser utilizada no \emph{live coding}, por exemplo, no SuperCollider (ver \autoref{apendice_delays}).

\subsection{Música Generativa}\label{sec:alg_complexo}

Brian Eno foi mencionado por \citeonline{ward_live_2004}, embora a inclusão pareça ser polisêmica. Chamamos  mesmo em um fragmento de texto de \citeonline{eno_music_1978}:

\begin{citacao}
O conceito de música projetada especificamente como pano-de-fundo ambiental foi um pioneirismo da indústria Muzak nos anos cinquenta, e tem sido conhecido genericamente como Muzak. As conotações que este termo carrega são particularmente associadas com um tipo de material que as indústria Muzak produz - melodias familiares arranjadas e orquestradas de maneira leve e derivativa. Compreensivelmente, isso leva ouvintes atentos (e compositores) a dispensar inteiramente o conceito de música ambiental como digna de atenção (\ldots) Se as empresas existentes de música enlatada se baseiam em regularizar ambientes cobrindo as idiossincrasias acústicas e atmosféricas, a Música AMbiental intenciona em melhorá-las. \textbf{Se o pano-de-fundo convencional é produzido removendo todo o sentido de dúvida e incerteza (e portanto o interesse genuíno) da música, a Música Ambiental retém essas qualidades} \cite[online]{eno_music_1978}
\footnote{\emph{The concept of music designed specifically as a background feature in the environment was pioneered by Muzak Inc. in the fifties, and has since come to be known generically by the term Muzak. The connotations that this term carries are those particularly associated with the kind of material that Muzak Inc. produces - familiar tunes arranged and orchestrated in a lightweight and derivative manner. Understandably, this has led most discerning listeners (and most composers) to dismiss entirely the concept of environmental music as an idea worthy of attention.(\ldots)Whereas the extant canned music companies proceed from the basis of regularizing environments by blanketing their acoustic and atmospheric idiosyncracies, Ambient Music is intended to enhance these. Whereas conventional background music is produced by stripping away all sense of doubt and uncertainty (and thus all genuine interest) from the music, Ambient Music retains these qualities.} Grifo nosso.}
\end{citacao} 

É interessante que neste contexto podemos fazer referência à \emph{Discreet Music} (1975) e \emph{Ambient 1: Music for Airports} (1978); tais peças fazem uso deliberado de laços de fitas que criam sistemas de atrasos (\emph{delays}). Se tais procedimento criativos não eram novos na época \footnote{Nesse sentido indicamos as peças de Steve Reich e Alvin Lucier e Pauline Oliveros já citadas.}; geralmente essas abordagens podem criar \emph{drones} que ficam variando em seu espectro.

Um exemplo interessante de \emph{live coding} com \emph{drones}, que utiliza pequenos impulsos sonoros no SuperCollider (ver \autoref{sc_drone}). Aqui o processo de escuta é contínuo, semelhante à práticas da música eletroacústica,onde algoritmos são pré-concebidos e modificados durante a performance.  Cole Ingraham utiliza um objeto que gera impulsos sonoros aperiódicos, e o mantêm como um plano sonoro que se mantêm, e vai sendo. Paralelamente, sons contínuos vai permeando este plano dos impulsos, a partir de senóides com um vibrato. É interessante notar que em alguns momentos, o \emph{live coder} deixa de programar para prestar atenção no resultado sonoro obtido, ao invés de ficar constantemente re-programando o código-fonte.

\begin{figure}
\begin{center}
\includegraphics[scale=0.5]{./imagens/sc_drone.png}
\caption{Improviso com \emph{drones} no SuperCollider \textbf{Fonte}: \url{https://www.youtube.com/watch?v=b8j4umQ2lIE}. }
\label{sc_drone}
\end{center}
\end{figure}

\subsection{Música eletrônica de Pista}\label{sec:musica_vanguarda_pista}

Somado a todos esses elementos, existe a imagem já descrita de um DJ que se utiliza destas técnicas e estéticas para controlar seu \emph{set} (embora seja possível assumir que esta personagem pode utilizar outros dispositivos, como sintetizadores e baterias eletrônicas, todos eles presentes em um computador) e promover uma música na qual a dança é elemento central. É importante deixar claro que uma cultura DJ possui suas peculiaridades dependentes de contexto social: por exemplo, um dj europeu/norte-americano está em uma configuração social diversa do dj latino-americano e africano. Se até o momento temos discutido práticas musicais, na sua maioria, a partir de pessoas que falam o inglês como sua primeira língua, é natural supor que esta cultura DJ que falamos no \emph{livecoding} faz parte de uma cultura anglófona.

Esse fenômeno de apropriação das MA, MG, MP e DJ, na cultura musical dos países rotulados como desenvolvidos, pode ser entendido a partir daquilo que Fernando  \citeonline{iazzetta_musica_2009} chamou de "sinergia de produções, que em sua diversidade compartilham dos mesmos elementos sociais e culturais" \cite[p.~152]{iazzetta_musica_2009}. Mais especificamente, trata-se de um fenômeno em sentido mais amplo das produções musicais usando o computador; mas buscarei encarar a questão no \emph{livecoding} da mesma maneira, como uma sinergia entre boates, casas noturnas e ambientes informais com ambientes de produção de conhecimento (universidades, escolas de música, faculdades de engenharia).

Primeiro é necessário esclarecer se a apropriação foi intencional ou não.  O discurso do manifesto de \citeonline{ward_live_2004} indica uma consciência parcial dos autores a respeito desta sinergia. No entanto é discutível se esse cruzamento de gêneros é mais um reflexo das transformações sociais e culturais que o computador trouxe consigo do que algo intencional: o uso deliberado de muitas ferramentas de um estúdio portátil, a expansão de possibilidades de produção musical através das redes de computadores, a troca de informações a respeito do uso de novos \emph{softwares} entre músicos acadêmicos e não-acadêmicos , trouxe à tona diferentes comunidades daquela ou outra tecnologia. De forma semelhante, no \emph{livecoding}, as novas possibilidades tecnológicas em contato com modos de fazer música já estabelecidos possiblitaram a emancipação de novas práticas, o que por sua vez, capacitou o cruzamento de estéticas (tudo isso, no entanto, visto apenas pela ótica das culturas de pessoas que falam o inglês como primeira língua).

Poderia ser questionado se o intercruzamento de gêneros musicais no \emph{livecoding} depende dos \emph{softwares}.  \citeauthoronline{iazzetta_musica_2009} responderia não, mas que dependem de uma articulação entre produções musicais e seus compositores, que carregam diferentes formações teóricas. Esse problema é colocada da seguinte forma:


Embora seja possível considerar uma "comunidade Max/MSP", ela está dentro de uma população contendo as comunidades "SuperCollider", "PureData", "ChucK", indicando apenas alguns. 

Dentro dessa população, surgem as pequenas comunidades de \emph{softwares} de \emph{live coding}, pela utilização e invenção de mini-linguagens pouco usadas, se comparadas com as do parágrafo acima. No contexto anglófono, essas pequenas comunidades apropriam um termo, segundo \citeonline{collins_algorave:_2014}, \emph{algorave}. 

Algorave é  um tipo de música eletronica de pista, que se utiliza de algoritmos, processos, e teorias generativas;  não se configura como uma música de pista normal, em seu processo de criação, mas reproduz alguns regimes de escuta, como \emph{dance}, \emph{drum'n'bass}, \emph{cyberpunk}, etc. Nesse processo de criação, é comum utilizar alguns \emph{softwares} já comentados, como o \emph{iXiLang} e o \emph{Tidal}, o que leva a crer na emancipação de ``comunidades iXiLang'' e ``comunidades Tidal''.

No entanto, \emph{algorave} é um termo anterior ao advento do \emph{live coding} e ao uso dos \emph{softwares} supracitados:

\begin{citacao}
\emph{Algorave} não é sustentado exclusivamente por \emph{live coders}, mas estes têm mantido uma forte presença em todos os eventos até agora. É assim talvez, porque a tradição do \emph{live coding} de projetar telas motiva todo o esforço; onde algoritmos não estão visíveis por períodos de tempo durante uma algorave, se corre o risco das coisas parecerem muito como um evento de música eletrônica padrão. \cite[p.~356]{collins_algorave:_2014} \footnote{Tradução nossa de \emph{Algorave is not exclusively a preserve of live coders, but they have maintained a strong presence at every event thus far. This is perhaps because the live coding tradition of projecting screens help motivates the whole endeavour; where algorithms are not made visible for periods during an algorave, we run the risk of things feeling much like a standard electronic music event.}}
\end{citacao}


\citeauthoronline{collins_algorave:_2014} apresenta dados a respeito da história da \emph{algorave}: em 1992 uma performance entitulada \emph{Cybernetic Composer} de Charles Ames; passando pelo \emph{Aphex Twin} (Richard David James), que  reinvindica em 1997 o termo (interessante do ponto de vista de gênero musical) \emph{live club algorithm}; em 2000 o \emph{Slub}, citado no inicio deste capitulo (na época Adrian Ward, Alex McLean), realizam performances, autodenominadas \emph{generative techno}, com abordagem \emph{gabba}; é interessante aqui o uso do termo \emph{club live coding}. Em 2001 é identificado a utilização de redes neurais para composição de padrões semelhantes ao \emph{drum'n'bass}. Em 2004 é fundado o TOPLAP, organização internacional  de \emph{live coding}, em uma casa noturna de Hamburgo. \footnote{\loccit{collins_algorave:_2014}.}

 



\chapter{Estudos de casos}\label{cap:estudos_de_caso}

Comparo três casos do pesquisador australiano Andrew Sorensen que considero simbólicos. A comparação será feita com as improvisações \emph{Study in Keith} (2009), 

\section{Study in Keith} 

\section{The Disklavier Sessions}

Neste trabalho, Sorensen controla um piano. Através da programação com o software \emph{Extempore}, Sorensen controla o mecanismo interno de um piano acústico fabricado pela Yamaha. Esse piano, batizado de \emph{Disklavier}, é peculiar por permitir a memorização (em componentes eletrônicos inseridos no corpo acústico) de uma performance.

É interessante nesta memorização, o processo de codificação dos eventos musicais em eventos MIDI.  

 Nesse vídeo, além do controle do instrumento musical
através de dispositivos adequados, o músico externaliza para o público o processo de
criação e edição do código-fonte que controla o dispositivo.

\section{Strange Places}

\chapter{Conclusão}


% ---
% Conclusão (outro exemplo de capítulo sem numeração e presente no sumário)
% ---em numeração e presente no sumário)
% ---
%\chapter*[Conclusão]{Conclusão}\addcontentsline{toc}{chapter}{Conclusão}\label{conclusao}

A pesquisa tomou um rumo diferente da proposta inicial, : um estudo que discutiria a questão da aparente intimidação experenciada por compositores no uso de linguagens de programação para composição musical, especificamente em ambientes de redes de computadores, bem como a elaboração de um \emph{software} original com base nessa reflexão. 

Para chegar neste aplicativo, tomei conhecimento de uma cena emergente no que é conhecido hoje como \emph{live coding} em navegadores de internet\footnote{\emph{Apple Safari}, \emph{Google Chrome}, \emph{Mozzilla Firefox}.}, tais como \emph{Gibber} \cite{roberts_gibber:_2012}, \emph{Vivace} \cite{vieira_vivace:_2015}, \emph{Wavepot}\footnote{\url{http://www.wavepot.com}, acessado em \today}, \emph{Html5Bytebeat}\footnote{\url{https://github.com/greggman/html5bytebeat}, acessado em \today.}. Uma reflexão a respeito das linguagens nestes aplicativos possibilitou o desenvolvimento de um aplicativo \emph{web} em conjunto com Luíz Schiavonni, professor da UFSJ (Universidade Federal de São João del-Rei), que foi chamado de \emph{Termpot}\footnote{\url{http://jahpd.githb.io/termpot}, acessado em \today.}. 

No entanto, o Programa de Pós-graduação em Artes, Cultura e Linguagens  (PPG-ACL/UFJF) proporcionou o contato com obras  de autores como \citeonline{kuhn_structure_1970}, \citeonline{feyerabend_against_1975} e \citeonline{santos_filosofia_2008} que foram significativas para aprimorar a metodologia; arrisquei-me a uma pesquisa orientada a paradigmas, aqui, paradigmas do \emph{livecoding}, considerando a existência cooperativa/competitiva de um conjunto deles em ambientes de produção de conhecimento (incluímos aqui a Universidade). Adicionalmente, a vivência com uma botânica dedicada em problemas de plano de manejo de uma espécie vegetal em Ubatuba/SP e um comunicador debruçado em problemas de gênero em Juiz de Fora/JF, me colocou na posição de refletir sobre uma questão, pessoalmente mais fundamental que a proposta inicial, de uma ecologia de gêneros musicais utilizando um fragmento de produções do \emph{livecoding}\footnote{Em comunicação pessoal com Tiago Rubini, ``a primeira menção à palavra 'gênero' se refere à questão de identidade de gênero. A segunda, sobre dinâmicas de gêneros musicais''. Infiro que uma dinâmica de gêneros musicais pressupõe um conjunto limitado de conhecimentos, que exige ser discutido para oferecer uma idéia das próprias limitações do conhecimento discutido.}. Autores citados acima não foram utilizados neste trabalho para discutir o \emph{livecoding}, mas foram importantes para a a concepção do seguinte triangulamento metodológico realizado: levantamento qualitativo do \emph{livecoding}, a partir de uma bibliografia básica disponível\footnote{\url{http://toplap.org/wiki/Videos,_Articles_and_Papers}. Acessado em \today.} e verificação quantitativa, através de levantamento de dados seletivo na internet que confirmasse o levantamento qualitativo em um nicho específico\footnote{Utilizando um SDK disponibilizado pelo \emph{Soundcloud}.}.


A partir de diferentes textos, derivei palavras-chave que carregam significados do objeto de pesquisa: \emph{live coding}, \emph{live-coding}, \emph{livecoding}, \emph{live code}, \emph{conversational programming}, \emph{on-the-fly programming}, \emph{live algorithm programming}; estas palavras-chaves apontam para uma prática de improvisação utilizando o computador, articulada de maneira colaborativa ou competitiva entre \emph{performers}, mediada por projetores visuais e acústicos (projetores e alto-falantes), e direcionadas para um público parcialmente passivo em ato de ouvir música e ver imagens (codificadas durante a improvisação). Este aspecto de improvisação com \emph{scripts} criados e editados no computador, segundo \citeonline{cox_coding_2004}, reside em uma capacidade de predizer (de forma aproximada) resultados complexos antes de codificar, isto é, o  executante deve ser capaz de realizar em um curto espaço de tempo uma formalização lógica de um comportamento audiovisual antes de sua programação em um \emph{script}: 

\begin{citacao}
Um programador é, portanto, capaz de predizer e especular sobre como o seu código irá se comportar em circunstâncias mais usuais. Como com qualquer coisa que é de autoria, a questão da subjetividade é inevitável, uma vez que qualquer resultado particular pode ser alcançado em diferentes (e muitas vezes concorrentes) maneiras. Nesse sentido, qualquer senso de improvisação depende de um entendimento preditivo de sistemas complexos e geradores. \cite[p.~169]{cox_coding_2004}\footnote{Tradução nossa de: \emph{A programmer is therefore able to predict and speculate upon how their code will behave in most usual circumstances. As with anything that is authored, the issue of subjectivity is unavoidable, since any particular result can be achieved in different (and often competing) ways. In this, any sense of improvisation relies on a preditive understanding of complex and generative systems}.} 
\end{citacao}

Colocado de outra forma por \citeonline{ruthmann_teaching_2010}

\begin{citacao}
Executar de maneira efetiva uma manipulação em tempo-real do código (live coding musical) para criar e formalizar música gerada requer ambos entendimentos musicais e computacionais. De uma perspectiva musical, é necessário entender o fluxo de como uma música generativa deve soar. Da perspectiva computacional, é preciso entender como o código deve ser ajustado e manipulado em tempo real para atingir mudanças aurais e musicais desejadas. \cite[p.~3-4]{ruthmann_teaching_2010}. \footnote{Tradução nossa de \emph{Performing effective real-time manipulation of code (musical live coding) to create and shape generated music requires both musical and computational understanding. From a musical perspective, one needs to understand how the ongoing, generative music should sound. From a computational perspective, one needs to understand how the code can be adjusted and manipulated in real time to achieve the aural and musical changes and outcomes one desires}.}
\end{citacao}

Esses conhecimentos pré-concebidos antes de codificar podem ter origem no conhecimento de mundo da Música daquele que realiza a improvisação; de diferentes maneiras, músicos-programadores adequam programação-partituras para determinados contextos do fazer musical. Expressa-se uma teoria musical que regula os algoritmos, esta por sua vez regulada por uma ecologia de gostos musicais adequadas para cada contexto. Interessado na dinâmica dos gêneros musicais que emergem de diferentes contextos de improvisação, coloquei-me a tarefa de discutir o que diferentes autores debruçados no assunto \emph{livecoding} entendem por música, como realizam, e seus discursos da respeito de realizações em público. 

Decidi recortar o assunto a partir de comparações limitadas ao âmbito musical por julgar-me incapaz de abordar diferentes linguagens artísticas como a literatura generativa e práticas derivadas de exibição simultanea da imagem renderizada e seu respectivo código; algumas menções foram feitas apenas para delimitar o que esté fora e o que está dentro de uma janela de discussão. Uma bibliografia básica foi levantada como forma de esclarecer origens e práticas musicais do \emph{livecoding}: \begin{inparaenum}[]
\item \citeonline{cox_aesthetics_2000},
\item \citeonline{cox_coding_2004},
\item \citeonline{collins_live_2003},
\item \citeonline{mclean_hacking_2004},
\item \citeonline{wang_--fly_2004},
\item \citeonline{ward_live_2004},
\item \citeonline{collins_live_2007},
\item \citeonline{rohrhuber_improvising_2009},
\item \citeonline{ruthmann_teaching_2010},
\item \citeonline{mclean_visualisation_2010},
\item \citeonline{magnusson_algorithms_2011},
\item \citeonline{magnusson_herding_2014},
\item \citeonline{magnusson_scoring_2014}
\end{inparaenum}.


O \emph{livecoding} pode existir em contextos tradicionalmente reservados para concertos ou em contextos informais que estimulam a sociabilização através da dança (os \emph{Night Clubs} de \citeonline{mclean_hacking_2004}); noto que, em um período de aproximadamente dez anos, ocorre um movimento de cooperação entre espaços acadêmicos e de entretenimento, onde emerge o termo \emph{algorave}\footnote{``Nenhuma conferência acadêmica está completa sem uma \emph{algorave}, uma chance de dançar algoritmos com velhos ou novos amigos. Teremos pelo menos um clube de noite, no excelente co-op Wharf Chambers $[$\url{http://www.wharfchambers.org/}$]$. em \url{http://iclc.livecodenetwork.org/cfp.html\#performance}. Tradução nossa de: \emph{No academic conference is complete without an algorave, a chance to dance to algorithms with friends new and old. We will have at least one club night, at the excellent Wharf Chambers co-op. More details to follow.}}. Por outro lado, estes mesmos espaços estimularam o desenvolvimento de novas áreas de pesquisa, entre elas, o \emph{livecoding} de ambientes virtuais, isto é, a prática sendo aplicada em pequenas ou grandes redes de computadores; ao mesmo tempo, o advento da biblioteca \emph{WebAudio API} possibilitou a experimentação do \emph{livecoding} na internet\footnote{Como os aplicativos \emph{Gibber}, \emph{Wavepot} e \emph{Html5Bytebeat}, \emph{Vivace}. Para mais informações a respeito, sugiro a leitura de "The Web Browser As Synthesizer And Interface" de \citeonline{roberts_web_2013} e "The Viability of the Web Browser as a Computer Music Platform"\citeonline{wyse_viability_2014}.}.

%No primeiro modo, existe uma pretensa sociabilização entre agentes. Por agentes me refiro aos membros do grupo de executantes e membros do grupo do público. Por sociabilização, entendo a interação entre membros. Esta separação ilustra uma característica observada (e praticada pelo autor deste trabalho) até o momento: em algumas performances ao vivo e vídeos da internet, ou em apresentações, a sociabilização entre membros executantes se caracteriza como ativa (modificação  da improvisação através da edição do código compartilhado ou por interferências sonoras), e a sociabilização entre executantes/público tem uma tendência a ser passiva (os membros do público são apenas convidados a observar e a escutar o que está sendo realizado). Tais agrupamentos ainda podem ser organizados em: \begin{inparaenum}[\itshape 1)\upshape] \item agrupamentos institucionalizados (como as \emph{Laptop orchestras}); e \item agrupamentos informais (\emph{solos}, \emph{duos}, \emph{trios}) \end{inparaenum}. 

Destes dois modos busquei fazer um mapeamento de pŕaticas musicais mencionadas por autores de manifestos e artigos mencionados no segundo parágrafo desta conclusão, bem como a localização geográfica e que práticas musicais são colocadas no plano de discussão em relação ao \emph{livecoding}: \begin{inparaenum}[\itshape 1)\upshape]
\item Música Algorítmica (MA),
\item Música Processual (MP), 
\item Música Generativa (MG),
\item e Música de Pista (ou o que que denominamos a partir do termo simplificado \emph{Disk Jockey}, (DJ)
\end{inparaenum}. Busquei averiguar até que ponto o uso dos termos referidos estão de acordo com definições compartilhadas, respaldados em um embate com autores como \citeonline{reich_music_1968}, \citeonline{eno_music_1978}, \citeonline{kramer_sonification_1999}, \citeonline{roads_times_2001}, \citeonline{wooler_framework_2005} \citeonline{malt_concepts_2006}, \citeonline{walker_auditory_2006}, \citeonline{essl_algorithmic_2007},  \citeonline{cope_prefacio_2008}, \citeonline{iazzetta_musica_2009}, \citeonline{mailman_agency_2013} e \citeonline{collins_algorave:_2014} \citeonline{casteloes_conversores_2015}. 

Em um segundo momento, levantamos dados pertinentes ao tema na rede social \emph{Soundcloud} como maneira de justificar o primeiro mapeamento das comunidades de gosto.

Realizando uma retrospectiva geopolítica dos autores, isto é, observando as localizações do globo terrestre em que foram escritos os textos supracitados no primeiro parágrafo deste capítulo, outros em \emph{passim}, mais um conjunto de dados que podem ser observados no \autoref{dados_sclivecoding} -- que descrevem essas regiões de maneira quantitativa em um período de tempo entre 2008 e 2015--, é possível confirmar uma tese que foi sendo construída no decorrer da pesquisa, mas que somente pode ser verbalizada no segundo parágrafo de ``A genealogia da moral'' de Nietzsche, ``o privilégio senhorial de dar nomes permitem-nos conceber a origem da linguagem ela mesma como uma manifestação de poder dos governantes''\footnote{Cf. \emph{On the Genealogy of Morality} Edited by Keith Ansell-Pearson. Translated by Carol Diethe. Cambridge. 2006. Tradução nossa do segundo parágrafo do primeiro ensaio \emph{The seigneurial pribilege of giving names even allow us to conceive of the origin of a language itself as a manifestation of the power of the rulers}.}, no sentido de que o \emph{live coding} possue esse nome, como \emph{programa de pesquisa}, que integra todo uma Epistemologia do Norte que que \citeauthoronline{santos_filosofia_2008} tanto comenta.

Na inglaterra identifico os autores como Alex McLean, Nick Collins Adrian Ward, e Dave Griftths. Nos EUA identifico as \emph{Laptop Orchestras}, em Stanford e Princeton, bem como compositores como James Harkins e Joshua Parmenter e Ge Wang;  na Austrália, tem sido notável o papel de Andrew Sorensen na \emph{Queensland University of Technology} utilizando o piano; no Brasil identifico os trabalhos de Bernardo Barros, André Damião, Antônio Goulart, Vilson Vieira, Geraldo Magela de Castro Rocha Junior, Caleb Mascarenhas Luporini, Daniel Penalva, Ricardo Fabbri, Renato Fabbri, Ricardo Brasileiro e Daniel Penalva e Flávio Luiz Schiavonni. 

No levantamento de dados do Soundcloud pude confirmar alguns dos pontos levantados no levantamento bibliográfico e questionar algumas afirmações feitas: por exemplo, na questão de qual país produz grande quantidade de livecoding confirmei posições de países falantes da língua inglesa, como Inglaterra e EUA, mas ao mesmo tempo, notei uma grande quantidade de produções anônimas, isto é, com localizações não identificadas, desacreditando em uma centralização de produção; ademais foi possível perceber uma miríade de produções em países como Alemanha, México e Japão. Na questão de gênero musical, articulado através de \emph{hashtags}\footnote{Para mais informações, sugiro a leitura de \url{https://en.wikipedia.org/wiki/Hashtag} e \url{https://en.wikipedia.org/wiki/Tag_(metadata)}, acessados em \today.}, pude confirmar uma hibridização de práticas musicais consideradas anteriormente distintas, e ao mesmo tempo, questionar uma centralização conceitual em um ou outro termo, como \begin{inparaenum}[\itshape a)\upshape]
\item \emph{algorithmic music}
\item \emph{algorave},
\item \emph{algopop},
\item \emph{bytebeat},
\item \emph{drone}
\item \emph{electronic music},
\item \emph{electroacoustic},
\item \emph{glitch},
\item \emph{noise},
\item \emph{whistling}
\end{inparaenum}.

O seguinte ponto-de-vista foi desenvolvido: cada termo utilizado no \emph{livecoding} carrega uma teoria musical e, por outro lado, costumes distintos do fazer musical no \emph{livecoding} estão relacionados com gêneros musicais, a partir  daquilo que  discute comunidades de gosto; tais comunidades de gosto foram investigadas em um nicho musical virtual específico, a saber, a rede social Soundcloud. 

Percebi o \emph{livecoding} como um campo de estudos em transição entre Música e Ciências da Computação; esse campo tem sido estimulado em países considerados como centros de referência, por exemplo, Inglaterra e EUA. Esse estímulo tem sido acompanhado por colaborações interdisciplinares formalizadas (isto é, reconhecidas institucionalmente); não coincidentemente, isso foi pressuposto por \citeonline{mathews_digital_1963} no desenvolvimento da família MUSIC N como única maneira de avançar neste campo de estudo\footnote{Lembro que o desenvolvimento de estudos musicais com CSIRAC (na Austrália),  antes mesmo do advento do MUSIC N, tem como um dos fatores de seu fracasso, segundo \citeonline{di_nunzio_genesi_2010}, devido a uma falta de cooperação entre músicos e cientistas da computação.}. No Brasil, no entanto, existem barreiras institucionais que impedem a colaboração interdisciplinar formal, porém estas mesmas barreiras forçam músicos interessados na área, a aprender a programar, mesmo com algumas barreiras do jargão técnico (muitas vezes intimidadoras); de maneira semelhante, cientistas (da computação e muitas vezes físicos) e não-acadêmicos brasileiros, com seu conhecimento de programação, exploraram comportamentos musicais em \emph{softwares} como forma de ultrapassar uma barreira impostas pelo jargão técnico dos músicos.


% ---
\postextual

% ---
% Bibliografia
% ---
\bibliography{main}

% ----------------------------------------------------------
% Glossário
% ----------------------------------------------------------
%
% Consulte o manual da classe abntex2 para orientações sobre o glossário.
%
%\glossary

% ----------------------------------------------------------
% Apêndices
% ----------------------------------------------------------

% Imprime uma página indicando o início dos apêndices
%\partapendices

% ---
% Inicia os apêndices
% ---
\begin{apendicesenv}

%\include{./apendice/dados_imagens}
%\include{./apendice/recursividade}
%\include{./apendice/webaudioapi_wavepotruntime}


\end{apendicesenv}


\phantompart
\printindex
%---------------------------------------------------------------------

\end{document}
