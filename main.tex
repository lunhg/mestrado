%% abtex2-modelo-trabalho-academico.tex, v-1.9.2 laurocesar

%%
%% This work may be distributed and/or modified under the
%% conditions of the LaTeX Project Public License, either version 1.3
%% of this license or (at your option) any later version.
%% The latest version of this license is in
%%   http://www.latex-project.org/lppl.txt
%% and version 1.3 or later is part of all distributions of LaTeX
%% version 2005/12/01 or later.
%%

%% This work has the LPPL maintenance status `maintained'.
%% 
%% The Current Maintainer of this work is the abnTeX2 team, led
%% by Lauro César Araujo. Further information are available on 
%% http://abntex2.googlecode.com/
%%
%% This work consists of the files abntex2-modelo-trabalho-academico.tex,
%% abntex2-modelo-include-comandos and abntex2-modelo-references.bib
%%
%------------------------------------------------------------------------
% ------------------------------------------------------------------------ 
% abnTeX2: Modelo de Trabalho Academico (tese de doutorado, dissertacao de
% mestrado e trabalhos monograficos em geral) em conformidade com 
% ABNT NBR 14724:2011: Informacao e documentacao - Trabalhos academicos -
% Apresentacao
% ------------------------------------------------------------------------
% ------------------------------------------------------------------------

\documentclass[
	% -- opções da classe memoir --
	12pt,				% tamanho da fonte
	openright,			% capítulos começam em pág ímpar (insere página vazia caso preciso)
	twoside,			% para impressão em verso e anverso. Oposto a oneside
	a4paper,			% tamanho do papel. 
	% -- opções da classe abntex2 --
	%chapter=TITLE,		        % títulos de capítulos convertidos em letras maiúsculas
	%section=TITLE,		        % títulos de seções convertidos em letras maiúsculas
	%subsection=TITLE,	        % títulos de subseções convertidos em letras maiúsculas
	%subsubsection=TITLE,           % títulos de subsubseções convertidos em letras maiúsculas
	% -- opções do pacote babel --
	english,			% idioma adicional para hifenização
    italian,                        % idioma adicional para hifenização
	brazil				% o último idioma é o principal do documento
	]{abntex2}

% ---
% Pacotes
% ---
% ---
% Pacotes básicos 
% ---
\usepackage[T1]{fontenc}
\usepackage[utf8]{inputenc}		
\usepackage{lmodern}			
\usepackage{lastpage}			
\usepackage{indentfirst}	

%\usepackage{tocstyle}
%\usetocstyle[tocgraduated]{KOMAlike}

\usepackage{color}				
\usepackage{graphicx}			
\usepackage{wallpaper}			
\usepackage{microtype} 		
\usepackage{url}
\usepackage[table,xcdraw]{xcolor}
\usepackage{tabularx}

\usepackage{minted}
\usepackage{amsmath,amssymb}
\usepackage{framed}
\usepackage[amsmath,framed]{ntheorem}
\usepackage{pdfpages}
\usepackage{xcolor}
%\usepackage{sclang-prettifier}
%\usepackage{xparse}
%\usepackage{ifmtarg}% http://ctan.org/pkg/ifmtarg
%\usepackage{textcase}

% ---
		
% ---
% Pacotes adicionais, usados apenas no âmbito do Modelo Canônico do abnteX2
% ---
\usepackage{lipsum}				% para geração de dummy text
% ---

%---
% Pacote para listas em uma linha
%---
\usepackage{paralist}

% ---
% Epigrade
% ---
\usepackage{epigraph}

% Front end para amsthm (\declaretheorem)
\usepackage{thmtools}    

% ---
% Pacotes de citações
% ---
\usepackage[brazilian,hyperpageref]{backref}	 % Paginas com as citações na bibl
\usepackage[alf]{abntex2cite}	% Citações padrão ABNT

%TODO
\usepackage[colorinlistoftodos]{todonotes}
%\usepackage{ntheorem}

\usepackage{tablefootnote}


%%%%%%%%%%% syntax highlight %%%%%%%%%%%%%%%%%%%%%%%%%%%%%%%%%%%%%%%%%%%%%%%%%%%%%
\usepackage{listings}
\definecolor{maroon}{rgb}{0.5,0,0}
\definecolor{darkgreen}{rgb}{0,0.5,0}
\definecolor{deepblue}{rgb}{0,0,0.5}
\definecolor{deepred}{rgb}{0.6,0,0}
\definecolor{purple}{rgb}{0.5,0,0.5}
\definecolor{deepgreen}{rgb}{0,0.5,0}
\definecolor{lightred}{rgb}{1,0.1,0.1}
%%%%%%%%%%%%%%%%%%%%%%%%%%%%%%%%%%%%%%%%%%%%%%%%%%%%%%% 

%\usepackage{tikz}
%\tikzset{
%  treenode/.style = {shape=rectangle, rounded corners,
%                     draw, align=center,
%                     top color=white, bottom color=blue!20},
%  root/.style     = {treenode, font=\Large, bottom color=red!30},
%  env/.style      = {treenode, font=\ttfamily\normalsize},
%  dummy/.style    = {circle,draw}
%}

%\usetikzlibrary{shadows}


% ---
% MACROS
% ---
\input{./macros}

% ---
% Informações de dados para CAPA e FOLHA DE ROSTO
% ---
\titulo{\emph{Live Coding}: um algoritmo gerador de uma sonoridade tonal em \emph{A Study in Keith} (2009)}
\autor{Guilherme Martins Lunhani}

\instituicao{Universidade Federal de Juiz De Fora -- UFJF
  \par
  Instituto de Artes e Design -- IAD
  \par
  Programa de Pós-Graduação em Artes, Cultura e Linguagens}

\orientador[Orientador: ]{Prof. Dr. Luiz Eduardo Castelões Pereira da Silva}

% \changes{Versão inicial }{2013/07/22 }{v0.0.3}
\tipotrabalho{Dissertação (Mestrado)}

% O preambulo deve conter o tipo do trabalho, o objetivo, 
% o nome da instituição e a área de concentração 
\preambulo{Dissertação apresentada ao Programa de Pós-Graduação em Artes, Cultura e Linguagens, para fins de obtenção do título de Mestre em Artes, Cultura e Linguagens da Universidade Federal de Juiz de Fora.}
%\EnableCrossrefs
%\CodelineIndex
%\RecordChanges

% ---
% Configurações de aparência do PDF final

% alterando o aspecto da cor azul
\definecolor{blue}{RGB}{41,5,195}

% informações do PDF
\makeatletter
\hypersetup{
     	%pagebackref=true,
		pdftitle={\@title}, 
		pdfauthor={\@author},
    	pdfsubject={\imprimirpreambulo},
	    pdfcreator={LaTeX with abnTeX2},
		pdfkeywords={abnt}{latex}{abntex}{abntex2}{trabalho acadêmico}, 
		colorlinks=true,       		% false: boxed links; true: colored links
    	linkcolor=blue,          	% color of internal links
    	citecolor=blue,        		% color of links to bibliography
    	filecolor=magenta,      		% color of file links
		urlcolor=blue,
		bookmarksdepth=4
}
\makeatother

%\newcommand{\todosautoresdelivecoding}{\begin{inparaenum}[]\item \citeonline{collins_live_2003},\item \citeonline{collins_generative_2003},\item \citeonline{collins_live_2003-1},\item \citeonline{wang_--fly_2004},\item \citeonline{ward_live_2004},\item \citeonline{blackwell_programming_2005},\item \citeonline{collins_live_2007},\item \citeonline{griffiths_fluxus:_2008},\item \citeonline{mclean_patterns_2009},\item \citeonline{rohrhuber_improvising_2009},\item \citeonline{mclean_visualisation_2010},\item \citeonline{magnusson_algorithms_2011},\item \citeonline{mccallum_show_2011},\item \citeonline{magnusson_herding_2014},\item \citeonline{magnusson_scoring_2014},\item \citeonline{collins_algorave:_2014},\item \citeonline{sorensen_livecodings_2014}\end{inparaenum}}
% --- 
% Espaçamentos entre linhas e parágrafos 
% --- 

% O tamanho do parágrafo é dado por:
\setlength{\parindent}{1.3cm}

% Controle do espaçamento entre um parágrafo e outro:
\setlength{\parskip}{0.2cm}  % tente também \onelineskip

% ---
% compila o indice
% ---
\makeindex

\makeindex

% ----
% Início do documento
% ----
\begin{document}
\pagenumbering{roman}
% Retira espaço extra obsoleto entre as frases.
\frenchspacing 

% ----------------------------------------------------------
% ELEMENTOS PRÉ-TEXTUAIS
% ----------------------------------------------------------
\pretextual

% ---
% Capa
% ---
\imprimircapa
% ---

% ---
% Folha de rosto
% (o * indica que haverá a ficha bibliográfica)
% ---
\imprimirfolhaderosto*
% ---

% ---
% Inserir a ficha bibliografica
% ---
% Isto é um exemplo de Ficha Catalográfica, ou ``Dados internacionais de
% catalogação-na-publicação''. Você pode utilizar este modelo como referência. 
% Porém, provavelmente a biblioteca da sua universidade lhe fornecerá um PDF
% com a ficha catalográfica definitiva após a defesa do trabalho. Quando estiver
% com o documento, salve-o como PDF no diretório do seu projeto e substitua todo
% o conteúdo de implementação deste arquivo pelo comando abaixo:
%
\begin{fichacatalografica}
    \includepdf{fig_ficha_catalografica.pdf}
\end{fichacatalografica}
%\begin{fichacatalografica}
%	\vspace*{\fill}					% Posição vertical
%	\hrule							% Linha horizontal
%	\begin{center}					% Minipage Centralizado
%	\begin{minipage}[c]{12.5cm}		% Largura
%	
%	\imprimirautor
%	
%	\hspace{0.5cm} \imprimirtitulo  / \imprimirautor. --
%	\imprimirlocal, 2016
%	
%	\hspace{0.5cm} \pageref{LastPage} p. ; 30 cm.\\
%	
%	\hspace{0.5cm} \imprimirorientadorRotulo~\imprimirorientador\\
%	
%	\hspace{0.5cm}
%	\parbox[t]{\textwidth}{\imprimirtipotrabalho~--~\imprimirinstituicao,
%	2016	
%	%\imprimirdata.
%	}\\
%	
%	\hspace{0.5cm}
%		1. Livecoding.
%		2. Study in Keith.
%		3. Sistemas criativos
%		I. Orientador: Prof. Dr. Luiz Eduardo Castelões Pereira da Silva
%		II. UFJF - Universidade Federal de Juiz de Fora.
%		III. Instituto de Artes e Design
%		IV. \imprimirtitulo \\ 			
%	
%	\end{minipage}
%	\end{center}
% end{fichacatalografica}
% ---


% ---
% Inserir errata
% ---
%\begin{errata}
%\end{errata}

% \includepdf{folhadeaprovacao_final.pdf}
%\input{./folha_aprovacao}

% ---
% Dedicatória
% ---
\include{./dedicatoria}
%\newpage
% ---
% Agradecimentos
% ---
%\begin{agradecimentos}
\newpage
\begin{flushright}
\huge{\textbf{Agradecimentos}}
\ \\
\small{Maiores agradecimentos a Deus.  
\ \\
Para uma família, pelo apoio, Jair, Olímpia e Júlia. 
\ \\
Aos moradores de rua de Juiz de Fora que deram sentido à fraqueza da pergunta deste trabalho.
\ \\
Aos Professores Dr. Luiz Eduardo Castelões, Dr. Alexandre Fenerich e Dr. Flávio Luiz Schiavonni, fundamentais no apoio institucional, na sugestão de leituras acadêmicas e extra-acadêmicas, na cobrança de prazos, nas críticas e nas conversas sobre Música e Computadores.
\ \\
À FAPEMIG por suprir a lacuna financeira em um momento delicado nas economias das Universidades Federais.
\\
Simone pelo Amor, e pela filosofia de Espinosa.
\ \\
Aos amigxs que estão (ou moraram em Juiz de Fora): Glerm Soares, Tiago Rubini, Anna Flávia, Dhiego e Luíza.
\ \\
Aos \emph{freakcoders} Daniel Penalva, Renato Fabbri e Vilson Vieira. }
\end{flushright}

\newpage
% ---

% ---
% Epígrafe do livecoding
% ---
\epigraph{Alguém poderia imaginar uma interface musical na qual um músico especifica o som resultante desejado, em uma linguagem descritiva, na qual poderia ser então traduzida em parâmetros de partículas e renderizados em som. Uma alternativa poderia especificar um exemplo: "Faça um som assim (arquivo de som), mas com pouco vibrato''}{Curtis \citeonline{roads_microsound_2001}}
\newpage

% ---
% RESUMOS
% ---

% resumo em português
\setlength{\absparsep}{18pt} % ajusta o espaçamento dos parágrafos do resumo
\begin{resumo}

Este documento discute uma versão sintetizada de uma técnica polivalente cujo nome é \emph{live coding}, suas construções históricas na Música, e uma simulação de improvisação tonal guiada por improvisação com linguagens de programação.

Na Introdução (ver p.~\pageref{cap:intro}) apresentamos uma definição de \emph{live coding}. A definição destaca o fazer musical, mas não exclúi outras potências artísticas. 

No \autoref{cap:introducao} (ver p.~\pageref{cap:introducao}) destacamos um mecanismo criativo desta técnica em dois contextos não musicais.

No \autoref{cap:protohistoria} (ver p.~\pageref{cap:protohistoria}) listamos  períodos de atividades musicais que prototiparam e formalizaram o mecanismo criativo do primeiro capítulo.

No \autoref{cap:estudos_de_caso} (ver p.~\pageref{cap:estudos_de_caso}) analisamos uma proposição musical, 
um vídeo intitulado \emph{A Study in Keith} de \citeonline{sorensen_keith_2009}, de acordo com o mecanismo mental do primeiro capítulo.

A contribuição deste trabalho para a musicologia brasileira é a organização historiográfica de uma técnica ainda pouco elaborada em português.

\vspace{\onelineskip}
\noindent
\textbf{Palavras-chaves}: Improvisação de códigos; Música Computacional;\emph{A Study in Keith}.
\end{resumo}

%%%%%%%%%% traduçoes resumo
% resumo em inglês
\begin{resumo}[Abstract]
 \begin{otherlanguage*}{english}
	This document presents a synthesized version of a versatile technique whose name is live coding, its historical buildings in music, and a   simulation of a tonal improvisation, guided by improvisation with programming languages.
	
In the Introduction (see p.~\pageref{cap:intro}), we present a definition of live coding. The definition highlights
a focus on music, but does not exclude other artistic powers.

In Chapter 1(see p.~\pageref{cap:introducao}), we highlight a creative mechanism of this technique in two unmusical contexts.

In Chapter 2 (see p.~\pageref{cap:protohistoria}), we listed periods of musical activities that prototyped and formalized the creative engine of the first chapter.

In Chapter 3 (see p.~\pageref{cap:estudos_de_caso}), we analyzed a musical proposition, a Sorensen and Swift's video entitled A Study in Keith (2009), according to the first mental mechanism chapter.

The contribution of this work to the Brazilian musicology is a historiographical organization of a technique still little developed in portuguese.
   \vspace{\onelineskip}
 
   \noindent 
   \textbf{Key-words}: Live Coding; Computer Music; \emph{A Study in Keith}.
 \end{otherlanguage*}
\end{resumo}

% resumo em francês 
\begin{comment}

\begin{resumo}[Résumé]
 \begi'n{otherlanguage*}{french}
    Il s'agit d'un résumé en français.
 
   \textbf{Mots-clés}: latex. abntex. publication de textes.
 \end{otherlanguage*}
\end{resumo}

% resumo em espanhol
\begin{resumo}[Resumen]
 \begin{otherlanguage*}{spanish}
   Este es el resumen en español.
  
   \textbf{Palabras clave}: latex. abntex. publicación de textos.
 \end{otherlanguage*}
\end{resumo}
% ---
\end{comment}


% ---
% inserir lista de tabelas
% ---
%\pdfbookmark[0]{\listtablename}{lot}
%\listoftables*
%\cleardoublepage
% ---
% inserir lista de abreviaturas e siglas
% ---
%\input{./siglas}
% ---
% inserir o sumario
% ---

\pdfbookmark[0]{\contentsname}{toc}
\tableofcontents*
\cleardoublepage

% ---
% inserir lista de ilustrações
% ---
\pdfbookmark[0]{\listfigurename}{lof}
\listoffigures*
\cleardoublepage

% ----------------------------------------------------------
% ELEMENTOS TEXTUAIS
% ----------------------------------------------------------
\textual
%\newpage
%\input{./aviso}

% ----------------------------------------------------------
% Introdução (exemplo de capítulo sem numeração, mas presente no Sumário)
% ----------------------------------------------------------
\input{./introducao}
\newpage

%------------------------------------------------------
% PRINCIPAL
% ----------------------------------------------------------
%\part{Ecologia de saberes no \emph{livecoding}}\label{parte1}
\pagenumbering{arabic}
\include{./01-Introducao}
%\newpage
\chapter{Definições Históricas da Improvisação de códigos}\label{cap:protohistoria}

Este capítulo desenvolve um contexto cronológico da improvisação de códigos, do ponto de vista musical. \citeonline{mori_pietro_2015} descreve um caso prematuro de \emph{live coding} na Itália, com o compositor Pietro Grossi \ver{sec:grossi}. As atividades dos grupos californianos \emph{The League of Automatic Composers}/\emph{The Hub} contextualizam o ambiente cultural estadounidense \ver{sec:baiasaofranscisco}. Deste ambiente, o compositor Ron Kuivila propõe em Amsterdã uma improvisação de códigos prototípica, sem projeção \ver{sec:kuivila}. No começo dos anos 2000, sete programadores ingleses respondem à crítica de \citeonline{schloss_dilemma_2003}, ou o papel cênico do músico durante uma apresentação com computadores \ver{sec:laptoptoplap}. 

\section{Pietro Grossi}\label{sec:grossi}

O compositor veneziano Pietro Grossi foi  um dos pioneiros da \emph{Computer Music} Italiana. Sacrificou questões timbrísticas e focou na \emph{responsividade} \ver{sec:reflex} e na \emph{telemática}\ver{sec:tel}. 

Segundo \citeonline[p.~126]{mori_pietro_2015}:

\begin{citacao}
\traducao{Grossi começou a se interessar por música computacional durante a primeira metade do anos 60, quando ele organizou um programa de rádio centrado em torno de uma "música inovadora" \cite{giomi_conversasioni_1999}. Contudo, a primeira experiência de Grossi com um computador foi em Milão, no Centro de Pesquisa Elétrica da Olivetti-General. Aqui, auxiliado por alguns técnicos internos e engenheiros, ele conseguiu compor e gravar alguns de seus primeiros trabalhos em música computacional. Eles foram, em sua maior parte, transcrições de música clássica ocidental. Contudo, houve algumas exceções, por exemplo, uma faixa chamada Mixed Paganini.}{Grossi began to be interested in computer music during the first half of the 1960s, when he hosted a radio program centred around “innovative music” in general (Giomi1999). However, the first Grossi's experience with calculator took place in Milan, in the Olivetti-General Electric Research centre. Here, aided by some internal technicians and engineers, he managed to compose and record some of his first computer music works. They were, for the most part, transcriptions of Western classical music. However, there were some exceptions, for example a track called Mixed Paganini.}
\end{citacao}

Uma cópia do disco \emph{GE-115 - Computer Concerto}, gravado no \emph{Studio di Fonologia musicale di Firenze} e lançado pela Olivetti em 1967, possui a seguinte descrição das composições de Grossi: \traducao{``Do lado A existem algumas transcrições de música clássica, e do lado B existem três canções originais. (\ldots) Este 7''$[$polegadas$]$ foi distribuído como presente de natal e de ano novo pela companhia Olivetti''.}{On side A there's transcribed classical music, on side B there are three original songs. (\ldots). This 7" was distributed as a christmas and new year gift by the Olivetti company.}\disponivelem{https://www.youtube.com/watch?v=ZQSP_wF7wSY}. No entanto, é necessária uma correção sobre o lado A, e um detalhe do lado B\disponivelem{https://www.discogs.com/Studio-Di-Fonologia-Musicale-Di-Firenze-GE-115-Computer-Concerto/release/575632}. As transcrições realizadas foram da \emph{Oferenda Musical BWV 1079} de J.S.Bach e o quinto dos 24 Caprichos de Nicolò Paganini. As peças originais de Grossi foram três:  \emph{i})\emph{Mixed Paganini} (derivado do 5$^o$ capricho) ; \emph{ii}) \emph{Permutations Of Five Sounds} e; \emph{iii}) \emph{Continuous}: 

\begin{citacao}
\traducao{Praticamente, Grossi modificou, auxiliado por alguns programas rudimentares, o material sonoro original. (\ldots) Uma coleção posterior dos Capricci de Paganini, gravado em Pisa, foi revista por Barry Truax na Computer Music Journal \cite{truax_barry_1984}. \cite[p.~126]{mori_pietro_2015}.}{Practically, Grossi modified, aided by some rudimental music programs, the original sound material. (\ldots) A later collection of Paganini’s Capricci, recorded in Pisa, was reviewed by Barry Truax on Computer Music Journal (Truax1984).} 
\end{citacao}

\subsection{Reflexividade}\label{sec:reflex}

Grossi não fica satisfeito com o trabalho, e a Olivetti não se interessa mais por suas pesquisas. Ao procurar emprego e novos espaços criativos, é contratado pelo \traducao{``Centro de pesquisa IBM, dentro do Comitê Nacional para a Pesquisa''}{IBM Research Centre in Pisa, inside the CNR Institute (Centro Nazionale per la Ricerca: National Research Committee)} \cite[p.~126]{mori_pietro_2015}. Ali desenvolveu, em linguagem FORTRAN, o DCMP (\emph{Digital Computer Music Program}), um programa integrado com um terminal de vídeo e um teclado alfanumérico, e segundo Mori, ao usar este terminal de áudio, o compositor escolheu deliberadamente abandonar o problema do timbre.  Esta abordagem parte de uma abordagem ``preguiçosa'' (\emph{prigo}). Grossi dizia sobre si mesmo, como ``uma pessoa que está consciente de que o seu tempo é limitado e não quer perder tempo em fazer coisas inúteis ou na espera de alguma coisa quando não é necessária.''\footnote{Tradução nossa de \emph{a person who is aware that his or her time is limited and do not want to waste time in doing useless things or in waiting for something when it is not necessary.}} (\idemibdem). 
Propomos substituir \emph{prigo} por \emph{reflexivo}, ou a \traducao{``habilidade de um programa manipular como dados algo que representa o estado do programa durante sua própria execução, o mecanismo para codificação de estados de execução é chamado \emph{reificação}\''.\cite[p.~1]{malefant_reflection_1996}.}{the ability of a program to manipulate as data something representing the state of the program during its own execution, the mechanism for encoding execution states as data being called reification.}

Este sentido "preguiçoso" ou reflexivo levou Grossi a advogar que novos timbres gerados por computador deveriam esperar por melhores implementações de \emph{hardware}:

\begin{citacao}
\traducao{(\ldots) o intéprete era capaz de produzir e reproduzir música em tempo real, digitando alguns comandos específicos e os parâmetros composicionais desejados. O som resultante vinha imediatamente depois da operação de decisão, sem qualquer atraso causado por cálculos. Haviam muitas escolhas de reprodução no programa: era possível salvar na memória do computador peças de músicas pré-existentes, para elaborar qualquer material sonoro no disco rígido, para administrar arquivos musicais e iniciar um processo de composição automático, baseado em algoritmos que trabalham com procedimentos ``pseudo-casuais''. Existia também uma abundância de escolhas para mudanças na estrutura da peça. Um dos mais importantes aspectos do trabalho de Grossi foi que todas intervenções eram instantâneas: o operador não tinha que esperar pelo computador terminar todas operações requisitadas, e depois ouvir os resultados. Cálculos de dados e reprodução sonoras eram simultâneos. \textbf{Esta simultaneidade não era comum no campo da \emph{Computer Music} daquele tempo, e Grossi deliberadamente escolheu trabalhar desta forma, perdendo muito no lado da qualidade sonora. Seu desejo era poder escutar os sons resultantes imediatamente} (\idemibdem).}{(\ldots) the performer was able to produce and reproduce music in real time by typing some specific commands and the desired composition's parameters. The sound result came out immediately after the operator's decision, without any delay caused by calculations. There were many reproduction choices inscribed in this software: it was possible to save on the computer memory pieces of pre-existing music, to elaborate any sound material in the hard disk, to manage the music archive and to start an automated music composition process based on algorithms that worked with “pseudo-casual” procedures. There were also plenty of choices for piece structure modifications. One of the most important aspects of Grossi’s work was that all the interventions were instantaneous: the operator had not to wait for the computer to finish all the requested operations and then hear the results. Data calculation and sound reproduction were simultaneous. This simultaneity was not common in the computer music field of that time and Grossi deliberately chose to work in this way, losing much on the sound quality’s side. His will was to listen to the sound result immediately.}
\end{citacao}

\citeonline[p.~127]{mori_pietro_2015} destaca o compositor consciente dos problemas técnicos, e de um descarte pelo pensamento timbrístico corrente na Europa:

\begin{citacao}
\traducao{O DCMP foi compilado na fase inicial do desenvolvimento de tecnologias computacionais. Naquele tempo, os recursos de cálculo eram escassos e, para obter a reprodução em tempo-real citada, era necessário pedir por pouca quantidade de dados. Contudo, o músico veneziano foi capaz escrever um programa muito leve, capaz de modificar somente os parâmetros necessários para um cálculo de recursos reduzidos: altura e duração. A síntese de timbres necessita de uma quantidade imensa de dados, e então a escolha foi descartá-la temporariamente, e todos os sons eram reproduzidos com o timbre de uma onda quadrada Esta forma de onda era gerada por extração do estado binário do \emph{pin} de saída da placa mãe que controla o programa. Essa saída tinha um único \emph{bit}, e então a onda sonora gerada era o resultado desta mudança do estado binário. Desta forma, o computador não emprega quaisquer recursos para calcular a síntese sonora, economizando-os para o processo de produção musical. Grossi não estava interessado na qualidade da saída sonora em sua primeira fase em Pisa. O que importava particularmente era a capacidade em trabalhar em tempo real, ou, em outras palavras, para ter a escolha de escutar imediatamente ao que ele escreveu no teclado do terminal de vídeo \apud{giomi_conversasioni_1999}{mori_pietro_2015}.}{The DCMP was compiled in the early phase of computer technology development. At that time, the calculation resources were low and, to obtain the just cited real time reproduction, it had to ask for very low quantity of data. Therefore, the Venetian musician chose to write very light software, able to modify only parameters that required a few calculation resources: pitch and duration. Timbre synthesis needed a big amount of data, so that choice was temporarily discarded and all the sounds were reproduced with square wave timbre. This waveform was generated by extracting the binary status of a motherboard's exit pin controlled by the software. This exit had only one bit, so the sound wave generated was the result of this bit status changing. In this way, the computer did not employ any resources for calculating the sound synthesis, saving them for music production process. Grossi was not very interested in the quality of sound output in this first phase in Pisa. What he cared particularly was to be able to work in real time, or, in other words, to have the choice to listen immediately to what he typed on the video terminal’s keyboard.}
\end{citacao}

\subsection{Telemática}\label{sec:tel}

É importante situar que a escolha deliberada para o DCMP é justificada nos anos 70. Até a metade da década, Grossi foi capaz de implementar melhorias de timbre, \traducao{``digitalmente controladas, mas com uma tecnologia de síntese analógica. Foi lançado em 1975 e foi chamado de TAU2'' (\emph{Terminale Audio 2$^a$ versione -- Terminal de Áudio 2$^a$ versão}) (\idemibdem).}{digitally controlled but with analog sound synthesis technology. It was launched in 1975 and called TAU2 (Terminale Audio 2a versione – Audio Terminal 2nd version)}. Esta tecnologia tinha um programa, o TAUMUS, uma modificação do DCMP, que podia tocar:

\begin{citacao}
\traducao{(\ldots) até doze vozes simultâneas. Estas doze vozes eram divididas em três grupos, compostos de quatro canais cada. O operador poderia atribuir um timbre diferente para cada grupo, que era modulado usando síntese aditiva com sete sobretons. Cada sobretom era controlado individualmente pelo programa.}{(\ldots) twelve different voices simultaneously. These twelve voices were divided in three groups, composed of four channels each. The operator could choose to assign a different timbre to every single group, which was modulated using additive synthesis with seven overtones. Every overtone could be controlled individually by software.}
\end{citacao}

Segundo \citeonline[p.~128]{mori_pietro_2015}, uma outra novidade do TAU2-TAUMUS, em relação às concepções do DCMP, era o conceito de modulação de modelos (\emph{modelli modulanti}), ou \traducao{``uma espécie de remendos que agem em um parâmetro musical''}{they were a sort of patches that acted on some musical parameter.}. É importante notar que, ao aplicar um remendo (\emph{patch}), através de comandos escritos com o teclado alfanumérico, o programa não interrompia o fluxo sonoro. \traducao{``Esta era uma inovação crítica do ponto de vista performativo, porque então Grossi era capaz de tocar, e interagir, em tempo real com o programa, ao escrever instruções no teclado sem parar o fluxo sonoro''.}{ This was a critical innovation under the performative point of view, because then Grossi was able to play and to interact in real time with the software, by typing instructions on the keyboard without stopping the sound flux.}

Grossi foi além deste problema reflexivo no final da década de 1970. O TAU2-TAUMUS sofreu uma considerável modificação para controlar um sistema digital-analógico remotamente. O novo programa foi batizado em 1986 de TELETAU e, segundo \citeonline[p.~128--129]{mori_pietro_2015}, possibilitava o acesso a um computador da CNR, em Pisa, com uma conexão da rede BITNET. No entanto o TELETAU não vingou por diversos motivos: falhas e bugs que aumentavam de maneira dramática a manutenção e custos; o alto custo de transmissão e, por último mas não menos, a baixa qualidade da saída sonora devido à lentidão da conexão de dados.


\begin{citacao}
\traducao{$[$Pietro$]$ Grossi fez sua primeira experiência do tipo durante uma conferência de tecnologia em Rimini em 1970, onde o músico reproduzia algumas de suas composições, bem como sons randômicos, empregando um terminal de vídeo conectado pelo telefone para o computador da CNR em Pisa. A RAI, empresa de radiodifusão italiana, emprestou suas pontes de rádio $[$Comunicação entre duas antenas$]$ para enviar sinais sonoros entre Pisa e Rimini. É como se fosse o primeiro experimento de telemática musical no mundo.\cite[p.~129]{mori_pietro_2015}}{Grossi made his first experience of this kind during a conference on technology in Rimini in 1970, where the musician reproduced many of his compositions and random sounds as well, by employing a video terminal connected via telephone to the CNR's computer in Pisa. RAI, the Italian public broadcasting company, lent its powerful FM radio bridges to send back sound signals from Pisa to Rimini. It is likely to be the first official experiment of musical telematics in the world.
}
\end{citacao}



\section{Baía de São Franscisco}\label{sec:baiasaofranscisco}

A prática musical com o computador, realizada na Costa Oeste dos EUA durante os anos 1970 e 1980, segundo \citeauthoronline{brown_indigenous_2013}, decorre de um contexto construído em torno do \emph{Mills College} em Oakland \cite[3$^o$ parágrafo]{brown_indigenous_2013}:

\traduzcitacao{Com o florescimento da indústria de computadores pessoais na Baía de São Franscisco, o acesso às novas tecnologias e pessoas que desenvolveram elas era talvez o melhor no mundo. Mas se para todos os jovens com fortunas como panos para suas mentes (e seus futuros), que perseguiam um excitamento aditivo na construção de máquinas eletrônicas, também existiam políticos utópicos que sonhavam com uma nova sociedade construída no livre e aberto acesso à informação, e na abrangente tecnologia baseada em sistemas inteligentes. Esta também é a cultura que deu ao mundo a música ``New Age'', uma versão aguada e comercializada das músicas com base em modos e drones que Terry Riley, Pauline Oliveros, e LaMonte Young inventaram durante os anos cinquenta e sessenta. Mas a música feita na Costa Oeste também incluiam improvisações barulhentas e despreocupadas, que sobraram das revoluções contra-culturais dos anos 60 \cite[1$^o$ parágrafo]{brown_indigenous_2013}}{With the flowering personal computer industry in the Bay Area, access to the new digital technologies and to the people who developed them was perhaps the best in the world. But for all the young men with fortunes in the back of their minds (and in their futures) who pursued the addictive excitement of building electronic machines, there were also the political utopians whose dream was of a new society built on the free and open access to information, and on a comprehensively designed technology based on embedded intelligence. This was also the culture that gave the world "New Age" music, a watered-down and commercialized version of the musics based on modes and drones that Terry Riley, Pauline Oliveros, and LaMonte Young invented here during the late fifties and early sixties. But West Coast music-making also included a free-wheeling, noisy, improvisational edge left over from the counter-cultural revolutions of the sixties.}



\subsection{The League of Automatic Composers}

Na segunda metade da década de setenta, Jim Horton começou a adquirir micro-controladores KIM-1\footnote{\emph{Keyboard Input Monitor}. Disponível em \url{http://www.6502.org/trainers/buildkim/kim.htm}.} com interesses musicais. Segundo \citeonline{brown_indigenous_2013}, não demorou para que outros compositores interessados comprassem. Discussões informais posteriores incluiram, além de Horton, David Behrman, John Bischoff, Rich Gold, Cathy Morton, Paul Robinson, e Paul Kalbach. Em 1977 e 1978  Horton colaborou com duas peças, apresentadas no \emph{Mills College}, que interligavam sistemas musicais elaborados com os microcontroladores \ver{fig:siskim1}. A primeira peça foi realizada com algoritmos inspirados nas teorias matemáticas de Leonard Euler (séc. XVIII). A segunda peça explorava a comunicação entre os microcontroladores, de forma que \traducao{``notas ocasionais da minha $[$Bischof$]$ máquina faziam a máquina de Jim transpor atividades melódicas de acordo com minha nota base'' \cite[5$^o$ parágrafo]{brown_indigenous_2013}}{the occasional tones of my $[$Bischof$]$ machine caused Jim’s machine to transpose its melodic activity according to my "key" note.}. Em 1978, Bischof, Gold e Horton formaram uma banda nas proximidades de Berkley. Posteriormente Behrman se junta ao trio. No dia 26 de Novembro gravam um \emph{Extended Play} (EP)\footnote{Gravação muito longa para um \emph{demo} e insuficiente para um disco de vinil da época.} de quatro faixas no \emph{Blind Lemmon}, um ponto de encontro musical fundado em 1958\disponivelem{http://www.chickenonaunicycle.com/Berkeley\%20Art.htm}. O disco foi lançado pela Lovely Music (NY) em 1980 como \emph{The Hub: Computer Network Music}.  Durante este tempo, foi formado o grupo \emph{``The League of Automatic Music Composers''}\footnote{Segundo \citeonline[6$^o$parágrafo]{brown_indigenous_2013}, o nome é uma referência ao grupo ``The League of Composers'' formado por Aaron Copland nos anos 20.}, que, além de  Bischof e Behrman, participavam Tim Perkis, Scot Gresham-Lancaster, Mark Trayle e Phil Stone. 

\begin{figure}[!h]
  \centering
  \includegraphics[scale=0.7]{imagens/siskim1.jpg}
  \caption{Sistema de música computacional de John Bischof \emph{circa} 1980. Foto: Eva Shoshanny\protect\footnotemark. \textbf{Fonte}: \citeonline{brown_indigenous_2013}.}
  \label{fig:siskim1}
\end{figure}

\footnotetext{Tradução de \emph{John Bischoff's KIM-1 computer music system circa 1980 photo: Eva Shoshany}}

%É interessante uma descrição da parceria entre Horton, Bischof e Perkis. Durante 1979, o trio realiza participações sociais com a exploração musical de \emph{hardwares}:

\begin{citacao}
Na primavera de 1979, montamos uma série quinzenal regular de apresentações informais sob os auspícios da \emph{Bay Center for the Performing Arts}. Todos outros domingos à tarde passávamos algumas horas configurando nossa rede de KIMs na sala \emph{Finnish Hall}, na Berkeley, e deixávamos a rede tocando, com retoques aqui e ali, por uma ou duas horas. Os membros da audiência poderiam ir e vir como quisessem, fazer perguntas, ou simplesmente sentar e ouvir. Este foi um evento comunitário do tipo em que outros compositores aparecem, tocando ou compartilhando circuitos eletrônicos que tinham projetado e construído. Um interesse na construção de instrumentos eletrônicos de todos os tipos parecia estar "no ar". Os eventos da sala \emph{Finn Hall} foram feitos para uma cena com paisagens sonoras geradas por computador misturado com os sons de grupos de dança folclórica ensaiando no andar de cima e as reuniões ocasionais do Partido Comunista na sala de trás do 'venerável e velho edifício'. A série durou cerca de 5 meses que eu me lembre. \cite[online]{brown_indigenous_2013}\footnote{Tradução nossa de: \emph{In the spring of 1979, we set up a regular biweekly series of informal presentations under the auspices of the East Bay Center for the Performing Arts. Every other Sunday afternoon we spent a few hours setting up our network of KIMs at the Finnish Hall in Berkeley and let the network play, with tinkering here and there, for an hour or two. Audience members could come and go as they wished, ask questions, or just sit and listen. This was a community event of sorts as other composers would show up and play or share electronic circuits they had designed and built. An interest in electronic instrument building of all kinds seemed to be "in the air." The Finn Hall events made for quite a scene as computer-generated sonic landscapes mixed with the sounds of folk dancing troupes rehearsing upstairs and the occasional Communist Party meeting in the back room of the venerable old building. The series lasted about 5 months as I remember.}}
\end{citacao}

Em 1980, Gold e Behrman abandonam o grupo, e Tim Perkis se junta. Este foi período em que o grupo solidifica suas atividades na região da Baía de São Franscisco. O trio (Horton, Bischof e Perkis) formaliza a comunicação entre os microcontroladores de cada membro  -- o que para a época era arriscado ao ponto de queimar componentes. Realizadas as conexões, tocavam ao menos três horas, tempo em que ouviam e ajustavam os sistemas\disponivelem{https://www.youtube.com/watch?v=HW0qax8M68A}\cite[7$^o$ parágrafo]{brown_indigenous_2013}. Outro evento de importância é a associação do grupo com a banda \emph{Rotary Club}, formada por alunos recém-formados da \emph{Mills College}: Sam Ashley, Kenneth Atchley, Ben Azarm, Barbara Golden, Jay Cloidt e Brian Reinbolt. O grupo \traducao{``baseava seu estilo de performace em torno de uma caixa de comutação projetada por Brian Reinbolt}{based their performance style around an automatic switching box designed by member Brian Reinbolt''.}\cite[8$^o$ parágrafo]{brown_indigenous_2013}. Em 1983 o grupo reduziu suas atividades, época em que Horton contraiu artrite degenerativa.

\begin{figure}[!h]
    \centering
    \includegraphics[scale=0.9]{imagens/perkis.jpg}
    \caption{Circuito do computador caseiro dedicado à síntese sonora de Tim Perkis, usado no começo dos anos 1980. Foto: Eva Shoshany\protect\footnotemark. \textbf{Fonte}: \citeonline{brown_indigenous_2013}}
    \label{fig:perkis}
  \end{figure}

\footnotetext{Tradução de \emph{Tim Perkis' homebuilt computer-driven sound synthesis circuitry used in early 1980s. photo: Eva Shoshany}.}

\citeonline[11$^o$ parágrafo]{brown_indigenous_2013} suas redes de composições, ou  \traducao{``ocasiões públicas para escuta compartilhada}{public occasions for shared listening''.}. O som era produzido por um sistema limitado, de \traducao{``baixa velocidade (1 \emph{MHz}) e poucos dados (8 \emph{bits})''}{slow speed (1 \emph{MHz}) and data width (8 \emph{bits})} com ênfase em uma artesania instrumental híbrida de performance. Em outras palavras, \traducao{``A ênfase estava na exploração da tecnologia em mãos -- que poderia ser adquirida pessoalmente ou construída a partir do zero, em vez do desejo incessante de melhores ferramentas''.}{The emphasis was on exploration of the technology at hand—technology that could be personally acquired or built from scratch—rather than the endless wish for better tools.}\cite[22$^o$ parágrafo]{brown_indigenous_2013}: 

\begin{citacao}
\traducao{Os membros da liga geralmente adaptavam composições solo para usar dentro da banda. Estes solos eram desenvolvidos independentemente por cada compositor, e eram tipicamente baseados em esquemas de algoritmos de um tipo ou outro. Existiam características de improvisação diferentes para muitas delas, bem como as músicas eram diferentes em detalhes. Teorias matemáticas, sistemas de afinação experimentais, algoritmos de inteligência artificial, projetos de instrumentos de improvisação, e performance interativa eram algumas das áreas exploradas nestes trabalhos (\ldots) Os solos tocavam simultaneamente no cenário de grupo, se tornando ``sub''-composições que interagem, cada uma enviando e recebendo dados pertinentes para o funcionamento musical. \cite[12$^o$ parágrafo]{brown_indigenous_2013}.}
{League members generally adapted solo compositions for use within the band. These solos were developed independently by each composer and were typically based on algorithmic schemes of one kind or another. There was a distinctly improvisational character to many of these as the music was always different in its detail. Mathematical theories of melody, experimental tuning systems, artificial intelligence algorithms, improvisational instrument design, and interactive performance were a few of the areas explored in these solo works. (\ldots) The solos, played simultaneously in the group setting, became interacting "sub"-compositions, each sending and receiving data pertinent to its musical functioning.}
\end{citacao}

\subsection{The Hub}

O primeiro \emph{The Hub} foi surgiu com o duo Bischoff e Perkis que desenvolveram trabalhos com outros grupos em 1986, como o duo formado por Chris Brown e Mark Trayle, o duo Scott Greham-Lancaster/Richard Zvonar, e o trio Phil Burk/Larry Polansky/Phil Stone. Bischoff pontua que o nome da banda era uma maneira simbólica de caracterizar um sistema musical centralizado, \traducao{``(\ldots) um pequeno microcontrolador como caixa de correio, para postar dados usados no controle de seus sitemas individuais, que eram então acessados por outro intérprete, para usar de qualquer maneira e em qualquer tempo que escolher''.}{(\ldots) a small microcomputer as a mailbox to post data used in controlling their individual music systems, which was then accessible to the other player to use in whatever way and at whatever time he chose.}. O computador centralizado original, \emph{Hub}, era um dos microcontroladores KIM-1 utilizados na época do \emph{The League}:  

\begin{citacao}
\traducao{\emph{The Hub} originalmente surgiu como uma maneira de limpar uma bagunça. (\ldots) Toda vez que nós ensaiamos, um conjunto complicado de conexões \emph{ad-hoc} entre computadores tinham de ser feitas. Isso criou um sistema com um comportamento rico e variado, mas sujeito a falhas, e trazer outros jogadores ficava difícil. Mais tarde, procuramos uma maneira de abrir o processo, para torná-lo mais fácil para os outros músicos tocarem no contexto de rede. O objetivo era criar uma nova maneira para pessoas fazerem música juntos. A solução bateu no ponto da facilidade de uso, e fornecimento de uma interface de usuário padrão, de modo que os jogadores poderiam conectar praticamente qualquer tipo de computador. \emph{The Hub} é um pequeno computador dedicado a passar mensagens entre os jogadores. Ele serve como uma memória comum, mantendo informações sobre a atividade de cada jogador que seja acessível para os computadores de outros jogadores \cite[seção 2.1]{brown_indigenous_2013}. }{The Hub originally came about as a way to clean up a mess. John Bischoff, (\ldots) Every time we rehearsed, a complicated set of ad-hoc connections between computers had to be made. This made for a system with rich and varied behavior, but it was prone to failure, and bringing in other players was difficult. Later we sought a way to open the process up, to make it easier for other musicians to play in the network situation. The goal was to create a new way for people to make music together. The solution hit upon had to be easy to use and provide a standard user interface, so that players could connect almost any type of computer. The Hub is a small computer dedicated to passing messages between players. It serves as a common memory, keeping information about each player's activity that is accessible to other players' computers.}
\end{citacao}

Em 1987, Nick Collins e Phil Niblock realizaram uma curadoria de performances telemáticas entre a \emph{Experimental Media} e \emph{The Clocktower} em Nova York. Participam os membros do \emph{The Hub}, divididos em dois trios, formados por John Bischoff/Tim Perkis/Mark Trayle e Chris Brown/Scot Gresham-Lancaster/Phil Stone. Cada trio possuiu um \emph{Hub} intercomunicável. As performances ``\emph{Simple Degradation}'', ``\emph{Borrowing and Stealing}'' e ``\emph{Vague Notions}'' ocorrem através da intercomunicação entre os \emph{Hubs}, de forma que o \traducao{``sexteto $[$está$]$ acusticamente divorciado mas informacionalmente ligado''.}{acoustically divorced, but informationally joined sextet.} \cite[seção 2.2]{brown_indigenous_2013}.

\section{Ron Kuivila}\label{sec:kuivila}

\citeonline{mclean_patterns_2009} comentam  a performance \emph{Water Surfaces}, realizada na edição de 1985 da STEIM \footnote{\emph{STudio for Electro-Instrumental Music}, disponível em \url{http://steim.org/about/}.}, em Amsterdã, como significativa para a concepção de uma improvisação de códigos (excluindo a tecnologia de projeção visual) . A performance chamou a atenção de, e foi incluída na primeira faixa do disco ``\emph{TOPLAP001 - A prehistory of live coding}'' (2007),  \footnote{Disponível em \url{http://toplap.org/wiki/TOPLAP_CDs}.}; uma nota sobre a performance descreve o seguinte: \traducao{``Esta obra usou programação FORTH ao vivo; Curtis \citeonline{roads_steim_1986} testemunhou e relatou a performance de Ron Kuivila feita na STEIM em Amsterdã, em 1985; a performance original termina com a quebra do sistema\ldots
''}{This work used live FORTH programming; Curtis Roads witnessed and reported a performance by Ron Kuivila at STEIM in 1985; the original performance apparently closed with a system crash\ldots}


\traduzcitacao{Ronald Kuivila programou um computador Apple II no palco para cirar sons densos, rodopiantes e métricos, disposto em camadas e dobravam sobre si. Considerando o equipamento usado, os sons eram surpreendentemente grandes em escala. Kuivila teve problemas em controlar a peça devido q problemas sistêmicos. Ele finalmente entrou em dificuldades técnicas e finalizou a performance \cite[p.~47]{roads_steim_1986}}{Ronald Kuivila programmed an Apple II computeronstage to create dense, whirling, metric sounds that layered in and folded over each other. Considering the equipment used, the sounds were often surprisingly gigantic in scale. Kuivila had trouble controlling the piece due to system problems. He finally gave in to technical difficulties and ended the performance}

%FORTH é uma linguagem de programação elaborada por Charles Moore (1938-). Entre seus paradigmas de programação, utiliza da \emph{reflexividade} como dispositivo de escrita e observação dos algoritmos elaborados.

Ge \citeonline{wang_historical_2005}, em uma comunicação pessoal com Curtis Roads, cita a seguinte declaração: \traducao{``Eu vi o \emph{software} FORTH de Ron Kuivila quebrar e queimar no palco em Amsterdã em 1985, mas antes disso, não fez uma música muito interessante. A performance consistiu de digitação''.}{I saw Ron Kuivila's Forth software crash and burn onstage in Amsterdam in 1985, but not before making some quite interesting music. The performance consisted of typing.}

Nenhuma fonte sonora foi encontrada disponível online. 

\section{Live coding}\label{sec:laptoptoplap}

O documento-manifesto ``\emph{Live Algorithm Programming and Temporary Organization for its Promotion}'', de \citeonline{ward_live_2004,mclean_patterns_2009} formaliza regras para uma improvisação de códigos \ver{sec:laptop}e posteriormente possibilita a construção da organização TOPLAP \ver{sec:toplap}.  Deste manifesto, selecionamos um ponto: o lema ``Show us your screens'' como uma síntese das regras práticas do \emph{live coding} \ver{sec:showusyourscreens}.

\subsection{LAPTOP}\label{sec:laptop}

``\emph{Live Algorithm Programming and Temporary Organization for its Promotion}'' \cite{ward_live_2004,blackwell_programming_2005} é um primeiro documento-manifesto sobre o \emph{live coding} como modalidade artística, e de suas regras práticas. O seu acrônimo LAPTOP representa o principal equipamento técnico utilizado. Este manifesto expõe o ambiente de performance característico do \emph{algorave} e um suporte ideológico para o \emph{Code DJing}. Ritos técnicos do improvisador, como por exemplo, a projeção do código, são justificados através do discurso de transparência e provável colaboração entre intérprete e público:

\begin{citacao}
O \emph{Livecoding} permite a exploração de espaços algorítmicos abstratos como uma improvisação intelectual. Como uma atividade intelectual, pode ser colaborativa. Codificação e teorização podem ser atos sociais. Se existe um público, revelar, provocar e desafiar eles com uma matemática complexa se faz com a esperança de que sigam, ou até mesmo participem da expedição. Estas questões são, de certa forma, independentes do computador, quando a valorização e exploração do algoritmo é o que importa. Outro experimento mental pode ser encarado com um DJ ao vivo codificando e escrevendo uma lista de instruções para o seu \emph{set} (feito com o iTunes, mas aparelhos reais funcionam igualmente bem). Eles passam ao HDJ $[$ \emph{Headphone Disk Jockey} $]$ de acordo com este conjunto de instruções, mas no meio do caminho modificam a lista. A lista está em um retroprojetor para que o público possa acompanhar a tomada de decisão e tentar obter um melhor acesso ao processo de pensamento do compositor. \cite[p.~245]{ward_live_2004} \footnote{Tradução nossa de: \emph{Live coding allows the exploration of abstract algorithm spaces as an intellectual improvisation. As an intellectual activity it may be collaborative. Coding and theorising may be a social act. If there is an audience, revealing, provoking and challenging them with the bare bone mathematics can hopefully make them follow along or even take part in the expedition. These issues are in some ways independent of the computer, when it is the appreciation and exploration of algorithm that matters.   Another thought experiment can be envisaged in which a live coding DJ writes down an instruction list for their set (performed with iTunes, but real decks would do equally well). They proceed to HDJ according to this instruction set, but halfway through they modify the list. The list is on an overhead projector so the audience can follow the decision making and try to get better access to the composer’s thought process.}}
\end{citacao}

Adiante podemos ver outros dois conceitos aglutinados: a Música de Processos, e a Música Generativa:


\begin{citacao}
Contudo, alguns músicos exploram suas idéias como processos de \emph{software}, muitas vezes ao ponto que o \emph{software} se torna a essência da música. Neste ponto, os músicos podem ser pensados como programadores explorando seu código manifestado como som. Isso não reduz seu papel principal como um músico, mas complementa, com a perspectiva única na composição de sua música. \textbf{Termos como ``música generativa'' e ``música de processos'' tem sido inventados e apropriados para descrever esta nova perspectiva de composição}. Muita coisa é feita das supostas propriedades da chamada ``música generativa'' que separa o compositor do resultado do seu trabalho.'' %Brian Eno compara o fazer da música generativa com o semear de sementes que são deixadas para crescer, e sugere abrir mão do controle dos nossos processos, deixando eles ``brincarem ao vento''. 
\footnote{\opcit[p.~245-246]{ward_live_2004}. Tradução nossa de \emph{Indeed, some musicians explore their ideas as software processes, often to the point that a software becomes the essence of the music. At this point, the musicians may also be thought of as programmers exploring their code manifested as sound. This does not reduce their primary role as a musician, but complements it, with unique perspective on the composition of their music. Terms such as “generative music” and “processor music” have been invented and appropriated to describe this new perspective on composition. Much is made of the alleged properties of so called “generative music” that separate the composer from the resulting work. }}.%Brian Eno likens making generative music to sowing seeds that are left to grow, and suggests we give up control to our processes, leaving them to “play in the wind”.}}
\end{citacao}

A Música como um Processo Gradual\footnote{\cfcite{reich_music_1968}} e a Música Generativa são referenciais possíveis na improvisação de códigos, mas não estão ligadas necessariamente como processo de escuta. A primeira é descrita por \citeonline[p.~128]{mailman_agency_2013} como processos determinísticos que agem sobre focos de quadros temporais. A segunda \traducao{``(\ldots) é sensitiva às circuntâncias, isso quer dizer que irá reagir diferentemente dependendo das suas condições iniciais, onde ocorre e assim por diante''}{Generative music is sensitive to circumstances, that is to say it will react differently depending on its initial condition, on where it's happening and so on.} \cite{eno_generative_1996}. Ambas diferem do processo de codificação descrito por \citeonline[p.~130]{McLean2011}:

\begin{citacao}
\traducao{Na codificação ao vivo a performance é o processo de desenvolvimento de \emph{software}, em vez de seu resultado. O trabalho não é gerado por um programa acabado, mas através de sua jornada de desenvolvimento do nada para um algoritmo complexo, gerando mudanças contínuas da forma musical ou visual ao longo do caminho. Isto contrasta com a arte generativa popularizada pela música geradora de Brian \citeonline{eno_generative_1996}. (\ldots) O resultado segue mais ou menos o mesmo estilo, com apenas algumas permutações, dando uma idéia das qualidades da peça. Isto é bem ilustrado pelo nosso estudo de caso de um artista-programador, que executa seu programa poucas vezes não para produzir novas obras, mas para obter diferentes perspectivas sobre o mesmo trabalho.}{In live coding the performanceis the \emph{process} of software development, rather than its outcome. The work is not generated by a finished program, but through its journey of development from nothing to a complex algorithm, generating continuously changing musical or visual form along the way. This is by contrast to \emph{generative} art popularised by the generative music of Brian \citeonline{eno_generative_1996} (\ldots)Output more or less follows the same style, with only a few permutations giving an idea of the qualities of the piece. This is well illustrated by our case study of an artist-programmer, who ran their program a few time not to produce new works, but to get different perspectives on the same work. }
\end{citacao}

\subsection{TOPLAP}\label{sec:toplap}

\citeonline[p.~246]{ward_live_2004} e \citeonline{ramsay_algorithms_2010} significam acrônimo LAPTOP e descrevem o surgimento da organização TOPLAP (ver \autoref{fig:TOPLAP}):

\begin{citacao}
\traducao{A organização TOPLAP (www.toplap.org), cuja sigla possui diversas interpretações, uma sendo \emph{Organização Temporária para a Proliferação da Programação de Algoritmos Ao Vivo}, foi criada para promover e explorar o \emph{live coding}. TOPLAP nasceu em um bar esfumaçado em Hamburgo a uma da manhã em 15 de Fevereiro de 2004.}{The organisation TOPLAP (www.toplap.org), whose acronym has a number of interpretations,  one being the Temporary Organisation for the Proliferation for Live Algorithm Programming, has been set up to promote and explore live coding. TOPLAP was born in a smoky Hamburg bar at 1am on Sunday 15th February 2004}
\end{citacao}

\begin{figure}[!h]
  \centering
  \includegraphics[scale=0.8]{imagens/TOPLAP.png}
  \caption{Definição do siginificado de TOPLAP. \textbf{Fonte}: \citeonline{ramsay_algorithms_2010}.}
  \label{fig:TOPLAP}
\end{figure}

O símbolo ``|'' é uma representação gráfica do operador lógico \emph{OR} (OU), bastante utilizado em estruturas condicionais. Isto é, \emph{Temporary }| \emph{Trasnational} | \emph{Terrestrial} | \emph{Transdimensional} significa que as letras ímpares ``T'', e ``P'' e ``A'', podem significar um ou outro termo indicado pelo algoritmo. Este comportamento é praticado por Nick Collins (1975-) para gerar pseudônimos como Click Nilson, ou Sick Lincoln. 

\subsection{\emph{Show us your screens}}\label{sec:showusyourscreens}

Além das performances inaugurais nos festivais Europeus, o manifesto Lubeck04, \traducao{``iniciado em um ônibus \emph{Ryanair}, em  Hamburgo para o aeroporto Lübeck''\cite[p.~247]{ward_live_2004}}{begun on a Ryanair transit bus from Hamburg to Lubeck airport}, mais conhecido como ``\emph{Show us your screens}'', prescreve algumas regras práticas do \emph{live coding}. 

\begin{citacao}
Exigimos:

• Acesso à mente do intérprete, para todo o instrumento humano.

• Obscurantismo é perigoso. Mostre-nos suas telas.

• Programas são instrumentos que podem modificar eles mesmos.

• O programa será transcendido - Língua Artificial é o caminho.

• O código deve ser visto assim como ouvido, códigos subjacentes visualizados bem como seu resultado visual.

• Codificação ao vivo não é sobre ferramentas. Algoritmos são pensamentos. Motosserras são ferramentas. É por isso que às vezes algoritmos são mais difíceis de perceber do que motosserras.

Reconhecemos contínuos de interação e profundidade, mas preferimos:

• Introspecção dos algoritmos.

• A externalização hábil de algoritmo como exibição expressiva/impressiva de destreza mental.

• Sem \emph{backup} (minidisc, DVD, safety net computer).

Nós reconhecemos que:

• Não é necessário para uma audiência leiga compreender o código para apreciar, tal como não é necessário saber como tocar guitarra para apreciar uma performance de guitarra.

• Codificação ao vivo pode ser acompanhada por uma impressionante exibição de destreza manual e a glorificação da interface de digitação.

• Performance envolve contínuos de interação, cobrindo talvez o âmbito dos controles, no que diz respeito ao parâmetro espaço da obra de arte, ou conteúdo gestual, particularmente direcionado para o detalhe expressivo. Enquanto desvios na tradicional taxa de reflexos táteis da expressividade, na música instrumental, não são aproximadas no código, por que repetir o passado? Sem dúvida, a escrita de código e expressão do pensamento irá desenvolver suas próprias nuances e costumes. 
\footnote{\loccit{ward_live_2004}. Tradução nossa de:\emph{We demand: \begin{inparaenum}[•]
\item Give us access to the performer's mind, to the whole human instrument.
\item Obscurantism is dangerous. Show us your screens.
\item Programs are instruments that can change themselves.
\item The program is to be transcended - Artificial language is the way.
\item Code should be seen as well as heard, underlying algorithms viewed as well as their visual outcome.
\item Live coding is not about tools. Algorithms are thoughts. Chainsaws are tools. That's why algorithms are
sometimes harder to notice than chainsaws.
\end{inparaenum}. We recognise continuums of interaction and profundity, but prefer:  \begin{inparaenum}[•]
\item Insight into algorithms
\item The skillful extemporisation of algorithm as an expressive/impressive display of mental dexterity
\item No backup (minidisc, DVD, safety net computer)
\end{inparaenum}. We acknowledge that: \begin{inparaenum}[•]
\item It is not necessary for a lay audience to understand the code to appreciate it, much as it is not necessary
to know how to play guitar in order to appreciate watching a guitar performance.
\item Live coding may be accompanied by an impressive display of manual dexterity and the glorification of the
typing interface.
\item Performance involves continuums of interaction, covering perhaps the scope of controls with respect to
the parameter space of the artwork, or gestural content, particularly directness of expressive detail. Whilst
the traditional haptic rate timing deviations of expressivity in instrumental music are not approximated in
code, why repeat the past? No doubt the writing of code and expression of thought will develop its own
nuances and customs.
\end{inparaenum}}}
\end{citacao}

O manifesto acima surgiu, entre outros motivos, como uma resposta ao artigo ``\emph{Using Contemporary Technology in Live Performance; the Dilemma of the Performer}'' \cite{schloss_dilemma_2003}. A crítica principal de \citeauthoronline{ward_live_2004} refere-se ao sétimo dos questionamentos sugeridos para uma performance de improvisação ao vivo com computadores. Isto é, em um contexto de embate acadêmico, o desafio colocado por \citeonline[p.~241]{schloss_dilemma_2003} foi um estímulo considerável para  emancipação da improvisação de códigos. É curioso notar que o problema e a intenção de Schloss eram opostas ao que foi proposto por \citeauthoronline{ward_live_2004}:


\begin{citacao}
\traducao{Para reiterar, agora que nós temos computadores rápidos o suficiente para execução ao vivo, nós temos novas possibilidades, e um novo problema. Do começo da evidência arqueológica da música até agora, música era tocada acusticamente, e sempre foi fisicamente evidente como o som era produzido; alí existia uma relação de proximidade entre gesto e resultado. Agora nós não temos mais que seguir as leis da física (ultimamente temos, mas não nos termos do que o observador vê), uma vez que nós temos completo poder do computador como intérprete e intermediário entre nosso corpo físico e o som produzido. \textbf{Por esta causa, a ligação entre gesto e resultado foi completamente perdido, se é que existe ligação. Isto significa que nós podemos ir além da relação de causa-e-efeito entre executante e instrumento que faz a mágica.} Mágica é bom; muita mágica é fatal.}{
To reiterate, now that we have fast enough computers toperform live, we have new possibilities, and a new problem.From the beginning of the archeological evidence of musicuntil now, music was played acoustically, and thus it wasalways physically evident how the sound was produced; there was a nearly one-to-one relationship between gesture andresult. Now we don’t have to follow the laws of physicsanymore (ultimately we do, but not in terms of what theobserver observes), because we have the full power of computers as interpreter and intermediary between our physicalbody and the sound production. Because of this, the link between gesture and result can be completely lost, if indeed there is a link at all. This means that we can go so far beyond the usual cause-and-effect relationship between performer and instrument that it seems like magic. Magic is great; too much magic is fatal
} 
\end{citacao}

A crítica de \citeonline[p.~239]{schloss_dilemma_2003}: \traducao{``considerar a visão do observador sobre os modos de performance das interações físicas e mapeamentos de gestos em som, para fazer uma performance convincente e efetiva''}{Its now necessary, (\ldots) ;to consider the observer’s view of the performer’s modes of physical interactions and mappings from gesture to sound, in order to make the performance convincing and effective.} era especificamente direcionada aos compositores que improvisam música computacional no palco com foco apenas no aspecto sonoro ou tecnológico. Sua questão tange a ausência de gestos referenciais, esforço físico, no caso de performances com dispositivos extendidos, o problema do movimento exagerado, e a expectativa cênica na performance musical:


\begin{citacao}
\traducao{1. Causa-e-efeito é importante, pelo menos para o observador/audiência em uma sala de concerto. 
\ \\
2.Corolário: Mágica na performance é bom. Muita mágica é fatal! (chato).
\ \\
3. Um componente visual é essencial para a audiência, tal como existe um aparato visual de entrada para parâmetros e gestos.
\ \\
4. Sutileza é importante. Grandes gestos são facilmente visíveis de longe, o que é bom, mas eles são movimentos de desenho animado se comparados à execução de um instrumento musical.
\ \\
5. Esforço é importante. Neste sentido, nós estamos em desvantagem de desempenho na performance musical com o computador.
\ \\
6. Improvisação no palco é bom, mas ``mimar'' o aparato no palco não é improvisação, é edição. É provavelmente mais apropriado fazer isso no estúdio antes do concerto, ou se durante o concerto, com o console no meio ou atrás da sala de concerto.
\ \\
7. Pessoas que representam devem representar. Um concerto de música de computador não é uma desculpa/oportunidade para um programador(a) se sentar no palco. Sua presença melhora ou impede o desempenho da representação?
}{1. Cause-and-effect is important, at least for the observer/audience in a live concert venue. 2. Corollary: Magic in a performance is good. Too much magic is fatal! (Boring). 3. A visual component is essential to the audience, such that there is a visual display of input parameters/gestures. The gestural aspect of the sound becomes easier to experience. 4. Subtlety is important. Huge gestures are easily visible from far away, which is nice, but they are cartoon- movements compared to playi
ng a musical instrument. 5. Effort is important. In this regard, we are handicapped in computer music performance. 6. Improvisation on stage is good, but “baby-sitting” the apparatus on stage is not improvisation, it is editing. It is probably more appropriate to do this either in the studio before the concert, or if at the concert, then at the console in the middle or back of the concert hall. 7. People who perform should be performers. A computer music concert is not an excuse/opportunity for a computer programmer to finally be on stage. Does his/her presence enhance the performance or hinder it?}
\end{citacao}

No item 3, é apontado uma justificativa para projeções visuais para a audiência, e não para o improvisador, o componente visual é essencial (substantificação provável da prática). Ward et al. vão no caminho oposto ao de Schloss, ao projetarem códigos-fonte. Para Schloss, isso é mimar o aparato (e o público) e tornar a apresentação pedante. No item 5, a representação cênica do esforço é importante, e neste sentido, é essencial para o ouvinte receber a música como performance. O item 7 restringe a atividade do programador como artista, e é, ao mesmo tempo, o gatilho para Ward et al. começarem a divulgar o \emph{live coding}.

\section{Discussão}

Oferecemos um cenário proto-histórico do ponto de vista, na Itália com o compositor Pietro Grossi \ver{sec:grossi}, nos EUA com Jim Horton, John Bischoff, Tim Perkis \ver{sec:baiasaofranscisco}, e na Holanda com Ron Kuivila \ver{sec:kuivila} que deram suporte ao pensamento promovido por \citeonline{ward_live_2004}.

O tipo de material sonoro utilizado nas peças de Grossi, se comparada com os trabalhos de Max Mathews \footnote{\cfcite{mathews_digital_1963,mathews_technology_1969,roads_interview_1980,park_interview_2009,di_nunzio_genesi_2010}}, estão debruçadas na resolução do problema de performance, contraposto à capacidade de processamento dos \emph{mainframes} da época. Para Grossi, com o problema da capacidade de processamento, os compositores deveriam esperar por melhores implementações técnicas dos engenheiros, e naquele contexto, o computador foi uma ferramenta de bricolagem para reprodução de uma música do séc. XVII-XX (\emph{circa}). Isto é, operações composicionais tradicionais como inversão, retrogradação, retrogradação da inversão, aceleração, diminuição, são reproduzidas como comandos de computador.

Nossas descrições sobre o contexto estadounidense centraram-se no trio formado por Jim Horton, John Bischoff, Tim Perkis, e posteriormente como um sexteto, formador por John Bischoff/Tim Perkis/Mark Trayle/Chris Brown/Scot Gresham-Lancaster/Phil Stone. Membros do grupo referem-se ao \emph{The Hub} como um \emph{sexteto acusticamente divorciado mas informacionalmente ligado.}. Os materiais composicionais são infomações interoperadas por um sistema comum, e então distribuídas entre os integrantes, que agem como \emph{jogadores}. 

Seguindo os desenvolvimentos tecnológicos da época, Ron Kuivila representa um uso peculiar do computador para a atividade de composição musical, ainda que dado ao aparente fracasso. No entanto, não descartamos a possibilidade de uma queima programada\footnote{Disponível em \href{http://www.tycomsystems.com/beos/bebook/The\%20Kernel\%20Kit/index.html}{http://www.tycomsystems.com/beos/bebook/The\%20Kernel\%20Kit/index.html}}.

Por último, descrevemos a formalização de uma improvisação de códigos. Esta formalização se deu a partir de um embate acadêmico levado a cabo por sete artistas-programadores ingleses \cite{ward_live_2004}. Um documento-manifesto foi produzido e estimulou a divulgação de uma forma de estética \emph{hacker} em Músicas Eletrônicas para Dançar. 
%\newpage
%\include{./02-Metodologia}
%\newpage
\chapter{Estudo de caso}\label{cap:estudos_de_caso}

A pesquisa desenvolvida nos capítulos anteriores sucitaram a seguinte pergunta: como investigar um caso musical de improvisação de códigos, um vídeo de \emph{A Study in Keith} de Andrew Sorensen (2012)?

Os trabalhos de \citeonline{Forth2010,McLean2011} possibilitaram investigar o vídeo dentro de um abordagem cognitivista, de forma que contextualizamos um método de análise sugerida por \citeonline[p.~117]{McLean2011}:

\begin{citacao}
\traducao{Aqui nós tomamos a perspectiva que uma propriedade conceitual é representada por um melhor \emph{exemplo simples} possível, ou \emph{protótipo}. (\ldots) Para fundamentar a discussão em música, considere uma peça de jazz, onde jazz é um conceito e uma composição particular é uma instância de um conceito. Um musicista, que explora os limites do jazz, encontrou uma peça para além das regras usuais do jazz. Através deste processo, os limites do gênero musical podem ser redefinidos em algum grau, ou se a peça está em um novo terreno particularmente fértil, um novo sub-gênero de jazz emerge. Contudo uma peça de música que não quebra limites, de alguma forma pode ser considerada não-criativa.}{Here we take the view that a conceptual property is represented by a single best possible example, or \emph{prototype}. In accordance with the theories reviewed in chapter 2, these prototypes arise through perceptual states, within the geometry of quality dimensions. To ground the discussion in music, consider a piece of jazz, where jazz is the concept and the particular composition is an instance of that concept. The musician, in exploring the boundaries of jazz, then finds a
piece beyond the usual rules of jazz. Through this process, the boundaries of a music genre
} 
\end{citacao}


\section{Metodologia de Análise}\label{sec:metodo}

Para análisar \emph{A Study in Keith} utilizamos um recorte do \emph{Quadro Conceitual de Sistemas Criativos}\footnote{Creative System Frameworks, \emph{ou CSF}, \cfcite{mclean_music_2006,Forth2010,McLean2011}.} \ver{tab:universodeconceitos}, e do modelo-baseem que este foi desenvolvido, o modelo de improvisação de Jeff \citeonline{pressing_improvisation_1987} \ver{sec:im}. É importante esclarecer que, para não confundirmos questões ontológicas,  substituímos \emph{conceito} por \emph{proposição da improvisação} ou simplismente \emph{proposição}\ver{sec:proposicao}. Renomeamos o termo \emph{Referente opcional} de Pressing por \emph{Referentes de A Study in Keith Zero, Um, Dois e Três}. A Linguagem é \emph{A Linguagem de Programação utilizada em A Study in Keith}.

%Buscamos desta forma discretizar a parte inicial do processo criativo de um artista-programador, de forma que trocamos conceito para proposição, universo de conceitos para universo de proposições, espaço conceitual para espaço de proposições, para discutir os três primeiros blocos de eventos sonoros, que formam a primeira sequência de eventos sonoros, formados a partir de um referente opcional

\newpage

\begin{table}[!h]
\caption{Definições formais do Universo de possibilidades de \citeonline{wiggins_framework_2006}, ou Universo de Conceitos por \citeonline{mclean_music_2006,Forth2010}. Neste trabalho, como quadro de proposições.}
\small
    \begin{tabular}{ | p{4.25cm} | p{5.25cm} | p{5.25cm} |}
    \hline 
    \hline 

    Representação
    & \tiny{Nome}     
    & \tiny{Significado} \\
    \hline

    $c$
    & \tiny{Conceito} 
    & \tiny{Uma instância de um conceito, abstrato ou concreto \cite{wiggins_framework_2006}}. \\
    \hline

    $\mathcal{U}$
    & \tiny{Universo de Conceitos} 
    & \tiny{Superconjunto não restrito de conceitos. \cite{wiggins_framework_2006}. ``Um universo de todos conceitos possíveis'' \cite{mclean_music_2006} \tablefootnote{Tradução de \emph{A universe of all possible concepts}.}}\\
    \hline

    $\mathcal{L}$
    & \tiny{Linguagem} 
    & \tiny{Linguagem utilizada para expressar regras.} \\
    \hline

    $\mathcal{A}$
    & \tiny{Alfabeto} 
    & \tiny{Alfabeto da linguagen que contêm caracteres apropriadospara expressão das regras} \\
    \hline

    $\mathcal{R}$
    & \tiny{Regras de validação} 
    & \tiny{Validam os conceitos em um universo, se apropriados ou não para o espaço trabalhado.} \\
    \hline

    $[[.]]$
    & \tiny{Função de interpretação} 
    & \tiny{``Uma função parcial de $\mathcal{L}$ para funções que resultam em números reais entre [0, 1] (\ldots) 0.5 $[$ou maior$]$ significa uma verdade booleana e menos que 0.5 siginifica uma falsidade booleana; a necessidade disso para valores reais se tornará clara abaixo'' \cite[p.~452]{wiggins_framework_2006}\tablefootnote{Tradução de \emph{(\ldots) a partial function from $\mathcal{L}$ to functions yielding real numbers in [0, 1]. (\ldots) 0.5 to mean Boolean true and less than 0.5 to mean Boolean false; the need for the real values will become clear below}.}}\\
    \hline

     $[[\mathcal{R}]]$
    & \tiny{Regras de validação} 
    & \tiny{``Uma função que interpreta $\mathcal{R}$, resultando em uma função indicando aderência ao conceito em $\mathcal{R}$''\tablefootnote{Tradução de \emph{A function interpreting $\mathcal{R}$, resulting in a function indicating adherence of a concept to $\mathcal{R}$}}} \\
    \hline

     $\mathcal{C} = [[\mathcal{R}]](\mathcal{U}) $
    & \tiny{Espaço Conceitual} 
    & \tiny{``Todos espaços conceituais são um subconjunto não-estrito de $\mathcal{U}$''\tablefootnote{Tradução de \emph{All conceptual spaces are non-strict subset}.}. Um subconjunto contido em $\mathcal{U}$ \cite{wiggins_framework_2006}. Uma função que interpreta $\mathcal{R}$, resultando em uma função que indica aderência ao conceito em $\mathcal{R}$ \tablefootnote{Tradução de \emph{A function interpreting $\mathcal{R}$, resulting in a function indicating adherence of a concept to $\mathcal{R}$}.} } \\
    \hline

    $\mathcal{T}$
    & \tiny{Regras de detecção} 
    & \tiny{``Regras definidas dentro de $\mathcal{L}$ para definir estratégias transversais para localizar conceitos dentro de $\mathcal{U}$'' \cite{mclean_music_2006}\tablefootnote{Tradução de \emph{Rules defined within $\mathcal{L}$ to define a traversal strategy to locate concepts within $\mathcal{U}$ }}} \\
    \hline

    $\mathcal{E}$
    & \tiny{Regras de qualidade} 
    & \tiny{``(\ldots) conjunto de regras que permitem-nos avaliar qualquer conceito que nós encontramos em $\mathcal{C}$ e determinar sua qualidade, de acordo com critérios que nós considerarmos apropriados'' \cite[p.453]{wiggins_framework_2006}\tablefootnote{Tradução de \emph{(\ldots) set of rules which allows us to evaluate any concept we find in C and determine its quality, according to whatever criteria we may consider appropriate.}}``Regras definidas dentro de $\mathcal{L}$ para avaliar a qualidade ou a desejabilidade do conceito $c$'' \cite{mclean_music_2006}\tablefootnote{Tradução de \emph{Rules defined within $\mathcal{L}$ which evaluate the quality or desirability of a concept $c$.}}}\\
    \hline

    $<<<\mathcal{R}, \mathcal{T}, \mathcal{E}>>>$
    & \tiny{Função de interpretação} 
    & \tiny{Uma regra necessária para definir o espaço conceitual, ``independentemente da ordem, mas também, ficcionalmente, enumerá-los em uma ordem particular, sob o controle de $\mathcal{T}$ -- isto é cricial para a simulação de um comportamento criativo de um $\mathcal{T}$ particular \cite{wiggins_framework_2006} \tablefootnote{Tradução de \emph{We need a means not just of defining the conceptual space, irrespective of order, but also, at least notionally, of enumerating it, in a particular order, under the control of $\mathcal{T}$ -- this is crucial to the simulation of a particular creative behaviour by a particular $\mathcal{T}$.}}. ``Uma função que interpreta a estratégia transversal $\mathcal{T}$, informada por $\mathcal{R}$ e $\mathcal{E}$ . Opera sobre um subconjunto ordenado de $mathcal{U}$ (do qual tem acesso randômico) e resulta em outro subconjunto ordenado de $\mathcal{U}$.''\tablefootnote{Tradução de \emph{A function interpreting the traversal strategy $\mathcal{T}$, informed by $\mathcal{R}$ and $\mathcal{E}$ . It operates upon anordered subset of $mathcal{U}$ (of which it has random access) and results in another ordered subset of $\mathcal{U}$.}}} \\
    \hline
    \hline
   
    \end{tabular}
\label{tab:universodeconceitos}
\end{table}


\subsection{O modelo de improvisação}\label{sec:im}

Segundo Pressing, o Modelo de Improvisação é ``um esboço para uma teoria geral da improvisação integrada com preceitos da Psicologia Cognitiva'' \cite[p.~2]{pressing_improvisation_1987}. Este modelo será utilizado para especificar elementos de uma performance exemplar, como o caso investigado neste trabalho. Por exemplo, uma improvisação particionada em diferentes sequências pode ser parcialmente mapeada em categorias, como blocos sonoros, referentes conceituais e normas estilísticas, conjuntos de objetivos e processos. Este nos pareceu um modelo mais transparente para o compositor, músico e intérprete. O que não quer dizer que é possível readequar ambos para nosso interesse. Um sumário sobre o modelo de improvisação é apresentado na \autoref{tab:modelo_improvisacao}. Por seu caráter lógico, parece ser uma possibilidade interessante, e assumiremos como tal.

\begin{table}[!h]
\caption{Definições formais do Modelo de improvisação de Jeff \citeonline{pressing_improvisation_1987}, segundo \citeonline[p.~2]{mclean_music_2006}.}
\small
    \begin{tabular}{ | p{6cm} | p{9cm} |}
    \hline 
    \hline 

    \tiny{Representação}   
    & \tiny{Significado} \\
    \hline

    $E'$
    & \tiny{Um bloco de eventos sonoros}\tablefootnote{\emph{A cluster of sound events}.} \\
    \hline

    $K'$
    & \tiny{Uma seqüência de blocos de eventos E, onde um bloco de eventos não se sobrepõe com o seguinte}\tablefootnote{A sequence of E event clusters, where event cluster onsets do not overlap with those of a following one}\\
    \hline

    $I'$
    & \tiny{Uma improvisação, particionada por interrupções em um número de K sequências}\tablefootnote{An improvisation, partitioned by interrupts into a number of K sequences} \\
    \hline

    $R'$
    & \tiny{Um referente opcional, tal como uma partitura ou uma norma estilística}\tablefootnote{An optional referent, such as a score or stylistic norm} \\
    \hline

    $G'$
    & \tiny{Um conjunto de objetivos }\tablefootnote{A set of current goals.} \\
    \hline

    $M'$
    & \tiny{Uma memória de longo prazo}\tablefootnote{Long term memory.} \\
    \hline

    $O'$
    & \tiny{Um conjunto de objetos}\tablefootnote{An array of objects.} \\
    \hline

    $F'$
    & \tiny{Um conjunto de características dos objetos}\tablefootnote{An array of objects Features.} \\
    \hline

    $P'$
    & \tiny{Um conjunto de processos}\tablefootnote{An array of Process} \\
    \hline
    \hline
   
    \end{tabular}
\label{tab:modelo_improvisacao}
\end{table}
  
%Nos diagramas abaixo, $C_\emph{\ldots}$ representamos qualquer proposição (que pode incluir outras). Entre os elementos iniciais (raízes, vermelho) e transitórios (nós, azul), ocorrem as ramificações (ramos, linhas pretas), isto é, a exploração da proposição dentro de outros conceitos.

%\begin{tikzpicture}
%  [
%    grow                    = right,
%    sibling distance        = 6em,
%    level distance          = 10em,
%    edge from parent/.style = {draw, -latex},
%    every node/.style       = {font=\footnotesize},
%    sloped
%  ]
%  \node [root] {\csf{U}{Música}}
%    child { node [env] {\csf{U}{pesquisa}}
%      child { node [env] {\csf{U}{livecoding}}}
%    }
%    child { node [env] {\csf{C}{\ldots}}};
%\end{tikzpicture}

%No primeiro capítulo, incluímos um subjconjunto neste Espaço Conceitual da Pesquisa \ver{app:A}). 

%\begin{tikzpicture}
%  [
%    grow                    = right,
%    sibling distance        = 6em,
%    level distance          = 10em,
%    edge from parent/.style = {draw, -latex},
%    every node/.style       = {font=\footnotesize},
%    sloped
%  ]
%  \node [root] {\csf{C}{pesquisa}}
%    child { node [env] {\csf{C}{livecoding}}
%      child { node [env] {\csf{C}{\ldots}}}
%      child { node [env] {\csf{C}{ICLC}}}
%    }
%    child { node [env] {\csf{C}{\ldots}}}; 
%\end{tikzpicture}

%Podemos inclur elementos históricos, o período transitório entre 1970 e 2000 (\emph{circa}), quando emanciparam as práticas e as regras heurísticas.  

%\begin{tikzpicture}
%  [
%    grow                    = right,
%    sibling distance        = 6em,
%    level distance          = 10em,
%    edge from parent/.style = {draw, -latex},
%    every node/.style       = {font=\footnotesize},
%    sloped
%  ]
%  \node [root] {\csf{C}{livecoding}}
%    child { node [env] {\csf{C}{Elementos Históricos}}
%      child {node [env] {\csf{C}{Proto-História}}}
%      child {node [env] {\csf{C}{Manifestos}}}
%    }
%    child { node [env] {\csf{C}{\ldots}}};
%\end{tikzpicture}

%Por último, \csf{C}{pesquisa} investiga o \emph{live coding} a partir de um caso específico:


%\begin{tikzpicture}
%  [
%    grow                    = right,
%    sibling distance        = 6em,
%    level distance          = 10em,
%    edge from parent/.style = {draw, -latex},
%    every node/.style       = {font=\footnotesize},
%    sloped
%  ]
%  \node [root] {\csf{C}{pesquisa}}
%    child { node [env] {\csf{C}{livecoding}}
%      child { node [env] {\csf{C}{\ldots}}}
%      child { node [env] {\csf{C}{Sessão de Improvisação}}
%        child { node [env] {\csf{C}{Study in Keith}}}
%        child { node [env] {\csf{C}{\ldots}}}
%      }
%    }
%    child { node [env] {\csf{C}{\ldots}}}; 
%\end{tikzpicture}

%\begin{example}{Representação do modelo de improvisação para \emph{Study in Keith}.}

%Da especificação \csf{C}{Study in Keith} derivamos blocos de eventos, sequências de blocos de eventos, interrupções, referentes opcionais, objetivos, um objeto que carrega uma memória de algo, objetos (musicais, sonoros, visuais, etc.) e processos

%\begin{tikzpicture}
%  [
%    grow                    = right,
%    sibling distance        = 4em,
%    level distance          = 12em,
%    edge from parent/.style = {draw, -latex},
%    every node/.style       = {font=\footnotesize},
%    sloped
%  ]
%  \node [root] {\footnotesize \csf{C}{Study in Keith}}
%    child { node [env] {\footnotesize \csf{E'}{Study in Keith}}}
%    child { node [env] {\footnotesize \csf{K'}{Study in Keith}}}
%    child { node [env] {\footnotesize \csf{I'}{Study in Keith}}}
%    child { node [env] {\footnotesize \csf{R'}{Study in Keith}}}
%    child { node [env] {\footnotesize \csf{G'}{Study in Keith}}}
%    child { node [env] {\footnotesize \csf{O'}{Study in Keith}}}
%    child { node [env] {\footnotesize \csf{F'}{Study in Keith}}}
%    child { node [env] {\footnotesize \csf{P'}{Study in Keith}}}; 

%\end{tikzpicture}
%\end{example}

%No entanto, exploramos apenas os conceitos envolvidos em um ciclo de bricolagem de um código, o que limita nossos resultados:

%\begin{example}{Especificação do modelo de improvisação para \emph{Study in Keith}.}

%Da representação derivamos uma sequências de blocos de eventos, uma interrupção, três referentes opcionais, um objetivo, e uma classe objetos (musicais, sonoros, visuais, etc.).

%\begin{tikzpicture}
% [
%    grow                    = right,
%    sibling distance        = 4em,
%    level distance          = 12em,
%    edge from parent/.style = {draw, -latex},
%    every node/.style       = {font=\footnotesize},
%    sloped
%  ]
%  \node [root] {\footnotesize \csf{C}{Study in Keith}}
%    child { node [env] {\footnotesize \pressingthree{K'}{Study in Keith}{0}}
%      child { node [env] {\footnotesize \pressingthree{E'}{Study in Keith}{0}}}
%      child { node [env] {\footnotesize \pressingthree{E'}{Study in Keith}{1}}}
%      child { node [env] {\footnotesize \pressingthree{E'}{Study in Keith}{2}}}
%    }
%    child { node [env] {\footnotesize \pressingthree{I'}{Study in Keith}{0}}}
%    child { node [env] {\footnotesize \pressingthree{G'}{Study in Keith}{0}}}
%    child { node [env] {\footnotesize \pressingthree{O'}{Study in Keith}{0}}}
%    child { node [env] {\footnotesize \csf{R'}{Study in Keith}}
%      child { node [env] {\footnotesize \pressingthree{R'}{Study in Keith}{0}}}
%      child { node [env] {\footnotesize \pressingthree{R'}{Study in Keith}{1}}}
%      child { node [env] {\footnotesize \pressingthree{R'}{Study in Keith}{2}}}
%    };

%\end{tikzpicture}
%\end{example}


%\section{Formalização}\label{sec:formaliza}

%O espaço conceitual do \emph{livecoding} é definido como uma função de interpretação das regras de validação (o que pode ser ou não considerado como próprio de uma categorização musical), de gosto (questões de estilo) e de localização transversal de conceitos (conceitos internos que permitem o cruzamento com outros conceitos). As regras de validação foram estudadas neste trabalho como as regras heurísticas do \emph{live coding}. Isto é, que conjunto de métodos são utilizados para caracterizar uma performance de \emph{live coding} como tal? Elementos históricos, e ideológicos (divulgados em manifestos), são levantados para responder esta pergunta. Por outro lado, este estudo abandonou a investigação das regras de gosto, tema que pode ser melhor explorado em trabalhos posteriores, a partir de Janotti \citeonline{janotti_jr._a_2003,sa_musica_2006,sa_se_2009}. A tarefa de localização transversal de conceitos é trabalhada no último capítulo. O espaço conceitual de \emph{Study in Keith} está contido no espaço conceitual do \emph{live coding} através da intersecção entre os conceitos deste último, com os espaços conceituais dos concertos \emph{Sun Bears}, de Keith Jarret, misturados. No entanto o espaço conceitual não será investigado em sua totalidade, e sim apenas uma sonoridade.

\section{\emph{A Study in Keith}: Proposição}\label{sec:proposicao}

Sorensen faz duas as descrições de uma mesma proposição, ou o \emph{Espaço conceitual de A Study in Keith}. Os Concertos \emph{Sun Bear} de Keith Jarret \ver{sec:sunbear} são citados como inspiradores da improvisação de códigos. Desta forma, existe um \emph{Referencial Zero de A Study in Keith}, ou \pressingthree{R}{ask}{0}.

\begin{citacao}
\traducao{\emph{A Study In Keith} é um trabalho para piano solo (NI's Akoustik Piano), inspirado nos concertos \emph{Sun Bear} de Keith Jarrett. Note que não existe som para os dois primeiros 2 minutos da performance, enquanto estruturas iniciais são construídas. \textbf{Não é bem Keith, mas inspirado por Keith}. \cite{sorensen_keith_2009}}{"A Study In Keith" is a work for solo piano (NI's Akoustik Piano) by Andrew Sorensen inspired by Keith Jarrett's Sun Bear concerts. Note that there is no sound for the first 2 minutes of the performance while initial structures are built. Not quite Keith, but inspired by Keith.}
\end{citacao}

\citeonline{sorensen_youtube_2014} indica outros referenciais, que chamamos de \emph{Referencial Um, Dois e Três de A Study in Keith}, \pressingthree{R}{ask}{1},\pressingthree{R}{ask}{2},\pressingthree{R}{ask}{3}, ou o piano virtual \emph{Akoustik Piano NI}, o ambiente de programação \emph{Impromptu} e a linguagem de programação \emph{Scheme}. Este último é a \emph{Linguagem de Programação de A Study in Keith}, ou \csf{L}{ask}:

\begin{citacao}
\traducao{\emph{A Study in Keith} é uma performance de programação ao vivo por Andrew Sorensen, inspirado nos concertos \emph{Sun Bear} de Keith Jarret. Toda a música que você ouve é gerada a partir do código do programa que é escrito e manipulado em \emph{tempo-real} durante a performance. O trabalho foi executado usando o ambiente de desenvolvimento $[$em linguagem$]$ Scheme $[$chamado$]$ Impromptu (\url{http://impŕomptu.moso.com.au}). Não é Keith, mas inspirado por Keith \cite{sorensen_youtube_2014}.
}
{
``A Study In Keith'' is a live programming performance by Andrew Sorensen inspired by Keith Jarrett's Sun Bear concerts. All of the music you hear is generated from the program code that is written and mani$[$p$]$ulated in real-time during the performance. The work was performed using the Impromptu Scheme software development environment (\url{http://impromptu.moso.com.au}). Not Keith, but inspired by Keith.
}
\end{citacao}


%\section{Objetivo}\label{sec:objetivo}

%Observação e análise de um comportamento criativo musical, de um improvisador-programador, que escreve uma programação-partitura, e realiza a manutenção de um pensamento musical tradicional. Mais especificamente, analisamos o contexto musical de uma simples sequência de blocos sonoros \pressingthree{K}{ask}{0}, gerada por uma função de interpretação $<<<$\csf{R}{ask}, \csf{T}{ask},\csf{E}{ask}$>>>$, cujo referente opcional direto, \pressingthree{R}{ask}{0}, são os Concertos \emph{Sun Bear} de Keith Jarret.

%\section{Justificativa}

%Parafraseamos \citeonline[p.~121]{McLean2011} ao afirmarmos que \traducao{Nosso estudo de caso é de alguma forma simplista e não é intenção ilustrar uma grande arte ou um grande código. Contudo delineia um processo criativo de classes, como efetuado pelo presente autor.}{Our case study is somewhat simplistic, and is not intended to illustrate either great art or great code. However it does trace a creative process of sorts, as carried out by the present author.}.


\section{Referentes Opcionais}\label{sec:sunbear}

Aqui foi possível elaborar uma solução possível ao problema enunciado como ``\emph{Study in Keith} não é \emph{Sun Bears}''. Em seguida tratamos do referencial um, \pressingthree{R}{ask}{1}, ou o timbre de piano utilizado \ver{sec:NI}, e de um ambiente de programação musical chamado \emph{Impromptu} \ver{sec:impromptu} como referencial dois \pressingthree{R}{ask}{2}. 

\subsection{Concertos Sun Bear}\label{sec:sunbearanal}

Os concertos \emph{Sun Bear} são originalmente dez LPs  de improvisações de Keith Jarret no Japão, produzidos pela \emph{ECM Records}\footnote{http://www.ecmrecords.com/} entre 1976 e 1978. Foram realizados e gravados como sessões de improvisação contínua, variando entre 31 a 43 minutos cada. Para cada dia, duas sessões de improvisação, em cidades diferentes. Kyoto, 5 de novembro\footnote{Disponível em \url{https://www.youtube.com/watch?v=T2TfIQNxhjc}.}; Osaka, 8 de novembro\disponivelem{https://www.youtube.com/watch?v=FC4iZ1wMoU8}; Nagoya, 12 de novembro\footnote{\url{https://www.youtube.com/watch?v=3a7ezm3D1jA}.}. Tokyo, 14 de novembro\disponivelem{https://www.youtube.com/watch?v=ZH8VIjjhPQ4}; Sapporo, 18 de Novembro\disponivelem{https://www.youtube.com/watch?v=BqYBT_HoG4M}.

%Um documento crítico impresso é mencionado na \emph{internet} como um antigo documento contendo notas discográficas \cite{rollingstone1985}. Seu acesso foi restrito durante a pesquisa, e não foi possível incluir alguma citação. Da mesma forma, não encontramos documentos análiticos específicos sobre a peça, mas uma tese de doutorado de Dariuz \citeonline{terefenko2004} auxiliou na compreensão harmônica do tema principal.

 Existem algumas notas discográficas compiladas por uma comunidade de fãs e críticos musicais estadounidenses. Duas notas sugerem uma descrição da forma musical aplicada por Keith Jarret: \traducao{``O tema de \emph{Kyoto Parte 1} é repetido por Keith Jarret no fim de \emph{Kyoto Parte 2}. Então podemos considerar o todo deste concerto como uma grande Suíte.''}{The theme of Kyoto Part 1 is repeated By Kj at the end of Kyoto Part 2. So we can consider the whole of this concert as one big Suite}\cite[p.~129]{jarret_discography_2014}. 

\traduzcitacao{Revisto por Richard S. Ginnel\footnote{Disponível em \url{http://www.mcana.org/formembersatlarge.html}.}: $[$--$]$ Este pacote gigantesco -- um conjunto de dez LPs agora comprimidos em uma caixa robusta de seis $[$embalagens de$]$ CDs -- foi ridicularizado uma vez como uma última viagem de ego, provavelmente por muitos que não tomaram um tempo para ouvir tudo. (\ldots) Ainda assim, o milagre é como esta caixa é consistentemente muito boa. \textbf{Na abertura de Kyoto, a meditação direcionada para o \emph{gospel}} está em plena atuação, ao nível de suas melhores performances solo em Bremen e Koln,\textbf{e os concertos Osaka e Nagoya possuem citações de primeira linha, geralmente do tipo \emph{folk}}, mesmo profundas, idéias líricas \cite[p.~130]{jarret_discography_2014}
}{
Review by Richard S. Ginell: $[$--$]$ This gargantuan package -- a ten-LP set now compressed into a chunky six-CD box -- once was derided as the ultimate ego trip, probably by many who didn't take the time to hear it all. You have to go back to Art Tatum's solo records for Norman Granz in the '50s to find another large single outpouring of solo jazz piano like this, all of it improvised on the wing before five Japanese audiences in Kyoto, Osaka, Nagoya, Tokyo, and Sapporo. Yet the miracle is how consistently good much of this giant box is.  In the opening Kyoto concert, Jarrett's gospel-driven muse is in full play, up to the level of his peak solo performances in Bremen and Koln, and the Osaka and Nagoya concerts have pockets of first-rate, often folk-like, even profound, lyrical ideas.
}

O \emph{gospel} e o \emph{folk} são categorizados como gêneros musicais nesta suíte que não possui pausas entre as partes (o improviso é contínuo, mas seccionado por transições). Seu motivo gerador é, segundo Uwe \citeonline{karcher2009}, \traducao{``Na verdade, a abertura não é realmente improvisada - ela é baseada em uma canção chamada \emph{Song Of The Heart}''}{Actually, the opening was not really improvised - it is based on a tune named \emph{Song Of The Heart}.}\disponivelem{https://www.youtube.com/watch?v=JgyRoQPDwM8}.

%\begin{figure}[!h]
%  \centering
  %\includegraphics[scale=0.5]{imagens/Jarret_intro.png}
%  \input{./Jarret}
%  \caption{Transcrição do motivo gerador do disco Kyoto, parte 1. \textbf{Fonte}: autor.}
%  \label{fig:Jarret_intro}
%\end{figure}

%É importante também mencionar que a fonte original de uma transcrição, utilizada para escuta e anotação, foi retirada do ar por violação de direitos autorais, o que desviou o foco de uma transcrição mais detalhada e correta. No final desta pesquisa, encontramos um outro vídeo, uma áudio-transcrição executada por Uwe \citeonline{karcher2009} que aponta erros na primeira transcrição acima, e também indica que este tema não é improvisado:

%\begin{citacao}
% Eu tenho interesse especial em transcrever os primeiros 10, 11 minutos (que são simplesmente fenomenais). Por fim, eu usei a Reprise que Keith jogou no final da Parte II.}{My transcription of the famous opening of the concert played by Keith in Kyoto on November 5th, 1976. Released in a 6-CD-Box-Set called "Sun Bear Concerts" (ECM). A must have for all (Jazz)piano enthusiasts! Actually,  I have been interested particularly in transcribing the first 10, 11 minutes (which are simply phenomenal). To come to an end, I used the Reprise which Keith played at the end of Part II. To memorize these 34 pages was pretty ambitious, but it worked :) Hope you enjoy!}
%\end{citacao}

\begin{figure}[!h]
  \centering
%  %\includegraphics[scale=0.5]{imagens/Jarret_intro.png}
  \input{./Jarret2}
  \caption{Transcrição do motivo gerador do disco Kyoto, parte 1. textbf{Fonte}: Uwe \citeonline{karcher2009}.}
  \label{fig:Jarret_intro2}
\end{figure}

A maneira como Jarret improvisou este concerto possibilita questionar como o código é improvisado por \citeonline{sorensen_keith_2009}: o resultado é de fato uma improvisação de códigos, ou existe um agenciamento onde o improvisador prepara um código?

\begin{example}{Redução da primeira sonoridade dos concertos \emph{Sun Bear}}\label{ex:schenker}

%O padrão cromático \^1\^b2-\^1-\^7b forma uma figura alternada com um ostinato no baixo.Em seguida, o padrão \^{b5}-\^4-\^3 cria uma cadência progressiva de contra-polo da dominante, subdominante da dominante e dominante. No entanto, é importante destacar que a análise foi feita para uma sonoridade muito específica, que não considera outras sequências da improvisação.

\emph{Song of the heart} apresenta três blocos de eventos iniciais: uma figura que alterna, a partir de um baixo, os intervalos nona menor, terça menor, segunda maior, e oitava, formando um ostinato. Nos compassos 3 a 5 aparecem uma nota que forma uma relação de trítono com o baixo. O acorde formado, um Sol bemol Maior (transcrito assim para facilitar a leitura), é expandido nos compassos 6 a 10, gerando uma figura cromática cuja transcrição apresenta a seguinte cadência: Sol Bemol Maior (com décima primeira aumentada adicionada, em terceira inversão), Fá Maior (que alterna o \^5 com o \^13 -- ver \cite{terefenko2004}) e Dó Maior com sétima menor (posição fundamental). Limitamo-nos a considerar a progressão do ponto de vista da exploração de um contra-polo seguido de uma subdominante da dominante e dominante.  

Tomando um \emph{blues} tradicional de 12 compassos, seguimos uma fórmula prática $C:~I^7~$ $\Rightarrow~IV^7~\Rightarrow~I^7~\Rightarrow~I^7$ $\Rightarrow~IV^7~\Rightarrow~IV^7~\Rightarrow~I^7~\Rightarrow~I^7~\Rightarrow~V^7~\Rightarrow~IV^7~\Rightarrow~I^7~$. É possível explorar ``seção plagal'' do padrão, ao separarmos os acordes 4 a 7 ($C:~I^7~\Rightarrow~IV^7~\Rightarrow~IV^7~\Rightarrow~I^7$), e transformarmos, por substituição de trítono, a primeira subdominante da sequência, ou $C: subV/IV$ (substituição por trítono da subdominante em Dó Maior). Isto é, a substituição-padrão, $C:~bV/V^7$ (acorde de quinto grau bemol da dominante com sétima), ou $C:~bII^7$ (acorde de segundo grau bemol com sétima), passa a ser operacionalizada como um contra-polo à tônica -- ver \cite{soares_luteria_2015}.  O que pode ser notado como $C:~(bV^7/V)/IV~\Rightarrow~IV^7$, ou $C:~bII^7/IV~\Rightarrow~IV^7$ (sequência do segundo grau bemol da subdominante para a subdominante do tom), pode ser simplificado como $C:~bV^7~\Rightarrow~IV^7$, ou uma sequência do quinto grau bemol para o quarto grau.  No entanto, a última transcrição de Jarret (considerando sua futura correção), suprime e transforma as sétimas no quinto grau bemol, o que caracteriza uma sonoridade de tensão progressiva (e ambígua) com um baixo pedal. Uma tríade do quinto grau bemol, uma tétrade do quarto grau com sexta e quarta (com sua sétima no baixo), e primeiro grau com sétima:  $C:~bV~\Rightarrow~IV^6_{4}~\Rightarrow~I^7$. O baixo pedal pode sugerir um 11 grau aumentado no primeiro acorde da sequência ($C:~_{11\#add}bV$). Uma notação sintética desta sequência sugere o padrão  $C:~_{11\#add}bV~\Rightarrow~_{5}IV^{4-5}~\Rightarrow~I^{7-8}$.
\centering{\input{./JarretSchenker}}
\end{example}


\subsection{NI-Akoustik Piano}\label{sec:NI}

\emph{A Study in Keith} pode ser observado como uma simulação de \emph{Disklavier Sessions} \citeonline{sorensen_disklavier_2013}. Este último caso não é citado por  \citeonline{sorensen_keith_2009} e \citeonline{sorensen_youtube_2014} mas a improvisação de códigos é semelhante àquela de \emph{A Study in Keith}.

\emph{A Study in Keith} utiliza o piano virtual \emph{Akoustic Piano NI} (\emph{Native Instruments})\footnote{Disponível em \url{http://www.native-instruments.com/en/company/}.}. Como um \emph{plugin} VST, os sons do instrumento virtual são gravações nota-a-nota de um piano acústico; a tomada de som é realizada em diversos pontos do tampo harmônico, de forma que detalha as ressonâncias do instrumento. São então amostrados digitalmente\disponivelem{http://www.native-instruments.com/en/products/komplete/keys/definitive-piano-collection/}.

\emph{Disklavier Sessions} utiliza um piano \emph{Disklavier} da Yamaha, um modelo que internalizou um computador registrador de eventos, que podem ser codificados e então são convertidos como movimentos do martelo do Piano:

%\begin{figure}[h]
% \centering
%  \includegraphics[scale=0.5]{imagens/disklavier.jpg}
%  \caption{Piano Disklavier de armário, com a parte interna exposta para exibir a placa-mãe. \textbf{Fonte}: wikimedia.org}
%  \label{fig:disklavier}
%\end{figure}

\begin{citacao}
\traducao{Em \emph{Disklavier Sessions} os programas escritos em tempo-real por Ben e Andrew geram um fluxo de dados de notas que é enviado para ser executado em um piano disklavier mecanizado. Assim como as alturas das notas, toda a performance do piano deve ser codificada na informação gerada pelo programa e enviada para o piano disklavier.}{In the Disklavier Sessions the programs beign written in real-time by Ben and Andrew are generating a live stream of note data which is sent to a mechanized disklavier piano to be performed. As well the individual note pitches all of the piano performance must be encoded into the information being generated by the program and sent to disklavier piano}
\end{citacao}

%Sorensen e Swift controlam os eventos MIDI e osconvertidos em ações do martelo do piano \ver{sec:eventos}. No momento, podemos dizer que são programados, em tempo-real, em um \emph{software}/Ambiente de programação nomeado como \emph{Impromptu}, cuja base de desenvolvimento é o \emph{Extempore}. 

\subsection{Ambiente e Linguagem: Impromptu}\label{sec:impromptu}

\begin{citacao}
\traducao{
Impromptu é uma linguagem e um ambiente de programação OSX\footnote{Sistema Operacional Mac OSX.} para compositores, artistas sonoros, VJ's e artistas gráficos  com um interesse em programação ao vivo ou $[$programação$]$ interativa. Impromptu  é um ambiente de linguagem Scheme, um membro da família das linguages Lisp. Impromptu é usado por artistas-programadores em performances de \emph{livecoding} em torno do mundo.
}{
Impromptu is an OSX programming language and environment for composers, sound artists, VJ's and graphic artists with an interest in live or interactive programming. Impromptu is a Scheme language environment, a member of the Lisp family of languages. Impromptu is used by artist-programmers in livecoding performances around the globe.\emph{Disponível em \url{http://impromptu.moso.com.au/}}}
\end{citacao}

Segundo \citeonline[p.~823]{sorensen_impromptu_2010}, o Impromptu é um ambiente de programação ciberfísico, análogo à \emph{partitura} tradicional. O ambiente suporta a compilação de pequenos trechos de códigos executáveis em linguagem \emph{Scheme} (\csf{L}{ask}). Nos termos de \citeonline{magnusson_algorithms_2011}, os algoritmos codificados nesta linguagem \emph{são o instrumento}:

\begin{citacao}
\traducao{Considere a analogia da partitura musical tradicional. A partitura provê uma especificação estática da intenção -- um programa de domínio estático. Musicistas, representam o domínio do processo, executam ações requeridas para realizar ou reificar a partitura. Finalmente, as ações no domínio do processo resultam em ondas sonoroas que são percebidas por uma audiência humana como música. Este estágio final é o nosso domínio real de trabalho. Agora considere um domínio de programação dinâmica no qual o compositor concebe e descreve uma partitura em \emph{tempo-real}. Nós geralmente chamamos este tipo de composição de improvisação. \textbf{Na improvisação o(a) musicista é envolvido em um circuito-fechado retroalimentado que envolve premeditação, movendo para ação casual e finalmente para reação, refinamento e reflexão.}}{
Consider the analogy of a traditional musical score. The score provides a static specification of intention – a static program domain. Musicians, representing the process domain, perform the actions required to realise or reify the score. Finally, the actions in the process domain result in sound waves which are perceived by a human audience as music. This final stage is our real-world task domain. Now consider a dynamic program domain in which a composer conceives of and describes a musical score in real-time. We commonly call this type of composition improvisation. In it, the improvising musician is involved in a feedback loop involving forethought, moving to causal action and finally toreaction, refinement and reflection.}
\end{citacao}

Existe uma restrição quanto ao nicho de usuários do \emph{software}, com suporte para usuários de computadores Apple. 

Para lidar com outros sistemas (como por exemplo, sistemas operacionais Linux) e outras arquiteturas de processamento (32bit e 64 bit), o projeto foi liberado como código-aberto, com o nome \emph{Extempore}.

\subsection{Extempore}

O \emph{Extempore} possui um sistema de \emph{programação ciberfísica} reflexiva \ver{sec:grossi}:

\begin{citacao}
\traducao{\emph{Extempore} é projetado para suportar um estilo de programação apelidado de $[$''$]$programação ciberfísica''. Programação ciberfísica suporta a noção de um programador humano operando como um agente ativo em uma rede distribuída em tempo-real de sistemas ambientalmente conscientes.} {Extempore is designed to support a style of programming dubbed 'cyberphysical' programming. Cyberphysical programming supports the notion of a human programmer operating as an active agent in a real-time distributed network of environmentally aware systems. \disponivelem{https://github.com/digego/extempore}. }
\end{citacao}

Entre suas características incluímos \disponivelem{http://benswift.me/2012/08/07/extempore-philosophy/}:

- Codificação  de alto-nível em linguagem Scheme;

- Processamento de Sinais Digitais (DSP) \footnote{Sobre DSP, \cfcite{smith_dsp_2012}.} em  tempo-real;

- Sequenciamento de áudio, baseado em notas, como o disparo de sons parametrizados em altura, intensidade e duração \disponivelem{http://benswift.me/2012/10/15/playing-an-instrument-part-i/};

As primeira e terceiras características serão explorada neste capítulo como base do processo de codificação na improvisação \emph{Study in Keith}.

\subsection{Scheme}\label{sec:scheme}

Esta subseção demonstra a \emph{Linguagem de A Study in Keith} \ver{tab:universodeconceitos}, ou o dialeto \emph{Scheme} da linguagem de programação \emph{Lisp}. Abaixo descrevemos uma característica pertinente à análise de \emph{Study in Keith}, a saber, sua expressão textual através de uma gramática generativa:

%\begin{citacao}
%\traducao{Um sistema chamado LISP (para Processador de LISta) foi desenvolvido para um computador IBM 704 pelo grupo de Inteligência Artificial no M.I.T. O sistema foi projetado para facilitar experimentos com um sistema proposto chamado ``Recebedor de conselhos'' $[$Advice Taker$]$, onde uma máquina pode ser instruída para lidar com sentenças declarativas, bem como imperativas, e poderia exibir um ``senso comum'' no desempenho de suas instruções. A proposta original para o \emph{Advice Taker} foi feita em novembro de 1958. O principal requerimento foi um sistema de programação para manipular expressões que representam sentenças formais, declarativas e imperativas, de modo que o sistema \emph{Advice Taker} pode fazer deduções. No curso do desenvolvimento, o sistema LISP passou por diversas simplificações e, eventualmente, se baseou em um esquema para representar funções recursivas parciais de certas classes de expressões simbólicas. Esta representação é independente do computador IBM 704, ou qualquer outro computador eletrônico, e agora parece útil expor o sistema, começando com a classe de expressões chamadas expressões-S e as chamadas funções-S \cite[seção 1]{mccarthy_recursive_1960}.}{A programming system called LISP (for LISt Processor) has been developed for the IBM 704 computer by the Artificial Intelligence group at M.I.T. The system was designed to facilitate experiments with a proposed system called the Advice Taker, whereby a machine could be instructed to handle declarative as well as imperative sentences and could exhibit ``common sense'' in carrying out its instructions. The original proposal [1] for the Advice Taker was made in November 1958. The main requirement was a programming system for manipulating expressions representing formalized declarative and imperative sentences so that the Advice Taker system could make deductions.In the course of its development the LISP system went through several stages of simplification and eventually came to be based on a scheme for representing the partial recursive functions of a certain class of symbolic expressions. This representation is independent of the IBM 704 computer, or of any other electronic computer, and it now seems expedient to expound the system by starting with the class of expressions called S-expressions and the functions called S-functions.}
%\end{citacao}

%A definição de funções-S foge do escopo de nossa pesquisa, mas ela pode ser compreendida de maneira intuitiva, a partir das expressões-S. \citeonline[seção~3]{mccarthy_recursive_1960} define expressões- S como ``átomos'' e listas de átomos, onde um átomo também pode ser uma lista de átomos. Existe uma classe de expressões simbólicas definida por parênteses. Dentro desta expressão simbólica são inseridos átomos (ver exemplo \ref{ex:s-expression}, \ref{ex:s-expression2} e \ref{ex:s-expression3}). 

%%%%%%%%%%%%%%%%%%%%%%%%%%%%%%%%%%%%%%%%%%%%%%%%%
\begin{example}{Expressão simbólica vazia}\label{ex:s-expression}
\begin{minted}[fontsize=\scriptsize]{cl}
( )
\end{minted}
\end{example}
%%%%%%%%%%%%%%%%%%%%%%%%%%%%%%%%%%%%%%%%%%%%%%%%%

Segundo \citeonline[seção~3]{mccarthy_recursive_1960}, listas são sentenças abstratas de átomos, que são os elementos constituintes de uma lista:

%%%%%%%%%%%%%%%%%%%%%%%%%%%%%%%%%%%%%%%%%%%%%%%%%
\begin{example}{Expressão simbólica com átomos}\label{ex:s-expression2}
\begin{minted}[fontsize=\scriptsize]{cl}
;;A     -> 
( A )

;;AB    -> 
( A B )

;;ABA   ->
( A B A )

;;ABAC  ->
( A B A C )
( A B A C A )
\end{minted}
\end{example}
%%%%%%%%%%%%%%%%%%%%%%%%%%%%%%%%%%%%%%%%%%%%%%%%%

Átomos também podem ser outras listas:

%%%%%%%%%%%%%%%%%%%%%%%%%%%%%%%%%%%%%%%%%%%%%%%%%
\begin{example}{Expressão simbólica com átomos}\label{ex:s-expression3}
\begin{minted}[fontsize=\scriptsize]{cl}
;;A = A
;;B = AB
;;C = BAB

;; ABA  -> 
( A ( A B ) A )
;; ABAC ->
( A ( A B ) A ( B A B ) )
( A ( A B ) A (( A B ) A ( A B )))
\end{minted}
\end{example}
%%%%%%%%%%%%%%%%%%%%%%%%%%%%%%%%%%%%%%%%%%%%%%%%%

Uma característica da \emph{Linguagem de A Study in Keith} (\csf{L}{ask}) é sua notação para expressar uma proposição como  ``some as unidades de uma lista'':

%%%%%%%%%%%%%%%%%%%%%%%%%%%%%%%%%%%%%%%%%%%%%%%%%
\begin{example}{Notação prefixada}
\begin{minted}[fontsize=\scriptsize]{cl}
;;A = 1
;;B = +
;;C = 2
;; ABA  -> BAA
( + 1 1 )  ;; = 2

;;ABAC -> BAAC -> 
( + 1 1 2 );; = 4

;;ABACA -> BAACA -> 
( + 1 1 2 1);; = 5
\end{minted}
\end{example}
%%%%%%%%%%%%%%%%%%%%%%%%%%%%%%%%%%%%%%%%%%%%%%%%%

Existe um vocabulário pré-definido para criação de sentenças, de forma que o significado do código possa ser legível para os propósitos desejados

%%%%%%%%%%%%%%%%%%%%%%%%%%%%%%%%%%%%%%%%%%%%%%%%%
\begin{example}{Notação Scheme}
\begin{minted}[fontsize=\scriptsize]{cl}
;; define A = 1
(define A 1)

;; define B = 2
(define B 2)

;; divisao na forma (lambda argumentos operacao) 
(define divide            ;; define nome da funcao
        (lambda (a b)     ;; argumentos da funcao (calculo lambda)
                (/ a b))  ;; o que faz a funcao
)

;; execucao descritiva
(divide A B)
\end{minted}
\end{example}
%%%%%%%%%%%%%%%%%%%%%%%%%%%%%%%%%%%%%%%%%%%%%%%%%

Para os propósitos deste trabalho, será útil apresentar um código musical fictício, como um protótipo de \emph{jazz} tonal. O exemplo abaixo alterna citações e códigos para contextualizarmos um pseudo-código descrito por \citeonline[p.~823-824]{sorensen_impromptu_2010}:

%%%%%%%%%%%%%%%%%%%%%%%%%%%%%%%%%%%%%%%%%%%%%%%%%
\begin{example}{Elaboração/Codificação Musical em Scheme}
Este exemplo é semelhante ao primeiro algoritmo de \emph{A Study in Keith} \ver{sec:eventos}:

\begin{citacao}
\traducao{
\small{Dois performers se apresentam no palco. Um violinista, em pé e parado, com seu arco preparado. Outro senta-se atrás do brilho da tela do \emph{laptop}. Uma projeção da tela do \emph{laptop} é projetada acima do palco, e mostra uma página em branco, com um simples cursor piscando. O musicista-programador começa a digitar \ldots}
}{
Two performers are present on stage. One, a violinist, stands paused, bow at the ready. Another sits behind the glow of a laptop screen. A projection of the laptop screen is cast above the stage showing a blank page with a single blinking cursor. The laptop musician begins to type ...
}
\end{citacao}

\begin{minted}{cl}
( play-sound ( now ) synth c3 soft minute)
\end{minted}

\begin{citacao}
\traducao{
\small{\ldots a expressao é avaliada, e lampeja no retroprojetor, para exibir a ação do executante. Um som etéreo sintetizado entra imediatamente no espaço e o violinista começa a improvisar em simpatia com a novidade da textura. O músico-programador, ouve o material temático fornecido pelo violinista e começa a delinear um processo generativo Markoviano para acompanhar o violino:}
}
{
(\ldots) the expression is evaluated and blinks on the overhead projection to display the performer’s action. An ethereal synthetic sound immediately enters the space and the violinist begins to improvise in sympathy with the newly evolving synthetic texture. The laptop performer, listens to the thematic material provided by the violinist and begins to outline a generative Markov process to accompany the violin ...
}
\end{citacao}


\begin{minted}[fontsize=\scriptsize]{cl}
( define chords
  ( lambda ( beat chord duration )
    ( for-each ( lambda ( pitch )
                   ( play synthj pitch soft duration ))
               chord )
    ( schedule (* metro * ( + beat duration )) chords
               (+ beat duration )
               ( random ( assoc chord (( Cmin7 Dmin7 )
                                       ( Dmin7 Cmin7 ))))
               duration )))

( chords (* metro * get-beat 4) Cmin7 4)
\end{minted}
%%%%%%%%%%%%%%%%%%%%%%%%%%%%%%%%%%%%%%%%%%%%%%%%%

\begin{citacao}
\traducao{\small{\ldots A função \emph{chords} é chamada no primeiro tempo e um nova barra de tempo, e uma simples progressão recursiva de acordes começa a suportar a performance melódica do violino. A função \emph{chords} cria um laço temporal, gerando uma sequência interminável de acordes de quatro tempos. Depois de poucos momentos de reflexão, o musicista-programador começa a modificar a função \emph{chords} para suportar uma progressão de acordes mais variada, com uma razão aleatória $[$em função$]$ da recursão temporal\ldots}}
{\ldots the “chords” function is called on the first beat of a new common time bar and a simple recursive chord progression begins supporting the melodic performance of the violin. The chord function loops through time, creating an endless generative sequence of four beat chords. After a few moments of reflection the laptop performer begins to modify the “chords” function to support a more varied chord progression with a randomised rate of temporal recursion\ldots}
\end{citacao}

%%%%%%%%%%%%%%%%%%%%%%%%%%%%%%%%%%%%%%%%%%%%%%%%%
\begin{minted}[fontsize=\scriptsize]{cl}
( define chords
  ( lambda ( beat chord duration )
    ( for-each ( lambda ( pitch )
                   ( play dls (+ 60 pitch) soft duration ))
               chord )
    ( schedule (* metro * ( + beat duration )) chords
               (+ beat duration )
               ( random ( assoc chord (( Cmin7 Dmin7 Bbmaj )
                                       ( Bbmaj Cmin7 )
                                       ( Dmin7 Cmin7 )))
               ( random (3 6))))))
               
( chords (* metro * get-beat 4) Cmin7 4)
\end{minted}
\end{example}
%%%%%%%%%%%%%%%%%%%%%%%%%%%%%%%%%%%%%%%%%%%%%%%%%

Este código é a \emph{estratégia transversal de A Study in Keith}, ou \csf{T}{ask}:

%%%%%%%%%%%%%%%%%%%%%%%%%%%%%%%%%%%%%%%%%%%%%%%%%
\begin{example}{Nome da estratégia transversal}
\verb|chords| é o nome da estratégia.

\begin{minted}[fontsize=\scriptsize]{cl}
;; Definicao de acordes
( define chords
...
)
\end{minted}
\end{example}
%%%%%%%%%%%%%%%%%%%%%%%%%%%%%%%%%%%%%%%%%%%%%%%%%

A função \verb|chords| é executada como um impulso musical, com um único acorde com os seguintes parâmetros: momento de execução, grau e qualidade do acorde, e duração do acorde:

\begin{example}{Estímulo inicial para a estratégia}
\begin{minted}[fontsize=\scriptsize]{cl}
; Execucao da funcao
( chords (* metro * get-beat 4) Cmin7 4)
\end{minted}
\end{example}


Adiante são definidas propriedades com termos do vocabulário da música tonal, ou, coloquialmente, batida (no sentido da posição de uma unidade de tempo em um pulso, \emph{tactus}), acorde (tríades, tétrades, formadas por relações de intervalos de terças maiores e menores), e duração (o quanto, em relação à unidade de tempo, este acorde irá durar):

%%%%%%%%%%%%%%%%%%%%%%%%%%%%%%%%%%%%%%%%%%%%%%%%%
\begin{example}{O que operacionaliza a estratégia}
\begin{minted}[fontsize=\scriptsize]{cl}
( ...
    ( lambda ( beat chord duration )
  ...
)    
\end{minted}
\end{example}
%%%%%%%%%%%%%%%%%%%%%%%%%%%%%%%%%%%%%%%%%%%%%%%%%%%

Existem duas estratégias internas na estratégia principal, cuja execução é realizada atavés de outras palavras-chaves. A palavra-chave \verb|for-each| realiza um laço iterativo para cada altura do acorde:

%%%%%%%%%%%%%%%%%%%%%%%%%%%%%%%%%%%%%%%%%%%%%%%%%%%%%%%%
\begin{example}{Laço iterativo para cada altura do acorde}
\begin{minted}[fontsize=\scriptsize]{cl}
;; Primeira estrategia interna 
;; Para cada acorde operacionalize cada altura
( for-each ( lambda ( pitch )
               ( play dls (+ 60 pitch) soft duration ))
           chord )
\end{minted}
\end{example}
%%%%%%%%%%%%%%%%%%%%%%%%%%%%%%%%%%%%%%%%%%%%%%%%%%%%%%%%%%

Para cada acorde \verb|chord|, é tocada uma nota (\verb|pitch|), com um centro em Dó 3 (MIDI 60), em piano (\verb|soft|) e uma duração padrão (\verb|duration|):

%%%%%%%%%%%%%%%%%%%%%%%%%%%%%%%%%
\begin{example}{Execução da nota}
\begin{minted}[fontsize=\scriptsize]{cl}
( play dls (+ 60 pitch) soft duration ))
\end{minted}
\end{example}
%%%%%%%%%%%%%%%%%%%%%%%%%%%%%%%%%%

A palavra-chave \verb|schedule| executa, recursivamente, um fluxo de acordes associados (\verb|random (assoc chord|), em resposta ao estímulo (\verb|( chords (* metro * get-beat 4) Cmin7 4)|). 

%%%%%%%%%%%%%%%%%%%%%%%%%%%%%%%%%%%%%%
\begin{example}{Fluxo de novos acordes}
\begin{minted}[fontsize=\scriptsize]{cl}
( schedule (* metro * ( + beat duration )) chords
               (+ beat duration )
               ( random ( assoc chord (( Cmin7 Dmin7 Bbmaj )
                                       ( Bbmaj Cmin7 )
                                       ( Dmin7 Cmin7 )))
               ( random (3 6))))
\end{minted}
\end{example}
%%%%%%%%%%%%%%%%%%%%%%%%%%%%%%%%%%%%%%%%

O momento de execução deste acorde depende da execução do acorde anterior

%%%%%%%%%%%%%%%%%%%%%%%%%%%%%%%%%%%%%%%%%%%%%%%%%%%%%%%
\begin{example}{Quando novos acordes serão computados}
\begin{minted}[fontsize=\scriptsize]{cl}
( schedule (* metro * ( + beat duration )) chords
                  ...
                 )
\end{minted}
\end{example}
%%%%%%%%%%%%%%%%%%%%%%%%%%%%%%%%%%%%%%%%%%%%%%%%%%%%%%%

Sendo que o acorde será executado logo em seguida que anterior terminar, com uma cadência harmônica escolhida dentre uma lista de cadências, com uma duração randômica entre três e seis unidades de tempo:

%%%%%%%%%%%%%%%%%%%%%%%%%%%%%%%%%%%%%%%%%%
\begin{example}{Propriedades de novos acordes}
\begin{minted}[fontsize=\scriptsize]{cl}
( schedule ... chords
               (+ beat duration )
               ( random ( ... ))
               ( random (3 6)))
\end{minted}
\end{example}
%%%%%%%%%%%%%%%%%%%%%%%%%%%%%%%%%%%%%%%%%%

A escolha de acordes é feita de maneira randômica, segundo uma lista de cadências predeterminadas. Neste ponto, podemos indicar de maneira mais explícita uma regra de qualidade \ver{tab:universodeconceitos}:

%%%%%%%%%%%%%%%%%%%%%%%%%%%%%%%%%%%%%%%%%%%%
\begin{example}{Propriedades de novos acordes}
\begin{minted}[fontsize=\scriptsize]{cl}
( random ( assoc chord (( Cmin7 Dmin7 Bbmaj )
                        ( Bbmaj Cmin7 )
                        ( Dmin7 Cmin7 )))
\end{minted}
\end{example}
%%%%%%%%%%%%%%%%%%%%%%%%%%%%%%%%%%%%%%%%%%%

Como definido pela função \verb|chords|, o acorde será tocado em um momento que depende do cronograma, cuja duração pode variar de 3 a 6 unidades de tempo. No caso, é prototipado um fluxo recursivo de acordes.

\begin{figure}
  \centering
  \includegraphics[scale=0.3]{imagens/markov.png}
  \caption{Distribuição, aproximada, de probabilidades de acontecimento com um conjunto de possíveis cadências tonais organizados como uma cadeia de Markov. \textbf{Fonte}: \citeonline{swift_playingII_2012}.}
   \label{fig:markov}
\end{figure}

No caso do bloco de código de \citeonline[p.~823-824]{sorensen_impromptu_2010}, são utilizadas os seguintes movimentos harmônicos: $I^{7+}~\Rightarrow ii^{7}~\Rightarrow~IV^{7}/IV$, e $IV^{7}/IV~\Rightarrow~I^{7}$ e $ii^{7}~\Rightarrow~I^{7}$.

%A partir de uma exploração destes referentes opcionais -- $[$\csf{R}{ask}{0},~\csf{R}{ask}{1},~\csf{R}{ask}{2}~$]$ -- foi possível classificar, além de uma linguagem \pressingthree{L}{ask}{0}, o algoritmo gerador da uma sonoridade tonal em \emph{Study in Keith}, ou \csf{T}{ask} \ver{sec:scheme}. Uma análise do algoritmo permite verificar algumas regras de qualidade \csf{E}{ask}. A interpretação $<<<$~\csf{R}{ask},~\csf{T}{ask},~\csf{E}{ask}~$>>>$ produziu uma sequência de blocos de eventos \pressingthree{K}{ask}{0} \ver{sec:eventos}. Outros blocos também são produzidos, porém nossa análise busca investigar o espaço conceitual que possibilitou os primeiros resultados em um ciclo de bricolagem.


\section{Blocos de Eventos}\label{sec:eventos}

Na seção anterior definimos qual é a proposição de \emph{A Study in Keith} e seus referenciais.

Seguiremos com a fase de codificação da estratégia transversal \ver{tab:universodeconceitos} de Sorensen, como regra de detecção \csf{T}{ask}, que possui uma regra de qualidade \csf{E}{ask} \ver{sec:define_instr}. Este espaço conceitual gera uma sequência de blocos de eventos \pressingthree{K}{ask}{0} \ver{tab:modelo_improvisacao}, como um contraponto de primeira espécie \ver{sec:define_instr}, sem relação alguma com o plano harmônico de \emph{Sun Bears} \ver{sec:proposicao}.

Uma nota sobre esta improvisação é feita pelo próprio Sorensen: nos primeiros dois minutos do vídeo (aproximadamente 1$'$53$''$), existe um silêncio característico do momento em que os primeiros códigos são escritos. Este comportamento, do tempo de codificação, ao tempo de ação musical, é similar em outros vídeos de Sorensen: \sorensen{An evening of livecoding at 53 Rusden Street}{https://vimeo.com/2433303}, \sorensen{Just for Fun}{https://vimeo.com/2433971}, \sorensen{A Study in Part}{https://vimeo.com/2434054}, \sorensen{Stained}{https://vimeo.com/2502546}, \sorensen{Transmissions in Sound}{Transmissions in Sound}, \sorensen{Antiphony}{https://vimeo.com/2503188},  \sorensen{Strange Places}{https://vimeo.com/2503257}, \sorensen{Orchestral}{https://vimeo.com/2579694}, \sorensen{UMDT}{https://vimeo.com/2579880}, \sorensen{Day of Triffords}{https://vimeo.com/2735394}, \sorensen{Face to Face}{https://vimeo.com/5690854}, \sorensen{BM\&E}{https://vimeo.com/7339135}, \sorensen{A Christimas Carol}{https://vimeo.com/8364077} \sorensen{Dancing Phalanges}{https://vimeo.com/8732631}, \sorensen{Livecoding Audio DSP}{https://vimeo.com/15585520}, \sorensen{Jazz Ensenble Study}{https://vimeo.com/15679078}, \sorensen{Variations on a Christmas Theme}{https://vimeo.com/18008372}.

\input{./04-Analise}

%\newpage

% ---
% Conclusão
% ---
\chapter*[Conclusão]{Conclusão}\addcontentsline{toc}{chapter}{Conclusão}\label{conclusao}

Este documento contextualizou o que é uma improvisação de códigos, e como podemos entender suas manifestações quando se codificam artefatos sonoros, visuais, corporais e têxteis. Diferente de outras definições, buscamos situar o \emph{live coding} como uma técnica de programação que possibilita elaborar proposições artísticas de qualquer tipo. \citeonline{ward_live_2004} definem a improvisação de códigos como \traducao{``atividade da escrita integral (ou partes) de um programa enquanto ele é executado''}{Live coding is the activity of writing (parts of ) a program while it runs}. \citeonline{blackwell_programming_2005} enfatizam a definição do ponto de vista da linguagem de programação como instrumento musical. \citeonline{mclean_hacking_2006} relata o \emph{live coding} como ferramenta para um \emph{Disk Jockey codificado}.  \citeonline{sorensen_keith_2009} definem \traducao{``uma prática de performance para o qual linguagens de computador definem o meio primário de expressão artística''.}{Live coding is a performance pratice for which computer languages define  the primary means of expression.}. Para \citeonline{sorensen_impromptu_2010}, \emph{live coding} envolve a premissa de uma programação-partitura audiovisual reativa: 

\traduzcitacao{
Livecoding é uma prática de arte computacional que envolve criação em tempo-real de programas de audiovisual generativo para performances multimídias interativas. Comumente as ações dos programadores são expostas para uma audiência por projeção do ambiente de edição. Performances de livecoding geralmente envolvem mais de um participante, e são geralmente iniciadas a partir de uma folha conceitual em branco   
\cite[p.~823]{sorensen_impromptu_2010}}{Livecoding [10, 50] is a computational arts practice that involves the real-time creation of generative audiovisual software for interactive multimedia performance. Commonly the programmers’ actions are exposed to the audience by projection of the editing environment. Livecoding performances often involve more than one participant, and are often commenced from a conceptual blank slate}

\citeonline{magnusson_algorithms_2011,collins_origins_2014} sintetizam o \emph{live coding} como improvisação audiovisual. \citeonline{sorensen_programming_2014} define como \traducao{``programar sistemas de tempo-real em tempo real''}{programming real-time systems in real-time}. Uma discussão intitulada ``\emph{Wtf is livecoding}''\disponivelem{http://lurk.org/groups/livecode/messages/topic/ofAxZpxsKFpDRLnoA48Bh} dificultou o próprio processo de definição, onde o compositor Nick Collins diz que o \traducao{``\emph{Live coding} celebra a efemeridade da própria definição''}{Live coding celebrates the ephemerality of definition itself}. Aqui fica explícito uma \emph{bricolagem} da técnica. \citeonline{sorensen_programming_2014} destaca que modificar alguma coisa é próprio da técnica, de forma que é possível extender essa bricolagem para proposições. Nick \citeonline{collins_origins_2014} situa essa questão da seguinte forma:

  \begin{figure}[h]
    \centering
    \includegraphics[scale=0.7]{imagens/live_coding_def.png}
    \caption{Definição de \emph{live coding}: ``Insira a definição aqui''. \textbf{Fonte}: \citeonline{collins_origins_2014}.}
    \label{fig:live_coding_def}
  \end{figure}

Neste ponto encontramos um desafio à metodologia de pesquisa acadêmica: se o termo que contextualiza um estudo de caso é variável por definição, como analisar este caso? 

Foi necessária uma revisão histórica para entender como a atividade de programar, no sentido de \emph{tocar um instrumento}, se construiu historicamente. Alí noções como \emph{reflexividade} e instrumentos acusticamente divorciados mas informacionalmente ligados, possibilitaram o artistas-programadores se auto-organizarem dentro de suas agremiações. Uma dessas agremiações merece destaque, e surgiu no seio acadêmico inglês. De certa forma esta agremiação prescreveu regras de conduta para improvisar códigos, bem aceitas entre diversos artistas-programadores. Desta agremiação que se tornou uma organização, selecionamos um caso exemplar que segue as regras estipuladas. Investigamos uma improvisação de códigos como uma simulação de uma improvisação instrumental. Esta improvisação instrumental, para facilitação do que pode ser harmonicamente simples, possibilitou destacar a diferença entre o que foi proposto e aquilo que foi plenamente executado.

Por outro lado, julgar \emph{A Study in Keith} apenas com base na sua simplicidade harmônica, se comparada ao pensamento harmônico inicial dos Concertos \emph{Sun Bear}, é obscurecer a possibilidade de uma metodologia composicional, neste caso, uma espécie de \traducao{desenvolvimento orientado a testes}{Test-driven development} musicais. Ao aprofundarmo-nos em uma escavação netnográfica \cite{mori_analysing_2015}, foi possível notar que o mesmo mecanismo de códigos é utilizado em uma performance com pianos acústicos, ou \emph{Disklavier Sessions} de 2011. Neste sentido, o recorte desta pesquisa foi determinar como um algoritmo musical é elaborado e codificado por um artista-programador, aplicado ao desenvolvimento de \emph{softwares}.


% Se por um lado a definição agrega definições, o que dificulta a tarefa inicial de descrever os fundamentos do objeto de pesquisa, por outro ilustra a improvisação de códigos como um \emph{Universo de conceitos}. Neste trabalho consideramos que definições ou performances de improvisação de códigos estão contidas em diferentes \emph{Espaços Conceituais} \cite{wiggins_framework_2006,mclean_music_2006}. Artistas-programadores (\emph{live coders}) transitam entre os Espaços Conceituais para criação de Sistemas Criativos (códigos, programas). Estes Sistemas Criativos são representados em diferentes Linguagens de Programação. Regras práticas conduzem o processo de escrita e exposição desta linguagem; mas não restringem o resultado (no caso da pesquisa, musical). Mas algumas categorizações musicais se destacam.  Neste sentido, selecionamos um exemplo simbólico, \emph{A Study in Keith} de Andrew \citeonline{sorensen_keith_2009}\footnote{Disponível em \url{https://vimeo.com/2433947}.}. Representa um caso particular que foge dos exemplos citados anteriormente, mas envolve a manutenção de uma tradição musical tonal através de um interessante esforço de \emph{replicação do estilo}. No \autoref{cap:introducao} selecionamos alguns exemplos afim de ilustrar nossa percepção (conhecimento-psicológico) do imaginário daquilo que \citeonline{McLean2011} chama de artistas-programadores. No \autoref{sec:protohistoria} buscamos levantar um conjunto de conhecimentos históricos. No \autoref{cap:metodologia}, discutimos um modelo de formalização conceitos, observados pelo prisma de Alex \citeonline{mclean_music_2006}, contraposto por \citeonline{thornton_quantitative_2007}, e rediscutido por \citeonline{Forth2010} \citeonline{McLean2011}. No \autoref{cap:estudos_de_caso}, organizamos conceitos de um algoritmo inicial de uma improvisação de códigos, \emph{A Study in Keith} de 2009, segundo este um Quadro Conceitual de Sistemas Criativos. Finalizamos este trabalho com o apêndice \autoref{app:A}, onde descrevemos um processo de organização de tais qualidades da improvisação de códigos. 


% ---
% Bibliografia
% ---
\bibliography{main}

% ----------------------------------------------------------
% Glossário
% ----------------------------------------------------------
%
% Consulte o manual da classe abntex2 para orientações sobre o glossário.
%
%\glossary

% ----------------------------------------------------------
% Apêndices
% ----------------------------------------------------------

% Imprime uma página indicando o início dos apêndices
%\partapendices

% ---
% Inicia os apêndices
% ---
%\begin{apendicesenv}
%\include{./A-nuvem}
%\include{./B-GROOVE}
%\end{apendicesenv}

\phantompart
\printindex
%---------------------------------------------------------------------
\end{document}