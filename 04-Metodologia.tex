\newpage

\begin{table}[!h]
\caption{Definições formais do Universo de possibilidades de \citeonline{wiggins_framework_2006}, ou Universo de Conceitos por \citeonline{mclean_music_2006,Forth2010}. Neste trabalho, como quadro de proposições.}
\small
    \begin{tabular}{ | p{4.25cm} | p{5.25cm} | p{5.25cm} |}
    \hline 
    \hline 

    Representação
    & \tiny{Nome}     
    & \tiny{Significado} \\
    \hline

    $c$
    & \tiny{Conceito} 
    & \tiny{Uma instância de um conceito, abstrato ou concreto \cite{wiggins_framework_2006}}. \\
    \hline

    $\mathcal{U}$
    & \tiny{Universo de Conceitos} 
    & \tiny{Superconjunto não restrito de conceitos. \cite{wiggins_framework_2006}. ``Um universo de todos conceitos possíveis'' \cite{mclean_music_2006} \tablefootnote{Tradução de \emph{A universe of all possible concepts}.}}\\
    \hline

    $\mathcal{L}$
    & \tiny{Linguagem} 
    & \tiny{Linguagem utilizada para expressar regras.} \\
    \hline

    $\mathcal{A}$
    & \tiny{Alfabeto} 
    & \tiny{Alfabeto da linguagen que contêm caracteres apropriadospara expressão das regras} \\
    \hline

    $\mathcal{R}$
    & \tiny{Regras de validação} 
    & \tiny{Validam os conceitos em um universo, se apropriados ou não para o espaço trabalhado.} \\
    \hline

    $[[.]]$
    & \tiny{Função de interpretação} 
    & \tiny{``Uma função parcial de $\mathcal{L}$ para funções que resultam em números reais entre [0, 1] (\ldots) 0.5 $[$ou maior$]$ significa uma verdade booleana e menos que 0.5 siginifica uma falsidade booleana; a necessidade disso para valores reais se tornará clara abaixo'' \cite[p.~452]{wiggins_framework_2006}\tablefootnote{Tradução de \emph{(\ldots) a partial function from $\mathcal{L}$ to functions yielding real numbers in [0, 1]. (\ldots) 0.5 to mean Boolean true and less than 0.5 to mean Boolean false; the need for the real values will become clear below}.}}\\
    \hline

     $[[\mathcal{R}]]$
    & \tiny{Regras de validação} 
    & \tiny{``Uma função que interpreta $\mathcal{R}$, resultando em uma função indicando aderência ao conceito em $\mathcal{R}$''\tablefootnote{Tradução de \emph{A function interpreting $\mathcal{R}$, resulting in a function indicating adherence of a concept to $\mathcal{R}$}}} \\
    \hline

     $\mathcal{C} = [[\mathcal{R}]](\mathcal{U}) $
    & \tiny{Espaço Conceitual} 
    & \tiny{``Todos espaços conceituais são um subconjunto não-estrito de $\mathcal{U}$''\tablefootnote{Tradução de \emph{All conceptual spaces are non-strict subset}.}. Um subconjunto contido em $\mathcal{U}$ \cite{wiggins_framework_2006}. Uma função que interpreta $\mathcal{R}$, resultando em uma função que indica aderência ao conceito em $\mathcal{R}$ \tablefootnote{Tradução de \emph{A function interpreting $\mathcal{R}$, resulting in a function indicating adherence of a concept to $\mathcal{R}$}.} } \\
    \hline

    $\mathcal{T}$
    & \tiny{Regras de detecção} 
    & \tiny{``Regras definidas dentro de $\mathcal{L}$ para definir estratégias transversais para localizar conceitos dentro de $\mathcal{U}$'' \cite{mclean_music_2006}\tablefootnote{Tradução de \emph{Rules defined within $\mathcal{L}$ to define a traversal strategy to locate concepts within $\mathcal{U}$ }}} \\
    \hline

    $\mathcal{E}$
    & \tiny{Regras de qualidade} 
    & \tiny{``(\ldots) conjunto de regras que permitem-nos avaliar qualquer conceito que nós encontramos em $\mathcal{C}$ e determinar sua qualidade, de acordo com critérios que nós considerarmos apropriados'' \cite[p.453]{wiggins_framework_2006}\tablefootnote{Tradução de \emph{(\ldots) set of rules which allows us to evaluate any concept we find in C and determine its quality, according to whatever criteria we may consider appropriate.}}``Regras definidas dentro de $\mathcal{L}$ para avaliar a qualidade ou a desejabilidade do conceito $c$'' \cite{mclean_music_2006}\tablefootnote{Tradução de \emph{Rules defined within $\mathcal{L}$ which evaluate the quality or desirability of a concept $c$.}}}\\
    \hline

    $<<<\mathcal{R}, \mathcal{T}, \mathcal{E}>>>$
    & \tiny{Função de interpretação} 
    & \tiny{Uma regra necessária para definir o espaço conceitual, ``independentemente da ordem, mas também, ficcionalmente, enumerá-los em uma ordem particular, sob o controle de $\mathcal{T}$ -- isto é cricial para a simulação de um comportamento criativo de um $\mathcal{T}$ particular \cite{wiggins_framework_2006} \tablefootnote{Tradução de \emph{We need a means not just of defining the conceptual space, irrespective of order, but also, at least notionally, of enumerating it, in a particular order, under the control of $\mathcal{T}$ -- this is crucial to the simulation of a particular creative behaviour by a particular $\mathcal{T}$.}}. ``Uma função que interpreta a estratégia transversal $\mathcal{T}$, informada por $\mathcal{R}$ e $\mathcal{E}$ . Opera sobre um subconjunto ordenado de $mathcal{U}$ (do qual tem acesso randômico) e resulta em outro subconjunto ordenado de $\mathcal{U}$.''\tablefootnote{Tradução de \emph{A function interpreting the traversal strategy $\mathcal{T}$, informed by $\mathcal{R}$ and $\mathcal{E}$ . It operates upon anordered subset of $mathcal{U}$ (of which it has random access) and results in another ordered subset of $\mathcal{U}$.}}} \\
    \hline
    \hline
   
    \end{tabular}
\label{tab:universodeconceitos}
\end{table}


\subsection{O modelo de improvisação}\label{sec:im}

Segundo Pressing, o Modelo de Improvisação é ``um esboço para uma teoria geral da improvisação integrada com preceitos da Psicologia Cognitiva'' \cite[p.~2]{pressing_improvisation_1987}. Este modelo será utilizado para especificar elementos de uma performance exemplar, como o caso investigado neste trabalho. Por exemplo, uma improvisação particionada em diferentes sequências pode ser parcialmente mapeada em categorias, como blocos sonoros, referentes conceituais e normas estilísticas, conjuntos de objetivos e processos. Este nos pareceu um modelo mais transparente para o compositor, músico e intérprete. O que não quer dizer que é possível readequar ambos para nosso interesse. Um sumário sobre o modelo de improvisação é apresentado na \autoref{tab:modelo_improvisacao}. Por seu caráter lógico, parece ser uma possibilidade interessante, e assumiremos como tal.

\begin{table}[!h]
\caption{Definições formais do Modelo de improvisação de Jeff \citeonline{pressing_improvisation_1987}, segundo \citeonline[p.~2]{mclean_music_2006}.}
\small
    \begin{tabular}{ | p{6cm} | p{9cm} |}
    \hline 
    \hline 

    \tiny{Representação}   
    & \tiny{Significado} \\
    \hline

    $E'$
    & \tiny{Um bloco de eventos sonoros}\tablefootnote{\emph{A cluster of sound events}.} \\
    \hline

    $K'$
    & \tiny{Uma seqüência de blocos de eventos E, onde um bloco de eventos não se sobrepõe com o seguinte}\tablefootnote{A sequence of E event clusters, where event cluster onsets do not overlap with those of a following one}\\
    \hline

    $I'$
    & \tiny{Uma improvisação, particionada por interrupções em um número de K sequências}\tablefootnote{An improvisation, partitioned by interrupts into a number of K sequences} \\
    \hline

    $R'$
    & \tiny{Um referente opcional, tal como uma partitura ou uma norma estilística}\tablefootnote{An optional referent, such as a score or stylistic norm} \\
    \hline

    $G'$
    & \tiny{Um conjunto de objetivos }\tablefootnote{A set of current goals.} \\
    \hline

    $M'$
    & \tiny{Uma memória de longo prazo}\tablefootnote{Long term memory.} \\
    \hline

    $O'$
    & \tiny{Um conjunto de objetos}\tablefootnote{An array of objects.} \\
    \hline

    $F'$
    & \tiny{Um conjunto de características dos objetos}\tablefootnote{An array of objects Features.} \\
    \hline

    $P'$
    & \tiny{Um conjunto de processos}\tablefootnote{An array of Process} \\
    \hline
    \hline
   
    \end{tabular}
\label{tab:modelo_improvisacao}
\end{table}