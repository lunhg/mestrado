É interessante notar que o o PIC do  \emph{livecoding} realiza um acesso constante às suas liminaridades.

Este acesso levou, nas primeiras críticas a este trabalho, aos seguintes questionamentos: o \emph{livecoding} apenas se configura como tal se projetarmos um código textual par aum público, e improvisarmos neste código? A utilização de linguagens visuais, como \emph{PD} ou \emph{Max/Msp}, ao vivo (\emph{live patching}) são um caso à parte ou devem ser incluídas no objeto de pesquisa? É pertinente chamar de \emph{live coding} uma performance que utiliza \emph{DAWs}\footnote{\emph{Digital Audio Workstation}.}\emph{Live}!\emph{Ableton}? Utilizar dispositivos diferentes do \emph{laptpop}, como corpos dançantes, em associação com dispositivos digitais, é \emph{livecoding}? Se tais perguntas apenas expõem as multiplicidades do \emph{livecoding}, dificultam uma análise de caso. 

Uma pergunta mais específica irá delinear o método de pesquisa, tomando por base, um \emph{universo de conceitos do livecoding} $\mathcal{U}_\emph{livecoding}$. Este, por sua vez, contêm infinitos espaços conceituais $\mathcal{C}$, com instâncias de conceitos $c$. Para efeito de estudo selecionamos dois espaços conceituais: $\mathcal{C}_\emph{livecoding}$ e $\mathcal{C}_\emph{Study in Keith}$, como definido na \autoref{eq:universo_pesquisa}.

\begin{equation}
\mathcal{U}_\emph{pesquisa} = [\ldots, \mathcal{C}_\emph{livecoding} \bigcup \mathcal{C}_\emph{Study in Keith}, \ldots]
\end{equation}\label{eq:universo_pesquisa}

Mais especificamente,\textbf{O que pode ser pressuposto musicalmente em uma sessão de \emph{livecoding} nomeada \emph{Study in Keith}?} Esta pergunta leva a especificar o espaço conceitual da pesquisa, como definido nas \autoref{eq:espaco_pesquisa}.

\begin{equation}
\mathcal{C}_\emph{pesquisa} = \mathcal{C}_\emph{livecoding} \bigcup \mathcal{C}_\emph{Study in Keith}
\end{equation}\label{eq:espaco_pesquisa}

Por sua vez, os espaços conceituais do \emph{livecoding} e de \emph{Study in Keith} são definidos como uma união entre as regras de validação, de gosto e de localização de conceitos no espaço dado:

\begin{equation}
\mathcal{C}_\emph{livecoding} = <<<\mathcal{R}_\emph{livecoding}, \mathcal{T}_\emph{livecoding}, \mathcal{E}_\emph{livecoding}>>>
\end{equation}\label{eq:livecoding}

\begin{equation}
\mathcal{C}_\emph{Study in Keith} = <<<\mathcal{R}_\emph{Study in Keith}, \mathcal{T}_\emph{Study in Keith},  \mathcal{E}_\emph{Study in Keith} >>> 
\end{equation}\label{eq:keith}

\section{Método de pesquisa}\label{conjunto_conhecimentos}

Antes de avançar com a aplicação das equações acima, propomos uma digressão sobre a multiplicidade de regras impostas pelo \emph{Universo de Possibilidades} ou \emph{Universo de Conceitos}.

Tomando o \emph{live coding} como uma feira de idéias, propomos pensar este Programa de Investigação científica de um ponto de vista social. Boaventura de Souza \citeonline{santos_abissal_2007,santos_filosofia_2008} entende  que pesquisador é forçado, na academia, a limitar aquilo que cada vez mais se expande. Santos discute isso da seguinte maneira:

\begin{citacao}
O saber só existe como pluralidade de saberes, tal como a ignorância só existe como pluralidade de ignorâncias. As possibilidades e os limites de compreensão e de acção de cada saber só podem ser conhecidas na medida em que cada saber se propuser uma comparação com outros saberes. Essa comparação é sempre uma versão contraída da diversidade epistemológica do mundo, já que esta é infinita. É, pois, uma comparação limitada, mas é também o modo de pressionar ao extremo os limites e, de algum modo, de os ultrapassar ou deslocar. Nessa comparação consiste o que designo por ecologia de saberes. (\ldots) Sendo sempre limitado o conjunto de saberes que integra a ecologia dos saberes há que definir como se consitituem esses conjuntos. \textbf{À partida, é possível um número ilimitado de ecologia de saberes, tão ilimitado quanto o da diversidade epistemológica do mundo. Cada exercício de ecologia de saberes implica uma selecção de saberes e um campo de interacção onde o exercício tem lugar}. \cite[p.~28-30]{santos_filosofia_2008}.
\end{citacao}


Para diferenciar a aplicação dos estudos sociais, defino, este trabalho, a \emph{ecologia de saberes} como \textbf{um estudo de um subconjunto de elementos categóricos no universo de conceitos $U$, e suas respectivas derivações}.

Derivando conceitos em $\mathcal{C}_\emph{livecoding}$, podemos recorrer às  \autoref{fig:nuvemlivecoding} e \autoref{tab:comparacao} do \autoref{cap:introducao}, nas páginas 13 e 14. Expomos na \autoref{eq:espaco_lc} um fragmento do espaço conceitual multidimensional do \emph{livecoding}.

\begin{equation}
\mathcal{C}_\emph{livecoding} = [\mathcal{Suj}_\emph{livecoding}, \mathcal{App}_\emph{livecoding},  \mathcal{Verb}_\emph{livecoding}, \mathcal{Adj}_\emph{livecoding}, \mathcal{Sub}_\emph{livecoding}, \ldots] \\ \mathcal{Suj}_\emph{livecoding} = [\ldots, "Collins", "Blackwell", "McLean", "Grossi", \ldots] \\ \mathcal{App}_\emph{livecoding} = [\ldots, "SuperCollider", "Gibber", "SonicPi", \ldots] \\ \mathcal{Verb}_\emph{livecoding} = [\ldots, explore, make, coding, perform \ldots] \\ \mathcal{Adj}_\emph{livecoding} = [\ldots, open, electronic, musical, live \ldots] \\ \mathcal{Sub}_\emph{livecoding} = [\ldots, University, environment, text, context, musician]
\end{equation}\label{eq:espaco_lc}

Um desses elementos, \emph{Grossi} destaca a importância de um estudo dos comportamentos históricos do \emph{livecoding}. Por este meio,poderemos observar o \emph{núcleo de pesquisa} e a \emph{heurística} \cite{lakatos_falsification_1970,neto_lakatos_2008} do \emph{live coding}, e separá-lo, por exemplo, da \emph{live Computer Music} (\autoref{eq:lcm}). Em outras palavras, irão permitir definir o conjunto $\mathcal{R}_\emph{livecoding}$ \todo{\tiny será que define $\mathcal{T}_\emph{livecoding}$ também?}

O mesmo processo pode ser aplicado à \emph{Study in Keith}, como na \autoref{eq:espaco_keith}.

\begin{equation}
\mathcal{C}_\emph{Study in keith} = [\mathcal{Suj}_\emph{Study in Keith}, \mathcal{App}_\emph{Study in Keith},  \mathcal{Verb}_\emph{Study in Keith}, \mathcal{Adj}_\emph{Study in Keith}, \mathcal{Sub}_\emph{Study in Keith}, \ldots] \\ \mathcal{Suj}_\emph{Study in keith} = [Andrew Sorensen, Keith Jarret, Jazz, Concertos Sun Bear] \\ \mathcal{App}_\emph{Study in Keith} = [LISP, Impromptu] \\ mathcal{Verb}_\emph{Study in Keith} = [replicação] \\ \mathcal{Adj}_\emph{Study in Keith} = [?], \mathcal{Sub}_\emph{Study in Keith} = [Concerto]
\end{equation}\label{eq:espaco_keith}

Por sua vez, \emph{Study in Keith} é um outro espaço conceitual que inclui o pianista e compositor Keith Jarret como sujeito. Por outro lado, se a necessidade de estudo histórico foi levantada, realizaremos uma pequana contextualização dos seus concertos \emph{Sun Bear} (1976-1979).




