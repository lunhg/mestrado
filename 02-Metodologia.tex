\chapter{Metodologia}\label{cap:metodologia}

%O espaço conceitual da pesquisa é virtualmente infinito. Do ponto de vista da filosofia do conhecimento, Boaventura de Souza \citeonline{santos_filosofia_2008} aponta comparar aquilo que foi limitado com seu contexto:

%\begin{citacao}
%O saber só existe como pluralidade de saberes, tal como a ignorância só existe como pluralidade de ignorâncias. As possibilidades e os limites de compreensão e de acção de cada saber só podem ser conhecidas na medida em que cada saber se propuser uma comparação com outros saberes. Essa comparação é sempre uma versão contraída da diversidade epistemológica do mundo, já que esta é infinita. É, pois, uma comparação limitada, mas é também o modo de pressionar ao extremo os limites e, de algum modo, de os ultrapassar ou deslocar. Nessa comparação consiste o que designo por ecologia de saberes. (\ldots) Sendo sempre limitado o conjunto de saberes que integra a ecologia dos saberes há que definir como se consitituem esses conjuntos. \textbf{À partida, é possível um número ilimitado de ecologia de saberes, tão ilimitado quanto o da diversidade epistemológica do mundo. Cada exercício de ecologia de saberes implica uma selecção de saberes e um campo de interacção onde o exercício tem lugar}. \cite[p.~28-30]{santos_filosofia_2008}.
%\end{citacao}

Carolina Di \citeonline{prospero_social_2015} descreve uma dinâmica das pluralidades conceituais do \emph{live coding} através dos estudos sobre \emph{liminaridade} de \citeonline{turner_comunnitas_1969}. Tais pluralidades são resultantes de períodos onde entidades (pessoas ou instituições) marginais produzem artefatos de conhecimento que podem gerar mudanças estruturais em um espaço de convivência: 

\begin{citacao}
Liminaridade, marginalidade e inferioridade estrutural são condições no qual são frequentemente gerados mitos, rituais, símbolos sistemas filosóficos e obras de arte. Estas formas culturais provêm aos homens um conjunto de  modelos que são, em um nível, reclassificações periódicas da realidade e da relação do homem com a sociedade, natureza e cultura. Mas, eles são mais do que classificações, uma vez que incitam à ação bem como ao pensamento. Cada uma dessas produções tem um caractere multivocal, tendo muitos significados, e cada um capaz de mover as pessoas. \apud[p.~71]{turner_comunnitas_1969}{prospero_social_2015}\footnote{Liminality, marginality, and structural inferiority are conditions in which are frequently generated myths, rituals, symbols, philosophical systems and works of art. These cultural forms providemen with a set of templates or models which are, at one level, periodical reclassifications of reality and man's relationship to society, nature and culture. But, they are more than classifications, since they incite to action as well as to thought. Each of these productions has a multivocal character, having many meanings, and each is capable of moving people}
\end{citacao}

Liminaridade é discutida por \citeonline[p.~96]{turner_comunnitas_1969b} através da coexistência, justaposta e alternada, de duas diferentes estruturas de relação humana:

\begin{citacao}
É como pensar que existem dois grandes "modelos" para a inter-relação humana, justapostas e alternadas. O primeiro é da sociedade como um sistema estruturado e diferenciado, e muitas vezes $[$como$]$ sistema hierárquico de posições político-jurídico-económicas com muitos tipos de avaliação, que separa os homens em termos de "mais" ou "menos". O segundo, que emerge reconhecidamente no período liminar, é da sociedade como um não-estruturado, ou comitato rudimentarmente estruturado e relativamente indiferenciado, a comunidade, ou mesmo comunhão de indivíduos iguais que juntos se submetem à autoridade geral dos anciãos rituais.\footnote{Tradução de \emph{It is as though there are here two major" models" for human interrelatedness,juxtaposed and alternating. The first is of society as a structured, differentiated, and often hierarchical system of politico-legal-economic positions with many types of evaluation, separating men in terms of" more" or "less." The second, which emerges recognizably in the liminal period, is of society as an unstructured or rudimentarily structured and relatively undifferentiated comitatus, community, or even communion of equal individuals who submit together to the general authority of the ritual elders.}}
\end{citacao}

Di prospero esclarece o que seria essa liminaridade no \emph{live coding}: 

\begin{citacao}
Liminaridade é apresentada de algumas formas no \emph{live coding}: novos projetos e propostas, a procura por desmistificar a relação com a tecnologia, tornar o código um artesanato ou um produto artístico, mas, mais do que isso, na construção de uma comunidade participativa, uma comunidade coletiva imaginada. \textbf{Liminaridade do espaço para se expressar e construir várias propostas que suscitam transformações não somente nos campos artísticos e culturais, mas também institucional}, a cena do \emph{live coding} envolve construir o mundo inteiro, um mundo da arte nos termos de \citeonline{becker_art_1982}. De Acordo com o autor, aquele que colabora na produção de uma obra de arte não o faz a partir do nada, mas em acordos passados ou costumes, convenções, que geralmente cobrem as decisões tomadas, e isso torna as coisas mais simples (\emph{Ibdem}, \emph{idem}). \footnote{Tradução nossa de: \emph{Liminality is present in some ways in live coding: new projects and proposals, the search to demystify the relationship with technology, making the code a craft or artistic product, but, more than anything, in the construction of its "participatory community", a collectively imagined community. Liminality of space to express themselves and build various proposals raises transformations not only in the artistic or cultural field but also institutional, the live coding scene involves building an entire world, an art world in terms of Becker (Becker 1982). According to the author, who cooperates in producing a work of art do not do it from nothing but rest on past agreements or custom / conventions, which usually cover the decisions to be taken, and this makes things simpler.}}
\end{citacao} 

Com base nisso, propomos que a investigação sobre uma liminaridade do \emph{live coding} seja construída sobre a noção de \emph{improvisação}, \emph{criatividade} e \emph{espaços conceituais}, no contexto de improvisação de Andrew Sorensen.  Para isso, recorremos ao artigo ``\emph{Music improvisation and creative systems}'' de Alex \citeonline{mclean_music_2006}.

Este artigo discute a improvisação musical com base em outros dois artigos, ``\emph{A preliminary framework for description, analysis and comparison of creative systems}'', de \citeonline{wiggins_framework_2006}, e ``\emph{Improvisation: methods and models}'' de Jeff \citeonline{pressing_improvisation_1987}. McLean propõe ``traduzir o Modelo de Improvisação para funcionar em um modelo computacional'' \cite[p.~5]{mclean_music_2006}\footnote{Tradução parcial de \emph{The above analysis highlights a number of areas for consideration while translating the IM to a working computer model.}} uma adaptação entre a \emph{Ferramenta de Estruturação de Sistemas Criativos}\footnote{\emph{Creative System Framework}.} de Wiggins, e o \emph{Modelo de Improvisação} de Pressing.

\section{Criatividade}\label{sec:sistemas_criativos}

\begin{citacao}
Por definição, criatividade cria, i.e., produz alguma coisa nova. Mas se estamos comprometidos com uma abordagem mecanicista do mundo -- nenhum milagre é permitido -- iremos acreditar que tudo o que ocorre é, em princípio,  previsível. Iremos acreditar também que qualquer coisa nova deve ser construída de componentes existentes. Isso implica que nada pode ser intrinsicamente novo. \cite[p.~2]{thornton_quantitative_2007}\footnote{Tradução de \emph{By definition, creativity creates, i.e., it produces something new. But if we are committed to a mechanistic account of the world — no miracles allowed — we believe that everything that occurs is predictable in principle. We also believe that any new thing must be constructed from existing components. This implies that nothing can ever be intrinsically new.}}
\end{citacao}

Ao discutirem o paradoxo de Boden, \citeonline[p.~450]{wiggins_framework_2006} e Thornton concordam que o ato criativo é sempre mediado por desenvolvimentos conceituais (ver \autoref{tab:interpretacao})

\begin{table}[!h]
\caption{Definições formais do Universo de criatividade para fins quantitativos.}
\small
    \begin{tabular}{ | p{6cm} | p{9cm} |}
    \hline 
    \hline 

    \tiny{Autor}
    & \tiny{Interpretação da criatividade}\\      
    \hline

    $Wiggins$
    & \tiny{\tabletraducao{Boden concebe o processo de criatividade como uma identificação e/ou localização de novos objetos conceituais em um espaço conceitual}{Boden conceives the process of creativity as the identification and/or location of new conceptual objects in a conceptual space.}.} \\
    \hline

    $Thornton$
    & \tiny{\tabletraducao{Qualquer ato criativo é fundado na conceitualização ou realização de um ponto dentro de um espaço conceitual particular}{Any creative act is thus founded on conceptualisation or the realisation of a point within a particular ‘conceptual space’}} \\
    \hline
    \hline
    \end{tabular}
\label{tab:interpretacao}
\end{table}

Podem ocorrer dois tipos de de conceitualização: a que identifica novos pontos no espaço conceitual (comportamento explorador), e a que gera novos pontos (comportamento transformador). Por outro lado, existem os conceitos do tipo pessoais (P-conceitos), elaborados por uma pessoa através da exploração, e os comportamentos do tipo históricos (H-conceitos), que envolvem o comportamento transformacional. Wiggins problematiza esta abordagem ao tocar a esfera social. Segundo o autor, ``pode ser possível, por exemplo, um comportamento ser P-criativo em uma sociedade, mas H-criativo em outra'' (\emph{Ibdem}, \emph{idem}). \footnote{Tradução de \emph{it would be possible, for example, for a creative behaviour to be only P-creative in one society, but H-creative in another.}}.

Também existe uma separação entre as ``idéias novas'' e aquilo que seria ``genuinamente criativo''.  Idéias novas envolvem os P-Conceitos e comportamentos exploradores. O que não quer dizer que, sejam desconsiderados como criativos: \traducao{Processos exploradores são considerados criativos se eles forem guiados de alguma forma por uma ``heurística'' ou ``mapas''.\citeonline[p.~3]{thornton_quantitative_2007}}{Exploratory processes are only to be considered creative if they are guided in some way by ‘heuristics’ or ‘maps’.} Por outro lado, ``a criatividade genuína'' ocorre através de regras delineadoras da transformação conceitual: \traducao{Uma mera idéia nova é uma que pode ser descrita e/ou produzida pelo mesmo conjunto de regras generativas como outras idéias familiares. Uma genuína, original ou idéia criativa é uma que não pode \apud[p.~3;p.~40]{thornton_quantitative_2007}{boden_creative_1990}}{A merely novel idea is one which can be described and/or produced by the same set of generative rules as are other, familiar ideas. A genuinely original, or creative, idea is one which cannot.}.

Para lidar formalmente com o tema ``criatividade'' e seus Espaços Conceituais, Wiggins realiza uma categorização de comportamentos critaivos em humanos, e em sistemas computacionais (ver \autoref{tab:criatividade}):

\begin{table}[!h]
\caption{Definições formais de criatividade por \citeonline[p.~451]{wiggins_framework_2006}}
\small
    \begin{tabular}{ | p{4cm} | p{11.25cm} |}
    \hline 
    \hline 

    \tiny{Criatividade} 
    & \tiny{``A performance de tarefas que, quando executados por um humano, são consideradas criativas''  \tablefootnote{Tradução de \emph{The performance of tasks which, if performed by a human, would be deemed creative.}.}} \\
    \hline

    \tiny{Computação criativa} 
    & \tiny{``O estudo e suporte, através de meios e métodos computacionais, do comportamento exibido por sistemas naturais e artificiais, que são considerados criativos''. \tablefootnote{Tradução de \emph{The study and support, through computational means and methods, of behaviour exhibited by natural and artificial systems, which would be deemed creative if exhibited by humans.}.}} \\
    \hline

    \tiny{Sistemas criativos} 
    & \tiny{``Uma coleção de processos, naturais ou automáticos, que são capazes de alcançarem ou simularem comportamentos que em humanos seria considerado criativo''} \\
    \hline

    \tiny{Comportamento Criativo} 
    & \tiny{``Um ou mais dos comportamentos exibidos por um sistema criativo''\tablefootnote{Tradução de \emph{One or more of the behaviours exhibited by a creative system.}}} \\
    \hline
   
    \end{tabular}
\label{tab:criatividade}
\end{table}


\section{Ferramenta de Estruturação de Sistemas Criativos}\label{sec:csf}

\begin{citacao}
O universo, $\mathcal{U}$, é um espaço multidimensional, no qual dimensões são capazes de representar qualquer coisa, e todos os possíveis conceitos distintos correspondentes àqueles pontos em $\mathcal{U}$ (\ldots) Para tornar a proposta um espaço-tipo possível, permitirei que $\mathcal{U}$ contenha todos os conceitos abstratos, bem como os concretos, e que é possível representar os artefatos tanto completos e incompletos \cite[p.~451]{wiggins_framework_2006}.\footnote{Tradução de \emph{The universe, U, is a multidimensional space, whose dimensions are capable of representing anything, and all possible distinct concepts correspond with distinct points in U. (\ldots) To make the proposal as state-spacelike as possible, I allow that U contains all abstract concepts as well as all concrete ones, and that it is therefore possible to represent both complete and incomplete artefacts}}
\end{citacao}

Wiggins esclarece que Boden não reconhece de forma explícita $\mathcal{U}$, ``ela borra a distinção entre as regras que determinam a adesão do espaço (\ldots) e outras disposições que possam permitir a construção e/ou detecção de um conceito representado por um ponto no espaço'' (\emph{Idem, ibdem}).

Espaços conceituais $\mathcal{C}$, finitos ou infinitos são definidos como restrições de um universo $\mathcal{U}$, caracterizando um conjunto não-determinístico de conhecimentos: \traducao{A noção-chave na teoria de Boden é aquele do espaço conceitual. Enquanto nenhuma definição formal é provida, é comum interpretar esta frase literalmente, tomando o espaço conceitual sendo um espaço de conceitualizações, ou representações de conceitos \cite[p~.7]{thornton_quantitative_2007}.}{The key notion in Boden’s theory is that of the conceptual space. While no formal definition has been provided, it is common to interpret the phrase literally, taking the conceptual space to be a space of conceptualisations or concept representations.}

\citeonline{mclean_music_2006} ainda descreve regras que validam concepções diferentes entre espaços conceituais $\mathcal{C}$ diversos em um Universo de Conceitos $\mathcal{U}$ (ver \autoref{tab:universodeconceitos}).

\begin{table}[!h]
\caption{Definições formais do Universo de possibilidades de \citeonline{wiggins_framework_2006}, ou Universo de Conceitos por \citeonline{mclean_music_2006}.}
\small
    \begin{tabular}{ | p{4.25cm} | p{5.25cm} | p{5.25cm} |}
    \hline 
    \hline 

    Representação
    & \tiny{Nome}     
    & \tiny{Significado} \\
    \hline

    $c$
    & \tiny{Conceito} 
    & \tiny{Uma instância de um conceito, abstrato ou concreto \cite{wiggins_framework_2006}}. \\
    \hline

    $\mathcal{U}$
    & \tiny{Universo de Conceitos} 
    & \tiny{Superconjunto não restrito de conceitos. \cite{wiggins_framework_2006}. ``Um universo de todos conceitos possíveis'' \cite{mclean_music_2006} \tablefootnote{Tradução de \emph{A universe of all possible concepts}.}}\\
    \hline

    $\mathcal{L}$
    & \tiny{Linguagem} 
    & \tiny{Linguagem utilizada para expressar regras.} \\
    \hline

    $\mathcal{A}$
    & \tiny{Alfabeto} 
    & \tiny{Alfabeto da linguagen que contêm caracteres apropriadospara expressão das regras} \\
    \hline

    $\mathcal{R}$
    & \tiny{Regras de validação} 
    & \tiny{Validam os conceitos em um universo, se apropriados ou não para o espaço trabalhado.} \\
    \hline

    $[[.]]$
    & \tiny{Função de interpretação} 
    & \tiny{``Uma função parcial de $\mathcal{L}$ para funções que resultam em números reais entre [0, 1] (\ldots) 0.5 $[$ou maior$]$ significa uma verdade booleana e menos que 0.5 siginifica uma falsidade booleana; a necessidade disso para valores reais se tornará clara abaixo'' \cite[p.~452]{wiggins_framework_2006}\tablefootnote{Tradução de \emph{(\ldots) a partial function from $\mathcal{L}$ to functions yielding real numbers in [0, 1]. (\ldots) 0.5 to mean Boolean true and less than 0.5 to mean Boolean false; the need for the real values will become clear below}.}}\\
    \hline

     $[[\mathcal{R}]]$
    & \tiny{Regras de validação} 
    & \tiny{``Uma função que interpreta $\mathcal{R}$, resultando em uma função indicando aderência ao conceito em $\mathcal{R}$''\tablefootnote{Tradução de \emph{A function interpreting $\mathcal{R}$, resulting in a function indicating adherence of a concept to $\mathcal{R}$}}} \\
    \hline

     $\mathcal{C} = [[\mathcal{R}]](\mathcal{U}) $
    & \tiny{Espaço Conceitual} 
    & \tiny{``Todos espaços conceituais são um subconjunto não-estrito de $\mathcal{U}$''\tablefootnote{Tradução de \emph{All conceptual spaces are non-strict subset}.}. Um subconjunto contido em $\mathcal{U}$ \cite{wiggins_framework_2006}. Uma função que interpreta $\mathcal{R}$, resultando em uma função que indica aderência ao conceito em $\mathcal{R}$ \tablefootnote{Tradução de \emph{A function interpreting $\mathcal{R}$, resulting in a function indicating adherence of a concept to $\mathcal{R}$}.} } \\
    \hline

    $\mathcal{T}$
    & \tiny{Regras de detecção} 
    & \tiny{``Regras definidas dentro de $\mathcal{L}$ para definir estratégias transversais para localizar conceitos dentro de $\mathcal{U}$'' \cite{mclean_music_2006}\tablefootnote{Tradução de \emph{Rules defined within $\mathcal{L}$ to define a traversal strategy to locate concepts within $\mathcal{U}$ }}} \\
    \hline

    $\mathcal{E}$
    & \tiny{Regras de qualidade} 
    & \tiny{``(\ldots) conjunto de regras que permitem-nos avaliar qualquer conceito que nós encontramos em $\mathcal{C}$ e determinar sua qualidade, de acordo com critérios que nós considerarmos apropriados'' \cite[p.453]{wiggins_framework_2006}\tablefootnote{Tradução de \emph{(\ldots) set of rules which allows us to evaluate any concept we find in C and determine its quality, according to whatever criteria we may consider appropriate.}}``Regras definidas dentro de $\mathcal{L}$ para avaliar a qualidade ou a desejabilidade do conceito $c$'' \cite{mclean_music_2006}\tablefootnote{Tradução de \emph{Rules defined within $\mathcal{L}$ which evaluate the quality or desirability of a concept $c$.}}}\\
    \hline

    $<<<\mathcal{R}, \mathcal{T}, \mathcal{E}>>>$
    & \tiny{Função de interpretação} 
    & \tiny{Uma regra necessária para definir o espaço conceitual, ``independentemente da ordem, mas também, ficcionalmente, enumerá-los em uma ordem particular, sob o controle de $\mathcal{T}$ -- isto é cricial para a simulação de um comportamento criativo de um $\mathcal{T}$ particular \cite{wiggins_framework_2006} \tablefootnote{Tradução de \emph{We need a means not just of defining the conceptual space, irrespective of order, but also, at least notionally, of enumerating it, in a particular order, under the control of $\mathcal{T}$ -- this is crucial to the simulation of a particular creative behaviour by a particular $\mathcal{T}$.}}. ``Uma função que interpreta a estratégia transversal $\mathcal{T}$, informada por $\mathcal{R}$ e $\mathcal{E}$ . Opera sobre um subconjunto ordenado de $mathcal{U}$ (do qual tem acesso randômico) e resulta em outro subconjunto ordenado de $\mathcal{U}$.''\tablefootnote{Tradução de \emph{A function interpreting the traversal strategy $\mathcal{T}$, informed by $\mathcal{R}$ and $\mathcal{E}$ . It operates upon anordered subset of $mathcal{U}$ (of which it has random access) and results in another ordered subset of $\mathcal{U}$.}}} \\
    \hline
    \hline
   
    \end{tabular}
\label{tab:universodeconceitos}
\end{table}

\section{O modelo de improvisação}\label{sec:im}

McLean realiza uma comparação entre o \emph{Universo de possibilidades} de Wiggins com o \emph{Modelo de Improvisação} de Pressing. No entanto, McLean argumenta que:

\begin{citacao}
Pressing discute comportamento criativo no contexto do Modelo de Improvisação, e de fato é parte da Ferramenta de Estruturação de Sistemas Criativos. (\ldots) Durante a transferência de notação do Modelo de Improvisação para a Ferramenta de Sistemas Criativos, nós consideramos improvisação musical de uma maneira clara e temos uma linguage comum na qual comparar com outros modelos \footnote{Tradução de \emph{However Pressing does discuss creative behaviour in the context of the IM, and indeed the CSF is in part. (\ldots) In transferring the IM to the notation of the CSF we may consider music improvisation in a clearer manner and have a common language in which to compare it with other models.}}.
\end{citacao}

Segundo Pressing, o Modelo de Improvisação é ``um esboço para uma teoria geral da improvisação integrada com preceitos da Psicologia Cognitiva (\ldots) teoria do comportamento de improvisação na música'' \cite[p.~2]{pressing_improvisation_1987}. 

Este modelo será utilizado para especificar performances exemplares, como o caso investigado neste trabalho, \emph{Study in Keith}. Por exemplo, uma improvisação particionada em diferentes sequências pode ser parcialmente mapeada em categorias, como blocos sonoros, referentes conceituais e normas estilísticas, conjuntos de objetivos e processos.

Um sumário sobre o modelo de improvisação é apresentado na \autoref{tab:modelo_improvisacao}.

\begin{table}[!h]
\caption{Definições formais do Modelo de improvisação de Jeff \citeonline{pressing_improvisation_1987}, segundo \citeonline[p.~2]{mclean_music_2006}.}
\small
    \begin{tabular}{ | p{6cm} | p{9cm} |}
    \hline 
    \hline 

    \tiny{Representação}   
    & \tiny{Significado} \\
    \hline

    $E'$
    & \tiny{Um bloco de eventos sonoros}\tablefootnote{\emph{A cluster of sound events}.} \\
    \hline

    $K'$
    & \tiny{Uma seqüência de blocos de eventos E, onde um bloco de eventos não se sobrepõe com o seguinte}\tablefootnote{A sequence of E event clusters, where event cluster onsets do not overlap with those of a following one}\\
    \hline

    $I'$
    & \tiny{Uma improvisação, particionada por interrupções em um número de K sequências}\tablefootnote{An improvisation, partitioned by interrupts into a number of K sequences} \\
    \hline

    $R'$
    & \tiny{Um referente opcional, tal como uma partitura ou uma norma estilística}\tablefootnote{An optional referent, such as a score or stylistic norm} \\
    \hline

    $G'$
    & \tiny{Um conjunto de objetivos }\tablefootnote{A set of current goals.} \\
    \hline

    $M'$
    & \tiny{Uma memória de longo prazo}\tablefootnote{Long term memory.} \\
    \hline

    $O'$
    & \tiny{Um conjunto de objetos}\tablefootnote{An array of objects.} \\
    \hline

    $F'$
    & \tiny{Um conjunto de características dos objetos}\tablefootnote{An array of objects Features.} \\
    \hline

    $P'$
    & \tiny{Um conjunto de processos}\tablefootnote{An array of Process} \\
    \hline
    \hline
   
    \end{tabular}
\label{tab:modelo_improvisacao}
\end{table}

\begin{figure}[!h]
  \centering
  \includegraphics[scale=0.7]{imagens/contido.png}
  \caption{Representação da justaposição  entre dois epaços conceituais. A região em marrom representa um grupo de conceitos transitórios, bem como os limites desta transição. \textbf{Fonte}: autor. }
  \label{fig:contido}
\end{figure}

\section{Diagramação dos espaços conceituais}

\newcommand{\csfeq}[2]{
\mathcal{#1}_\emph{#2}
}

\newcommand{\unionspaces}[6]{
\csfeq{#1}{#2} = \csfeq{#3}{#4} \bigcup \csfeq{#5}{#6}
}

\newcommand{\listspaces}[9]{
\csfeq{#1}{#2}~=~[\csfeq{#3}{#2},~\csfeq{#4}{#2},~\csfeq{#5}{#2},~\csfeq{#6}{#2},~\csfeq{#7}{#2},~\csfeq{#8}{#2},~\csfeq{#9}{#2}
}

Formalmente, a figura acima pode ser representada como na \autoref{eq:def} , se desconsiderarmos qualquer outros espaços conceituais.

\begin{example}{Representação formal da \autoref{fig:contido}}
\begin{equation}
\unionspaces{C}{Study in Keith}{C}{live coding}{C}{Sun Bears}
\label{eq:def}
\end{equation}
\end{example}

Este grupo também pode ser descrito como uma lista de propriedades como na \autoref{eq:def2}:

\begin{example}{Representação formal das propriedades da \autoref{fig:contido}}
\begin{equation}
\listspaces{C}{SK}{E'}{K'}{I'}{R'}{G'}{M'}{O'}{F'},~\csfeq{P'}{SK}]
\label{eq:def}
\end{equation}
\end{example}
  
Nos diagramas abaixo, $C_\emph{\ldots}$ representa qualquer espaço conceitual abstrato (que pode incluir outro previamente apresentado). Entre os elementos iniciais (raízes, vermelho) e transitórios (nós, azul), ocorrem as ramificações (ramos, linhas pretas), isto é, a exploração de conceitos dentro de outros conceitos. De um lado, a aplicação de regras de validação sobre o universo conceitual da pesquisa (tudo aquilo que foi produzido em dois anos de mestrado) gerou o espaço conceitual desta tese. Estas regras de validação foram, em sua maior parte, os processos de orientação e qualificação. Em outras palavras, \csf{C}{pesquisa}$=[[$\csf{R}{pesquisa}$]]($\csf{U}{pesquisa}$)$.

\begin{example}{Representação do universo conceitual da \emph{pesquisa}}

O Universo de Conceitos da pesquisa, \csf{U}{pesquisa}, é um recorte do universo conceitual da música, \csf{U}{música}:

\begin{tikzpicture}
  [
    grow                    = right,
    sibling distance        = 6em,
    level distance          = 10em,
    edge from parent/.style = {draw, -latex},
    every node/.style       = {font=\footnotesize},
    sloped
  ]
  \node [root] {\csf{U}{Música}}
    child { node [env] {\csf{U}{pesquisa}}
      child { node [env] {\csf{U}{livecoding}}}
    }
    child { node [env] {\csf{C}{\ldots}}};
\end{tikzpicture}

No primeiro capítulo, incluímos um subjconjunto neste Espaço Conceitual da Pesquisa. Este subconjunto é constituído pelos termos representados na \autoref{fig:nuvemlivecoding} (p.~\pageref{fig:nuvemlivecoding}), e no \autoref{app:A} (p.~\pageref{app:A}). 

\begin{tikzpicture}
  [
    grow                    = right,
    sibling distance        = 6em,
    level distance          = 10em,
    edge from parent/.style = {draw, -latex},
    every node/.style       = {font=\footnotesize},
    sloped
  ]
  \node [root] {\csf{C}{pesquisa}}
    child { node [env] {\csf{C}{livecoding}}
      child { node [env] {\csf{C}{\ldots}}}
      child { node [env] {\csf{C}{ICLC}}}
    }
    child { node [env] {\csf{C}{\ldots}}}; 
\end{tikzpicture}

Podemos inclur elementos históricos, o período transitório entre 1970 e 2000 (\emph{circa}), onde emanciparam as práticas e as regras heurísticas.  

\begin{tikzpicture}
  [
    grow                    = right,
    sibling distance        = 6em,
    level distance          = 10em,
    edge from parent/.style = {draw, -latex},
    every node/.style       = {font=\footnotesize},
    sloped
  ]
  \node [root] {\csf{C}{livecoding}}
    child { node [env] {\csf{C}{Elementos Históricos}}
      child {node [env] {\csf{C}{Proto-História}}}
      child {node [env] {\csf{C}{Manifestos}}}
    }
    child { node [env] {\csf{C}{\ldots}}};
\end{tikzpicture}

Por último, \csf{C}{pesquisa} investiga o \emph{live coding} a partir de um caso específico:

\begin{tikzpicture}
  [
    grow                    = right,
    sibling distance        = 6em,
    level distance          = 10em,
    edge from parent/.style = {draw, -latex},
    every node/.style       = {font=\footnotesize},
    sloped
  ]
  \node [root] {\csf{C}{pesquisa}}
    child { node [env] {\csf{C}{livecoding}}
      child { node [env] {\csf{C}{\ldots}}}
      child { node [env] {\csf{C}{Sessão de Improvisação}}
        child { node [env] {\csf{C}{Study in Keith}}}
        child { node [env] {\csf{C}{\ldots}}}
      }
    }
    child { node [env] {\csf{C}{\ldots}}}; 
\end{tikzpicture}
\end{example}

Por outro lado \csf{C}{Study in Keith} pode ser definido pelo modelo de improvisação de Pressing (\autoref{tab:modelo_improvisacao}, \pageref{tab:modelo_improvisacao}).

\begin{example}{Representação do modelo de improvisação para \emph{Study in Keith}.}
\begin{tikzpicture}
  [
    grow                    = right,
    sibling distance        = 6em,
    level distance          = 10em,
    edge from parent/.style = {draw, -latex},
    every node/.style       = {font=\footnotesize},
    sloped
  ]
  \node [root] {\footnotesize \csf{C}{Study in Keith}}
    child { node [env] {\footnotesize \csf{E'}{Study in Keith}}}
    child { node [env] {\footnotesize \csf{K'}{Study in Keith}}}
    child { node [env] {\footnotesize \csf{I'}{Study in Keith}}}
    child { node [env] {\footnotesize \csf{R'}{Study in Keith}}}
    child { node [env] {\footnotesize \csf{G'}{Study in Keith}}}
    child { node [env] {\footnotesize \csf{O'}{Study in Keith}}}
    child { node [env] {\footnotesize \csf{F'}{Study in Keith}}}
    child { node [env] {\footnotesize \csf{P'}{Study in Keith}}}; 
\end{tikzpicture}
\end{example}

\section{Formalização}

O espaço conceitual do \emph{livecoding} é definido como uma função de interpretação das regras de validação (o que pode ser ou não considerado como próprio de uma categorização musical), de gosto (questões de estilo) e de localização transversal de conceitos (conceitos internos que permitem o cruzamento com outros conceitos):

\begin{example}{Delimitação de regras para o \emph{live coding} e para \emph{Study in Keith}.}
\begin{equation}
\mathcal{C}_\emph{livecoding} = <<<\mathcal{R}_\emph{livecoding}, \mathcal{T}_\emph{livecoding},  \mathcal{E}_\emph{livecoding} >>> 
\end{equation}

As regras de validação foram estudadas neste trabalho como as regras heurísticas do \emph{live coding}. Isto é, que conjunto de métodos são utilizados para caracterizar uma performance de \emph{live coding} como tal? Elementos históricos, e ideológicos (divulgados em manifestos), são levantados para responder esta pergunta .

\begin{equation}
\mathcal{R}_\emph{live coding} = \mathcal{R}_\emph{Proto-história} \bigcup  \mathcal{R}_\emph{Manifestos}
\end{equation}

Por outro lado, este estudo abandonou a investigação das regras de gosto, tema que pode ser melhor explorado em trabalhos posteriores, a partir de \citeonline{janotti_jr._a_2003,sa_musica_2006,sa_se_2009}.

A tarefa de localização transversal de conceitos é trabalhada no último capítulo. O espaço conceitual de \emph{Study in Keith} está contido no espaço conceitual do \emph{live coding} através da união entre os conceitos deste último, com os espaços conceituais dos concertos \emph{Sun Bears}, de Keith Jarret, \csf{C}{Sun Bears} $\subset$ \csf{C}{live coding}. 

\end{example}

Expomos na equação \ref{eq:espaco_lc} o espaço conceitual multidimensional do \emph{Study in Keith}. Isto é, a aplicação de regras de validação do \emph{live coding} e regras de validação dos Concertos \emph{Sun Bear}:

\begin{example}{Aplicação}\label{eq:espaco_lc}
O espaço conceitual não-estrito da pesquisa é um função de interpretação das regras de validação do \emph{live coding}, e das regras de validação do disco \emph{Sun Bears}, sobre o Universo de conceitos do \emph{livecoding}
\begin{equation}
\mathcal{C}_\emph{pesquisa} = [[\mathcal{R}_\emph{live coding} \bigcup \mathcal{R}_\emph{Sun Bears}]](\mathcal{U}_\emph{live coding} )
\end{equation}
\end{example}