\chapter{Metodologia de análise de uma Improvisação musical de códigos}\label{cap:metodologia}

Este capítulo contextualiza a \emph{criatividade} dentro do pensamento proposto por um improvisador programador \cite{McLean2011}. Apresentaremos uma definição de criatividade \ver{sec:criatividade} para contextualizar o \traducao{Quadro Conceitual de Sistemas Criativos}{Creative System Frameworks, \emph{ou CSF}.} \ver{sec:csf}. Em seguida especificamos este quadro para os propósitos específicos desta pesquisa \ver{sec:diagrama} e formalizamos o último capítulo \ver{sec:formaliza}.

\section{Criatividade}\label{sec:criatividade}

Para analisar uma improvisação de códigos, nos termos de um processo criativo com um instrumento musical (computador), definimos \emph{criatividadade} a partir de uma proposição do livro ``The Creative Mind: myths and mechanisms'' de Margaret \citeonline{boden_creative_1990},  \ver{sec:diferencas}. Em seguida discutimos a formação de \emph{imagens mentais}, durante um processo criativo idealizado \ver{sec:imagem_mental}, afim de discutir um mecanismo teórico para realização destas imagens mentais em performances de improvisadores-programadores \ver{sec:tidal}. 

\subsection{O paradoxo da criatividade}\label{sec:diferencas}

É possível discutir quais valores a palavra \emph{criatividade} carrega. Se por um lado faltam-nos documentos que comprovem qual é sua definição comum, sugerimos que uma definição cíclica (``a criatividade cria'') é uma maneira de pensar que, socialmente, a avaliação de um comportamento criativo está baseada em sua capacidade de \emph{gerar novidades}. Para \citeonline[p.~2]{thornton_quantitative_2007}, Boden discute que esta concepção de criatividade cria um paradoxo quando colocada sob a perspectiva da lógica mecanicista do computador:

\begin{citacao}
\traducao{O ponto inicial de Boden para o desenvolvimento de sua sua explicação, é a observação de que o conceito de criatividade contém um paradoxo. Por definição, criatividade cria, i.e., produz alguma coisa nova. Mas se estamos comprometidos com uma abordagem mecanicista do mundo -- nenhum milagre é permitido -- iremos acreditar que tudo o que ocorre é, em princípio, previsível. Iremos acreditar também que qualquer coisa nova deve ser construída de componentes existentes. Isso implica que nada pode ser intrinsicamente novo.}{Boden’s starting point for the development of her account is the observation that the concept of creativity contains a paradox. By definition, creativity creates, i.e., it produces something new. But if we are committed to a mechanistic account of the world — no miracles allowed — we believe that everything that occurs is predictable in principle. We also believe that any new thing must be constructed from existing components. This implies that nothing can ever be intrinsically new.}
\end{citacao}


 De outro ponto de vista, o mecanismo de julgamento do que é ou não é novidade parte de uma identificação do que é possível e do que não é, de forma que a impossibilidade é valorizada: \traducao{Para justificar a classificação de uma idéia criativa... alguém deve identificar os princípios generativos com respeito ao que é impossível}{To justify calling an idea creative... one must identify the generative principles with respect to which it is impossible.} \apud[p.~40;p.~3]{boden_creative_1990}{thornton_quantitative_2007}. Essa capacidade gerativa do aparentemente impossível pode estar conectada com uma inversão desta idéia comum de criatividade. Para Boden, \traducao{Qualquer ato criativo é fundado na conceitualização ou realização de um ponto dentro de um espaço conceitual particular (\emph{idem}, \emph{ibidem})}{Any creative act is thus founded on conceptualisation or the realisation of a point within a particular ‘conceptual space’.}. 

Para Thornton e \citeonline[p.~450--451]{wiggins_framework_2006}, na primeira edição de seu livro, Boden não explica como é o método de acesso aos espaços conceituais. Por outro lado, Boden oferece uma taxonomia de tipos de criativadade. 

As nomenclaturas derivadas são explicadas como uma divisão de duas classificações da criatividade \ver{fig:ortogonal}. A primeira classificação divide a criatividade em criatividade-psicológica, ou criatividade-pessoal (\emph{P-creativity}), e criatividade-histórica (\emph{H-creativity}). A segunda classificação separa criatividade em criatividade-exploradora e criatividade-transformacional.

\begin{figure}
\centering
\begin{tikzpicture}[scale=2.5]
\tikzstyle{every node}=[draw,shape=circle];
\path (0:0cm) node (v0) {\tiny $Criatividade$};
\path (0:1.25cm) node (v1) {\tiny $Criatividade-H$};
\path (2*90:1.25cm) node (v2) {\tiny $Criatividade-P$};
\path (3*90:1.25cm) node (v3) {\tiny $Criatividade Exp.$};
\path (90:1.25cm) node (v4) {\tiny $Criatividade Trans.$};
\draw (v0) -- (v1)
(v0) -- (v2)
(v0) -- (v3)
(v0) -- (v4);
\end{tikzpicture}
\caption{Classificação da criatividade : 1) criatividade-psicológica/criatividade-histórica; 2) critividade exploradora/criatividade transformacional. \textbf{Fonte}: autor com base em \citeonline{wiggins_framework_2006}.}
\label{fig:ortogonal}
\end{figure}

\begin{citacao}
\traducao{A distinção é entre o sentido de criar um conceito que nunca foi criado antes $[$criatividade-P$]$, e um conceito que nunca foi criado antes por um criador específico $[$criatividade-H$]$. Esta distinção será tangencialmente relevante para meu argumento aqui, mas antes de prosseguir, eu noto que esta não é uma simples escolha binária, mas ao invés, uma contextualização multidimensional: pode ser possível, por exemplo, para um comportamento criativo ser P-criativo em uma sociedade, mas H-criativo em outra; deste ponto de vista da segunda sociedade, apenas importam comportamentos H-criativos. (\ldots) no trabalho de Boden, existe uma distinção entre criatividade exploradora e transformacional, que será relevante aqui, e então merece alguma explicação. Boden concebe o processo de criatividade como uma identificação e/ou localização de novos objetos conceituais em um espaço conceitual.}{The distinction is between the sense of creating a concept which has never been created before at all, and a concept which has never been created before by a specific creator. This distinction will be only tangentially relevant to my argument here, but before proceeding, I note that this is not a simple binary choice, but rather multi-dimensional, context-based one: it would be possible, for example, for a creative behaviour to be only P-creative in one society, but H-creative in another; from the point of view of the second society, only the H-creativity matters. (\ldots) in Boden’s work, there is the distinction between exploratory and transformational creativity, which is directly relevant here, and so needs a little more explanation. \textbf{Boden conceives the process of creativity as the identification and/or location of new conceptual objects in a conceptual space}.}  
\end{citacao}

Para Wiggins, Boden chama de \emph{comportamento criativo explorador} a exploração de possibilidades completas ou parciais de um conceito. Já o \emph{comportamento criativo transformacional} pressupõe regras governadoras deste espaço conceitual em exploração, e busca transformar tais regras através de métodos. Para Boden,  são socialmente valorizados os conceitos-H e os processos transformacionais, enquanto o comportamento explorador e a criatividade-P são consideradas como característicos de crianças mais novas, ou pensamentos personalistas. Wiggins dá maior importância para a classificação exploradora/transformacional, e não desvaloriza o comportamento explorador em relação à transformação..  \citeonline[p.~3--4]{thornton_quantitative_2007} pontua que Boden admite que em algumas situações, uma criativade-exploradora não é menos criativa que uma criativade-transformadora. O comportamento explorador, como uma \emph{exploração guiada}, é muito útil em atividades onde se requer \traducao{(\ldots) a utilização de heurísticas e mapas para identificar conceitos valiosos dentre de um espaço conceitual existente}{the use of heuristics and maps to identify valuable concepts within an existing conceptual space} \ver{sec:showusyourscreens}. Neste sentido, os comportamentos criativos \emph{apenas transformadores} desenvolvem novos espaços conceituais que serão úteis para o comportamento explorador guiado. 

\begin{citacao}
\traducao{De fato, na primeira edição $[$\citeonline{boden_creative_1990}$]$, ela não oferece uma explicação do número de diferentes tipos de criatividade que ela identificou. Parece que sua intenção era distinguir os dois tipos notados, espaços conceituais devem ter uma característica generativa. E isso certamente é a interpretação comum. Ainda no sumário 'em uma casca de noz' de sua teoria, foi adicionado um prólogo à segunda edição (Boden, 2003), e em (Boden, 1998), ela coloca que sua explicação distingue três principais formas de criatividade, sendo exploração, transformação e \emph{combinação}. É somada à incerteza a observação que somente a definição forte $[$generalizadora$]$ da definição possue o poder de resolver o paradoxo da criatividade}{In fact, in the first edition, she offers no final count of the number of different types of creativity she has identified. It seems to be her intention to distinguish the two types noted, conceptual space must be generative in character. and this is certainly a common interpretation. Yet in the ‘nutshell’ summary of her theory, added as a prologue to the second edition (Boden, 2003), and in (Boden, 1998), she states that her account distinguishes three main forms of creativity, these being exploration, transformation and combination. Adding to the uncertainty is the observation that only the strong definition has the power to resolve the creativity paradox, arguably forcing us to recognise not two forms of creativity, or three, but one: transformation.}
\end{citacao}

Neste ponto \citeonline[p.~451]{wiggins_framework_2006} apresenta uma definição cíclica e generalizadora de criatividade, \traducao{A performance de tarefas que, quando executados por um humano, são consideradas criativas}{The performance of tasks which, if performed by a human, would be deemed creative}, e subdivide em quatro definições auxiliares: 

\begin{table}[!h]
\caption{Definições formais de criatividade por \citeonline[p.~451]{wiggins_framework_2006}}
\small
    \begin{tabular}{ | p{4cm} | p{11.25cm} |}
    \hline 
    \hline 

    \tiny{Criatividade} 
    & \tiny{``O estudo e suporte, através de meios e métodos computacionais, do comportamento exibido por sistemas naturais e artificiais, que podem ser considerados criativos se exibidos em humanos.''  \tablefootnote{Tradução de \emph{‘The study and support, through computational means and methods, of behaviour exhibited by natural and artificial systems, which would be deemed creative if exhibited by humans’’.}.}} \\
    \hline

    \tiny{Computação criativa} 
    & \tiny{``O estudo e suporte, através de meios e métodos computacionais, do comportamento exibido por sistemas naturais e artificiais, que são considerados criativos''. \tablefootnote{Tradução de \emph{The study and support, through computational means and methods, of behaviour exhibited by natural and artificial systems, which would be deemed creative if exhibited by humans.}.}} \\
    \hline

    \tiny{Sistemas criativos} 
    & \tiny{``Uma coleção de processos, naturais ou automáticos, que são capazes de alcançarem ou simularem comportamentos que em humanos seriam considerados criativos''} \\
    \hline

    \tiny{Comportamento Criativo} 
    & \tiny{``Um ou mais dos comportamentos exibidos por um sistema criativo''\tablefootnote{Tradução de \emph{One or more of the behaviours exhibited by a creative system.}}} \\
    \hline
    \hline
   
    \end{tabular}
\label{tab:criatividade}
\end{table}

\subsection{Criatividade, códigos e imagens mentais}\label{sec:imagem_mental}

Para \citeonline[p.~24--25]{McLean2011}, um comportamento criativo pode ser descrito como o processo de elaboração de uma \emph{imagem mental}. Mais especificamente, a imagem mental é um símbolo, e seu significado é governado por regras gramaticais e sociais (códigos). Por outro lado, McLean estabelece o conceito de \emph{imagem mental} como uma hierarquia de códigos simbólicos, visuais e gramaticais, como descrita pela teoria da Codificação Dual \apud[p.~25--29]{paivio_dual_1990}{McLean2011}:


\begin{citacao}
\traducao{Seu $[$Paivio$]$ argumento não é que existem dois códigos, mas sim que existe uma hierarquia de códigos, que se ramificam no topo em códigos lingüísticos discretos e códigos de percepção contínua, que Paivio nomeia como \emph {logogens} e \emph{imagens} respectivamente. (\ldots) \textbf{A explicação oferecida pela teoria da Codificação Dual é que existem sistemas de símbolos distindos, mas integrados, para linguagens e figuras}.}{His contention is not that there are two codes, but rather that there is a hierarchy of code, which branch at the top into discrete linguistic codes and contionuous perceptual codes, which Paivio names \emph{logogens} and \emph{imagens} respectively (\ldots) The explanation offered by Dual Coding Theory is that there are distinct, yet integrated symbol systems for imagery and language.}
\end{citacao}

Por exemplo, quando falamos em \emph{improvisação de códigos}, podemos imaginar uma pessoa escrevendo um texto em um computador. Dependendo do grau de intimidade com os símbolos aprendidos, pode ser que esta imagem seja mais ou menos específica, e , outras possibilidades de imagens mentais surgem. A própria atividade de descrição de uma imagem mental é uma \emph{estratégia} de conversão. No caso do improvisador-programador, que converte sua imagem mental em em código de computador, chamaremos de \emph{estratégia transversal}.

Neste processo, o improvisador-programador recorre aos padrões e estilos de escrita de uma linguagem. Entre padrões e estilos, são utilizados comentários, que explicam textualmente a imagem e seu resultado; da disposição espacial do código, ou  nomes de variáveis e funções que sugerem imagens mentais específicas. No caso de uma linguagem de programação de propósito geral, algumas regras definidas pela comunidade desenvolvedora da linguagem devem ser seguidas. No entanto, já vimos que o improvisador-programador é estimulado a recorrer às linguagens artificiais, ou linguagens de domínio específico (DSL) -- ``O programa será transcendido - Língua Artificial é o caminho'' \ver{sec:showusyourscreens}. Isto é, é parte da prática do improvisador-programador criar linguagens específicas para uma classe de imagens mentais específicas. 

Um exemplo de exploração de um processo transformacional será ilustrado a partir de uma descrição de \citeonline[p.~119]{McLean2011},  com o método da \emph{prática reflexiva} de um artista-plástico (no caso, o pintor Paul Klee). Supondo a discretização do processo transformacional, o artista cria uma imagem mental do que irá fazer, um esboço que delimita um campo de atividade (espaço conceitual), a atividade de pintura, e o resultado percebido.  Na fase de implementação do esboço, o pintor percebe que aquilo que foi considerado adequado no esboço, é inadequado para a situação prática. Neste momento, o artista reage ao resultado e re-elabora novas estratégias de conversão entre a imagem mental e o resultado. O processo continua até que o ofício seja considerado completo (\emph{obra}). Este não é o mesmo processo utilizado pelo(a) improvisador(a)-programador(a). O material, e o objetivo são diversos do(a) artista plástico(a). Seu material é o texto, mas não o texto discursivo, e sim o texto que descreve uma rotina de tarefas computacionais. Seu objetivo não é chegar a uma obra finalizada, mas sim não descontinuar o processo. McLean chama esse método de transformacional de programação por bricolagem \ver{fig:processo_criativo}. Em ambos os casos, na prática reflexiva e na bricolagem, são necessários conceitos que suportem a intenção (e vontade) do ofício. Conceitos elaborados através de restrições de idéias podem ser sistematizados, com o auxílio de esquemas visuais. Esta abordagem é muito semelhante à posição defendida por McLean, onde conceitos podem ser representados geometricamente. No caso do(a) improvisador(a)-programador(a), \emph{espaços conceituais} são reelaborados indefinidamente, começando por uma \emph{estratégia transversal}, passando para a observação, reação, e reformulação da imagem mental.

 \begin{figure}[h]
  \centering
  \includegraphics[scale=0.5]{imagens/processo_criativo.png}
  \caption{Modelo de bricolagem para o processo criativo realizado por um artista-programador. \textbf{Fonte}: \citeonline[p.~122]{McLean2011}. }
  \label{fig:processo_criativo}
\end{figure}

\begin{citacao}
\traducao{A Figura 6.2 $[$\autoref{fig:processo_criativo}$]$ caracteriza a programação por bricolagem como um laço retroalimentado envolvendo o algoritmo escrito, sua interpretação, e a percepção do programador e sua reação do resultado ou comportamento $[$do algoritmo$]$. (\ldots). No começo o programador tem um conceito meio-formado que só atinge consistência interna através do processo de ser expresso como um algoritmo. O laço interno é onde o programador elabora o objetivo de suas imaginações, e o laço externo é onde essa trajetória está fundamentada na pragmática do que elas realmente têm que fazer. Através deste processo ambos algoritmos e conceitos são desenvolvidos até que o programador sinta que um se aplica com o outro, ou de outra forma julga o processo criativo finalizado.
}{
Figure 6.2 characterises bricolage programming as a creative feedback loop encompassing the written algorithm, its interpretation, and the programmer’s perception and reaction to its output or behaviour. (\ldots). At the beginning, the programmer may have a half-formed concept, which only reaches internal consistency through the process of being expressed as an algorithm. The inner loop is where the programmer elaborates upon their imagination of what might be, and the outer where this trajectory is grounded in the pragmatics of what they have actually made. Through this process both algorithm and concept are developed, until the programmer feels they accord with one another, or otherwise judges the creative process to be finished.
}  
\end{citacao}

 Para McLean, DSLs ``provêm termos padronizados para descrever demandas particulares em um domínio de uma tarefa''. Isto é, uma linguagem customizada para atividades artísticas pode ser planejada, ou analizada, através de um \emph{Quadro de estruturação das Dimensões Cognitivas da Notação} \apud[p.~95--97]{church_cognitive_2008}{McLean2011}. Estas não são dimensões autônomas, o que diverge do foco de nosso trabalho. Uma situação comum na improvisação de códigos, é maleabilidade de escrita de um algoritmo. No caso, McLean define como a interdependência entre viscosidade e notação secundária: a possibilidade de diferentes soluções para o mesmo resultado. Por exemplo, em um \emph{patch} de PD, cuja imagem mental é uma onda quadrada. Podemos utilizar objetos nativos ou objetos extendidos \ver{fig:pd}. Uma outra interdependência de dimensões, entre as Operações Mentais Difíceis e a não-Invisibilidade de Dependências Escondidas, é notória do ponto de vista pedagógico. Podem afastar o compositor de seu objetivo musical, mais ou menos estruturado em uma teoria. DSLs como PureData, Max/MSP, CSound, SuperCollider, Tidal, praticam a invisibilidade de dependências em diferentes graus. Porém, entre os improvisadores de código, é considerado virtuosismo o praticante recorrer às Operações Mentais difíceis com o mínimo possível de Invisibilidade de Dependências. Por exemplo, utilizar linguagens de baixo nível como C \ver{sec:concerto}, ou de alto nível como Perl \cite{mclean_hacking_2004} e ainda assim, elaborar sons e imagens de maneira criativa. \ver{tab:dimensoes}:

\begin{table}[!h]
\caption{Dimensões cognitivas da Notação para linguagens de programação. \textbf{Fonte}: \apud{church_cognitive_2008}{McLean2011}.}
\small
    \begin{tabular}{ | p{7cm}| p{7cm} |}
    \hline 
    \hline 

    \tiny \textbf{Dimensão} & \textbf{Significado} \\
    \hline 
    \hline 

    \tiny \textbf{Abstração}  
    & \tiny \tabletraducao{Disponibilidade de mecanismos de abstração}{Avaliability of abstraction mechanisms} \\
    \hline

    \tiny \textbf{Dependências escondidas}

    & \tiny \tabletraducao{Invisibilidade de ligações importantes entre entidades.}{Invisibility of important links between entities.}\\
    \hline
    
    \tiny \textbf{Compromisso prematuro}  
    & \tiny \tabletraducao{Restrição na ordem de execução das coisas.}{Constraints on the order of doing things.} \\
  \hline

    \tiny \textbf{Notação secundária}  
    & \tiny \tabletraducao{Notação diversa da sintaxe formal.}{Notation other than formal syntax.} \\
    \hline

    \tiny \textbf{Viscosidade}  
    & \tiny \tabletraducao{Resistência à mudança.}{Resistance to change.} \\
    \hline

    \tiny \textbf{Proximidade de mapeamento}  
    & \tiny \tabletraducao{Proximidade de representação para o domínio-alvo.}{Closeness of representation to target domain.} \\
    \hline

    \tiny \textbf{Consistência}  
    & \tiny \tabletraducao{Semânticas similares são expressadas em formas sintáticas similares.}{Similar semantics are expressed in similar syntatic forms} \\
    \hline

    \tiny \textbf{Dispersividade}  
    & \tiny \tabletraducao{Prolixidade da linguagem.}{Verbosity of language.} \\
    \hline

    \tiny \textbf{Tendência ao erro}  
    & \tiny \tabletraducao{Probabilidade de erros.}{Likelihood of mistakes.} \\
    \hline

    \tiny \textbf{Operações mentais difíceis}  
    & \tiny \tabletraducao{Demanda de recursos cognitivos.}{Demand on cognitive resources.} \\
    \hline

    \tiny \textbf{Provisoriedade}  
    & \tiny \tabletraducao{Grau de compromisso com ações e marcos.}{Degree of commitment to actions or marks.} \\
    \hline
    
    \tiny \textbf{Função de expressividade}  
    & \tiny \tabletraducao{medida em que o efeito de um componente pode ser inferida.}{Extent to which the purpose of a component may be inferred.} \\
    \hline
    \hline
   
    \end{tabular}
\label{tab:dimensoes}
\end{table} 

\begin{figure}[!h]
  \centering
  \includegraphics[scale=0.7]{imagens/pd.png}
  \caption{Exemplo de uma caracteristica de viscosidade e notação secundária no PureData. \textbf{Fonte}: autor. }
  \label{fig:pd}
\end{figure}

 
\subsection{Comportamento Criativo: Bricolagem como estratégia transversal}\label{sec:tidal}

Vamos ilustrar um pequeno processo da \emph{estratégia transversal} com o \emph{software} Tidal.
Segundo \citeonline[p.~2]{mclean_tidal_2010}, \emph{Tidal} é uma linguagem de composição generativa, onde \traducao{padrões podem ser compostos de numerosos subpadrões em uma variedade de maneiras e para uma profundidade arbitrária, para produzir $[$partes$]$ inteiras complexas de partes simples}{patterns may be composed of numerous subpatterns in a variety of ways and to arbitrary depth, to produce complex wholes from simple parts}. Amostras sonoras representam imagens mentais de suas fontes (por exemplo ``sn'' para \emph{snare}, caixa-clara), com ritmos organizados com o auxílio de símbolos delimitadores de tempo (como espaço, `` ``, e colchetes, ``$[$'',``$]$'', \verb|{| e \verb|}|). Ritmos podem ser revertidos (\verb|rev|), diminuidos e aumentados (\verb|slow|, \verb|density|), iterados (\verb|every|) para recombinação permutação, padrões mais complexos (\verb|can|), como o algoritmode bjorklund que simula ritmos tradicionais\footnote{\cfcite{toussaint_euclidean_2005}}. Efeitos de panoramização, atraso (\emph{delay}), filtros e comunicação de rede. No Exemplo \ref{ex:tidal}. a imagem mental é a demanda da linguagem, que é produzir Música Eletrônica para Dançar.

\begin{citacao}
\traducao{Tidal é uma linguagem de padrões embebida em uma linguagem de programação Haskell, consistindo de representação de padrão, uma biblioteca de padrões geradores e combinadores, um $[$mecanismo$]$ de agendamento de eventos e uma interface para programar ao vivo. Esta é uma extensiva re-escrita de um trabalho anterior introduzido sobre o título \emph{Petrol} $[$\citeonline{mclean_petrol_2010}$]$. Extensões incluem melhoramentos de representação de padrão e um uma integração totalmente configurável do protocolo Open Sound Control $[$\citeonline{osc}$]$ \cite{mclean_tidal_2010}
}
{Tidal is a pattern language embedded in the Haskell programminglanguage, consisting of pattern representation, a library of pattern generators and combinators, an event scheduler and programmer’s live coding interface. This is an extensive re-write of earlier work introduced under the working title of Petrol [15]. Extensions include improved pattern representation and fully configurable integration with the Open Sound Control (OSC) protocol [16]
}
\end{citacao}

\begin{example}{Exemplo de Estratégia Transversal}\label{ex:tidal}

Imagem mental: um \emph{loop} sincopado, mas bastante regular, descrito em um compasso. Em uma ``partitura-mental'', estruturamos o primeiro tempo com um baixo, que volta a tocar na segunda semicolcheia do terceiro tempo. No Segundo tempo, silêncio. No quarto tempo uma caixa aberta:

{%
\parindent 0pt
\noindent
\ifx\preLilyPondExample \undefined
\else
  \expandafter\preLilyPondExample
\fi
\def\lilypondbook{}%
\includegraphics{53/lily-86c766d2-1}%
% eof

\ifx\postLilyPondExample \undefined
\else
  \expandafter\postLilyPondExample
\fi
}

O padrão acima pode ser elaborado em uma voz (\verb|d1|), que redireciona (\$) a função que toca amostras sonoras (\verb|sound|). Esta função lê uma corrente de caracteres (\textbf{string}) separados por um espaço em branco. Espaços em branco são delimitadores temporais. Cada subdivisão temporal é representada por delimitadores como $[$ e $]$. 

\begin{minted}{haskell}
-- Eletronic Dance Music, BPM = 120 
-- tempo 1 - baixo            (bass)
-- tempo 2 - silencio         (silence)
-- tempo 3 - silencio + baixo
-- tempo 4 - caixa            (sn e sn:4)
d1 \$ (sound "bass3 silence [silence bass3] sn:4")
\end{minted}

Sonoramente, é útil para começar. Mas uma Música Eletrônica para Dançar requer mais elementos. Seguiremos com mais dois passos. Podemos complementar os ritmos com uma caixa e um baixos mais secos no segundo e terceiro tempo.

\input{./tidal2}

\begin{minted}{haskell}
-- Eletronic Dance Music, BPM = 120 
-- tempo 1 - baixo            (bass)
-- tempo 2 - silencio         (silence)
-- tempo 3 - silencio + house
-- tempo 4 - caixa            (sn e sn:4)
d1 \$ (sound "bass3 sn [silence house] sn:4")
\end{minted}

É possível também fazer com que este padrão reduza seu tempo pela metade a cada quatro tempos, :

{%
\parindent 0pt
\noindent
\ifx\preLilyPondExample \undefined
\else
  \expandafter\preLilyPondExample
\fi
\def\lilypondbook{}%
\input{2b/lily-de0a8ae3-systems.tex}
\ifx\postLilyPondExample \undefined
\else
  \expandafter\postLilyPondExample
\fi
}

\begin{minted}{haskell}
-- Eletronic Dance Music, BPM = 120
-- com uma caixa seca no segundo tempo
-- e uma caixa aberta no quarto tempo
-- A cada 4 tempos, o ritmo diminui pela metade 
-- e depois volta ao normal.
d1 \$ every 4 (density 0.5) (sound "bass3 sn [silence house] sn:4")
\end{minted}
\end{example}

Para \citeonline[p.~130]{McLean2011}, esta estratégia criativa, de programar ``no momento'', a partir de um arquivo de texto em branco, com uma imagem mental do resultado sonoro (ou visual), é caracterizada pela  bricolagem. No início do exemplo acima, o programador elabora um meio-conceito do que quer fazer, cuja expressão apenas ganha existência através da codificação \ver{fig:processo_criativo}. As fases de observação, e reação levam o improvisador programador à reconceitualização, e um novo código é escrito. No entanto, ao invés de finalizar, o improvisador segue desenvolvendo.

\section{Quadro Conceitual de sistemas criativos}\label{sec:csf}

Uma maneira adequada de descrever um sistema criativo (ou parte dele) considera um \emph{Universo de Conceitos}:

\begin{citacao}
O universo, $\mathcal{U}$, é um espaço multidimensional, no qual dimensões são capazes de representar qualquer coisa, e todos os possíveis conceitos distintos correspondentes àqueles pontos em $\mathcal{U}$ (\ldots) Para tornar a proposta um espaço-tipo possível, permitirei que $\mathcal{U}$ contenha todos os conceitos abstratos, bem como os concretos, e que é possível representar os artefatos tanto completos e incompletos \cite[p.~451]{wiggins_framework_2006}.\footnote{Tradução de \emph{The universe, U, is a multidimensional space, whose dimensions are capable of representing anything, and all possible distinct concepts correspond with distinct points in U. (\ldots) To make the proposal as state-spacelike as possible, I allow that U contains all abstract concepts as well as all concrete ones, and that it is therefore possible to represent both complete and incomplete artefacts}}
\end{citacao}

Wiggins esclarece que Boden não reconhece de forma explícita $\mathcal{U}$, ``ela borra a distinção entre as regras que determinam a adesão do espaço (\ldots) e outras disposições que possam permitir a construção e/ou detecção de um conceito representado por um ponto no espaço'' (\emph{Idem, ibdem}). Espaços conceituais $\mathcal{C}$, finitos ou infinitos são definidos como restrições de um universo $\mathcal{U}$, caracterizando um conjunto não-determinístico de conhecimentos, representações, e conceitos:

\begin{citacao}
\traducao{A noção-chave na teoria de Boden é aquele do espaço conceitual. Enquanto nenhuma definição formal é provida, é comum interpretar esta frase literalmente, tomando o espaço conceitual sendo um espaço de conceitualizações, ou representações de conceitos \cite[p~.7]{thornton_quantitative_2007}.}{The key notion in Boden’s theory is that of the conceptual space. While no formal definition has been provided, it is common to interpret the phrase literally, taking the conceptual space to be a space of conceptualisations or concept representations.}
\end{citacao}

\citeonline[p.~452]{wiggins_framework_2006} considera  \traducao{(\ldots) um universo, $\mathcal{U}$, um espaço multidimensional, cujas dimensões são capazes de representar qualquer coisa, e todos possíveis conceitos distintos correspondentes com distintos pontos em $\mathcal{U}$.}{The universe, U, is a multidimensional space,whose dimensions are capable of representing anything, and all possible distinct concepts correspond with distinct points in U}. Uma incomensurabilidade é evitada através de uma restrição por meio de quatro axiomas. O primeiro axioma (Universalidade) estabelece que o universo $\mathcal{U}$ pode conter tantos conceitos bem definidos (completos), parcialmente definidos (incompletos), e  o mais incompleto dos conceitos. Os dois primeiros são representados pela letra $c$, enquanto o último é representado por \small{T}. O segundo axioma (Não-identidade dos conceitos), estabelece que dois conceitos em  $\mathcal{U}$ são mutuamente diferentes entre si ($c_1 \neq c_2$), e não são um Universo. Esta correção restringe a recursividade de conceitos, o que daria ênfase ao comportamento explorador e anularia a importância do comportamento transformacional. O terceiro axioma (Inclusão Universal 1) define que \emph{espaços conceituais} $\mathcal{C}$, que contêm instâncias de conceitos $c$, são subconjuntos não-estritos do conunto $\mathcal{U}$. O quarto axioma (Inclusão Universal 2) estabelece que espaços conceituais $\mathcal{C}$ também contem o conceito \small{T}.






 Wiggins define o Universo de Conceitos (\csf{U}{x}) como um conjunto não estrito dos \emph{Espaços Conceituais} (\csf{C}{x}) de Margaret \citeonline{boden_creative_1990}. Isto é, um Universo de Conceitos a respeito de alguma coisa, no nosso caso da improvisação de códigos \ver{eq:ul}. 

\begin{equation}
\mathcal{U}_\emph{livecoding} = [\mathcal{C}_\emph{Tecelagem}, \mathcal{C}_\emph{Audiovisual}, \mathcal{C}_\emph{Dança} \mathcal{C}_\emph{Música}, \ldots, ?]
\end{equation}\label{eq:ul}

\begin{table}[!h]
\caption{Definições formais do Universo de possibilidades de \citeonline{wiggins_framework_2006}, ou Universo de Conceitos por \citeonline{mclean_music_2006}.}
\small
    \begin{tabular}{ | p{4.25cm} | p{5.25cm} | p{5.25cm} |}
    \hline 
    \hline 

    Representação
    & \tiny{Nome}     
    & \tiny{Significado} \\
    \hline

    $c$
    & \tiny{Conceito} 
    & \tiny{Uma instância de um conceito, abstrato ou concreto \cite{wiggins_framework_2006}}. \\
    \hline

    $\mathcal{U}$
    & \tiny{Universo de Conceitos} 
    & \tiny{Superconjunto não restrito de conceitos. \cite{wiggins_framework_2006}. ``Um universo de todos conceitos possíveis'' \cite{mclean_music_2006} \tablefootnote{Tradução de \emph{A universe of all possible concepts}.}}\\
    \hline

    $\mathcal{L}$
    & \tiny{Linguagem} 
    & \tiny{Linguagem utilizada para expressar regras.} \\
    \hline

    $\mathcal{A}$
    & \tiny{Alfabeto} 
    & \tiny{Alfabeto da linguagen que contêm caracteres apropriadospara expressão das regras} \\
    \hline

    $\mathcal{R}$
    & \tiny{Regras de validação} 
    & \tiny{Validam os conceitos em um universo, se apropriados ou não para o espaço trabalhado.} \\
    \hline

    $[[.]]$
    & \tiny{Função de interpretação} 
    & \tiny{``Uma função parcial de $\mathcal{L}$ para funções que resultam em números reais entre [0, 1] (\ldots) 0.5 $[$ou maior$]$ significa uma verdade booleana e menos que 0.5 siginifica uma falsidade booleana; a necessidade disso para valores reais se tornará clara abaixo'' \cite[p.~452]{wiggins_framework_2006}\tablefootnote{Tradução de \emph{(\ldots) a partial function from $\mathcal{L}$ to functions yielding real numbers in [0, 1]. (\ldots) 0.5 to mean Boolean true and less than 0.5 to mean Boolean false; the need for the real values will become clear below}.}}\\
    \hline

     $[[\mathcal{R}]]$
    & \tiny{Regras de validação} 
    & \tiny{``Uma função que interpreta $\mathcal{R}$, resultando em uma função indicando aderência ao conceito em $\mathcal{R}$''\tablefootnote{Tradução de \emph{A function interpreting $\mathcal{R}$, resulting in a function indicating adherence of a concept to $\mathcal{R}$}}} \\
    \hline

     $\mathcal{C} = [[\mathcal{R}]](\mathcal{U}) $
    & \tiny{Espaço Conceitual} 
    & \tiny{``Todos espaços conceituais são um subconjunto não-estrito de $\mathcal{U}$''\tablefootnote{Tradução de \emph{All conceptual spaces are non-strict subset}.}. Um subconjunto contido em $\mathcal{U}$ \cite{wiggins_framework_2006}. Uma função que interpreta $\mathcal{R}$, resultando em uma função que indica aderência ao conceito em $\mathcal{R}$ \tablefootnote{Tradução de \emph{A function interpreting $\mathcal{R}$, resulting in a function indicating adherence of a concept to $\mathcal{R}$}.} } \\
    \hline

    $\mathcal{T}$
    & \tiny{Regras de detecção} 
    & \tiny{``Regras definidas dentro de $\mathcal{L}$ para definir estratégias transversais para localizar conceitos dentro de $\mathcal{U}$'' \cite{mclean_music_2006}\tablefootnote{Tradução de \emph{Rules defined within $\mathcal{L}$ to define a traversal strategy to locate concepts within $\mathcal{U}$ }}} \\
    \hline

    $\mathcal{E}$
    & \tiny{Regras de qualidade} 
    & \tiny{``(\ldots) conjunto de regras que permitem-nos avaliar qualquer conceito que nós encontramos em $\mathcal{C}$ e determinar sua qualidade, de acordo com critérios que nós considerarmos apropriados'' \cite[p.453]{wiggins_framework_2006}\tablefootnote{Tradução de \emph{(\ldots) set of rules which allows us to evaluate any concept we find in C and determine its quality, according to whatever criteria we may consider appropriate.}}``Regras definidas dentro de $\mathcal{L}$ para avaliar a qualidade ou a desejabilidade do conceito $c$'' \cite{mclean_music_2006}\tablefootnote{Tradução de \emph{Rules defined within $\mathcal{L}$ which evaluate the quality or desirability of a concept $c$.}}}\\
    \hline

    $<<<\mathcal{R}, \mathcal{T}, \mathcal{E}>>>$
    & \tiny{Função de interpretação} 
    & \tiny{Uma regra necessária para definir o espaço conceitual, ``independentemente da ordem, mas também, ficcionalmente, enumerá-los em uma ordem particular, sob o controle de $\mathcal{T}$ -- isto é cricial para a simulação de um comportamento criativo de um $\mathcal{T}$ particular \cite{wiggins_framework_2006} \tablefootnote{Tradução de \emph{We need a means not just of defining the conceptual space, irrespective of order, but also, at least notionally, of enumerating it, in a particular order, under the control of $\mathcal{T}$ -- this is crucial to the simulation of a particular creative behaviour by a particular $\mathcal{T}$.}}. ``Uma função que interpreta a estratégia transversal $\mathcal{T}$, informada por $\mathcal{R}$ e $\mathcal{E}$ . Opera sobre um subconjunto ordenado de $mathcal{U}$ (do qual tem acesso randômico) e resulta em outro subconjunto ordenado de $\mathcal{U}$.''\tablefootnote{Tradução de \emph{A function interpreting the traversal strategy $\mathcal{T}$, informed by $\mathcal{R}$ and $\mathcal{E}$ . It operates upon anordered subset of $mathcal{U}$ (of which it has random access) and results in another ordered subset of $\mathcal{U}$.}}} \\
    \hline
    \hline
   
    \end{tabular}
\label{tab:universodeconceitos}
\end{table}

\citeonline{mclean_music_2006} ainda descreve regras que validam concepções diferentes entre espaços conceituais $\mathcal{C}$ diversos em um Universo de Conceitos $\mathcal{U}$ (ver \autoref{tab:universodeconceitos}). McLean realiza uma comparação entre o \emph{Universo de possibilidades} de Wiggins com o \emph{Modelo de Improvisação} de Pressing \ver{sec:im}. No entanto, McLean argumenta que:

\begin{citacao}
Pressing discute comportamento criativo no contexto do Modelo de Improvisação, e de fato é parte do Quadro conceitual de Sistemas Criativos. (\ldots) Durante a transferência de notação do Modelo de Improvisação para a Ferramenta de Sistemas Criativos, nós consideramos improvisação musical de uma maneira clara e temos uma linguagem comum na qual comparar com outros modelos \footnote{Tradução de \emph{However Pressing does discuss creative behaviour in the context of the IM, and indeed the CSF is in part. (\ldots) In transferring the IM to the notation of the CSF we may consider music improvisation in a clearer manner and have a common language in which to compare it with other models.}}.
\end{citacao}


\subsection{O modelo de improvisação}\label{sec:im}

Segundo Pressing, o Modelo de Improvisação é ``um esboço para uma teoria geral da improvisação integrada com preceitos da Psicologia Cognitiva (\ldots) teoria do comportamento de improvisação na música'' \cite[p.~2]{pressing_improvisation_1987}. Este modelo será utilizado para especificar elementos de uma performance exemplar, como o caso investigado neste trabalho. Por exemplo, uma improvisação particionada em diferentes sequências pode ser parcialmente mapeada em categorias, como blocos sonoros, referentes conceituais e normas estilísticas, conjuntos de objetivos e processos. Este nos pareceu um modelo mais transparente para o compositor, músico e intérprete. O que não quer dizer que é possível readequar ambos para nosso interesse. Um sumário sobre o modelo de improvisação é apresentado na \autoref{tab:modelo_improvisacao}. Por seu caráter lógico, parece ser uma possibilidade interessante, e assumiremos como tal.

\begin{table}[!h]
\caption{Definições formais do Modelo de improvisação de Jeff \citeonline{pressing_improvisation_1987}, segundo \citeonline[p.~2]{mclean_music_2006}.}
\small
    \begin{tabular}{ | p{6cm} | p{9cm} |}
    \hline 
    \hline 

    \tiny{Representação}   
    & \tiny{Significado} \\
    \hline

    $E'$
    & \tiny{Um bloco de eventos sonoros}\tablefootnote{\emph{A cluster of sound events}.} \\
    \hline

    $K'$
    & \tiny{Uma seqüência de blocos de eventos E, onde um bloco de eventos não se sobrepõe com o seguinte}\tablefootnote{A sequence of E event clusters, where event cluster onsets do not overlap with those of a following one}\\
    \hline

    $I'$
    & \tiny{Uma improvisação, particionada por interrupções em um número de K sequências}\tablefootnote{An improvisation, partitioned by interrupts into a number of K sequences} \\
    \hline

    $R'$
    & \tiny{Um referente opcional, tal como uma partitura ou uma norma estilística}\tablefootnote{An optional referent, such as a score or stylistic norm} \\
    \hline

    $G'$
    & \tiny{Um conjunto de objetivos }\tablefootnote{A set of current goals.} \\
    \hline

    $M'$
    & \tiny{Uma memória de longo prazo}\tablefootnote{Long term memory.} \\
    \hline

    $O'$
    & \tiny{Um conjunto de objetos}\tablefootnote{An array of objects.} \\
    \hline

    $F'$
    & \tiny{Um conjunto de características dos objetos}\tablefootnote{An array of objects Features.} \\
    \hline

    $P'$
    & \tiny{Um conjunto de processos}\tablefootnote{An array of Process} \\
    \hline
    \hline
   
    \end{tabular}
\label{tab:modelo_improvisacao}
\end{table}

\begin{figure}[!h]
  \centering
  \includegraphics[scale=0.7]{imagens/contido.png}
  \caption{Representação da justaposição  entre dois epaços conceituais. A região em marrom representa um grupo de conceitos transitórios, bem como os limites desta transição. \textbf{Fonte}: autor. }
  \label{fig:contido}
\end{figure}

\section{Diagramação dos espaços conceituais}\label{sec:diagrama}

\newcommand{\csfeq}[2]{
\mathcal{#1}_\emph{#2}
}

\newcommand{\unionspaces}[6]{
\csfeq{#1}{#2} = \csfeq{#3}{#4} \bigcup \csfeq{#5}{#6}
}

\newcommand{\listspaces}[9]{
\csfeq{#1}{#2}~=~[\csfeq{#3}{#2},~\csfeq{#4}{#2},~\csfeq{#5}{#2},~\csfeq{#6}{#2},~\csfeq{#7}{#2},~\csfeq{#8}{#2},~\csfeq{#9}{#2}
}

Formalmente, a figura acima pode ser representada como na \autoref{eq:def} , se desconsiderarmos qualquer outros espaços conceituais.

\begin{example}{Representação formal da \autoref{fig:contido}}
\begin{equation}
\unionspaces{C}{Study in Keith}{C}{live coding}{C}{Sun Bears}
\label{eq:def}
\end{equation}
\end{example}

Este grupo também pode ser descrito como uma lista de propriedades como na \autoref{eq:def2}:

\begin{example}{Representação formal das propriedades da \autoref{fig:contido}}
\begin{equation}
\listspaces{C}{SK}{E'}{K'}{I'}{R'}{G'}{M'}{O'}{F'},~\csfeq{P'}{SK}]
\label{eq:def}
\end{equation}
\end{example}
  
Nos diagramas abaixo, $C_\emph{\ldots}$ representa qualquer espaço conceitual abstrato (que pode incluir outro previamente apresentado). Entre os elementos iniciais (raízes, vermelho) e transitórios (nós, azul), ocorrem as ramificações (ramos, linhas pretas), isto é, a exploração de conceitos dentro de outros conceitos. De um lado, a aplicação de regras de validação sobre o universo conceitual da pesquisa (tudo aquilo que foi produzido em dois anos de mestrado) gerou o espaço conceitual desta tese. Estas regras de validação foram, em sua maior parte, os processos de orientação e qualificação. Em outras palavras, \csf{C}{pesquisa}$=[[$\csf{R}{pesquisa}$]]($\csf{U}{pesquisa}$)$.

\begin{example}{Representação do universo conceitual da \emph{pesquisa}}

O Universo de Conceitos da pesquisa, \csf{U}{pesquisa}, é um recorte do universo conceitual da música, \csf{U}{música}:

\begin{tikzpicture}
  [
    grow                    = right,
    sibling distance        = 6em,
    level distance          = 10em,
    edge from parent/.style = {draw, -latex},
    every node/.style       = {font=\footnotesize},
    sloped
  ]
  \node [root] {\csf{U}{Música}}
    child { node [env] {\csf{U}{pesquisa}}
      child { node [env] {\csf{U}{livecoding}}}
    }
    child { node [env] {\csf{C}{\ldots}}};
\end{tikzpicture}

No primeiro capítulo, incluímos um subjconjunto neste Espaço Conceitual da Pesquisa \ver{app:A}). 

\begin{tikzpicture}
  [
    grow                    = right,
    sibling distance        = 6em,
    level distance          = 10em,
    edge from parent/.style = {draw, -latex},
    every node/.style       = {font=\footnotesize},
    sloped
  ]
  \node [root] {\csf{C}{pesquisa}}
    child { node [env] {\csf{C}{livecoding}}
      child { node [env] {\csf{C}{\ldots}}}
      child { node [env] {\csf{C}{ICLC}}}
    }
    child { node [env] {\csf{C}{\ldots}}}; 
\end{tikzpicture}

Podemos inclur elementos históricos, o período transitório entre 1970 e 2000 (\emph{circa}), onde emanciparam as práticas e as regras heurísticas.  

\begin{tikzpicture}
  [
    grow                    = right,
    sibling distance        = 6em,
    level distance          = 10em,
    edge from parent/.style = {draw, -latex},
    every node/.style       = {font=\footnotesize},
    sloped
  ]
  \node [root] {\csf{C}{livecoding}}
    child { node [env] {\csf{C}{Elementos Históricos}}
      child {node [env] {\csf{C}{Proto-História}}}
      child {node [env] {\csf{C}{Manifestos}}}
    }
    child { node [env] {\csf{C}{\ldots}}};
\end{tikzpicture}

Por último, \csf{C}{pesquisa} investiga o \emph{live coding} a partir de um caso específico:

\begin{tikzpicture}
  [
    grow                    = right,
    sibling distance        = 6em,
    level distance          = 10em,
    edge from parent/.style = {draw, -latex},
    every node/.style       = {font=\footnotesize},
    sloped
  ]
  \node [root] {\csf{C}{pesquisa}}
    child { node [env] {\csf{C}{livecoding}}
      child { node [env] {\csf{C}{\ldots}}}
      child { node [env] {\csf{C}{Sessão de Improvisação}}
        child { node [env] {\csf{C}{Study in Keith}}}
        child { node [env] {\csf{C}{\ldots}}}
      }
    }
    child { node [env] {\csf{C}{\ldots}}}; 
\end{tikzpicture}
\end{example}

Por outro lado \csf{C}{Study in Keith} pode ser definido pelo modelo de improvisação de Pressing (\autoref{tab:modelo_improvisacao}, \pageref{tab:modelo_improvisacao}).

\begin{example}{Representação do modelo de improvisação para \emph{Study in Keith}.}
\begin{tikzpicture}
  [
    grow                    = right,
    sibling distance        = 6em,
    level distance          = 10em,
    edge from parent/.style = {draw, -latex},
    every node/.style       = {font=\footnotesize},
    sloped
  ]
  \node [root] {\footnotesize \csf{C}{Study in Keith}}
    child { node [env] {\footnotesize \csf{E'}{Study in Keith}}}
    child { node [env] {\footnotesize \csf{K'}{Study in Keith}}}
    child { node [env] {\footnotesize \csf{I'}{Study in Keith}}}
    child { node [env] {\footnotesize \csf{R'}{Study in Keith}}}
    child { node [env] {\footnotesize \csf{G'}{Study in Keith}}}
    child { node [env] {\footnotesize \csf{O'}{Study in Keith}}}
    child { node [env] {\footnotesize \csf{F'}{Study in Keith}}}
    child { node [env] {\footnotesize \csf{P'}{Study in Keith}}}; 
\end{tikzpicture}
\end{example}

\section{Formalização}\label{sec:formaliza}

O espaço conceitual do \emph{livecoding} é definido como uma função de interpretação das regras de validação (o que pode ser ou não considerado como próprio de uma categorização musical), de gosto (questões de estilo) e de localização transversal de conceitos (conceitos internos que permitem o cruzamento com outros conceitos). As regras de validação foram estudadas neste trabalho como as regras heurísticas do \emph{live coding}. Isto é, que conjunto de métodos são utilizados para caracterizar uma performance de \emph{live coding} como tal? Elementos históricos, e ideológicos (divulgados em manifestos), são levantados para responder esta pergunta. Por outro lado, este estudo abandonou a investigação das regras de gosto, tema que pode ser melhor explorado em trabalhos posteriores, a partir de \citeonline{janotti_jr._a_2003,sa_musica_2006,sa_se_2009}. A tarefa de localização transversal de conceitos é trabalhada no último capítulo. O espaço conceitual de \emph{Study in Keith} está contido no espaço conceitual do \emph{live coding} através da união entre os conceitos deste último, com os espaços conceituais dos concertos \emph{Sun Bears}, de Keith Jarret, misturados. No entanto o espaço conceitual não será investigado em sua totalidade, e sim apenas uma sonoridade.