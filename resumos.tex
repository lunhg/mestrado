% resumo em português
\setlength{\absparsep}{18pt} % ajusta o espaçamento dos parágrafos do resumo
\begin{resumo}
Apresento o seguinte tema: quais categorizações músicais estão incluídas em , improvisações feitas com computadores, mais especificamente em uma prática conhecida por \textit{live coding}.

Descrevo como uma prática multidisciplinar e assimétrica em sua variedade. De fato, observo o \emph{live coding} como um Programa de Investigação Científica emergente em Nortes políticos/econômicos. Este programa, multidisciplinar, enquadra Ciências da Computação e Artes (Música, Audiovisual, Dança). O método de pesquisa, mais observado no corpus de textos (e neste trabalho), até o momento, se apresentou demasiadamente cirúrgico.

Nesse sentido, realizo a atividade (preliminar) de abordar  o \emph{live coding} como uma \emph{ecologia de saberes}, buscando na medida do possível, diminuir a assimetria epistemológica notificada. Direciono, na Parte I, uma discussão sobre assimetrias de definições do \emph{live coding}, de significados, de precurssores, e de práticas recentes. Direciono, na Parte II, uma discussão sobre as assimetrias de gênero musical, em um ambiente restrito, uma mídia social conhecida como \emph{Soundcloud}.

\textbf{Palavras-chaves}: \textit{Livecoding}. Improvisação. Gêneros Musicais.
\end{resumo}

%%%%%%%%%% traduçoes resumo
\begin{comment}
% resumo em inglês
\begin{resumo}[Abstract]
 \begin{otherlanguage*}{english}
   This is the english abstract.

   \vspace{\onelineskip}
 
   \noindent 
   \textbf{Key-words}: latex. abntex. text editoration.
 \end{otherlanguage*}
\end{resumo}

% resumo em francês 
\begin{resumo}[Résumé]
 \begin{otherlanguage*}{french}
    Il s'agit d'un résumé en français.
 
   \textbf{Mots-clés}: latex. abntex. publication de textes.
 \end{otherlanguage*}
\end{resumo}

% resumo em espanhol
\begin{resumo}[Resumen]
 \begin{otherlanguage*}{spanish}
   Este es el resumen en español.
  
   \textbf{Palabras clave}: latex. abntex. publicación de textos.
 \end{otherlanguage*}
\end{resumo}
% ---
\end{comment}