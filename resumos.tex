% resumo em português
\setlength{\absparsep}{18pt} % ajusta o espaçamento dos parágrafos do resumo
\begin{resumo}

Este documento é uma nota pedagógica para interessados em improvisação de códigos. Indicamos eventos históricos ligados à Música, e situamos a análise de um trecho exemplar (e pedagógico) desta história. Através do Quadro Conceitual de Sistemas Criativos de \citeonline{mclean_music_2006,wiggins_framework_2006,McLean2011} analisamos o primeiro algoritmo gerador de uma sonoridade tonal em \emph{A Study in Keith} de Andrew \citeonline{sorensen_keith_2009}. Por outro lado, consideramos este exemplo representativo por vestir o programador com a figura de intérprete musical, e ao mesmo tempo, manter o espírito do pianista concertista em um ambiente rico de concepções musicais.

\vspace{\onelineskip}
\noindent
\textbf{Palavras-chaves}: \textit{Livecoding}. \emph{Study in Keith}. Sistemas criativos.
\end{resumo}

%%%%%%%%%% traduçoes resumo
% resumo em inglês
\begin{comment}
\begin{resumo}[Abstract]
 \begin{otherlanguage*}{english}

   \vspace{\onelineskip}
 
   \noindent 
   \textbf{Key-words}: latex. abntex. text editoration.
 \end{otherlanguage*}
\end{resumo}

% resumo em francês 
\begin{comment}

\begin{resumo}[Résumé]
 \begi'n{otherlanguage*}{french}
    Il s'agit d'un résumé en français.
 
   \textbf{Mots-clés}: latex. abntex. publication de textes.
 \end{otherlanguage*}
\end{resumo}

% resumo em espanhol
\begin{resumo}[Resumen]
 \begin{otherlanguage*}{spanish}
   Este es el resumen en español.
  
   \textbf{Palabras clave}: latex. abntex. publicación de textos.
 \end{otherlanguage*}
\end{resumo}
% ---
\end{comment}