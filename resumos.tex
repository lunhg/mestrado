% resumo em português
\setlength{\absparsep}{18pt} % ajusta o espaçamento dos parágrafos do resumo
\begin{resumo}

Este documento é uma nota para interessados em improvisação de códigos. Talvez mais para aqueles interessados em procedimentos de pedagogia musical do código do que uma análise completa de uma ou duas obras. Indicamos eventos históricos da improvisação de códigos ligados à Música para situar a análise de um trecho exemplar (e pedagógico) desta história. Buscamos no Sistema Criativo de \citeonline{mclean_music_2006,wiggins_framework_2006} analisar o primeiro algoritmo gerador de uma sonoridade tonal em \emph{A Study in Keith} de Andrew \citeonline{sorensen_keith_2009}. Buscamos oferecer um texto em língua portuguesa sobre o \emph{live coding}, através de um exemplo que consideramos representativo por: i) uma manutenção de conhecimentos musicais tonais; ii) sua pedagogia de programação através da música; iii) uma possibilidade de transposição destes algoritmos para o piano.

\vspace{\onelineskip}
\noindent
\textbf{Palavras-chaves}: \textit{Livecoding}. \emph{Study in Keith}. Sistemas criativos.
\end{resumo}

%%%%%%%%%% traduçoes resumo
% resumo em inglês
\begin{comment}
\begin{resumo}[Abstract]
 \begin{otherlanguage*}{english}

   \vspace{\onelineskip}
 
   \noindent 
   \textbf{Key-words}: latex. abntex. text editoration.
 \end{otherlanguage*}
\end{resumo}

% resumo em francês 
\begin{comment}

\begin{resumo}[Résumé]
 \begi'n{otherlanguage*}{french}
    Il s'agit d'un résumé en français.
 
   \textbf{Mots-clés}: latex. abntex. publication de textes.
 \end{otherlanguage*}
\end{resumo}

% resumo em espanhol
\begin{resumo}[Resumen]
 \begin{otherlanguage*}{spanish}
   Este es el resumen en español.
  
   \textbf{Palabras clave}: latex. abntex. publicación de textos.
 \end{otherlanguage*}
\end{resumo}
% ---
\end{comment}