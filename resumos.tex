
% resumo em português
\setlength{\absparsep}{18pt} % ajusta o espaçamento dos parágrafos do resumo
\begin{resumo}

Primeiramente, Fora Temer. 

Este documento apresenta uma versão sintetizada de uma técnica polivalente cujo nome é \emph{live coding}, suas construções históricas na Música, e uma simulação de improvisação tonal guiada por improvisação com linguagens de programação.

Na Introdução (ver p.~\pageref{cap:intro}) apresentamos uma definição de \emph{live coding}. A definição destaca o fazer musical, mas não exclúi outras potências artísticas. 

No \autoref{cap:introducao} (ver p.~\pageref{cap:introducao}) destacamos um mecanismo criativo desta técnica em dois contextos não musicais.

No \autoref{cap:protohistoria} (ver p.~\pageref{cap:protohistoria}) listamos  períodos de atividades musicais que prototiparam e formalizaram o mecanismo criativo do primeiro capítulo.

No \autoref{cap:estudos_de_caso} (ver p.~\pageref{cap:estudos_de_caso}) analisamos uma proposta de ação, 
um vídeo entitulado \emph{A Study in Keith}, de \citeonline{sorensen_keith_2009}, a partir de uma ferramenta de análise deste mecanismo.

Uma é a contribuição deste trabalho para a musicologia brasileira: um documento que organiza conceitos básicos de uma técnica ainda pouco elaborada em território nacional.

\vspace{\onelineskip}
\noindent
\textbf{Palavras-chaves}: Improvisação de códigos. Música Computacional, \emph{A Study in Keith}.
\end{resumo}

%%%%%%%%%% traduçoes resumo
% resumo em inglês
\begin{comment}
\begin{resumo}[Abstract]
 \begin{otherlanguage*}{english}

   \vspace{\onelineskip}
 
   \noindent 
   \textbf{Key-words}: latex. abntex. text editoration.
 \end{otherlanguage*}
\end{resumo}

% resumo em francês 
\begin{comment}

\begin{resumo}[Résumé]
 \begi'n{otherlanguage*}{french}
    Il s'agit d'un résumé en français.
 
   \textbf{Mots-clés}: latex. abntex. publication de textes.
 \end{otherlanguage*}
\end{resumo}

% resumo em espanhol
\begin{resumo}[Resumen]
 \begin{otherlanguage*}{spanish}
   Este es el resumen en español.
  
   \textbf{Palabras clave}: latex. abntex. publicación de textos.
 \end{otherlanguage*}
\end{resumo}
% ---
\end{comment}