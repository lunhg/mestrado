% resumo em português
\setlength{\absparsep}{18pt} % ajusta o espaçamento dos parágrafos do resumo
\begin{resumo}
Este trabalho é um esqueleto da dissertação para a qualificação de mestrado em Artes Visuais, Música e Tecnologia. Apresentarei propostas de divisões de capítulos relacionadas ao seguinte tema: que tipo de músicas são, em categorias de gênero musical, improvisações feitas com computadores, uma prática que é conhecida por \textit{live coding}. Este trabalho é um resultado parcial da pesquisa realizada. Descreverei brevemente um conjunto de conhecimentos que adquiri no percurso do ler, escrever, pensar, programar e conversar com os Professores Dr. Luiz Eduardo Castelões e Dr. Alexandre Fenerich: algumas definições, grupos, \textit{softwares} e categorias de gêneros musicais encontradas na prática em questão.

Não é difícil entender que os primeiros computadores possuíam restrições técnicas que impediam uma performance imediatista, parecida com um instrumento musical tradicional. O manejo com máquina mais antigas por diversos pesquisadores, necessitava de um tempo consideravelmente maior que aquele que podemos experenciar: para que um código de computador pudesse resultar em som emitido por alto-falantes de um \emph{mainframe}, foi necessária a emergência de uma comunidade de programadores, cientistas e compositores. No \emph{live coding}, para que a mesma tarefa fosse  realizada de maneira relativamente trivial, outros grupos emergiram nos anos 80 e, mais recentemente, nos últimos 15 anos.

Um grau de vivacidade com o computador, no seu uso, é presente no dia-a-dia e, por quê não, como música? Quais questões são pertinentes ao fazer musical?  Observações e apontamentos preliminares dos problemas colocados nos dois parágrafos anteriores, foram planejados para 3 capítulos: \begin{inparaenum}[\itshape 1)\upshape]
\item discussão sobre o que é \textit{livecoding} (Parte I); 
\item um panorama histórico da música realizada pelo computador, como pano de fundo para algumas questões apontadas no capítulo anterior (Parte I);
\item análise quantitativa de questões de gênero musical abordados no Capítulo 1 (Parte II).
\end{inparaenum}

No sumário serão utilizados os algarismos Romanos I e II para separar as partes desta dissertação e algarismos arábicos para os capítulos, seções e subseções.

\textbf{Palavras-chaves}: \textit{Livecoding}. Improvisação. Gêneros Musicais.
\end{resumo}

%%%%%%%%%% traduçoes resumo
\begin{comment}
% resumo em inglês
\begin{resumo}[Abstract]
 \begin{otherlanguage*}{english}
   This is the english abstract.

   \vspace{\onelineskip}
 
   \noindent 
   \textbf{Key-words}: latex. abntex. text editoration.
 \end{otherlanguage*}
\end{resumo}

% resumo em francês 
\begin{resumo}[Résumé]
 \begin{otherlanguage*}{french}
    Il s'agit d'un résumé en français.
 
   \textbf{Mots-clés}: latex. abntex. publication de textes.
 \end{otherlanguage*}
\end{resumo}

% resumo em espanhol
\begin{resumo}[Resumen]
 \begin{otherlanguage*}{spanish}
   Este es el resumen en español.
  
   \textbf{Palabras clave}: latex. abntex. publicación de textos.
 \end{otherlanguage*}
\end{resumo}
% ---
\end{comment}